\chapter{Requisites}

\section{Bindu (Marking)}

Before use, a new robe must be marked with
(three) dot(s), blue (-green), black or brown
in colour, saying, either out loud or mentally:
“Imaṃ bindukappaṃ karomi.” (×3)
(“I make this properly marked.”)
[cf. Vin,IV,120]

\section{Adhiṭṭhāna (Determining)}

“Imaṃ *saṅghāṭiṃ* adhiṭṭhāmi.”
(“I determine this outer robe.”)

For ‘saṅghāṭiṃ’ substitute item as appropriate:
*uttarā-saṅgaṃ (upper robe)
*antara-vāsakaṃ (lower robe)
*pattaṃ (alms bowl)
*nisīdanaṃ (sitting-cloth)
*kaṇḍu-paṭicchādiṃ (skin-eruption covering cloth)
*vassika-sāṭikaṃ (rains cloth)
*paccattharaṇaṃ (sleeping cloth)
*mukha-puñchana-colaṃ (handkerchief)
*parikkhāra-colaṃ (small requisite). [Sp,III,643f]

♦ T h e f ir st th r e e a r tic le s m u st b e p r o p e r l y
marked before being determined for use. Only
one of each of these items may be determined

at any one time.
♦The rains cloth may be used only during the
[Sp,III,644]
four months of the Rains.
♦There is no limit to the number of articles
which may be determined in each of the last
three categories above, e.g.:
“Imāni *paccattharaṇāni* adhiṭṭhāmi.”
(“I determine these sleeping cloths.”)

Substitute ‘mukhapuñchana-colāni’ (handkerchiefs) or ‘parikkhāra-colāni’ (small requi[Sp,III,645]
sites) as appropriate.
• Articles are determined either by touching

the article and mentally reciting the relevant
Pali passage, or by uttering the Pali passage
w ith o u t to u ch in g t he a rt icl e. I n t he la t te r
case, if the article is beyond forearm’s length:
“Imaṃ” →“etaṃ” ; “imāni” →“etāni”
(this)

(that)

(these)

(those)
[Sp,III,643]

\section{Paccuddharaṇa (Relinquishing)}

When an outer robe, upper robe, lower robe,
alms bowl or sitting-cloth is to be replaced,
the article already determined must first be
relinquished from use:
“Imaṃ saṅghāṭiṃ paccuddharāmi.”
(“I relinquish this outer robe.”) [Sp,III,643]

Substitute the appropriate item for ‘saṅghāṭiṃ’.

♦Apart from relinquishing from use, a deter-

mined article ceases to be determined if it is
given to another, is stolen, is taken on trust by
a friend, or has a large visible hole in it.

\section{Vikappana (Sharing Ownership)}


There are varied practices about sharing ownership. Here are the most common
ways.

There are two formalae for sharing ownership
in the presence of the second owner:
(i) In the presence of the receiving bhikkhu,
and with the article within forearm’s length:
“Imaṃ cīvaraṃ tuyhaṃ vikappemi.”
(“I share this robe with you.”)

“Imāni cīvarāni tuyhaṃ vikappemi.”
(“… these robes…”)

“Imaṃ pattaṃ tuyhaṃ vikappemi.”
(“… this bowl…”)

“Ime patte tuyhaṃ vikappemi.”
(“… these bowls…”)

• When the receiving bhikkhu is the senior:
“tuyhaṃ” →“ā yasmato”
• When it is shared with more than one bhikkhu:
“tuyhaṃ” →“tumhākaṃ”
• When the article is beyond forearm’s length:

“imaṃ” →“etaṃ”; “imāni” →“etāni”;
“ime” →“ete”
[Vin,IV,122]

(ii) In the presence of the receiving bhikkhu
(who is named, e.g., “Uttaro”), and with the
article within forearm’s length, one says to
another bhikkhu:
“Imaṃ cīvaraṃ uttarassa bhikkhuno vikappemi.”
(“I share this robe with Uttaro Bhikkhu.”)

• When the receiving bhikkhu is the senior:
“uttarassa bhikkhuno” →“āyasmato uttarassa”
• If it is shared with a novice:
“uttarassa bhikkhuno”→“uttarassa sāmaṇerassa”
♦To share a bowl: “cīvaraṃ” → “pattaṃ”
• If more than one article is to be shared sub-

stitute the plural form as in (i) above.
• When the item is beyond forearm’s length
[Vin,IV,122]
substitute as in (i) above.
(iii) In the absence of the receiving bhikkhus,
say to a witness:
“Imaṃ cīvaraṃ vikappanatthāya tuyhaṃ dammi.”
(“I give this robe to you for the purpose of sharing.”)

The witness should then ask the original
owner the names of two bhikkhus or novices
who are his friends or acquaintances:
“Ko te mitto vā sandiṭṭho vā.”
(“Who is your friend or acquaintance?”)

After the original owner tells their names, e.g.,
“Uttaro bhikkhu ca tisso sāmaṇero ca”
(“Bhikkhu Uttaro and Sāmaṇera Tisso ”)

The witness then says:
“Ahaṃ tesaṃ dammi.”
(“I give it to them.”)
or

“Ahaṃ uttarassa bhikkhuno ca tissassa
sāmaṇerassa dammi.”
(“I give it to Bhikkhu Uttarro and Sāmaṇera Tisso.)
[Vin,IV,122]

♦To share a bowl: “cīvaraṃ” →“pattaṃ”
• If more than one article is to be shared sub-

stitute the plural form as in (i) above.
• When the item is beyond forearm’s length
substitute as in (i) above.

\section{Vikappana-paccuddharaṇa (Relinquishing Shared Ownership)}

Before actually using the shared article, the
other bhikkhu must relinquish his share.
If the other bhikkhu is senior, and the article
is within forearm’s length:
“Imaṃ cīvaraṃ mayhaṃ santakaṃ paribhuñja
vā visajjehi vā yathāpaccayaṃ vā karohi.”
(“This robe of mine: you may use it, give it away, or
[cf. Kv,122]
do as you wish with it.”)

When more than one robe is being relinquished:
“imaṃ cīvaraṃ” →“imāni cīvarāni”
“santakaṃ” →“santakāni”
When the second owner is junior:

“paribhuñja” →“paribhuñjatha”
“visajjehi” →“visajjetha”
“karohi” →“karotha”
I f th e ar ti cle ( s) i s ( a re ) b e yo n d fo re a rm ’ s
length, change case accordingly:
“Imaṃ” →“etaṃ” ; “imāni” →“etāni”
(this)

(that)

(these)

(those)

• T o r e s ci n d t h e s h a r e d o w n e r s h i p o f c a s e

<4a.iii> above, the witness says:
“Tesaṃ santakaṃ paribhuñja vā vissajjehi vā
yathāpaccayaṃ vā karohi.”
(“Use what is theirs, give it away or do as you like
with it.”)

♦To rescind the shared ownership of a bowl:
“cīvaraṃ” →“pattaṃ”

and alter according to <4a.i> above.
♦ Th e p r a c t ic e o f s o m e c o m m u n it i e s w h e n

sharing ownership of a bowl is that permission is not required before using it. However,
if the first owner wishes to determine a
shared bowl, the second owner should relinquish it first.


