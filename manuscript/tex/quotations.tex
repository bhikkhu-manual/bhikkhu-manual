\chapter{Quotations}

\section*{Admonishing Another Bhikkhu}

Before admonishing another bhikkhu, a bhikkhu should investigate five conditions
in himself and establish five other conditions in himself.

He should investigate thus: ‘Am I one who practises purity in bodily action?;…
purity in speech?; is the heart of good-will established in me towards my
fellows?; am I one who has heard the Teachings, practised them, and penetrated
them with insight?; is the Discipline known and thoroughly understood by me?’

And he should establish these five conditions in himself: he should speak at the
right time, speak of facts, and speak gently, and he should speak only
profitable words, and with a kindly heart.

\suttaRef{A.V.78}

\section*{Wrong Livelihood for a Layperson}

Trade in weapons, in human beings, in animal flesh, in liquor, in poison.

\suttaRef{A.III.207}

\ifhandbookedition
\vspace*{-\baselineskip}
\fi

\section*{Dealing with Doubtful Matters (Kālāma-sutta)}

Be not led by report, by tradition, by hearsay, by the authority of texts, by
mere logic, by inference, by considering appearances, by its agreement with an
established theory, by seeming possibilities, by the idea ‘This is our teacher’.

\suttaRef{A.I.189}

\ifhandbookedition
\vspace*{-\baselineskip}
\fi

\section*{The Gradual Teaching}

Talk on: generosity and giving; morality; the ease and happiness of heavenly
realms; the disadvantages of sensual pleasures; the benefits of renouncing
sensual pleasures.

\suttaRef{D.I.148}

\ifhandbookedition
\vspace*{-\baselineskip}
\fi

\section*{The Three Cravings and the Four Attachments}

Craving for sensuality, craving for becoming, craving for non-becoming.

Attachment to sensuality, to views, to conduct and custom, and attachment to the
idea of self.

\suttaRef{D.III.230; M.I.66}

\ifhandbookedition
\vspace*{-\baselineskip}
\fi

\section*{The Three Universal Characteristics of Existence}

All conditioned phenomena are subject to change.
All conditioned phenomena are unsatisfactory.
All things are not-self.

\suttaRef{S.IV.1; Dhp.277–9}

\ifhandbookedition
\vspace*{-\baselineskip}
\fi

\section*{The Three Kinds of Suffering}

The suffering of pain (\emph{dukkha-dukkhatā}).
The suffering of conditioned phenomena (\emph{saṅkhāra-dukkhatā}).
The suffering of change (\emph{vipariṇāma-dukkhatā}).

\suttaRef{D.III.216; S.IV.259}

\ifhandbookedition
\vspace*{-\baselineskip}
\fi

\section*{The Three Characteristics of Conditioned Experience}

Its arising is apparent. Its passing away is apparent. While it persists,
alteration is apparent.

\suttaRef{A.I.152}

\ifhandbookedition
\vspace*{-1.1\baselineskip}
\enlargethispage{\baselineskip}
\fi

\section*{The Four Nutriments}

‘All beings are maintained by nutriment.’ The Four Nutriments are coarse,
material food; sense-contact food; mental-intention food; and consciousness
food.

\suttaRef{D.III.228; M.I.48; S.II.101}

\ifhandbookedition
\vspace*{-1.1\baselineskip}
\fi

\section*{The Four Bases of Judgement}

Judging and basing faith on form and outward appearance, on reputation and
beauty of speech, on ascetic and self-denying practices, on teaching and
righteous behaviour.

\suttaRef{A.II.71}

\ifhandbookedition
\vspace*{-1.1\baselineskip}
\fi

\section*{The Five Facts to be Frequently Contemplated}

I am subject to decay and I cannot escape it. I am subject to disease and I
cannot escape it. I am subject to death and I cannot escape it. There will be
division and separation from all that I love and hold dear. I am the owner of my
actions -- whatever I do, whether good or bad, I must be heir to it.

\suttaRef{A.III.71}

\ifhandbookedition
\vspace*{-\baselineskip}
\fi

\section*{The Five Qualities for a New Bhikkhu to Establish}

Restraint in accordance with the Pāṭimokkha; restraint of the senses; restraint
as regards talking; love of solitude; cultivation of right views.

\suttaRef{A.III.138}

\ifhandbookedition
\vspace*{-\baselineskip}
\fi

\section*{The Five Ways of Restraint (Saṃvara)}

Restraint by the Monastic Code of Discipline, by mindfulness, by knowledge, by
patience, by energy and effort.

\suttaRef{Vism. 7}

\ifhandbookedition
\vspace*{-\baselineskip}
\fi

\section*{The Six Attributes of Dhamma}

The Dhamma is well expounded by the Blessed One,
Apparent here and now, timeless, encouraging investigation,
Leading inwards, to be experienced individually by the wise.

\suttaRef{M.I.37; A.III.285}

\ifhandbookedition
\vspace*{-\baselineskip}
\fi

\section*{The Seven Qualities of a Wholesome Friend}

% (Kalyāṇamitta)

That individual is endearing; worthy of respect; cultured and worthy of
emulation; a good counsellor; a patient listener; capable of discussing profound
subjects; and is one who never exhorts groundlessly, not leading or spurring one
on to a useless end.

\suttaRef{A.IV.31}

\ifhandbookedition
\vspace*{-\baselineskip}
\fi

\section*{The Seven Things Favourable to Mental Development (Sappāya)}

% (Sappāya)

Suitable abode, location, speech, companion, food, climate, and posture.

\suttaRef{Vism. 127}

\ifhandbookedition
\vspace*{-\baselineskip}
\fi

\section*{The Seven Conditions Leading to the Welfare of the Sangha}

\emph{(The Mahā Parinibbāna Sutta introduces five sets of seven conditions on
  this topic. The first two are listed below.)}

(1) To hold regular and frequent meetings.

(2) To meet in harmony, to do the duties of the Sangha in harmony, and to disperse
in harmony.

(3) To introduce no revolutionary rules, break up no established rules, but
to train oneself in accordance with the prescribed training rules.

(4) To honour and respect those elders of long experience, the fathers and leaders
of the Sangha, and to deem them worthy of listening to.

(5) Not to fall under the influence of craving.

(6) To delight in forest dwelling.

(7) To establish oneself in mindfulness, with this thought: ‘May disciplined monks
who have not yet come, come here; and may those who have already come live in
comfort’.

% Also at: \suttaRef{A.IV.20–21}

Seven further conditions that lead to no decline:

(1) Not to be fond of activities;\\
(2) not to be fond of gossip;\\
(3) not to be fond of sleeping;\\
(4) not to be fond of socializing;\\
(5) not to have evil desires;\\
(6) not to have evil friends;\\
(7) not to be prematurely satisfied and rest content with early success.

\suttaRef{D.II.77–78}

\section*{The Eight Worldly Conditions (Loka-dhammā)}

Gain and loss, fame and obscurity, praise and blame, happiness and suffering.

\suttaRef{A.IV.157}

\ifhandbookedition
\vspace*{-\baselineskip}
\fi

\section*{The Eight Gifts of a Good Person (Sappurisa-dāna)}

To give clean things; to give well-chosen things; to give at the appropriate
time; to give proper things; to give with discretion; to give regularly; to calm
one's mind on giving; to be joyful after giving.

\suttaRef{A.IV.243}

\ifhandbookedition
\vspace*{-\baselineskip}
\fi

\section*{The Ten Perfections (Pāramī)}

Generosity; morality; renunciation; wisdom; energy; patience; truthfulness;
determination; loving-kindness; equanimity.

\suttaRef{Buddhavaṃsa v.6}

\ifhandbookedition
\vspace*{-\baselineskip}
\fi

\section*{The Ten Wholesome Courses of Action}

(1)~To avoid the destruction of life and aim for the welfare of all lives.

(2)~To avoid taking what belongs to others.

(3)~To avoid sexual misconduct.

(4)~To avoid lying, not knowingly speaking a lie for the sake of any advantage.

(5)~To avoid malicious speech, to unite the discordant, to encourage the united, and
to utter speech that makes for harmony.

(6)~To avoid harsh language and speak gentle, courteous and agreeable words.

(7)~To avoid frivolous talk; to speak at the right time, in accordance with facts,
what is useful, moderate and full of sense.

(8)~To be without covetousness.

(9)~To be free from ill-will, thinking, ‘Oh, that these beings were free from hatred
and ill-will, and would lead a happy life free from trouble’.

(10)~To possess right view, such as that gifts and offerings are not fruitless and
that there are results of wholesome and unwholesome actions.

\suttaRef{M.I.287; A.V.266; 275–278}

\ifhandbookedition
\vspace*{-\baselineskip}
\fi

\section*{The Ten Topics for Talk among Bhikkhus}

(1)~Talk favourable to wanting little;
(2)~to contentment;
(3)~to seclusion;
(4)~to not mingling together;
(5)~to strenuousness;
(6)~to good conduct;
(7)~to concentration;
(8)~to understanding and insight;
(9)~to deliverance;
(10)~and talk favourable to the knowledge and vision of deliverance.

\suttaRef{M.I.145; M.III.113; A.V.129}

\ifhandbookedition
\vspace*{-\baselineskip}
\fi

\section*{The Thirteen Austerities (Dhutaṅgā)}

(1)~Wearing rag-robes;
(2)~possessing only 3 robes;
(3)~eating only alms-food;
(4)~collecting alms-food house-to-house;
(5)~eating only at one sitting;
(6)~eating only from the bowl;
(7)~not accepting late-come food;
(8)~living in the forest;
(9)~living at the foot of a tree;
(10)~living in the open;
(11)~living in a cemetery;
(12)~being satisfied with whatever dwelling is offered;
(13)~abstaining from lying down to sleep.

\suttaRef{Vism. 59–83}

\ifhandbookedition
\clearpage
\fi

\section*{The Thirty-Eight Highest Blessings}

(1) Not to associate with fools;\\
(2) to associate with the wise;\\
(3) to honour those worthy of honour;\\
(4) living in a good environment;\\
(5) having formerly done meritorious deeds;\\
(6) setting oneself in the right course;\\
(7) having extensive learning;\\
(8) having skill and knowledge;\\
(9) being accomplished in discipline;\\
(10) being well-spoken;\\
(11) being supportive of mother and father;\\
(12) cherishing one's children;\\
(13) cherishing one's spouse;\\
(14) having an uncomplicated livelihood;\\
(15) being generous;\\
(16) having right conduct;\\
(17) rendering aid to relatives;\\
(18) behaving blamelessly;\\
(19) abstaining from and avoiding evil;\\
(20) abstaining from intoxicants;\\
(21) persevering in virtue;\\
(22) being respectful;\\
(23) being humble;\\
(24) being content;\\
(25) having gratitude;\\
(26) hearing the Dhamma;\\
(27) being patient;\\
(28) being amenable to correction;\\
(29) seeing spiritual seekers;\\
(30) discussing the Dhamma;\\
(31) having strenuous self-control;\\
(32) living the holy life;\\
(33) seeing the Noble Truths;\\
(34) realizing Nibbāna;\\
(35) being unshakable;\\
(36) being free from sorrow;\\
(37) having a mind undefiled;\\
(38) having a mind which is secure.

Those who have done these things see no defeat and go in safety everywhere: to
them these are the highest blessings.

\suttaRef{Snp. 259–268}

\ifhandbookedition
\vspace*{-\baselineskip}
\fi

\section*{The Root of All Things}

Rooted in desire are all things.
Born of attention are all things.
Arising from contact are all things.
Converging on feeling are all things.
Headed by concentration are all things.
Dominated by mindfulness are all things.
Surmountable by wisdom are all things.
Yielding deliverance as essence are all things.
Merging in the Deathless are all things.
Terminating in Nibbāna are all things.

When questioned by other wanderers, you should answer them thus.

\suttaRef{A.V.106}

