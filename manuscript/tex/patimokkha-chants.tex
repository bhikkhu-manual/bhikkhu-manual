\chapter{Pāṭimokkha Chants}

\chapter{Verses on the Training Code}

\begin{leader}
  [Handa mayaṃ ovāda-pāṭimokkha-gāthāyo bhaṇāmase]
\end{leader}

Sabba-pāpassa akaraṇaṃ\\
Kusalassūpasampadā\\
Sacitta-pariyodapanaṃ\\
Etaṃ buddhāna sāsanaṃ\\
Khantī paramaṃ tapo tītikkhā\\
Nibbānaṃ paramaṃ vadanti buddhā\\
Na hi pabbajito parūpaghātī\\
Samaṇo hoti paraṃ viheṭhayanto\\
Anūpavādo anūpaghāto\\
Pāṭimokkhe ca saṃvaro\\
Mattaññutā ca bhattasmiṃ\\
Pantañca sayan'āsanaṃ\\
Adhicitte ca āyogo\\
Etaṃ buddhāna sāsanaṃ

\chapter{Sīl’uddesa-pāṭho Uposath’āvasāne Sajjhāyitabbo}

Bhāsitam idaṃ tena Bhagavatā jānatā passatā\\
arahatā sammā-sambuddhena,\\
Sampanna-sīlā bhikkhave viharatha\\
sampanna-pāṭimokkhā,\\
Pāṭimokkha-saṃvara-saṃvutā viharatha\\
ācāra-gocara-sampannā,\\
Aṇu-mattesu vajjesu bhaya-dassāvī\\
samādāya sikkhatha sikkhāpadesū-ti.\\
Tasmā-tih’amhehi sikkhitabbaṃ,\\
Sampanna-sīlā viharissāma sampannapāṭimokkhā,\\
Pāṭimokkha-saṃvara-saṃvutā viharissāma\\
ācāra-gocara-sampannā,\\
Aṇu-mattesu vajjesu bhaya-dassāvī\\
samādāya sikkhissāma sikkhāpadesū-ti,\\
Evañ hi no sikkhitabbaṃ.

\chapter{The Verses of Tāyana}

\begin{leader}
  [Handa mayaṃ tāyana-gāthāyo bhaṇāmase]
\end{leader}

\begin{twochants}
  Chinda sotaṃ parakkamma & kāme panūda brāhmaṇa \\
  Nappahāya muni kāme & n'ekattam-upapajjati \\
  Kayirā ce kayirāthenaṃ & daḷham-enaṃ parakkame \\
  Sithilo hi paribbājo & bhiyyo ākirate rajaṃ \\
  Akataṃ dukkaṭaṃ seyyo & pacchā tappati dukkaṭaṃ \\
  Katañca sukataṃ seyyo & yaṃ katvā nānutappati \\
  Kuso yathā duggahito & hattham-evānukantati \\
  Sāmaññaṃ dupparāmaṭṭhaṃ & nirayāyūpakaḍḍhati \\
  Yaṃ kiñci sithilaṃ kammaṃ & saṅkiliṭṭhañca yaṃ vataṃ \\
  Saṅkassaraṃ brahma-cariyaṃ & na taṃ hoti mahapphalan'ti \\
\end{twochants}

\chapter{Sāmaṇera Sikkhā}

Anuññāsi kho Bhagavā,\\
Sāmaṇerānaṃ dasa sikkhā-padāni,\\
Tesu ca sāmaṇerehi sikkhituṃ:\\
Pāṇātipātā veramaṇī,\\
Adinn’ādānā veramaṇī,\\
Abrahma-cariyā veramaṇī,\\
Musā-vādā veramaṇī,\\
Surā-meraya-majja-pamādaṭṭhānā veramaṇī,\\
Vikāla-bhojanā veramaṇī,\\
Nacca-gīta-vādita-visūka-dassanā veramaṇī,\\
Mālā-gandha-vilepana-dhāraṇa-maṇḍanavibhūsanaṭṭhānā veramaṇī,\\
Uccā-sayana-mahā-sayanā veramaṇī,\\
Jāta-rūpa-rajata-paṭiggahaṇā veramaṇī-ti,

Anuññāsi kho Bhagavā,\\
Dasahi aṅgehi samannāgataṃ sāmaṇeraṃ\\
nāsetuṃ.\\
Katamehi dasahi?\\
Pāṇātipātī hoti,\\
Adinn’ādāyī hoti,\\
Abrahma-carī hoti,\\
Musā-vādī hoti,\\
Majja-pāyī hoti,\\
Buddhassa avaṇṇaṃ bhāsati,\\
Dhammassa avaṇṇaṃ bhāsati,\\
Saṅghassa avaṇṇaṃ bhāsati,\\
Micchā-diṭṭhiko hoti,\\
Bhikkhunī-dūsako hoti,\\
Anuññāsi kho Bhagavā,\\
Imehi dasahi aṅgehi samannāgataṃ\\
sāmaṇeraṃ nāsetun-ti.

Anuññāsi kho Bhagavā,\\
Pañcahi aṅgehi samannāgatassa sāmaṇerassa\\
daṇḍa-kammaṃ kātuṃ.\\
Katamehi pañcahi?\\
Bhikkhūnaṃ alābhāya parisakkati,\\
Bhikkhūnaṃ anatthāya parisakkati,\\
Bhikkhūnaṃ anāvāsāya parisakkati,\\
Bhikkhū akkosati paribhāsati,\\
Bhikkhū bhikkhūhi bhedeti,\\
Anuññāsi kho Bhagavā,\\
Imehi pañcahi aṅgehi samannāgatassa\\
sāmaṇerassa daṇḍa-kammaṃ kātun-ti.

