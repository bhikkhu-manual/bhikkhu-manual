\chapter{Useful Notes}

\section{Invitation to Request}

An invitation to request (pavāraṇā), unless
otherwise specified, lasts at most four months.
One may make requests of blood-relatives (but
not in-laws) without receiving an invitation.
One may give special help to one’s parents as
well as to one’s steward and to anyone prepar[Vin,IV,101–104]
ing to become a bhikkhu.
2. Abusive Speech
The bases of abuse are rank of birth, personal
name, clan name, work, art, disease, physical
appearance, mental stains, faults, and other
bases. There are both direct abuse and sarcasm
and ridicule. Abusive speech may be a base for
either expiation (or wrong-doing) or, when
[Vin,IV,4–11]
only teasing, for dubbhāsita.
3. Unallowable Meats
The flesh of humans (this is a base for thullaccaya), elephants, horses, dogs, snakes, lions,
tigers, leopards, bears, and panthers.
[Vin,I,218–219]

Also unallowable is flesh incompletely cooked,

and meat from an animal seen, heard or suspected to have been killed in order that its meat
be offered to bhikkhus.
[Vin,I,218–219]

\section{Admonishing Another Bhikkhu}

Before admonishing another bhikkhu, a
bhikkhu should investigate five conditions in
himself and establish five other conditions in
himself.
He should investigate thus: “Am I one who
practises purity in bodily action?;… purity in
speech?; is the heart of good-will established
in me towards my fellows?; am I one who has
heard the Teachings, practised them, and penetrated them with insight?; is the Discipline
known and thoroughly understood by me?”
And he should establish these five conditions
in himself: he should speak at the right time,
speak of facts, and speak gently, and he
should speak only profitable words, and with
[A,V,78]
a kindly heart.

\section{Wrong Livelihood for a Layperson}

Trade in weapons, in human beings, in animal
flesh, in liquor, in poison.
[A,III,207]
6. Dealing with Doubtful Matters (Kālāmasutta)
Be not led by report, by tradition, by hearsay,
by the authority of texts, by mere logic, by
inference, by considering appearances, by its

a g r e e m e n t w it h a n e s ta b l i s h e d t h e o r y , b y
seeming possibilities, by the idea ‘This is our
teacher’.
[A,I,189]

\section{The Gradual Teaching}

Talk on: generosity and giving; morality; the
ease and happiness of heavenly realms; the disadvantages of sensual pleasures; the benefits of
renouncing sensual pleasures.
[Vin,I,15; D,I,148]

\section{The Last Words of the Buddha}

“Handadāni bhikkhave āmantayāmi vo,
vayadhammā saṅkhārā,
appamādena sampādetha”,
ayaṃ tathāgatassa pacchimā vācā.
“Now, take heed bhikkhus, I caution you thus:
decline-and-disappearance is the nature of all
conditions. Therefore strive on ceaselessly,
d i s c e r n i n g a n d a le rt ! ” T h e s e a re t h e f in a l
[D,II,156]
words of the Tathāgata.

\section{The 3 Cravings and the 4 Attachments}

Craving for sense-experience, craving for
being, craving for non-existence.
Attachment to sensuality, to views, to conduct
an d cu stom, and attachmen t to the way of
self.
[D,III,230; M,I,66]
10. The 3 Universal Characteristics of Experience
Every condition is necessarily impermanent
and must change and become otherwise.

Every cond ition is ne cess arily su ff ering, a
burden.
No-thing is the subject of experience.
[S,IV,1; Dhp,vv.277–9]

\section{The Three Kinds of Suffering}

The suffering of what is unpleasant or painful.
The suffering of what is subject to change and
so must become otherwise.
The suffering of experience determined by
conditions that determine oneself.
[D,III,216; S,IV,259]

\section{The 3 Characteristics of Conditioned Experience}

Its a rising is a ppa re nt. Its pa ssing a wa y is
apparent. While it persists, alteration or
change is apparent.
[A,I,152]

\section{The Four Nutriments}

“All beings are maintained by nutriment.” The
fou r Nu triments a re coa rse, material food;
sense-contact food; mental-intention food;
and consciousness food. [D,III,228; M,I,48; S,II,101]

\section{The Four Bases of Judgement}

Judging and basing faith on form and outward
appearance, on reputation and beauty of
speech, on ascetic and self-denying practices,
on teaching and righteous behaviour. [A,II,71]

\section{The Five Facts to be Frequently Contemplated}

“I am subject to decay and I cannot escape it.”
“I am subject to disease and I cannot escape it.”
“I am subject to death and I cannot escape it.”
“There will be division and separation from
all that I love or hold dear.”
“I am the owner of my actions—whatever I do,
whether good or bad, I must be heir to it.”
[A,III,71]

\section{The Five Qualities for a New Bhikkhu to Establish}

Restraint in accordance with the Pātimokkha;
restraint of the senses; restraint as regards
talking; love of solitude; cultivation of right
[A,III,138]
views.

\section{The Five Ways of Restraint}

Restraint by the Monastic Code of Discipline,
by mindfulness, by knowledge, by patience, by
energy and effort.
[Vism. 7]

\section{The Six Attributes of Dhamma}

Dhamma is well-expounded by the Awakened
One. It is visible here and now, non-temporal,
inviting one to come and see, leading onward
and inward, directly experiencable by the wise.
[M,I,37; A,III,285]

\section{The Seven Qualities of a Wholesome Friend}

(Kalyāṇamitta)

That individual is endearing; worthy of
respect; cultured and worthy of emulation; a
good counsellor; a patient listener; capable of
discussing profound subjects; and is one who
nev er e xhorts grou ndle ssly, not leading or
spurring one on to a useless end.
[A,IV,31]

\section{The 7 Things Favourable to Mental Development}

(Sappāya)

Suitable abode, location, speech, companion,
[Vism. 127]
food, climate, and posture.

\section{The Seven Conditions Leading to the Welfare of the Sangha}

(i) To hold regular and frequent meetings.
To meet in harmony, to do the duties of the
Sangha in harmony, and to disperse in harmony. To introduce no revolutionary ordinance,
break up no established ordinance, but to train
oneself in accordance with the prescribed training rules. To honour and respect those elders of
long experience, the fathers and leaders of the
Sangha, and to deem them worthy of listening
to. Not to fall under the influence of craving. To
delight in forest dwelling. To establish oneself
in mindfulness, with this thought: ‘May disciplined monks who have not yet come, come
here; and may those who have already come
live in comfort’.
(ii) Not to be fond of activities; not to be
fond of gossip; not to be fond of sleeping; not
to be fond of society; not to have evil desires;

not to have evil friends; not to be prematurely
satisfied and rest content with early success.
[D,II,77–78; A,IV,20–21]

\section{The Eight Utensils (aṭṭha-parikkhārā)}

The three robes, the bowl, a razor/sharp
knife, needle, belt, water-filter. [Ja,I,65; Da,I, 206]

\section{The Eight Worldly Conditions (Lokadhammā)}

G a in a n d l o s s , p ro m i n e n c e a n d o b s c u r it y ,
praise and blame, happiness and suffering.
[A,IV,157]

\section{The Eight Gifts of a Good Person}

To give clean things; to give well-chosen
things; to give at the appropriate time; to give
proper things; to give with discretion; to give
regularly; to calm one’s mind on giving; to be
joyful after giving.
[A,IV,243]

\section{The Ten Perfections}

Generosity; morality; renunciation; wisdom;
energy; patience; earnest-truth; determina[BV,v.6]
tion; loving-kindness; equanimity.

\section{The Ten Wholesome Courses of Action}

To avoid the destruction of life and be anxious for the welfare of all lives. To avoid taking what belongs to others. To avoid sexual
misconduct. To av oid lying, not knowingly
speaking a lie for the sake of any advantage.

To avoid malicious speech, to unite the
discordant, to encourage the united, and to
utter speech that makes for harmony. To
avoid harsh language and speak gentle, courteous and agreeable words. To avoid frivolous
talk; to speak at the right time, in accordance
with facts, what is useful, moderate and full
of sense. To be without covetousness. To be
free from ill-will, thinking, “Oh, that these
beings were free from hatred and ill-will, and
would lead a happy life free from trouble”. To
possess right view, such as that gifts and
offerings are not fruitless and that there are
results of wholesome and unwholesome
actions.
[M,I,287; A,V,266; 275–278]

\section{The Ten Reflections for One Gone Forth}

“I have come to a status different from that of
a layman (‘classlessness’). My livelihood is
bound up with others. I now have a way to
behave different to a layperson. Does my conduct lead to self-reproach? Does my conduct
lead to reproach from fellows in the holy life?
There must be separation from all that is dear
to me. I am the owner of my actions—whatever I do, whether good or bad, I must be heir
to it. How has my passing of the nights and
days been? Do I delight in a solitary place or
not? Have I developed any extraordinary qualities such that, when questioned in my latter
days by my fellows in the holy life, I shall not
be confounded?”
[A,V,57]

\section{The Ten Topics for Talk among Bhikkhus}

Talk favourable to wanting little; to contentment; to seclusion; to not mingling together;
to strenuousness; to good conduct; to concentration; to understanding and insight; to
deliverance; and talk favourable to the knowledge and vision of deliverance.
[M,I,145; M,III,113; A,V,129]

\section{The Thirteen Dhutaṅgā}

Wearing rag-robes; possessing only 3 robes;
eating only alms-food; collecting alms-food
house-to-house; eating only at one sitting; eating onl y from the bowl; not accepting latecome food; living in the forest; living at the
foot of a tree; living in the open; living in a
cemetery; being satisfied with whatever dwelling is offered; abstaining from lying down to
[Vism. 59–83]
sleep.

\section{The 38 Highest Blessings}

Not to associate with fools; to associate with
the wise; to honour those worthy of honour;
living in a good environment; having formerly
done meritorious deeds; setting oneself in the

right course; having extensive learning; having
skill and knowledge; being accomplished in
discipline; being well-spoken; being supportive
of mother and father; cherishing one’s children; cherishing one’s spouse; having an
uncomplicated livelihood; being generous;
having right conduct; rendering aid to relatives; behaving blamelessly; abstaining from
and avoiding evil; abstaining from intoxicants;
persevering in virtue; being respectful; being
humble; being content; having gratitude; hearing the Dhamma; being patient; being amenable to correction; seeing monks; discussing the
Dhamma; having strenuous self control; living
the holy life; seeing the Noble Truths; realizing
Nibbana; being unshakable; being free from
sorrow; having a mind undefiled; having a
mind which is secure.
Those who have done these things see no
defeat and go in safety everywhere: to them
these are the highest blessings.

[Sn.259–268]

