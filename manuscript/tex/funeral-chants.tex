\chapter{Funeral Chants}

\chapter{Dhamma-saṅgaṇī-mātikā}

Kusalā dhammā.\\
Akusalā dhammā.\\
Abyākatā dhammā.\\
Sukhāya vedanāya sampayuttā dhammā.\\
Dukkhāya vedanāya sampayuttā dhammā.\\
Adukkhamasukhāya vedanāya sampayuttā\\
dhammā.\\
Vipākā dhammā.\\
Vipāka-dhamma-dhammā.\\
N’eva vipāka na vipāka-dhamma-dhammā.\\
Upādinn’upādāniyā dhammā.\\
Anupādinn’upādāniyā dhammā.\\
Anupādinnānupādāniyā dhammā.\\
Saṅkiliṭṭha-saṅkilesikā dhammā.\\
Asaṅkiliṭṭha-saṅkilesikā dhammā.\\
Asaṅkiliṭṭhāsaṅkilesikā dhammā.\\
Savitakka-savicārā dhammā.\\
Avitakka-vicāra-mattā dhammā.\\
Avitakkāvicārā dhammā.\\
Pīti-saha-gatā dhammā.\\
Sukha-saha-gatā dhammā.\\
Upekkhā-saha-gatā dhammā.\\
Dassanena pahātabbā dhammā.\\
Bhāvanāya pahātabbā dhammā.\\
N’eva dassanena na bhāvanāya\\
pahātabbā dhammā.\\
Dassanena pahātabba-hetukā dhammā.\\
Bhāvanāya pahātabba-hetukā dhammā.\\
N’eva dassanena na bhāvanāya\\
pahātabba-hetukā dhammā.\\
Ācaya-gāmino dhammā.\\
Apacaya-gāmino dhammā.\\
N’ev’ācaya-gāmino nāpacaya-gāmino dhammā.\\
Sekkhā dhammā.\\
Asekkhā dhammā.\\
N’eva sekkhā nāsekkhā dhammā.\\
Parittā dhammā.\\
Mahaggatā dhammā.\\
Appamāṇā dhammā.\\
Paritt’ārammaṇā dhammā.\\
Mahaggat’ārammaṇā dhammā.\\
Appamāṇ’ārammaṇā dhammā.\\
Hīnā dhammā.\\
Majjhimā dhammā.\\
Paṇītā dhammā.\\
Micchatta-niyatā dhammā.\\
Sammatta-niyatā dhammā.\\
Aniyatā dhammā.\\
Magg’ārammaṇā dhammā.\\
Magga-hetukā dhammā.\\
Maggādhipatino dhammā.\\
Uppannā dhammā.\\
Anuppannā dhammā.\\
Uppādino dhammā.\\
Atītā dhammā.\\
Anāgatā dhammā.\\
Paccuppannā dhammā.\\
Atīt’ārammaṇā dhammā.\\
Anāgat’ārammaṇā dhammā.\\
Paccuppann’ārammaṇā dhammā.\\
Ajjhattā dhammā.\\
Bahiddhā dhammā.\\
Ajjhatta-bahiddhā dhammā.\\
Ajjhatt’ārammaṇā dhammā.\\
Bahiddh’ārammaṇā dhammā.\\
Ajjhatta-bahiddh’ārammaṇā dhammā.\\
Sanidassana-sappaṭighā dhammā.\\
Anidassana-sappaṭighā dhammā.\\
Anidassanāppaṭighā dhammā.

\chapter{Vipassanā-bhūmi-pāṭho}

Pañcakkhandhā:\\
Rūpakkhandho, vedanākkhandho,\\
Saññākkhandho, saṅkhārakkhandho,\\
Viññāṇakkhandho.\\
Dvā-das’āyatanāni:\\
Cakkhv-āyatanaṃ rūp’āyatanaṃ,\\
Sot’āyatanaṃ sadd’āyatanaṃ,\\
Ghān’āyatanaṃ gandh’āyatanaṃ,\\
Jivh’āyatanaṃ ras’āyatanaṃ\\
Kāy’āyatanaṃ phoṭṭhabb’āyatanaṃ\\
Man’āyatanaṃ dhamm’āyatanaṃ.\\
Aṭṭhārasa dhātuyo:\\
Cakkhu-dhātu rūpa-dhātu cakkhu-viññāṇa-dhātu,\\
Sota-dhātu sadda-dhātu sota-viññāṇa-dhātu,\\
Ghāna-dhātu gandha-dhātu ghāna-viññāṇa-dhātu,\\
Jivhā-dhātu rasa-dhātu jivhā-viññāṇa-dhātu,\\
Kāya-dhātu phoṭṭhabba-dhātu kāya-viññāṇa-dhātu,\\
Mano-dhātu dhamma-dhātu mano-viññāṇa-dhātu.\\
Bā-vīsat’indriyāni:\\
Cakkhu’ndriyaṃ sot’indriyaṃ ghān’indriyaṃ\\
jivh’indriyaṃ kāy’indriyaṃ man’indriyaṃ,\\
Itth’indriyaṃ puris’indriyaṃ jīvit’indriyaṃ,\\
Sukh’indriyaṃ dukkh’indriyaṃ\\
somanass’indri-yaṃ domanass’indriyaṃ\\
upekkh’indriyaṃ,\\
Saddh’indriyaṃ viriy’indriyaṃ sat’indriyaṃ\\
samādh’indriyaṃ paññ’indriyaṃ,\\
Anaññātañ-ñassāmī-t’indriyaṃ aññ’indriyaṃ\\
aññātāv’indriyaṃ.\\
Cattāri ariya-saccāni:\\
Dukkhaṃ ariya-saccaṃ,\\
Dukkha-samudayo ariya-saccaṃ,\\
Dukkha-nirodho ariya-saccaṃ,\\
Dukkha-nirodha-gāminī paṭipadā ariya-saccaṃ.\\
Avijjā-paccayā saṅkhārā,\\
Saṅkhāra-paccayā viññāṇaṃ,\\
Viññāṇa-paccayā nāma-rūpaṃ,\\
Nāma-rūpa-paccayā saḷ-āyatanaṃ,\\
Saḷ-āyatana-paccayā phasso,\\
Phassa-paccayā vedanā,\\
Vedanā-paccayā taṇhā,\\
Taṇhā-paccayā upādānaṃ,\\
Upādāna-paccayā bhavo,\\
Bhava-paccayā jāti,\\
Jāti-paccayā jarā-maraṇaṃ soka-parideva-dukkha-domanass’upāyāsā sambhavanti.\\
Evam-etassa kevalassa dukkhakkhandhassa samudayo hoti.\\
Avijjāya tv-eva asesa-virāga-nirodhā\\
saṅkhāra-nirodho,\\
Saṅkhāra-nirodhā viññāṇa-nirodho,\\
Viññāṇa-nirodhā nāma-rūpa-nirodho,\\
Nāma-rūpa-nirodhā saḷ-āyatana-nirodho,\\
Saḷ-āyatana-nirodhā phassa-nirodho,\\
Phassa-nirodhā vedanā-nirodho,\\
Vedanā-nirodhā taṇhā-nirodho,\\
Taṇhā-nirodhā upādāna-nirodho,\\
Upādāna-nirodhā bhava-nirodho,\\
Bhava-nirodhā jāti-nirodho,\\
Jāti-nirodhā jarā-maraṇaṃ soka-parideva-dukkha-domanass’upāyāsā nirujjhanti.\\
Evam-etassa kevalassa dukkhakkhandhassa nirodho hoti.

\chapter{Paṭṭhāna-mātikā-pāṭho}

Hetu-paccayo, ārammaṇa-paccayo,\\
adhipati-paccayo, anantara-paccayo,\\
samanantara-paccayo, saha-jāta-paccayo,\\
aññam-añña-paccayo, nissaya-paccayo,\\
upanissaya-paccayo, pure-jāta-paccayo,\\
pacchā-jāta-paccayo, āsevana-paccayo,\\
kamma-paccayo, vipāka-paccayo,\\
āhāra-paccayo, indriya-paccayo,\\
jhāna-paccayo, magga-paccayo,\\
sampayutta-paccayo, vippayutta-paccayo,\\
atthi-paccayo, n’atthi-paccayo,\\
vigata-paccayo, avigata-paccayo.

\chapter{Paṃsu-kūla}

(For the living)

Aciraṃ vat’ayaṃ kāyo,\\
Paṭhaviṃ adhisessati.\\
Chuḍḍho apeta-viññāṇo,\\
Niratthaṃ va kaliṅgaraṃ.

(For the dead)

Aniccā vata saṅkhārā\\
Uppāda-vaya-dhammino;\\
Uppajjitvā nirujjhanti,\\
Tesaṃ vūpasamo sukho.

Sabbe sattā maranti ca\\
Mariṃsu ca marissare\\
Tath’evāhaṃ marissāmi\\
N’atthi me ettha saṃsayo.

Addhuvaṃ jīvitaṃ,\\
Dhuvaṃ maraṇaṃ,\\
Avassaṃ mayā maritabbaṃ\\
Maraṇapariyosānaṃ me jīvitaṃ.\\
Jīvitam m’eva aniyataṃ,\\
Maraṇaṃ niyataṃ,\\
Maraṇaṃ niyataṃ.


