\chapter{Guidelines}

(i) The ten reasons for the establishing the Patimokkha:

For the excellence of the Sangha; for the wellbeing of the Sangha; for the
control of ill-controlled bhikkhus; for the comfort of wellbehaved bhikkhus; for
the restraint of the āsavā in this present state; for protection against the
āsavā in a future state; to give confidence to those of little faith; to
increase the confidence of the faithful; to establish the True Dhamma; to
support the Vinaya.

[Vin.III.20; A.V.70]

(ii) The Four Great Standards (Mahāpadesa)

(a) Whatever things are not prohibited as unallowable but agree with things that
are unallowable, being opposed to things that are allowable — such things are unsuitable.

(b) Whatever things are not prohibited as unallowable but agree with things that
are allowable, being opposed to things that are unallowable — such things are suitable.

(c) Whatever things are not permitted as allowable but agree with things that
are unallowable, being opposed to things that are allowable — such things are unsuitable.

(d) Whatever things are not permitted as allowable but agree with things that
are allowable, being opposed to things that are unallowable — such things are suitable.

[Vin.I.250]

(iii) If there is some obstacle to [the practice of the training rules], due to time and place, the rules should be upheld indirectly and not given up entirely, for otherwise there will be no principles (for discipline). A community without principles for discipline cannot last long…

[Entrance to the Vinaya, I.230]

