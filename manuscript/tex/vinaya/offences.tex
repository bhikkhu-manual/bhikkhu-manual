\chapter{Offences}

\ifhandbookedition
\vspace*{-\baselineskip}
\fi

\section{Āpatti-paṭidesanā (Confession of Offences)}

\subsection{Six reasons for āpatti}

(1) Lack of shame; (2) ignorance of the rule; (3) in doubt
but goes ahead; (4) thinks he ought when he ought not; (5) thinks he ought not
when he ought; (6) acts without thinking (i.e. absent-mindedly).

\ifhandbookedition
\vspace*{-0.5\baselineskip}
\fi

\subsection{Four conditions for exemption from āpatti}

A bhikkhu who is (1) insane, (2) delirious, (3) suffering intense pain, or (4)
the original perpetrator.

\subsection{The kinds of āpatti}

\textbf{(a)} Those that cannot be remedied (\emph{pārājika}).

\textbf{(b)} Those that can be remedied:

Heavy offences (\emph{saṅghādisesa}), confessed to a Sangha.

Light offences, confessed to another bhikkhu: \emph{thullaccaya} (grave
offences), \emph{pācittiya} (offences of expiation), \emph{pāṭidesanīya}
(offences to be confessed), \emph{dukkaṭa} (offences of wrongdoing), and
\emph{dubbhāsita} (offences of wrong speech).

\subsection{Method of confessing light offences}
\label{confessing-light-offences}

\instr{(Thai Formula)}

Before the general confession any known offences should be specified. Two
bhikkhus with the same offence should not confess that offence together. To do
so is a \emph{dukkaṭa} offence.\suttaRef{Vin.IV.122}

{\centering

\begin{tabular}{@{}ll@{}}
\prul{SB:} & Senior Bhikkhu\\
\prul{JB:} & Junior Bhikkhu\\
\end{tabular}

}

\ifhandbookedition
\enlargethispage{\baselineskip}
\fi

\instr{Junior bhikkhu is confessing:}

\hangindent=25pt%
\parbox{22pt}{\prul{JB:}} Ahaṁ bhante sambahulā nānā-vatthukāyo \emph{thullaccayāyo} āpattiyo āpanno tā paṭidesemi.\\ \emph{I, ven. sir, having many times fallen into grave offences with different bases, these I confess.}

\hangindent=25pt%
\parbox{22pt}{\prul{SB:}} Passasi āvuso.\\ \emph{Do you see, friend?}

\hangindent=25pt%
\parbox{22pt}{\prul{JB:}} Āma bhante passāmi.\\ \emph{Yes, ven. sir, I see.}

\hangindent=25pt%
\parbox{22pt}{\prul{SB:}} Āyatiṁ āvuso saṁvareyyāsi.\\ \emph{In future, friend, you should be restrained.}

\hangindent=25pt%
\parbox{22pt}{\prul{JB:}} Sādhu suṭṭhu bhante saṁvarissāmi. (×3)\\ \emph{It is well indeed, ven. sir. I shall be restrained.}

\instr{Senior bhikkhu is confessing:}

\hangindent=25pt%
\parbox{22pt}{\prul{SB:}} Ahaṁ āvuso sambahulā nānā-vatthukāyo \emph{thullaccayāyo} āpattiyo āpanno tā paṭidesemi.\\ \emph{I, friend, having many times fallen into grave offences with different bases, these I confess.}

\hangindent=25pt%
\parbox{22pt}{\prul{JB:}} Passatha bhante.\\ \emph{Do you see, ven. sir?}

\hangindent=25pt%
\parbox{22pt}{\prul{SB:}} Āma āvuso passāmi.\\ \emph{Yes, friend, I see.}

\hangindent=25pt%
\parbox{22pt}{\prul{JB:}} Āyatiṁ bhante saṁvareyyātha.\\ \emph{In future, ven. sir, you should be restrained.}

\hangindent=25pt%
\parbox{22pt}{\prul{SB:}} Sādhu suṭṭhu āvuso saṁvarissāmi. (×3)\\ \emph{It is well indeed, friend. I shall be restrained.}

This formula is repeated replacing ‘\emph{thullaccayāyo}’ with, in turn, ‘\emph{pācittiyāyo}’,
‘\emph{dukkaṭāyo}’, ‘\emph{dubbhāsitāyo}’.

With ‘\emph{dubbhāsitāyo}’ omit ‘\emph{nānā-vatthukāyo}’.

When confessing two offences of the same class:\\
‘sambahulā’ (\emph{many}) → ‘dve’ (\emph{twice})

When confessing a single offence:

‘Sambahulā nānā-vatthukāyo \emph{thullaccayāyo} āpattiyo āpanno tā paṭidesemi.’\\
→ ‘Ekaṁ \emph{thullaccayaṁ} āpattiṁ āpanno taṁ paṭidesemi.’

Replace, as appropriate, ‘\emph{thullaccayaṁ}’ with ‘\emph{pācittiyaṁ}’, ‘\emph{dukkaṭaṁ}’, ‘\emph{dubbhāsitaṁ}’.

\section{Nissaggiya Pācittiya}

When confessing a \emph{nissaggiya pācittiya} (‘expiation with forfeiture’)
offence, substitute ‘\emph{nissaggiyāyo pācittiyāyo}’ for
‘\emph{thullaccayāyo}’, or ‘\emph{nissaggiyaṁ pācittiyaṁ}’ for
‘\emph{thullaccayaṁ}’ in the formula at sec.\ref{confessing-light-offences}
above.

However, before confessing, the article in question must be forfeited to another
bhikkhu or to a Sangha. \suttaRef{Vin.III.196f}

\subsection[NP 1 (‘extra robe’)]{Nissaggiya Pācittiya 1 (‘extra robe’)}
\label{np-1-extra-robe}

On the eleventh dawn of keeping one ‘extra robe’, within forearm's length,
forfeiting to a more senior bhikkhu:

‘Idaṁ me \emph{bhante} cīvaraṁ dasāhātikkantaṁ nissaggiyaṁ. Imāhaṁ āyasmato nissajjāmi.’\\
‘\emph{This extra robe, ven. sir, which has passed beyond the ten day (limit), is
  to be forfeited by me: I forfeit it to you.}’

More than one robe, within forearm's length:

‘Imāni me bhante cīvarāni dasāhātikkantāni nissaggiyāni. Imānāhaṁ āyasmato nissajjāmi.’

If forfeiting to a Sangha: ‘āyasamato’ → ‘saṅghassa’ 

If forfeiting to a group of bhikkhus:\\
‘āyasamato’ → ‘āysamantānaṁ’

If senior bhikkhu: ‘bhante’ → ‘āvuso’

If beyond forearm's length:

\begin{tabular}{@{}lll@{}}
‘idaṁ’ (\emph{this}) & → & ‘etaṁ’ (\emph{that})\\
‘imāhaṁ’             & → & ‘etāhaṁ’\\
‘imāni’ (\emph{these}) & → & ‘etāni’ (\emph{those})\\
‘imānāhaṁ’             & → & ‘etānāhaṁ’\\
\end{tabular}

\ifhandbookedition
\vspace*{-0.5\baselineskip}

\enlargethispage{\baselineskip}
\fi

\subsubsection{Returning the robe}
\label{np-1-returning-the-robe}

‘Imaṁ cīvaraṁ āyasmato dammi.’\\
‘\emph{I give this robe to you.}’ \suttaRef{Vin.III.197}

For returning more than one robe:\\
‘imaṁ’ → ‘imāni’ ; ‘cīvaraṁ’ → ‘cīvarāni’

This formula for returning the article(s) also applies in NP. 2, 3, 6, 7, 8, 9,
10 below.

\subsection[NP 2 (‘separated from’)]{Nissaggiya Pācittiya 2 (‘separated from’)}

‘Idaṁ me bhante cīvaraṁ ratti-vippavutthaṁ aññatra bhikkhu-sammatiyā
nissaggiyaṁ. Imāhaṁ āyasmato nissajjāmi.’

‘\emph{This robe, ven. sir, which has stayed separate (from me) for a night
  without the consent of the bhikkhus, is to be forfeited by me: I forfeit it to
  you.}’ \suttaRef{Vin.III.199–200}

If multiple robes:\\
‘cīvaraṁ’ → ‘dvicīvaraṁ’/‘ticīvaraṁ’ (two-/three-robes)

% For other variants, see sec.\ref{np-1-extra-robe} above.\\
% For returning the robe(s) see sec.\ref{np-1-returning-the-robe} above.
 
\subsection[NP 3 (‘over-kept cloth’)]{Nissaggiya Pācittiya 3 (‘over-kept cloth’)}

‘Idaṁ me bhante akāla-cīvaraṁ māsātikkantaṁ nissaggiyaṁ. Imāhaṁ āyasmato nissajjāmi.’

‘\emph{This, ven. sir, ‘out of season’ robe, which has passed beyond the month
  (limit), is to be forfeited by me: I forfeit it to you.}’ \suttaRef{Vin.III.205}

For more than one piece of cloth:

‘Imāni me bhante akāla-cīvarāni māsātikkantāni nissaggiyāni. Imānāhaṁ āyasmato nissajjāmi.’

% For other variants, see sec.\ref{np-1-extra-robe} above.\\
% For returning the robe(s) see sec.\ref{np-1-returning-the-robe} above.

\subsection[NP 6 (‘asked for’)]{Nissaggiya Pācittiya 6 (‘asked for’)}

‘Idaṁ me bhante cīvaraṁ aññātakaṁ gahapatikaṁ aññatra samayā viññāpitaṁ
nissaggiyaṁ. Imāhaṁ āyasmato nissajjāmi.’

‘\emph{This robe, ven. sir, which has been asked from an unrelated householder at
  other than the proper occasion, is to be forfeited by me: I forfeit it to you.}’
\suttaRef{Vin.III.213}

For more than one piece of cloth:

‘Imāni me bhante cīvarāni aññātakaṁ gahapatikaṁ aññatra samayā viññāpitāni
nissaggiyāni. Imānāhaṁ āyasmato nissajjāmi.’

% For other variants, see sec.\ref{np-1-extra-robe} above.\\
% For returning the robe(s) see sec.\ref{np-1-returning-the-robe} above.

\subsection[NP 7 (‘beyond limit’)]{Nissaggiya Pācittiya 7 (‘beyond limit’)}

‘Idaṁ me bhante cīvaraṁ aññātakaṁ gahapatikaṁ upasaṁkamitvā tat'uttariṁ
viññāpitaṁ nissaggiyaṁ. Imāhaṁ āyasmato nissajjāmi.’

‘\emph{This robe, ven. sir, which has been asked for beyond the limitation from
  an unrelated householder, is to be forfeited by me: I forfeit it to you.}’
\suttaRef{Vin.III.214–215}

For more than one piece of cloth:

‘Imāni me bhante cīvarāni aññātakaṁ gahapatikaṁ tat'uttariṁ viññāpitāni
nissaggiyāni. Imānāhaṁ āyasmato nissajjāmi.’

% For other variants, see sec.\ref{np-1-extra-robe} above.\\
% For returning the robe(s) see sec.\ref{np-1-returning-the-robe} above.

\subsection[NP 8 (‘instructing’)]{Nissaggiya Pācittiya 8 (‘instructing’)}
\label{np-8-instructing}

‘Idaṁ me bhante cīvaraṁ pubbe appavārito aññātakaṁ gahapatikaṁ upasaṅkamitvā
cīvare vikappaṁ āpannaṁ nissaggiyaṁ. Imāhaṁ āyasmato nissajjāmi.’

‘\emph{This robe, ven. sir, which has been instructed about after having
  approached an unrelated householder without prior invitation is to be
  forfeited by me: I forfeit it to you.}’ \suttaRef{Vin.III.217}

% For other variants, see sec.\ref{np-1-extra-robe} above.\\
% For returning the robe(s) see sec.\ref{np-1-returning-the-robe} above.

\subsection[NP 9 (‘instructing’)]{Nissaggiya Pācittiya 9 (‘instructing’)}

For a robe (robe-cloth) received after making instructions to two or more
householders. Use formula of sec.\ref{np-8-instructing} above but change:

‘aññātakaṁ gahapatikaṁ’ → ‘aññātake gahapatike’

For returning the robe(s) see sec.\ref{np-1-returning-the-robe} above.

\suttaRef{Vin.III.219}

\subsection[NP 10 (‘reminding’)]{Nissaggiya Pācittiya 10 (‘reminding’)}

‘Idaṁ me bhante cīvaraṁ atireka-tikkhattuṁ codanāya atireka-chakkhattuṁ ṭhānena
abhinipphāditaṁ nissaggiyaṁ, imāhaṁ āyasmato nissajjāmi.’

‘\emph{This robe, ven. sir, which has been obtained by inciting more
  than three times, by standing more than six times, is to be forfeited by me: I
  forfeit it to you.}’ \suttaRef{Vin.III.223}

% For other variants, see sec.\ref{np-1-extra-robe} above.\\
% For returning the robe(s) see sec.\ref{np-1-returning-the-robe} above.

\subsection[NP 18 (‘gold and silver’)]{Nissaggiya Pācittiya 18 (‘gold and silver’)}

‘Ahaṁ bhante rūpiyaṁ paṭiggahesiṁ, idaṁ me nissaggiyaṁ. Imāhaṁ saṅghassa
nissajjāmi.’

‘\emph{Ven. sirs, I have accepted money. This is to be forfeited by me: I
  forfeit it to the Saṅgha.}’

To be forfeited to the Sangha only. \suttaRef{Vin.III.238}

\subsection[NP 19 (‘monetary exchange’)]{Nissaggiya Pācittiya 19 (‘monetary exchange’)}

‘Ahaṁ bhante nānappakārakaṁ rūpiyasaṁvohāraṁ samāpajjiṁ, idaṁ me nissaggiyaṁ.
Imāhaṁ saṅghassa nissajjāmi.’

‘\emph{Ven. sirs, I have engaged in various kinds of trafficking with money.
  This (money) is to be forfeited by me: I forfeit it to the Saṅgha.}’

To be forfeited to the Sangha only. \suttaRef{Vin.III.240}

\subsection[NP 20 (‘buying and selling’)]{Nissaggiya Pācittiya 20 (‘buying and selling’)}

‘Ahaṁ bhante nānappakārakaṁ kayavikkayaṁ samāpajjiṁ, idaṁ me nissaggiyaṁ. Imāhaṁ
āyasmato nissajjāmi.’

‘\emph{Ven. sir, I have engaged in various kinds of buying and selling. This
  (gain) of mine is to be forfeited by me: I forfeit it to you.}’ \suttaRef{Vin.III.242}

If forfeiting to a Sangha: ‘āyasmato’ → ‘saṅghassa’

If forfeiting to a group of bhikkhus:\\
‘āyasmato’ → ‘āyasmantānaṁ’

% For other variants, see sec.\ref{np-1-extra-robe} above.

\subsection[NP 21 (‘extra bowl’)]{Nissaggiya Pācittiya 21 (‘extra bowl’)}

‘Ayaṁ me bhante patto dasāhātikkanto nissaggiyo. Imāhaṁ āyasmato nissajjāmi.’

‘\emph{This bowl, ven. sir, which has passed beyond the ten-day (limit), is to be
  forfeited by me: I forfeit it to you.}’

% For other variants, see sec.\ref{np-1-extra-robe} above.

For returning the bowl:

‘Imaṁ pattaṁ āyasmato dammi.’\\
‘\emph{I give this bowl to you.}’ \suttaRef{Vin.III.243–244}

\subsection[NP 22 (‘new bowl’)]{Nissaggiya Pācittiya 22 (‘new bowl’)}

‘Ayaṁ me bhante patto ūnapañca-bandhanena pattena cetāpito nissaggiyo. Imāhaṁ
saṅghassa nissajjāmi.’

‘\emph{This bowl, ven. sirs, which has been exchanged for a bowl that has less
  than five mends, is to be forfeited by me: I forfeit it to the Sangha.}’

To be forfeited to the Sangha only. \suttaRef{Vin.III.246}

\subsection[NP 23 (‘kept tonics’)]{Nissaggiya Pācittiya 23 (‘kept tonics’)}

‘Idaṁ me bhante bhesajjaṁ sattāhātikkantaṁ nissaggiyaṁ. Imāhaṁ āyasmato
nissajjāmi.’

‘\emph{This tonic, ven. sir, which has passed beyond the seven-day
  (limit), is to be forfeited by me: I forfeit it to you.}’

Tonics can be returned, but not for consumption:

‘Imaṁ bhesajjaṁ āyasmato dammi.’\\
‘\emph{I give this tonic to you.}’ \suttaRef{Vin.III.251}

\subsection[NP 25 (‘snatched back’)]{Nissaggiya Pācittiya 25 (‘snatched back’)}

‘Idaṁ me bhante cīvaraṁ bhikkhussa sāmaṁ datvā acchinnaṁ nissaggiyaṁ. Imāhaṁ
āyasmato nissajjāmi.’

‘\emph{This robe, ven. sir, which has been snatched back after having given it
  myself to a bhikkhu, is to be forfeited by me: I forfeit it to you.}’

\suttaRef{Vin.III.255}

% For other variants, see sec.\ref{np-1-extra-robe} above.

\subsection[NP 28 (‘urgent’)]{Nissaggiya Pācittiya 28 (‘urgent’)}

‘Idaṁ me bhante acceka-cīvaraṁ cīvara-kālasamayaṁ atikkāmitaṁ nissaggiyaṁ.
Imāhaṁ āyasmato nissajjāmi.’

‘\emph{This robe-offered-in-urgency, ven. sir, has passed beyond the
  robe-season, is to be forfeited by me: I forfeit it to you.}’

\suttaRef{Vin.III.262}

% For other variants, see sec.\ref{np-1-extra-robe} above.

\subsection[NP 29 (‘wilderness abode’)]{Nissaggiya Pācittiya 29 (‘wilderness abode’)}

‘Idaṁ me bhante cīvaraṁ atireka-chā-rattaṁ vippavutthaṁ aññatra
bhikkhu-sammatiyā nissaggiyaṁ. Imāhaṁ āyasmato nissajjāmi.’

‘\emph{This robe, ven. sir, which has stayed separate (from me) for a night
  without the consent of the bhikkhus, is to be forfeited by me: I forfeit it to
  you.}’

\suttaRef{Vin.III.264}

% For other variants, see sec.\ref{np-1-extra-robe} above.

\subsection[NP 30 (‘diverted gain’)]{Nissaggiya Pācittiya 30 (‘diverted gain’)}

‘Idaṁ me bhante jānaṁ saṅghikaṁ lābhaṁ pariṇataṁ attano pariṇāmitaṁ nissaggiyaṁ.
Imāhaṁ āyasmato nissajjāmi.’

‘\emph{This gain belonging to the Saṅgha, ven. sir, which has been (already)
  diverted (to someone), (and) which has been knowingly diverted to myself
  (instead), is to be forfeited by me: I forfeit it to you.}’

% For other variants, see sec.\ref{np-1-extra-robe} above.

To return the article: ‘Imaṁ āyasmato dammi.’

\suttaRef{Vin.III.266}

\ifhandbookedition
\clearpage
\fi

\section{Saṅghādisesa}

\textbf{(i)} A bhikkhu who has committed \emph{saṅghādisesa} must first inform
one or more bhikkhus, and then inform a Sangha of at least four bhikkhus of his
fault(s) and ask to observe \emph{mānatta} (penance). When the Sangha has given
\emph{mānatta} to that bhikkhu, he recites the formula undertaking
\emph{mānatta} and then practises the appropriate duties for six days and
nights. When the bhikkhu has completed practising \emph{mānatta}, he requests
rehabilitation (\emph{abbhāna}) in the presence of a Sangha of at least twenty
bhikkhus.

\textbf{(ii)} A bhikkhu who has committed \emph{saṅghādisesa} and deliberately
concealed it must first live in \emph{parivāsa} (probation) for the number of
days that the offence was concealed. When the bhikkhu has completed his time
living in \emph{parivāsa}, he requests \emph{mānatta} and then follows the
procedure outlined in (i) above.

