\chapter{Offences}

\section{Āpatti-paṭidesanā (Confession of Offences)}

\subsection{Six reasons for āpatti}

(1) Lack of shame; (2) ignorance of the rule; (3) in doubt
but goes ahead; (4) thinks he ought when he ought not; (5) thinks he ought not
when he ought; (6) acts without thinking (i.e. absent-mindedly).

\subsection{Four conditions for exemption from āpatti}

A bhikkhu who is (1) insane, (2) delirious, (3) suffering intense pain, or (4)
the original perpetrator.

\subsection{The kinds of āpatti}

\textbf{(a)} Those that cannot be remedied (\emph{pārājika}).

\textbf{(b)} Those that can be remedied:

Heavy offences (\emph{saṅghādisesa}), confessed to a Sangha.

Light offences, confessed to another bhikkhu: \emph{thullaccaya} (grave
offences), \emph{pācittiya} (offences of expiation), \emph{pāṭidesanīya}
(offences to be confessed), \emph{dukkaṭa} (offences of wrongdoing), and
\emph{dubbhāsita} (offences of wrong speech).

\subsection{Method of confessing light offences}
\label{confessing-light-offences}

Before the general confession any known offences should be specified. Two
bhikkhus with the same offence should not confess that offence together. To do
so is a \emph{dukkaṭa} offence.\\
\mbox{}\suttaRef{Vin.IV.122}

The more junior bhikkhu confesses first, going through the different offence
classes. Then the senior bhikkhu does likewise:

\subsubsection{Thai Formula}

{\centering

\begin{tabular}{@{}ll@{}}
\prul{SAB:} & Senior Acknowledging Bhikkhu\\
\prul{JCB:} & Junior Confessing Bhikkhu\\
\end{tabular}

}

\hangindent=25pt%
\parbox{22pt}{\prul{JCB:}} Āhaṃ bhante sambahulā nānā-vatthukāyo \emph{thullacca yā yo} āpattiyo āpanno tā paṭidesemi.\\ \emph{I, ven. sir, having many times fallen into grave offences with different bases, these I confess.}

\hangindent=25pt%
\parbox{22pt}{\prul{SAB:}} Passasi āvuso?\\ \emph{Do you see, friend?}

\hangindent=25pt%
\parbox{22pt}{\prul{JCB:}} Āma bhante passāmi.\\ \emph{Yes, ven. sir, I see.}

\hangindent=25pt%
\parbox{22pt}{\prul{SAB:}} Āyatiṃ āvuso saṃvareyyāsi.\\ \emph{In future, friend, you should be restrained.}

\hangindent=25pt%
\parbox{22pt}{\prul{JCB:}} Sādhu suṭṭhu bhante saṃvarissāmi. (×3)\\ \emph{It is well indeed, ven. sir. I shall be restrained.}

\hangindent=25pt%
\parbox{22pt}{\prul{SCB:}} Āhaṃ āvuso sambahulā nānā-vatthukāyo \emph{thullacca yā yo} āpattiyo āpanno tā paṭidesemi.\\ \emph{I, friend, having many times fallen into grave offences with different bases, these I confess.}

\hangindent=25pt%
\parbox{22pt}{\prul{JAB:}} Passatha bhante?\\ \emph{Do you see, ven. sir?}

\hangindent=25pt%
\parbox{22pt}{\prul{SCB:}} Āma āvuso passāmi.\\ \emph{Yes, friend, I see.}

\hangindent=25pt%
\parbox{22pt}{\prul{JAB:}} Āyatiṃ bhante saṃvareyyātha\\ \emph{In future, ven. sir, you should be restrained.}

\hangindent=25pt%
\parbox{22pt}{\prul{SCB:}} Sādhu suṭṭhu āvuso saṃvarissāmi. (×3)\\ \emph{It is well indeed, friend. I shall be restrained.}

This formula is repeated replacing ‘\emph{thullacca yā yo}’ with, in turn, ‘\emph{pācittiyāyo}’,
‘\emph{dukkaṭāyo}’, ‘\emph{dubbhāsitāyo}’.

With ‘\emph{dubbhāsitāyo}’ omit ‘\emph{nānā-vatthukāyo}’.

When confessing two offences of the same class:\\
‘sambahulā’ (\emph{many}) → ‘dve’ (\emph{twice})

When confessing a single offence:\\
‘Sambahulā nānā-vatthukāyo \emph{thullacca yā yo} āpattiyo āpanno tā paṭidesemi.’\\
→ ‘Ekaṃ \emph{thullacca yaṃ} āpattiṃ āpanno taṃ paṭidesemi.’

Replace, as appropriate, ‘\emph{thullaccayaṃ}’ with ‘\emph{pācittiyaṃ}’, ‘\emph{dukkaṭaṃ}’, ‘\emph{dubbhāsitaṃ}’.

\subsubsection{Sri Lankan Formula}

\hangindent=25pt%
\parbox{22pt}{\prul{JCB:}} Okāsa, ahaṃ bhante, sabbā āpattiyo ārocemi.\\
Dutiyam-pi ahaṃ bhante, sabbā āpattiyo ārocemi.\\
Tatiyam-pi ahaṃ bhante, sabbā āpattiyo ārocemi.\\
\emph{I ven. sir, declare all offences. For the second time… For the third time…}

\hangindent=25pt%
\parbox{22pt}{\prul{SAB:}} Sādhu, sādhu.\\ \emph{It is good, it is good.}

\hangindent=25pt%
\parbox{22pt}{\prul{JCB:}} Okāsa ahaṃ bhante, sambahulā nānā-vatthukā āpattiyo āpajjiṃ, tā tumha-mūle paṭidesemi.\\ \emph{I, ven. sir, having many times fallen into many different offences with different bases, these I confess.}

\hangindent=25pt%
\parbox{22pt}{\prul{SAB:}} Passasi āvuso tā āpattiyo?\\ \emph{Do you see, friend, those offences?}

\hangindent=25pt%
\parbox{22pt}{\prul{JCB:}} Āma bhante passāmi.\\ \emph{Yes, ven. sir, I see.}

\hangindent=25pt%
\parbox{22pt}{\prul{SAB:}} Āyatiṃ āvuso saṃvareyyāsi.\\ \emph{In the future, friend, you should be restrained.}

\hangindent=25pt%
\parbox{22pt}{\prul{JCB:}} Sādhu suṭṭhu bhante āyatiṃ saṃvarissāmi.\\
Dutiyam-pi sādhu suṭṭhu bhante āyatiṃ saṃvarissāmi.\\
Tatiyam-pi sādhu suṭṭhu bhante āyatiṃ saṃvarissāmi.\\
\emph{It is well indeed, ven. sir, in future I shall be restrained. For the second time…For the third time…}

\hangindent=25pt%
\parbox{22pt}{\prul{SAB:}} Sādhu, sādhu.\\ \emph{It is good, it is good.}

\hangindent=25pt%
\parbox{22pt}{\prul{JCB:}} Okāsa ahaṃ bhante,\\
sabbā tā garukāpattiyo āvikaromi.\\
Dutiyam-pi okāsa ahaṃ bhante,\\
sabbā tā garukāpattiyo āvikaromi.\\
Tatiyam-pi okāsa ahaṃ bhante,\\
sabbā tā garukāpattiyo āvikaromi.\\
\emph{Ven. sir, I reveal all heavy offences. For the second time… For the third time…}

This final declaration is only used in some communities. Also, some communities
will acknowledge with a ‘\emph{Sādhu}’ after each declaration rather than as
shown above. That is, after each ‘\emph{ārocemi}’ and each
‘\emph{saṃvarissāmi}’.

\subsubsection{Sri Lankan Formula for same base offences}

\hangindent=25pt%
\parbox{22pt}{\prul{JCB:}} Okāsa ahaṃ bhante, desanādukkaṭāpattiṃ āpajjiṃ, taṃ tumha-mūle paṭidesemi.\\ \emph{I, ven. sir, confess an offence of wrong-doing through having confessed the same-based offences.}

\hangindent=25pt%
\parbox{22pt}{\prul{SAB:}} Passasi āvuso taṃ āpaṭṭiṃ?\\ \emph{Do you see, friend, that offence?}

\hangindent=25pt%
\parbox{22pt}{\prul{JCB:}} Āma bhante passāmi.\\ \emph{Yes, ven. sir, I see.}

\hangindent=25pt%
\parbox{22pt}{\prul{SAB:}} Āyatiṃ āvuso saṃvareyyāsi.\\ \emph{In the future, friend, you should be restrained.}

\hangindent=25pt%
\parbox{22pt}{\prul{JCB:}} Sādhu suṭṭhu bhante āyatiṃ saṃvarissāmi. Dutiyam-pi sādhu suṭṭhu … . Tatiyam-pi … saṃvarissāmi.\\ \emph{It is well indeed, ven. sir, in future I shall be restrained. For the second time… For the third time…}

\hangindent=25pt%
\parbox{22pt}{\prul{SAB:}} Sādhu, sādhu.\\ \emph{It is good, it is good.} \suttaRef{cf. Vin.II.102}

\section{Nissaggiya Pācittiya}

When confessing a \emph{nissaggiya pācittiya} (‘expiation with forfeiture’)
offence, substitute ‘\emph{nissaggiyāyo pācittiyāyo}’ for
‘\emph{thullaccayāyo}’, or ‘\emph{nissaggiyaṃ pācittiyaṃ}’ for
‘\emph{thullaccayaṃ}’ in the formula at sec.\ref{confessing-light-offences}
above.

However, before confessing, the article in question must be forfeited to another
bhikkhu or to a Sangha. \suttaRef{Vin.III.196f}

\subsection{Nissaggiya Pācittiya 1 (‘extra robe’)}
\label{np-1-extra-robe}

On the eleventh dawn of keeping one ‘extrarobe’, within forearm's length,
forfeiting to a more senior bhikkhu:

‘Idaṃ me \emph{bhante} cīvaraṃ dasāhātikkantaṃ nissaggiyaṃ, imāhaṃ āyasmato nissajjāmi.’\\
‘\emph{This extra robe, ven. sir, which has passed beyond the ten day (limit) is
  to be forfeited by me: I forfeit it to you.}’

More than one robe, within forearm's length:

‘Imāni me bhante, cīvarāni dasāhātikkantāni nissaggiyāni, imānāhaṃ āyasmato nissajjāmi.’

If forfeiting to a Sangha: ‘āyasamato’ → ‘saṅghassa’ 

If forfeiting to a group of bhikkhus:\\
‘āyasamato’ → ‘āysamantānaṃ’

If senior bhikkhu: ‘bhante’ → ‘āvuso’

If beyond forearm's length:

\begin{tabular}{@{}lll@{}}
‘idaṃ’ (\emph{this}) & → & ‘etaṃ’ (\emph{that}) \\
‘imāhaṃ’             & → & ‘etāhaṃ’             \\
‘imāni’ (\emph{these}) & → & ‘etāni’ (\emph{those})\\
‘imānāhaṃ’             & → & ‘etānāhaṃ’\\
\end{tabular}

\suttaRef{Vin.III.197}

\vspace*{-\baselineskip}

\subsection{Returning the robe}
\label{np-1-returning-the-robe}

‘Imaṃ cīvaraṃ āyasmato dammi.’\\
‘\emph{I give this robe to you.}’ \suttaRef{Vin.III.197}

For returning more than one robe:\\
‘imaṃ’ → ‘imāni’ ; ‘cīvaraṃ’ → ‘cīvarāni’

This formula for returning the article(s) also applies in NP. 2, 3, 6, 7, 8, 9,
10 below.

\subsection{Nissaggiya Pācittiya 2 (‘separated from’)}

‘Idaṃ me bhante cīvaraṃ ratti-vippavutthaṃ aññatra bhikkhu-sammatiyā
nissaggiyaṃ. Imāhaṃ āyasmato nissajjāmi.’\\
‘\emph{This robe, ven. sir, which has stayed separate (from me) for a night
  without the consent of the bhikkhus, is to be forfeited by me: I forfeit it to
  you.}’\\
\mbox{}\suttaRef{Vin.III.199–200}

If multiple robes:\\
‘cīvaraṃ’ → ‘dvicīvaraṃ’/‘ticīvaraṃ’ (two-/three-robes)

% For other variants, see sec.\ref{np-1-extra-robe} above.\\
% For returning the robe(s) see sec.\ref{np-1-returning-the-robe} above.
 
\subsection{Nissaggiya Pācittiya 3 (‘over-kept cloth’)}

‘Idaṃ me bhante akāla-cīvaraṃ māsātikkantaṃ nissaggiyaṃ, imāhaṃ āyasmato nissajjāmi.’\\
‘\emph{This, ven. sir, ‘out of season’ robe, which has passed beyond the month
  limit, is to be forfeited by me: I forfeit it to you.}’\\
\mbox{}\suttaRef{Vin.III.205}

For more than one piece of cloth:

‘Imāni me bhante akāla-cīvarāni māsātikkantāni nissaggiyāni. Imānāhaṃ āyasmato nissajjāmi.’

% For other variants, see sec.\ref{np-1-extra-robe} above.\\
% For returning the robe(s) see sec.\ref{np-1-returning-the-robe} above.

\subsection{Nissaggiya Pācittiya 6 (‘asked for’)}

‘Idaṃ me bhante cīvaraṃ aññātakaṃ gahapatikaṃ aññatra samayā viññāpitaṃ
nissaggiyaṃ, imāhaṃ āyasmato nissajjāmi.’\\
‘\emph{This robe, ven. sir, which has been asked from an unrelated householder at
  other than the proper occasion, is to be forfeited by me: I forfeit it to you.}’
\suttaRef{Vin.III.213}

For more than one piece of cloth:

‘Imāni me bhante cīvarāni aññātakaṃ gahapatikaṃ aññatra samayā viññāpitāni
nissaggiyāni. Imānāhaṃ āyasmato nissajjāmi.’

% For other variants, see sec.\ref{np-1-extra-robe} above.\\
% For returning the robe(s) see sec.\ref{np-1-returning-the-robe} above.

\subsection{Nissaggiya Pācittiya 7 (‘beyond limit’)}

‘Idaṃ me bhante cīvaraṃ aññātakaṃ gahapatikaṃ \emph{upasaṃkamitvā} tat'uttariṃ
viññāpitaṃ nissaggiyaṃ, imāhaṃ āyasmato nissajjāmi.’\\
‘\emph{This robe, ven. sir, which has been asked for beyond the limitation from
  an unrelated householder, is to be forfeited by me: I forfeit it to you.}’
\suttaRef{Vin.III.214–215}

‘\emph{Upasaṃkamitvā}’ is included in Sri Lanka.

For more than one piece of cloth:

‘Imāni me bhante cīvarāni aññātakaṃ gahapatikaṃ tat'uttariṃ viññāpitāni
nissaggiyāni. Imānāhaṃ āyasmato nissajjāmi.’

% For other variants, see sec.\ref{np-1-extra-robe} above.\\
% For returning the robe(s) see sec.\ref{np-1-returning-the-robe} above.

\subsection{Nissaggiya Pācittiya 8 (‘instructing’)}
\label{np-8-instructing}

‘Idaṃ me bhante cīvaraṃ pubbe appavārito aññātakaṃ gahapatikaṃ upasaṃkamitvā
cīvare vikappaṃ āpannaṃ nissaggiyaṃ. Imāhaṃ āyasmato nissajjāmi.’\\
‘\emph{This robe, ven. sir, which has been instructed about after having
  approached an unrelated householder without prior invitation is to be
  forfeited by me: I forfeit it to you.}’\\
\mbox{}\suttaRef{Vin.III.217}

% For other variants, see sec.\ref{np-1-extra-robe} above.\\
% For returning the robe(s) see sec.\ref{np-1-returning-the-robe} above.

\vspace*{-\baselineskip}

\subsection{Nissaggiya Pācittiya 9 (‘instructing’)}

For a robe (robe-cloth) received after making instructions to two or more
householders. Use formula of sec.\ref{np-8-instructing} above but change:

‘aññātakaṃ gahapatikaṃ’ → ‘aññātake gahapatike’

For returning the robe(s) see sec.\ref{np-1-returning-the-robe} above. \suttaRef{Vin.III.219}

\subsection{Nissaggiya Pācittiya 10 (‘reminding’)}

‘Idaṃ me bhante cīvaraṃ atireka-tikkhattuṃ codanāya atireka-chakkhattuṃ ṭhānena
abhinipphāditaṃ nissaggiyaṃ, imāhaṃ āyasmato nissajjāmi.’\\
‘\emph{This robe, ven. sir, which has been effected/obtained by inciting more
  than three times, by standing more than six times, is to be forfeited by me: I
  forfeit it to you.}’\\
\mbox{}\suttaRef{Vin.III.223}

% For other variants, see sec.\ref{np-1-extra-robe} above.\\
% For returning the robe(s) see sec.\ref{np-1-returning-the-robe} above.

\vspace*{-\baselineskip}

\subsection{Nissaggiya Pācittiya 18 (‘gold and silver’)}

‘Ahaṃ bhante rūpiyaṃ paṭiggahesiṃ. Idaṃ me nissaggiyaṃ. Imāhaṃ saṅghassa
nissajjāmi.’\\
‘\emph{Ven. sirs, I have accepted money. This is to be forfeited by me: I
  forfeit it to the Saṅgha.}’

To be forfeited to the Sangha only. \suttaRef{Vin.III.238}

\subsection{Nissaggiya Pācittiya 19 (‘monetary exchange’)}

‘Ahaṃ bhante nānappakārakaṃ rūpiyasaṃvohāraṃ samāpajjiṃ. Idaṃ me nissaggiyaṃ.
Imāhaṃ saṅghassa nissajjāmi.’\\
‘\emph{Ven. sirs, I have engaged in various kinds of trafficking with money.
  This (money) is to be forfeited by me: I forfeit it to the Saṅgha.}’

To be forfeited to the Sangha only. \suttaRef{Vin.III.240}

\subsection{Nissaggiya Pācittiya 20 (‘buying and selling’)}

‘Ahaṃ bhante nānappakārakaṃ kayavikkayaṃ samāpajjiṃ, idaṃ me nissaggiyaṃ, imāhaṃ
āyasmato nissajjāmi.’\\
‘\emph{Ven. sir, I have engaged in various kinds of buying and selling. This
  (gain) of mine is to be forfeited by me, I forfeit it to you.}’\\
\mbox{}\suttaRef{Vin.III.242}

If forfeiting to a Sangha: ‘āyasmato’ → ‘saṅghassa’

If forfeiting to a group of bhikkhus:\\
‘āyasmato’ → ‘āyasmantānaṃ’

% For other variants, see sec.\ref{np-1-extra-robe} above.

\subsection{Nissaggiya Pācittiya 21 (‘extra bowl’)}

‘Ayaṃ me bhante patto dasāhātikkanto nissaggiyo, imāhaṃ āyasmato nissajjāmi.’\\
‘\emph{This bowl, ven. sir, which has passed beyond the ten-day (limit) is to be
  forfeited by me: I forfeit it to you.}’

% For other variants, see sec.\ref{np-1-extra-robe} above.

For returning the bowl:

‘Imaṃ pattaṃ āyasmato dammi.’\\
‘\emph{I give this bowl to you.}’ \suttaRef{Vin.III.243–244}

\subsection{Nissaggiya Pācittiya 22 (‘new bowl’)}

‘Ayaṃ me bhante patto ūnapañca-bandhanena pattena cetāpito nissaggiyo. Imāhaṃ
saṅghassa nissajjāmi.’\\
‘\emph{This bowl, ven. sirs, which has been exchanged for a bowl that has less
  than five mends, is to be forfeited by me: I forfeit it to the Sangha.}’

To be forfeited to the Sangha only. \suttaRef{Vin.III.246}

\subsection{Nissaggiya Pācittiya 23 (‘kept medicines’)}

‘Idaṃ me bhante bhesajjaṃ sattāhātikkantaṃ nissaggiyaṃ. Imāhaṃ āyasmato
nissajjāmi.’\\
‘\emph{This medicine, ven. sir, which has been passed beyond the seven-day
  (limit), is to be forfeited by me: I forfeit it to you.}’

Medicine can be returned, but not for consumption:

‘Imaṃ bhesajjaṃ āyasmato dammi.’\\
‘\emph{I give this medicine to you.}’ \suttaRef{Vin.III.251}

\subsection{Nissaggiya Pācittiya 25 (‘snatched back’)}

‘Idaṃ me bhante cīvaraṃ bhikkhussa sāmaṃ datvā acchinnaṃ nissaggiyaṃ. Imāhaṃ
āyasmato nissajjāmi.’\\
‘\emph{This robe, ven. sir, which has been snatched back after having given it
  myself to a bhikkhu, is to be forfeited by me: I forfeit it to you.}’

\suttaRef{Vin.III.255}

% For other variants, see sec.\ref{np-1-extra-robe} above.

\vspace*{-\baselineskip}

\subsection{Nissaggiya Pācittiya 28 (‘urgent’)}

‘Idaṃ me bhante acceka-cīvaraṃ cīvara-kālasamayaṃ atikkāmitaṃ nissaggiyaṃ.
Imāhaṃ āyasmato nissajjāmi.’\\
‘\emph{This robe-offered-in-urgency, ven. sir, has passed beyond the
  robe-season, is to be forfeited by me: I forfeit it to you.}’ \suttaRef{Vin.III.262}

% For other variants, see sec.\ref{np-1-extra-robe} above.

\subsection{Nissaggiya Pācittiya 29 (‘wilderness abode’)}

‘Idaṃ me bhante cīvaraṃ atireka-chā-rattaṃ vippavutthaṃ aññatra
bhikkhu-sammatiyā nissaggiyaṃ. Imāhaṃ āyasmato nissajjāmi.’\\
‘\emph{This robe, ven. sir, which has stayed separate (from me) for a night
  without the consent of the bhikkhus, is to be forfeited by me: I forfeit it to
  you.}’ \suttaRef{Vin.III.264}

% For other variants, see sec.\ref{np-1-extra-robe} above.

\subsection{Nissaggiya Pācittiya 30}

‘Idaṃ me bhante jānaṃ saṅghikaṃ lābhaṃ pariṇataṃ attano pariṇāmitaṃ nissaggiyaṃ,
imāhaṃ āyasmato nissajjāmi.’\\
‘\emph{This gain belonging to the Saṅgha, ven. sir, which has been (already)
  diverted (to someone), (and) which has been knowingly diverted to myself
  (instead) is to be forfeited by me: I forfeit it to you.}’

% For other variants, see sec.\ref{np-1-extra-robe} above.

To return the article: ‘Imaṃ āyasmato dammi.’ \suttaRef{Vin.III.266}

\section{Saṅghādisesa}

\textbf{(i)} A bhikkhu who has committed \emph{saṅghādisesa} must first inform
one or more bhikkhus, and then inform a Sangha of at least four bhikkhus of his
fault(s) and ask to observe \emph{mānatta}. When the Sangha has given
\emph{mānatta} to that bhikkhu, he recites the formula undertaking
\emph{mānatta} and then practises the appropriate duties for six days and
nights. When the bhikkhu has completed practising \emph{mānatta}, he requests
rehabilitation (\emph{abbhāna}) in the presence of a Sangha of at least twenty
bhikkhus.

\textbf{(ii)} A bhikkhu who has committed \emph{saṅghādisesa} and deliberately
concealed it must first live in \emph{parivāsa} (probation) for the number of
days that the offence was concealed. When the bhikkhu has completed his time
living in \emph{parivāsa}, he requests \emph{mānatta} and then follows the
procedure outlined in (i) above.

