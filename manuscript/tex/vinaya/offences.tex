\chapter{Offences}

\section{Āpatti-paṭidesanā (Confession of Offences)}

(i) The six reasons for āpatti:
Lack of shame; ignorance of the rule; in doubt
b u t g o e s a h e a d ; t h in k s h e o u g h t w h e n h e
ought not; thinks he ought not when he
o u g h t ; a ct s w it h o u t t h in k in g ( i . e . a b s e n t mindedly).
(ii) There is no āpatti for:
A bhikkhu who is insane, delirious, suffering
intense pain, or the original perpetrator.
(iii) The kinds of āpatti:
(a) Those that cannot be remedied (pārājika).
(b) Those that can be remedied:
—Heavy offences (saṅghādisesa),
confessed to a Sangha.
—Light offences, confessed to another bhikkhu:
thullaccaya (grave offences), pācittiya
(offences of expiation), pāṭidesanīya (offences
to be confessed), dukkaṭa (offences of wrongdoing), and dubbhāsita (offences of wrong
speech).

(iv) Method of confessing light of fences:
♦ Be fore the general confession any know n
offences should be specified. Two bhikkhus
with the same offence should not confess that
offence together. To do so is a dukkaṭa
[Vin,IV,122]
offence.
♦ T h e m or e j u n i or b hi k k h u c o n fe ss e s fir st ,
going through the different offence classes.
Then the senior bhikkhu does likewise:
(a) Thai Formula
♦Junior Confessing Bhikkhu:

‘Āhaṃ bhante sambahulā nānā-vatthukāyo
*thullacca yā yo* āpattiyo āpanno tā
paṭidesemi.’
(‘I, ven. sir, having many times fallen into grave
offences with different bases, these I confess.’)

Senior Acknowledging Bhikkhu:
‘Passasi āvuso?’
(‘Do you see, friend?’)

JCB:

‘Āma bhante passāmi.’
(‘Yes, ven. sir, I see.’)

SAB:

‘Āyatiṃ āvuso saṃvareyyāsi’

(‘In future, friend, you should be restrained.’)

JCB: ‘Sādhu suṭṭhu bhante saṃvarissāmi.’ (×3)
(‘It is well indeed, ven. sir. I shall be restrained.’)

♦Senior Confessing Bhikkhu:

‘Āhaṃ āvuso sambahulā nānā-vatthukāyo
*thullacca yā yo* āpattiyo āpanno tā paṭidesemi.’
(‘I, friend, having many times fallen into grave
offences with different bases, these I confess.’)

Junior Acknowledging Bhikkhu:
‘Passatha bhante?’
(‘Do you see, ven. sir?’)

SCB:

‘Āma āvuso passāmi.’
(‘Yes, friend, I see.’)

JAB:

‘Āyatiṃ bhante saṃvareyyātha’

(‘In future, ven. sir, you should be restrained.’)

SCB: ‘Sādhu suṭṭhu āvuso saṃvarissāmi.’(×3)
(‘It is well indeed, friend. I shall be restrained.’)

• This formula is repeated replacing
‘thullaccayāyo’ with, in turn, ‘pācittiyāyo’,
‘dukkaṭāyo’, ‘dubbhāsitāyo’.
♦ With ‘dubbhāsitāyo’ omit ‘nānā-vatthukāyo’.
• When confessing two offences of the same class:
‘sambahulā’ (man y) →‘dve’ (twice)
• When confessing a single offence:
‘Sambahulā nānā-vatthukāyo thullacca yā yo
āpattiyo āpanno tā paṭidesemi.’
→‘Ekaṃ *thullacca yaṃ* āpattiṃ āpanno taṃ
paṭidesemi.’
Replace, as appropriate, ‘thullaccayaṃ’ with
‘pācittiyaṃ’, ‘dukkaṭaṃ’, ‘dubbhāsitaṃ’.

(b) Sri Lankan Formula.
Junior Confessing Bhikkhu:
‘Okāsa, ahaṃ bhante,
Sabbā āpattiyo ārocemi.
Dutiyam-pi ahaṃ bhante,
Sabbā āpattiyo ārocemi.
Tatiyam-pi ahaṃ bhante,
Sabbā āpattiyo ārocemi.’
(‘I ven. sir, declare all offences. For the second
time… For the third time…’)

Senior Acknowledging Bhikkhu:
/‘Sādhu, sādhu.’
(‘It is good, it is good.’)

JCB:
‘Okāsa ahaṃ bhante,
Sambahulā nānā-vatthukā āpattiyo āpajjiṃ,
Tā tumha-mūle paṭidesemi.’
(‘I, ven. sir, having many times fallen into many
different offences with different bases, these I confess.’)

SAB:

‘Passasi āvuso tā āpattiyo?’
(‘Do you see, friend, those offences?’)

JCB:

‘Āma bhante passāmi.’
(‘Yes, ven. sir, I see.’)

SAB:

‘Āyatiṃ āvuso saṃvareyyāsi.’

(‘In the future, friend, you should be restrained.’)

JCB: ‘Sādhu suṭṭhu bhante āyatiṃ saṃvarissāmi.
Dutiyam-pi sādhu suṭṭhu bhante āyatiṃ
saṃvarissāmi.
Tatiyam-pi sādhu suṭṭhu bhante āyatiṃ

saṃvarissāmi.’
(‘It is well indeed, ven. sir, in future I shall be
restrained. For the second time…For the third time…’)

SAB:

/‘Sādhu, sādhu.’
(‘It is good, it is good.’)

/JCB:

‘Okāsa ahaṃ bhante,
Sabbā tā garukāpattiyo āvikaromi.
Dutiyam-pi okāsa ahaṃ bhante,
Sabbā tā garukāpattiyo āvikaromi.
Tatiyam-pi okāsa ahaṃ bhante,
Sabbā tā garukāpattiyo āvikaromi.’

(‘Ven. sir, I reveal all heavy offences. For the second
time… For the third time…’)/

• This final declaration is only used in some

communities. Also, some communities will
acknowledge with a ‘Sādhu’ after each declaration rather than as shown above. That is
after each ‘ārocemi’ and each ‘saṃvarissāmi’.
(c) Sri Lankan Formula for same base offences.
JCB:

‘Okāsa ahaṃ bhante,
Desanādukkaṭāpattiṃ āpajjiṃ,
Taṃ tumha-mūle paṭidesemi.’

(‘I, ven. sir, confess an offence of wrong-doing
through having confessed the same-based offences.’)

SAB:

‘Passasi āvuso taṃ āpaṭṭiṃ?’
(‘Do you see, friend, that offence?’)

JCB:

‘Āma bhante passāmi.’
(‘Yes, ven. sir, I see.’)

SAB:

‘Āyatiṃ āvuso saṃvareyyāsi.’

(‘In the future, friend, you should be restrained.’)

JCB: ‘Sādhu suṭṭhu bhante āyatiṃ saṃvarissāmi.
Dutiyam-pi sādhu suṭṭhu … .
Tatiyam-pi … saṃvarissāmi.’
(‘It is well indeed, ven. sir, in future I shall be
restrained. For the second time… For the third time…’)

SAB:

‘Sādhu, sādhu.’
(‘It is good, it is good.’)

[cf. Vin,II,102]

\section{Nissaggiya Pācittiya}

When confessing a nissaggiya pācittiya (‘expiation with forfeiture’) offence, substitute ‘nissaggiyāyo pācittiyāyo’ for ‘thullaccayāyo’, or
‘nissaggiyaṃ pācittiyaṃ’ for ‘thullaccayaṃ’ in
the formula <5.iv> above.
♦However, before confessing, the article in
question must be forfeited to another bhikkhu
or to a Sangha.
[Vin,III,196f]

(i) Nissaggiya Pācittiya 1 (‘extra robe’)
On the eleventh dawn of keeping one ‘extrarobe’, within forearm's length, forfeiting to a
more senior bhikkhu:
‘Idaṃ me *bhante* cīvaraṃ dasāhātikkantaṃ
nissaggiyaṃ, imāhaṃ āyasmato nissajjāmi.’
(‘This extra robe, ven. sir, which has passed beyond the
ten day (limit) is to be forfeited by me: I forfeit it to you.’)

• More than one robe, within forearm's
length:
‘Imāni me bhante, cīvarāni dasāhātikkantāni
nissaggiyāni, imānāhaṃ āyasmato nissajjāmi.’
• If forfeiting to a Sangha:
‘āyasamato’ →‘saṅghassa’ [Vin,III,197]
• If forfeiting to a group of bhikkhus:
‘āyasamato’ →‘āysamantānaṃ’
• If senior bhikkhu: ‘bhante’ →‘āvuso’
• If beyond forearm's length:
‘idaṃ’ (this) →‘etaṃ’(that)
‘imāhaṃ’ →‘etāhaṃ’
‘imāni’(these) →‘etāni’ (those)
‘imānāhaṃ’ →‘etānāhaṃ’

(ii) Returning the robe
‘Imaṃ cīvaraṃ āyasmato dammi.’
(‘I give this robe to you.’)

[Vin,III,197]

• For returning more than one robe:
‘imaṃ’ →‘imāni’ ; ‘cīvaraṃ’ →‘cīvarāni’
♦This formula for returning the article(s) also

applies in Nis. Pāc. 2, 3, 6, 7, 8, 9, 10 below.

(iii) Nissaggiya Pācittiya 2 (‘separated from’)
‘Idaṃ me bhante cīvaraṃ ratti-vippavutthaṃ
aññatra bhikkhu-sammatiyā nissaggiyaṃ.
Imāhaṃ āyasmato nissajjāmi.’
(‘This robe, ven. sir, which has stayed separate (from
me) for a night without the consent of the bhikkhus,

is to be forfeited by me: I forfeit it to you.’)
[Vin,III,199–200]

• If multiple robes: ‘cīvaraṃ’ → ‘dvicīvaraṃ’/

‘ticīvaraṃ’ (two- / three-robes)
♦ For other variants, see <6.i> above.
♦ For returning the robe(s) see <6.ii> above.

(iv) Nissaggiya Pācittiya 3 (‘over-kept cloth’)
‘Idaṃ me bhante akāla-cīvaraṃ
māsātikkantaṃ nissaggiyaṃ,
imāhaṃ āyasmato nissajjāmi.’
(‘This, ven. sir, ‘out of season’ robe, which has
passed beyond the month limit, is to be forfeited
by me: I forfeit it to you.’)
[Vin,III,205]

• For more than one piece of cloth:

‘Imāni me bhante akāla-cīvarāni
māsātikkantāni nissaggiyāni.
Imānāhaṃ āyasmato nissajjāmi.’
♦ For other variants, see <6.i> above.
♦ For returning the robe(s) see <6.ii> above.

(v) Nissaggiya Pācittiya 6 (‘asked for’)
‘Idaṃ me bhante cīvaraṃ aññātakaṃ
gahapatikaṃ aññatra samayā viññāpitaṃ
nissaggiyaṃ, imāhaṃ āyasmato nissajjāmi.’
(‘This robe, ven. sir, which has been asked from an
unrelated householder at other than the proper
occasion, is to be forfeited by me: I forfeit it to you.’)
[Vin,III,213]

• For more than one piece of cloth:

‘Imāni me bhante cīvarāni aññātakaṃ
gahapatikaṃ aññatra samayā viññāpitāni
nissaggiyāni. Imānāhaṃ āyasmato nissajjāmi.’
♦ For other variants, see <6.i> above.
♦ For returning the robe(s) see <6.ii> above.

(vi) Nissaggiya Pācittiya 7 (‘beyond limit’)
‘Idaṃ me bhante cīvaraṃ aññātakaṃ
gahapatikaṃ *upasaṃkamitvā* tat'uttariṃ
viññāpitaṃ nissaggiyaṃ,
imāhaṃ āyasmato nissajjāmi.’
(‘This robe, ven. sir, which has been asked for beyond
the limitation from an unrelated householder, is to be
forfeited by me: I forfeit it to you.’)
[Vin,III,214–215]

• ‘upasaṃkamitvā’ is included in Sri Lanka.
• For more than one piece of cloth:

‘Imāni me bhante cīvarāni aññātakaṃ
gahapatikaṃ tat'uttariṃ viññāpitāni
nissaggiyāni. Imānāhaṃ āyasmato nissajjāmi.’
♦ For other variants, see <6.i> above.
♦ For returning the robe(s) see <6.ii> above.

(vi) Nissaggiya Pācittiya 8 (‘instructing’)
‘Idaṃ me bhante cīvaraṃ pubbe appavārito
aññātakaṃ gahapatikaṃ upasaṃkamitvā
cīvare vikappaṃ āpannaṃ nissaggiyaṃ.
Imāhaṃ āyasmato nissajjāmi.’
(‘This robe, ven. sir, which has been instructed about
after having approached an unrelated householder

without prior invitation is to be forfeited by me: I
[Vin,III,217]
forfeit it to you.’)

♦ For other variants, see <6.i> above.
♦ For returning the robe(s) see <6.ii> above.

(vii) Nissaggiya Pācittiya 9 (‘instructing’)
For a robe (robe-cloth) received after making
instructions to two or more householders. Use
formula of <6.vi> above but change:
‘aññātakaṃ gahapatikaṃ’
→‘aññātake gahapatike’
♦ For returning the robe(s) see <6.ii> above.
[Vin,III,219]

(viii) Nissaggiya Pācittiya 10 (‘reminding’)
‘Idaṃ me bhante cīvaraṃ atireka-tikkhattuṃ
codanāya atireka-chakkhattuṃ ṭhānena
abhinipphāditaṃ nissaggiyaṃ,
imāhaṃ āyasmato nissajjāmi.’
(‘This robe, ven. sir, which has been effected/
obtained by inciting more than three times, by
standing more than six times, is to be forfeited
by me: I forfeit it to you.’)
[Vin,III,223]

♦ For other variants, see <6.i> above.
♦ For returning the robe(s) see <6.ii> above.

(ix) Nissaggiya Pācittiya 18 (‘gold and silver’)
‘Ahaṃ bhante rūpiyaṃ paṭiggahesiṃ.
Idaṃ me nissaggiyaṃ.
Imāhaṃ saṅghassa nissajjāmi.’

(‘Ven. sirs, I have accepted money. This is to be
forfeited by me: I forfeit it to the Saṅgha.’)
[Vin,III,238]

♦ To be forfeited to the Sangha only.

(x) Nissaggiya Pācittiya 19 (‘monetar y exchange’)
‘Ahaṃ bhante nānappakārakaṃ rūpiyasaṃvohāraṃ samāpajjiṃ. Idaṃ me
nissaggiyaṃ. Imāhaṃ saṅghassa nissajjāmi.’
(‘Ven. sirs, I have engaged in various kinds of
trafficking with money. This (money) is to be
forfeited by me: I forfeit it to the Saṅgha.’)
[Vin,III,240]

♦ To be forfeited to the Sangha only.

(xi) Nissaggiya Pācittiya 20 (‘buying and selling’)
‘Ahaṃ bhante nānappakārakaṃ
kayavikkayaṃ samāpajjiṃ, idaṃ me
nissaggiyaṃ, imāhaṃ āyasmato nissajjāmi.’
(‘Ven. sir, I have engaged in various kinds of buying
and selling. This (gain) of mine is to be forfeited by
me, I forfeit it to you.’)
[Vin,III,242]

• If forfeiting to a Sangha:
‘āyasmato’ →‘saṅghassa’
• If forfeiting to a group of bhikkhus:
‘āyasmato’ →‘āyasmantānaṃ’
♦ For other variants, see <6.i> above.

(xii) Nissaggiya Pācittiya 21 (‘extra bowl’)
‘Ayaṃ me bhante patto dasāhātikkanto
nissaggiyo, imāhaṃ āyasmato nissajjāmi.’

(‘This bowl, ven. sir, which has passed beyond the
ten-day (limit) is to be forfeited by me:
I forfeit it to you.’)
[Vin,III,243–244]

♦ For other variants, see <6.i> above.
♦ For returning the bowl:

‘Imaṃ pattaṃ āyasmato dammi.’
(‘I give this bowl to you.’)

(xiii) Nissaggiya Pācittiya 22 (‘new bowl’)
‘Ayaṃ me bhante patto ūnapañcabandhanena pattena cetāpito nissaggiyo.
Imāhaṃ saṅghassa nissajjāmi.’
(‘This bowl, ven. sirs, which has been exchanged for
a bowl that has less than five mends, is to be forfeited
by me: I forfeit it to the Sangha.’)
[Vin,III,246]

♦ To be forfeited to the Sangha only.

(xiv) Nissaggiya Pācittiya 23 (‘kept medicines’)
‘Idaṃ me bhante bhesajjaṃ sattāhātikkantaṃ
nissaggiyaṃ. Imāhaṃ āyasmato nissajjāmi.’
(‘This medicine, ven. sir, which has been passed
beyond the seven-day (limit), is to be forfeited by me:
[Vin,III,251]
I forfeit it to you.’)

♦ Medicine can be returned, but not for consumption:
‘Imaṃ bhesajjaṃ āyasmato dammi.’
(‘I give this medicine to you.’)

(xiii) Nissaggiya Pācittiya 25 (‘snatched back’)
‘Idaṃ me bhante cīvaraṃ bhikkhussa sāmaṃ

datvā acchinnaṃ nissaggiyaṃ.
Imāhaṃ āyasmato nissajjāmi.’
(‘This robe, ven. sir, which has been snatched back
after having given it myself to a bhikkhu, is to be
forfeited by me: I forfeit it to you.’) [Vin,III,255]

♦ For other variants, see <6.i> above.

(xv) Nissaggiya Pācittiya 28 (‘urgent’)
‘Idaṃ me bhante acceka-cīvaraṃ cīvara-kālasamayaṃ atikkāmitaṃ nissaggiyaṃ.
Imāhaṃ āyasmato nissajjāmi.’
(‘This robe-offered-in-urgency, ven. sir, has passed
beyond the robe-season, is to be forfeited by me: I
forfeit it to you.’)
[Vin,III,262]

♦ For other variants, see <6.i> above.

(xvi) Nissaggiya Pācittiya 29 (‘wilderness abode’)
‘Idaṃ me bhante cīvaraṃ atireka-chā-rattaṃ
vippavutthaṃ aññatra bhikkhu-sammatiyā
nissaggiyaṃ. Imāhaṃ āyasmato nissajjāmi.’
(‘This robe, ven. sir, which has stayed separate (from
me) for a night without the consent of the bhikkhus,
is to be forfeited by me: I forfeit it to you.’) [Vin,III,264]

♦ For other variants, see <6.i> above.

(xvii) Nissaggiya Pācittiya 30
‘Idaṃ me bhante jānaṃ saṅghikaṃ lābhaṃ
pariṇataṃ attano pariṇāmitaṃ nissaggiyaṃ,
imāhaṃ āyasmato nissajjāmi.’
(‘This gain belonging to the Saṅgha, ven. sir, which

has been (already) diverted (to someone), (and)
which has been knowingly diverted to myself
(instead) is to be forfeited by me: I forfeit it to you.’)
[Vin,III,266]

♦ For other variants, see <6.i> above.
♦ To return the article:

‘Imaṃ āyasmato dammi.’

\section{Saṅghādisesa}

(i) A bhikkhu who has committed saṅghādisesa must first inform one or more bhikkhus,
a n d t he n i n f or m a S a n g h a of a t l e a s t f ou r
bhikkhus of his fault(s) and ask to observe
mānatta. When the Sangha has given mānatta
to that bhikkhu, he recites the formula undertaking mānatta and then practises the
appropriate duties for six days and nights.
When the bhikkhu has completed practising
mānatta, he requests rehabilitation (abbhāna)
in the presence of a Sangha of at least twenty
bhikkhus.
(ii) A bhikkhu who has committed saṅghādisesa and deliberately concealed it must first
live in parivāsa (probation) for the number of
days that the offence was concealed. When
the bhikkhu has completed his time living in
parivāsa, he requests mānatta and then follows the procedure outlined in (i) above.

