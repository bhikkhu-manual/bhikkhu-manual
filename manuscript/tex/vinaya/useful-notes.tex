\chapter{Useful Notes}

\section*{Invitation to Request}

An invitation to request (\emph{pavāraṇā}), unless otherwise specified, lasts at
most four months. One may make requests of blood-relatives (but not in-laws)
without receiving an invitation. One may give special help to one's parents as
well as to one's steward and to anyone preparing to become a bhikkhu.

\suttaRef{Vin.IV.101–104}

\section*{Abusive Speech}

The bases of abuse are rank of birth, personal name, clan name, work, art,
disease, physical appearance, mental stains, faults, and other bases. There are
both direct abuse and sarcasm and ridicule. Abusive speech may be a base for
either expiation (or wrong-doing) or, when only teasing, for \emph{dubbhāsita}.

\suttaRef{Vin.IV.4–11}

\section*{Unallowable Meats}

The flesh of humans (this is a base for \emph{thullaccaya}), elephants, horses,
dogs, snakes, lions, tigers, leopards, bears, and panthers.\\
\mbox{}\suttaRef{Vin.I.218–219}

Also unallowable is flesh incompletely cooked, and meat from an animal seen,
heard or suspected to have been killed in order that its meat be offered to
bhikkhus.\\
\mbox{}\suttaRef{Vin.I.218–219}

\section*{The Eight Utensils (aṭṭha-parikkhārā)}

The three robes, the bowl, a razor/sharp knife, needle, belt, water-filter.\\
\mbox{}\suttaRef{Ja.I.65; Da.I.206}

\vspace*{-\baselineskip}
