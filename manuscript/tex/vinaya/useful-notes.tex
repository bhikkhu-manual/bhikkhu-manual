\chapter{Useful Notes}

\section*{Invitation to Request}

An invitation to request (\emph{pavāraṇā}), unless otherwise specified, lasts at
most four months. One may make requests of blood-relatives (but not in-laws)
without receiving an invitation. One may give special help to one's parents as
well as to one's steward and to anyone preparing to become a bhikkhu.

\suttaRef{Vin.IV.101–104}

\section*{Hatthapāsa Distance}

The \emph{hatthapāsa} distance (arm's reach) is measured from the trunk of the
body, until the tip of the fingers of one's extended hands, about 1.25 metres.

The distance is measured horizontally, the vertical elevation is not taken into
account.

\section*{Days and Dawns}

The Vinaya terminology for \emph{one day} is the period of time between two
dawnrises.

If one extends a hand and is able to see the lines in the palm, the dawnrise
\emph{has already passed}.

One may also use a clock and the time of \emph{nautical twilight}. While the Sun
is between 12 and 6 degrees below the horizon, dawnrise \emph{has not yet passed}.

\section*{Seven-day Allowable Period}

There is sometimes a confusion of terms, as the \emph{seven-day tonics} are
permitted to be kept until the seventh \emph{dawnrise}, not for a seven-day
period, which is already past the seventh dawnrise.

The factor of \emph{effort} here is keeping the tonic past the seventh dawnrise
after receiving it.

\emph{Perception} is not a factor, if one thinks the seventh dawnrise hasn't
passed, but it has, it is nonetheless a \emph{nissaggiya pācittiya} offence.

The offence is to be confessed by the bhikkhu who received the items. If he has
traveled away since, and the items are no longer with him to be forfeited, he
may confess the offence, and the other bhikkhus may forfeit the items.

\clearpage

\section*{Mixing Allowables}

The day on which food, one-day, seven-day and lifetime allowables are received
should be kept in mind, as consuming a mixture can be a \emph{pācittiya}
offence.

The mixture takes on the shortest lifetime of the ingredients. 

The combinations are described in the \emph{Mahāvagga}:

\begin{tabular}{@{}lllll@{}}
  a. & 1d juice & rec. that morning & & \\
     &  + food     & rec. that morning & \(\rightarrow\) & that morning\\
  \hline
  b. & 7d tonic & rec. that morning & & \\
     & + food   & rec. that morning & \(\rightarrow\) & that morning\\
  \hline
  c. & lifetime medicine & rec. that morning & & \\
     & + food            & rec. that morning & \(\rightarrow\) & that morning\\
  \hline
  d. & 7d tonic & rec. sometime & & \\
     & + juice  & rec. that day & \(\rightarrow\) & until dawn\\
  \hline
  e. & lifetime medicine & rec. sometime & & \\
     & + juice           & rec. that day & \(\rightarrow\) & until dawn\\
  \hline
  f. & lifetime medicine & rec. sometime & & \\
     & + 7d tonic        & rec. sometime & \(\rightarrow\) & 7 days\\
\end{tabular}

\suttaRef{Mv. VI.40.3.}

\section*{Unallowable Meats}

The flesh of humans (this is a base for \emph{thullaccaya}), elephants, horses,
dogs, snakes, lions, tigers, leopards, bears, and panthers.\\
\mbox{}\suttaRef{Vin.I.218–219}

Also unallowable is flesh incompletely cooked, and meat from an animal seen,
heard or suspected to have been killed in order that its meat be offered to
bhikkhus.\\
\mbox{}\suttaRef{Vin.I.218–219}

\section*{Abusive Speech}

The bases of abuse are rank of birth, personal name, clan name, work, art,
disease, physical appearance, mental stains, faults, and other bases. There are
both direct abuse and sarcasm and ridicule. Abusive speech may be a base for
either expiation (or wrong-doing) or, when only teasing, for \emph{dubbhāsita}.

\suttaRef{Vin.IV.4–11}

\section*{The Eight Utensils (aṭṭha-parikkhārā)}

The three robes, the bowl, a razor/sharp knife, needle, belt, water-filter.\\
\mbox{}\suttaRef{Ja.I.65; Da.I.206}