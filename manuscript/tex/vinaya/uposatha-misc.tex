\section{Uposatha-day for Sāmaṇeras and Lay-followers}

\subsection{Ten Precepts for Sāmaṇeras (Thai tradition)}

% TODO: included in Sāmaṇera Sikkhā in Patimokkha Chants

‘Pāṇātipātā veramaṇī.\\
Adinnādānā veramaṇī.\\
Abrahma-cariyā veramaṇī.\\
Musā-vādā veramaṇī.\\
Surā-meraya-majja-pamādaṭṭhānā veramaṇī.\\
Vikāla-bhojanā veramaṇī.\\
Nacca-gīta vādita visūka-dassanā veramaṇī.\\
Mālā-gandha vilepana dhāraṇa maṇḍana vibhūsanaṭṭhānā veramaṇī.\\
Uccā-sayana mahā-sayanā veramaṇī.\\
Jātarūpa-rajata paṭiggahaṇā veramaṇī.’

{\itshape

‘I undertake the precept to refrain from:

\begin{packeditemize}

\item destroying living beings.
\item taking that which is not given.
\item any kind of intentional sexual behaviour.
\item false speech.
\item intoxicating drinks and drugs that lead to carelessness.
\item eating at wrong times.
\item dancing, singing, music and going to entertainments.
\item perfumes, beautification and adornment.
\item lying on a high or luxurious sleeping place.
\item accepting gold or silver.’

\end{packeditemize}

}

After the tenth precept, the bhikkhu:

‘Imāni dasa sikkhā-padāni samādiyāmi.’

The sāmaṇera repeats this three times.

\suttaRef{Vin.I.83–84}

\subsection{Eight Precepts (Thai Tradition)}

% TODO: included in Other Procedures

After bowing three times, with hands in \emph{añjali}, the laypeople recite the
following request:

‘\emph{Mayaṃ} bhante ti-saraṇena saha aṭṭha sīlāni \emph{yācāma}.\\
Dutiyam-pi mayaṃ bhante…\\
Tatiyam-pi mayaṃ bhante…’

‘\emph{We, ven. sir, request the 3 Refuges and the 5 Precepts.\\
  For the second time… For the third time…}’

As an individual, or one on behalf of a group:

‘Mayaṃ’ → ‘Ahaṃ’ ; ‘yācāma’ → ‘yācāmi’

\emph{Bhikkhu}: ‘Namo…’ (×3)\\
\emph{Laypeople}: repeat.

\emph{Bhk}: ‘Buddhaṃ saraṇaṃ gacchāmi.\\
Dhammaṃ saraṇaṃ gacchāmi.\\
Saṅghaṃ saraṇaṃ gacchāmi.\\
Dutiyam-pi Buddhaṃ saraṇaṃ gacchāmi.\\
Dutiyam-pi Dhammaṃ saraṇaṃ gacchāmi.\\
Dutiyam-pi Saṅghaṃ saraṇaṃ gacchāmi.\\
Tatiyam-pi Buddhaṃ saraṇaṃ gacchāmi.\\
Tatiyam-pi Dhammaṃ saraṇaṃ gacchāmi.\\
Tatiyam-pi Saṅghaṃ saraṇaṃ gacchāmi.’

\emph{Laypeople}: repeat line by line.

\emph{Bhk}: ‘Ti-saraṇa-gamanaṃ niṭṭhitaṃ.’\\
‘\emph{This completes the going to the 3 Refuges.}’

\emph{Laypeople}: ‘Āma bhante.’\\
‘\emph{Yes, ven. sir.}’

Then the bhikkhu recites, with the laypeople repeating line by line:

‘Pāṇātipātā veramaṇī sikkhā-padaṃ samādiyāmi.\\
Adinnādānā veramaṇī sikkhā-padaṃ samādiyāmi.\\
Abrahma-cariyā veramaṇī sikkhā-padaṃ samādiyāmi.\\
Musāvādā veramaṇī sikkhā-padaṃ samādiyāmi.\\
Surā-meraya-majja-pamādaṭṭhānā veramaṇī sikkhā-padaṃ samādiyāmi.\\
Vikāla-bhojanā veramaṇī sikkhā-padaṃ samādiyāmi.\\
Nacca-gīta vādita visūka-dassana mālāgandha vilepana dhāraṇa maṇḍana
vibhūsanaṭṭhānā veramaṇī sikkhā-padaṃ samādiyāmi.\\
Uccā-sayana mahā-sayanā veramaṇī sikkhā-padaṃ samādiyāmi.’

\suttaRef{A.IV.248–250}

{\itshape

‘I undertake the precept to refrain from:

\begin{packeditemize}

\item destroying living beings.
\item taking that which is not given.
\item any kind of intentional sexual behaviour.
\item false speech.
\item intoxicating drinks and drugs that lead to carelessness.
\item eating at wrong times.
\item dancing, singing, music and going to entertainments.
\item perfumes, beautification and adornment.
\item lying on a high or luxurious sleeping place.
\item accepting gold or silver.’

\end{packeditemize}

}

\emph{Bhk}: ‘Imāni aṭṭha sikkhā-padāni samādiyāmi’

\emph{Laypeople}: ‘Imāni aṭṭha sikkhā-padāni samādiyāmi’ (×3)\\
‘\emph{I undertake the Eight Precepts.}’

The bhikkhu then chants:

‘Imāni aṭṭha sikkhā-padāni\\
Sīlena sugatiṃ yanti,\\
Sīlena bhoga-sampadā,\\
Sīlena nibbutiṃ yanti,\\
Tasmā sīlaṃ visodhaye.’

‘\emph{These Eight Precepts\\
Have morality as a vehicle for happiness,\\
Have morality as a vehicle for good fortune,\\
Have morality as a vehicle for liberation,\\
Let morality therefore be purified.}’

The Laypeople may respond with:

‘Sādhu, sādhu, sādhu!’

Alternatively, the laypeople may chant:

‘Imaṃ aṭṭh'aṅga-samannāgataṃ buddhapaññattaṃ uposathaṃ, imañ-ca rattiṃ imañca
divasaṃ, samma-deva abhirakkhituṃ samādiyāmi.’

\emph{Bhk}: ‘Imāni aṭṭha sikkhā-padāni, ajj'ekaṃ rattin-divaṃ, uposatha (sīla)
vasena sādhukaṃ (katvā appamādena) rakkhitabbāni.’

\emph{Laypeople}: ‘Āma bhante.’

\emph{Bhk}:

‘Sīlena sugatiṃ yanti,\\
Sīlena bhoga-sampadā,\\
Sīlena nibbutiṃ yanti,\\
Tasmā sīlaṃ visodhaye.’

Having undertaken the Eight Precepts, layfollowers may stay overnight. The next
morning they will take their leave from the bhikkhu, who responds:

‘Yassa dāni tumhe kālaṃ maññatha.’\\
‘\emph{Please do what is appropriate at this time.}’

