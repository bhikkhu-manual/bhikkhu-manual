\chapter{Other Procedures}

\section{Nissaya (Dependence)}
\label{nissaya}

Taking dependence happens either individually or with the whole community before
the Rains Retreat and Winter Retreat. It is frequently preceded by \emph{Asking
  for Forgiveness}, see p.\pageref{asking-forgiveness} for preparation.

The bhikkhu:

\vspace*{\parskip}

\begin{paritta}
  ‘Ācariyo me bhante hohi,\\
  āyasmato nissāya vacchāmi.’ (×3)
\end{paritta}

\emph{(Ven. sir, may you be my teacher. I shall stay dependent on the ven. one.)}

The Ācariya:

\vspace*{\parskip}

\begin{paritta}
  ‘Sādhu, lahu, opāyikaṃ, paṭirūpaṃ,\\
  pāsādikena sampādehi.’
\end{paritta}

\emph{(It is good; …convenient; …suitable; …proper; … you should endeavour to
  conduct yourself in a good manner.)} \suttaRef{Vin.I.60–61}

The bhikkhu:

\vspace*{\parskip}

\begin{paritta}
  ‘Sādhu bhante. Ajja-t-agge-dāni thero mayhaṃ bhāro, aham-pi therassa bhāro.’ (×3)
\end{paritta}

\emph{(It is good, ven. sir. From this day onwards the Elder will be my burden
  and I shall be the burden of the Elder.)} \suttaRef{Sp.V.977}

At the end, bow three times and sit with feet folded on one side. The senior
monk may offer advice and encouragement in the practice.

\section{Kappiya-karaṇa (Making Fruit Allowable)}

For fruit with seeds or vegetables that can grow again, the bhikkhu says:

‘Kappiyaṃ karohi’ ‘\emph{Make it allowable.}’

The lay person, while ‘marking’ (cutting, tearing or burning) the fruit, etc., responds:

‘Kappiyaṃ bhante.’ ‘\emph{It is allowable, ven. sir.}’ \suttaRef{Sp.IV.767–768}

\section{Entering Town after Midday}

Leave can be taken in one's own language, or in Pali:

‘Vikāle gāmappavesanaṃ āpucchāmi.’\\
\emph{(I take leave to go to the town at the ‘wrong time’.)} \suttaRef{Kv.140}

\section{Saṅghadāna-apalokana (Sharing Saṅghadāna)}

After \emph{saṅghadāna} is offered, a bhikkhu, other than the senior
\emph{Thera}, recites:

\vspace*{\parskip}

\begin{paritta}
‘Yagghe bhante saṅgho jānātu.\\
Ayaṃ paṭhama-bhāgo \emph{therassa} pāpuṇāti.\\
Avasesā bhāgā amhākañc’eva pāpuṇanti.\\
Bhikkhūnañca sāmaṇerānaṃ gahaṭṭhānaṃ\\
Te yathāsukhaṃ paribhuñjantu.’ (×3)
\end{paritta}

‘therassa’ → ‘mahātherassa’

‘\emph{May the Saṅgha hear me. The first portion (of this offering) goes to the
  Elders. The remainder is for the rest of us here.}’

The Sangha responds: ‘Sādhu.’ \suttaRef{Sp.VII.1405–1409}

\section{Paṃsukūla-cīvara (Taking Rag-cloth)}

‘Imaṃ paṃsukūla-cīvaraṃ assāmikaṃ mayhaṃ pāpuṇāti.’\\
‘\emph{This rag-cloth, which is ownerless, has reached me.}’

\ifhandbookedition
\clearpage
\fi

\section{Desanā}

\subsection{Requesting permission to speak on Vinaya}

\vspace*{-0.5\baselineskip}

\begin{instruction}
  After bowing three times, with hands joined in añjali, addressing the senior
  elder:
\end{instruction}

‘Okāsaṃ me bhante thero detu vinaya-kathaṃ kathetuṃ.’

‘Namo… (×3); Buddhaṃ Dhammaṃ Saṅghaṃ namassāmi.’

‘Vinayo sāsanassa āyū'ti. Karotu me āyasmā okāsaṃ ahan-taṃ vattukāmo.’

‘\emph{Ven. sir, please give permission to speak on Vinaya… Vinaya is the life
  of the religion. I ask for permission from the ven. one: I wish to speak about
  the Vinaya.}’

Reply: ‘Karomi āyasmato okāsaṃ.’\\
‘\emph{I give you the opportunity, venerable.}’ \suttaRef{Vin.I.113}

\subsection{Requesting permission to speak on Dhamma}

\vspace*{-0.5\baselineskip}

\begin{instruction}
  After bowing three times, with hands joined in añjali, addressing the senior
  bhikkhu:
\end{instruction}

‘Okāsaṃ me bhante thero detu dhamma-kathaṃ kathetuṃ.’

‘Namo… (×3); Buddhaṃ Dhammaṃ Saṅghaṃ namassāmi.’

‘Apārutā tesaṃ amatassa dvārā ye sotavantā pamuñcantu saddhaṃ.’

% TODO: add note on Apārutā...
% Gavesako: This phrase is used particularly by LP Sumedho and only him. Usually in Thailand there are other phrases used to start a Dhamma talk (e.g. Ito paraṃ sakkaccaṃ dhammo sotabbo'ti. After this you should attentively listen to the Dhamma.). Anyone can pick any Pali phrase from the Suttas actually. 

‘\emph{Ven. sir, please give permission to speak on Dhamma… Open are the doors
  to the Deathless. May all those who have ears release their faith.}’

\section{Añjali}

Chanting and making formal requests is done with the hands in añjali.
This is a gesture of respect, made by placing the palms together
directly in front of the chest, with the fingers aligned and pointing
upwards.

\section{Requesting a Dhamma Talk}

\vspace*{-0.2\baselineskip}

\begin{instruction}
  After bowing three times, with hands joined in añjali,\\
  recite the following:
\end{instruction}

\vspace*{\parskip}

\begin{paritta}
Brahmā ca lokādhipatī sahampati\\
Katañjalī anadhivaraṃ ayācatha

Santīdha sattāpparajakkha-jātikā\\
Desetu dhammaṃ anukampimaṃ pajaṃ
\end{paritta}

\begin{instruction}
  (Bow three times again)
\end{instruction}

\ifhandbookedition
\clearpage
\fi

\begin{english}
The Brahmā god Sahampati, Lord of the world,\\
With palms joined in reverence, requested a favour:

`Beings are here with but little dust in their eyes,\\
Pray, teach the Dhamma out of compassion for them.'
\end{english}

\suttaRef{Buddhavaṃsa 1}

\section{Acknowledging the Teaching}

After the talk, the person next in seniority after the speaker, chants:

‘Handa mayaṃ dhamma-kathāya / ovāda-kathāya sādhukāraṃ dadāmase.’

\emph{Now let us express our approval of this Dhamma teaching / exhortation.}

If an exhortation, use \emph{ovāda-kathāya} instead of \emph{dhamma-kathāya}.

The listeners, together:

‘Sādhu, sādhu, sādhu. Anumodāmi.’\\
\emph{It is well, I appreciate it.}

\subsection{After the talk on Vinaya or Dhamma}

When the talk is concluded, the speaker chants:

‘Ayaṃ dhamma- / vinaya- / dhammavinaya-kathā sādh'āyasmantehi saṃrakkhetabbā'ti.’

‘\emph{This talk on Dhamma / Vinaya / Dhammavinaya should be well-preserved by you, ven. sirs.}’

The person next in seniority after the speaker:

‘Handa mayaṃ dhamma- / vinaya- / dhammavinaya-kathāya sādhukāraṃ dadāmase.’

‘\emph{Now let us make the act of acknowledging this Dhamma / Vinaya / Dhammavinaya talk.}’

The listeners, together:

‘Sādhu, sādhu, sādhu. Anumodāmi.’

\ifhandbookedition
\clearpage
\fi

\section[Three Refuges \& the Five Precepts]{Requesting the Three Refuges\newline \& the Five Precepts (Thai Tradition)}

\label{three-refuges}

\begin{instruction}
  After bowing three times, with hands joined in añjali,\\
  recite the appropriate request.
\end{instruction}

\ifhandbookedition
\enlargethispage{\baselineskip}
\fi

\prul{For a group from a monk}

Mayaṃ bhante tisaraṇena saha\\\vin pañca sīlāni yācāma\\
Dutiyampi mayaṃ bhante tisaraṇena saha\\\vin pañca sīlāni yācāma\\
Tatiyampi mayaṃ bhante tisaraṇena saha\\\vin pañca sīlāni yācāma

\prul{For oneself from a monk}

Ahaṃ bhante tisaraṇena saha\\\vin pañca sīlāni yācāmi\\
Dutiyampi ahaṃ bhante tisaraṇena saha\\\vin pañca sīlāni yācāmi\\
Tatiyampi ahaṃ bhante tisaraṇena saha\\\vin pañca sīlāni yācāmi

\prul{For a group from a nun}

Mayaṃ ayye tisaraṇena saha\\\vin pañca sīlāni yācāma\\
Dutiyampi mayaṃ ayye tisaraṇena saha\\\vin pañca sīlāni yācāma\\
Tatiyampi mayaṃ ayye tisaraṇena saha\\\vin pañca sīlāni yācāma

\prul{For oneself from a nun}

Ahaṃ ayye tisaraṇena saha\\\vin pañca sīlāni yācāmi\\
Dutiyampi ahaṃ ayye tisaraṇena saha\\\vin pañca sīlāni yācāmi\\
Tatiyampi ahaṃ ayye tisaraṇena saha\\\vin pañca sīlāni yācāmi

\prul{For a group from a layperson}

Mayaṃ mitta tisaraṇena saha\\\vin pañca sīlāni yācāma\\
Dutiyampi mayaṃ mitta tisaraṇena saha\\\vin pañca sīlāni yācāma\\
Tatiyampi mayaṃ mitta tisaraṇena saha\\\vin pañca sīlāni yācāma

\ifhandbookedition
\enlargethispage{\baselineskip}
\fi

\prul{For oneself from a layperson}

Ahaṃ mitta tisaraṇena saha\\\vin pañca sīlāni yācāmi\\
Dutiyampi ahaṃ mitta tisaraṇena saha\\\vin pañca sīlāni yācāmi\\
Tatiyampi ahaṃ mitta tisaraṇena saha\\\vin pañca sīlāni yācāmi

\begin{english}
  We/I, Venerable Sir/Sister/Friend,\\
  request the Three Refuges and the Five Precepts.

  For the second time,\\
  we/I, Venerable Sir/Sister/Friend,\\
  request the Three Refuges and the Five Precepts.

  For the third time,\\
  we/I, Venerable Sir/Sister/Friend,\\
  request the Three Refuges and the Five Precepts.
\end{english}

\begin{instruction}
  Repeat, after the leader has chanted ‘Namo tassa’ three times.
\end{instruction}

Namo tassa bhagavato arahato sammāsambuddhassa (×3)

\begin{english}
  Homage to the Blessed, Noble, and Perfectly Enlightened One.
\end{english}

Buddhaṃ saraṇaṃ gacchāmi\\
Dhammaṃ saraṇaṃ gacchāmi\\
Saṅghaṃ saraṇaṃ gacchāmi

\begin{english}
  To the Buddha I go for refuge.\\
  To the Dhamma I go for refuge.\\
  To the Saṅgha I go for refuge.
\end{english}

Dutiyampi buddhaṃ saraṇaṃ gacchāmi\\
Dutiyampi dhammaṃ saraṇaṃ gacchāmi\\
Dutiyampi saṅghaṃ saraṇaṃ gacchāmi

\begin{english}
  For the second time\ldots
\end{english}

Tatiyampi buddhaṃ saraṇaṃ gacchāmi\\
Tatiyampi dhammaṃ saraṇaṃ gacchāmi\\
Tatiyampi saṅghaṃ saraṇaṃ gacchāmi

\begin{english}
  For the third time\ldots
\end{english}

\begin{instruction}
  Leader:
\end{instruction}

[Tisaraṇa-gamanaṃ niṭṭhitaṃ]\\
\emph{This completes the going to the Three Refuges.}

\begin{instruction}
  Response:
\end{instruction}

Āma bhante / Āma ayye / Āma mitta\\
\emph{Yes, Venerable Sir / Sister / Friend.}

\begin{instruction}
  Repeat each precept after the leader.
\end{instruction}

{\raggedright

\begin{packedenumerate}
  \item Pāṇātipātā veramaṇī sikkhāpadaṃ samādiyāmi\\
    \emph{I undertake the precept to refrain from taking the life of any living~creature.}
  \item Adinnādānā veramaṇī sikkhāpadaṃ samādiyāmi\\
    \emph{I undertake the precept to refrain from taking that which is not given.}
  \item Kāmesu micchācārā veramaṇī sikkhāpadaṃ samādiyāmi\\
    \emph{I undertake the precept to refrain from sexual misconduct.}
  \item Musāvādā veramaṇī sikkhāpadaṃ samādiyāmi\\
    \emph{I undertake the precept to refrain from lying.}
  \item Surāmeraya-majja-pamādaṭṭhānā veramaṇī sikkhāpadaṃ samādiyāmi\\
    \emph{I undertake the precept to refrain from consuming intoxicating drink and drugs which lead to carelessness.}
\end{packedenumerate}

}

\begin{instruction}
  Leader:
\end{instruction}

[Imāni pañca sikkhāpadāni\\
Sīlena sugatiṃ yanti\\
Sīlena bhogasampadā\\
Sīlena nibbutiṃ yanti\\
Tasmā sīlaṃ visodhaye]

\begin{english}
  These are the Five Precepts;\\
  virtue is the source of happiness,\\
  virtue is the source of true wealth,\\
  virtue is the source of peacefulness ---\\
  Therefore let virtue be purified.
\end{english}

\begin{instruction}
  Response:
\end{instruction}

Sādhu, sādhu, sādhu.

\begin{instruction}
  (Bow three times)
\end{instruction}

\clearpage

\section[Three Refuges \& the Eight Precepts]{Requesting the Three Refuges\newline \& the Eight Precepts (Thai Tradition)}

\label{eight-precepts}

\begin{instruction}
  After bowing three times, with hands joined in añjali,\\
  recite the appropriate request.
\end{instruction}

\ifhandbookedition
\enlargethispage{\baselineskip}
\fi

\prul{For a group from a monk}

Mayaṃ bhante tisaraṇena saha\\\vin aṭṭha sīlāni yācāma\\
Dutiyampi mayaṃ bhante tisaraṇena saha\\\vin aṭṭha sīlāni yācāma\\
Tatiyampi mayaṃ bhante tisaraṇena saha\\\vin aṭṭha sīlāni yācāma

\prul{For oneself from a monk}

Ahaṃ bhante tisaraṇena saha\\\vin aṭṭha sīlāni yācāmi\\
Dutiyampi ahaṃ bhante tisaraṇena saha\\\vin aṭṭha sīlāni yācāmi\\
Tatiyampi ahaṃ bhante tisaraṇena saha\\\vin aṭṭha sīlāni yācāmi

\prul{For a group from a nun}

Mayaṃ ayye tisaraṇena saha\\\vin aṭṭha sīlāni yācāma\\
Dutiyampi mayaṃ ayye tisaraṇena saha\\\vin aṭṭha sīlāni yācāma\\
Tatiyampi mayaṃ ayye tisaraṇena saha\\\vin aṭṭha sīlāni yācāma

\prul{For oneself from a nun}

Ahaṃ ayye tisaraṇena saha\\\vin aṭṭha sīlāni yācāmi\\
Dutiyampi ahaṃ ayye tisaraṇena saha\\\vin aṭṭha sīlāni yācāmi\\
Tatiyampi ahaṃ ayye tisaraṇena saha\\\vin aṭṭha sīlāni yācāmi

\prul{For a group from a layperson}

Mayaṃ mitta tisaraṇena saha\\\vin aṭṭha sīlāni yācāma\\
Dutiyampi mayaṃ mitta tisaraṇena saha\\\vin aṭṭha sīlāni yācāma\\
Tatiyampi mayaṃ mitta tisaraṇena saha\\\vin aṭṭha sīlāni yācāma

\ifhandbookedition
\enlargethispage{\baselineskip}
\fi

\prul{For oneself from a layperson}

Ahaṃ mitta tisaraṇena saha\\\vin aṭṭha sīlāni yācāmi\\
Dutiyampi ahaṃ mitta tisaraṇena saha\\\vin aṭṭha sīlāni yācāmi\\
Tatiyampi ahaṃ mitta tisaraṇena saha\\\vin aṭṭha sīlāni yācāmi

\begin{english}
  We/I, Venerable Sir/Sister/Friend,\\
  request the Three Refuges and the Eight Precepts.

  For the second time,\\
  We/I, Venerable Sir/Sister/Friend,\\
  request the Three Refuges and the Eight Precepts.

  For the third time,\\
  We/I, Venerable Sir/Sister/Friend,\\
  request the Three Refuges and the Eight Precepts.
\end{english}

\begin{instruction}
  Repeat, after the leader has chanted ‘Namo tassa’ three times.
\end{instruction}

Namo tassa bhagavato arahato sammāsambuddhassa (×3)

\begin{english}
  Homage to the Blessed, Noble, and Perfectly Enlightened One.
\end{english}

Buddhaṃ saraṇaṃ gacchāmi\\
Dhammaṃ saraṇaṃ gacchāmi\\
Saṅghaṃ saraṇaṃ gacchāmi

\begin{english}
  To the Buddha I go for refuge.\\
  To the Dhamma I go for refuge.\\
  To the Saṅgha I go for refuge.
\end{english}

Dutiyampi buddhaṃ saraṇaṃ gacchāmi\\
Dutiyampi dhammaṃ saraṇaṃ gacchāmi\\
Dutiyampi saṅghaṃ saraṇaṃ gacchāmi

\begin{english}
  For the second time\ldots
\end{english}

Tatiyampi buddhaṃ saraṇaṃ gacchāmi\\
Tatiyampi dhammaṃ saraṇaṃ gacchāmi\\
Tatiyampi saṅghaṃ saraṇaṃ gacchāmi

\begin{english}
  For the third time\ldots
\end{english}

\begin{instruction}
  Leader:
\end{instruction}

[Tisaraṇa-gamanaṃ niṭṭhitaṃ]\\
\emph{This completes the going to the Three Refuges.}

\begin{instruction}
  Response:
\end{instruction}

Āma bhante / Āma ayye / Āma mitta\\
\emph{Yes, Venerable Sir / Sister / Friend.}

\begin{instruction}
  Repeat each precept after the leader.
\end{instruction}

\ifhandbookedition
\enlargethispage{-\baselineskip}
\fi

{\raggedright

\begin{packedenumerate}
  \item Pāṇātipātā veramaṇī sikkhāpadaṃ samādiyāmi\\
  \emph{I undertake the precept to refrain from taking the life of any living~creature.}
  \item Adinnādānā veramaṇī sikkhāpadaṃ samādiyāmi\\
  \emph{I undertake the precept to refrain from taking that which is not given.}
  \item Abrahmacariyā veramaṇī sikkhāpadaṃ samādiyāmi\\
  \emph{I undertake the precept to refrain from any intentional sexual activity.}
  \item Musāvādā veramaṇī sikkhāpadaṃ samādiyāmi\\
  \emph{I undertake the precept to refrain from lying.}
  \item Surāmeraya-majja-pamādaṭṭhānā veramaṇī sikkhāpadaṃ samādiyāmi\\
  \emph{I undertake the precept to refrain from consuming intoxicating drink and drugs which lead to carelessness.}
  \item Vikālabhojanā veramaṇī sikkhāpadaṃ samādiyāmi.\\
  \emph{I undertake the precept to refrain from eating at inappropriate times.}
  \item Nacca-gīta-vādita-visūkadassanā mālā-gandha-vilepana-dhāraṇa-maṇḍana-vibhūsanaṭṭhānā veramaṇī sikkhāpadaṃ samādiyāmi.\\
  \emph{I undertake the precept to refrain from entertainment, beautification, and~adornment.}
  \item Uccāsayana-mahāsayanā veramaṇī sikkhāpadaṃ samādiyāmi.\\
  \emph{I undertake the precept to refrain from lying on a high or luxurious sleeping place.}
\end{packedenumerate}

}

\suttaRef{A.IV.248–250}

\ifhandbookedition
\clearpage
\fi

\begin{instruction}
  Leader:
\end{instruction}

[Imāni aṭṭha sikkhāpadāni samādiyāmi]

\begin{instruction}
  Response:
\end{instruction}

Imāni aṭṭha sikkhāpadāni samādiyāmi (×3)

\begin{english}
  I undertake these Eight Precepts.
\end{english}

\begin{instruction}
  Leader:
\end{instruction}

[Imāni aṭṭha sikkhāpadāni\\
Sīlena sugatiṃ yanti\\
Sīlena bhogasampadā\\
Sīlena nibbutiṃ yanti\\
Tasmā sīlaṃ visodhaye]

\begin{english}
  These are the Eight Precepts;\\
  virtue is the source of happiness,\\
  virtue is the source of true wealth,\\
  virtue is the source of peacefulness ---\\
  Therefore let virtue be purified.
\end{english}

\begin{instruction}
  Response:
\end{instruction}

Sādhu, sādhu, sādhu.

\begin{instruction}
  (Bow three times)
\end{instruction}

\prul{Alternative ending for undertaking Uposatha precepts}

\begin{instruction}
  The laypeople may chant:
\end{instruction}

‘Imaṃ aṭṭh'aṅga-samannāgataṃ\\
buddhapaññattaṃ uposathaṃ, imañca rattiṃ\\
imañca divasaṃ, samma-deva abhirakkhituṃ samādiyāmi.’

\begin{instruction}
  Leader:
\end{instruction}

‘Imāni aṭṭha sikkhāpadāni,\\
ajj'ekaṃ rattin-divaṃ, uposatha (sīla)\\
vasena sādhukaṃ (katvā appamādena) rakkhitabbāni.’

\begin{instruction}
  Response:
\end{instruction}

‘Āma bhante.’

\begin{instruction}
  Leader:
\end{instruction}

‘Sīlena sugatiṃ yanti,\\
Sīlena bhoga-sampadā,\\
Sīlena nibbutiṃ yanti,\\
Tasmā sīlaṃ visodhaye.’

\ifhandbookedition
\clearpage
\fi

\subsection{Asking Forgiveness of The Triple Gem}

\instr{(Men Chant)}\\\relax
Ahaṃ buddhañ ca dhammañ ca saṅghañ ca saraṇaṃ gato\\
upāsakattaṃ desesiṃ bhikkhu-saṅghassa sammukhā.

\instr{(Women Chant)}\\\relax
Ahaṃ buddhañ ca dhammañ ca saṅghañ ca saraṇaṃ gatā\\
upāsikattaṃ desesiṃ bhikkhu-saṅghassa sammukhā.

Etaṃ me saraṇaṃ khemaṃ,\\
etaṃ saraṇam uttamaṃ\\
etaṃ saraṇam āgamma sabba-dukkhā pamuccaye.\\
Yathā-balaṃ careyyāhaṃ sammā-sambuddha-sāsanaṃ

\sidepar{m.}%
dukkha-nissaraṇass' eva bhāgī assaṃ anāgate.

\sidepar{w.}%
dukkha-nissaraṇass' eva bhāginissaṃ anāgate.

Kāyena vācāya va cetasā vā\\
buddhe kukammaṃ pakataṃ mayā yaṃ\\
buddho paṭigghaṅhātu accayantaṃ\\
kālantare saṃvarituṃ va buddhe

Kāyena vācāya va cetasā vā\\
dhamme kukammaṃ pakataṃ mayā yaṃ\\
dhammo paṭigghaṅhātu accayantaṃ\\
kālantare saṃvarituṃ va dhamme

Kāyena vācāya va cetasā vā\\
saṅghe kukammaṃ pakataṃ mayā yaṃ\\
saṅgho paṭigghaṅhātu accayantaṃ\\
kālantare saṃvarituṃ va sanghe

\subsection{Taking Leave after Uposatha}

Having undertaken the Eight Precepts, lay followers may stay overnight at the monastery. The next
morning they will take their leave from the senior monk:

\instr{Laypeople:}

Handa dāni mayaṃ bhante āpucchāma\\
bahukiccā mayaṃ bahukaraṇīyā

\instr{Senior monk:}

‘Yassa dāni tumhe kālaṃ maññatha.’\\
‘\emph{Please do what is appropriate at this time.}’

\section{Disrobing}

After the bhikkhus who are to witness the disrobing have assembled, the bhikkhu
who will disrobe should first confess his offences and ask for forgiveness. Then, wearing all his three
robes, with his \emph{saṅghāti} on his left shoulder:

Bow three times.

‘Namo tassa bhagavato arahato\\
sammā-sambuddhassa’ (×3)

Optionally, one may chant \emph{Recollection After Using the Requisites}
(p.\pageref{recollection-after-using}).

Bow three times.

Recite in Pali and in his own language:

‘Sikkhaṃ paccakkhāmi. Gihī'ti maṃ dhāretha.’\\
\emph{I give up the training. May you regard me as a layman.}

He may state this once, three times, or as many times as he needs to assure
himself that he is now a layman and no longer a bhikkhu. If two or more are
disrobing, they should state this passage separately.

The former bhikkhu then withdraws to change into lay clothes. When he returns,
he may request the \emph{Three Refuges and Five Precepts}.

