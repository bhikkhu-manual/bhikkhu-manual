\chapter{Other Procedures}

\section{Nissaya (Dependence)}

The bhikkhu:

‘Ācariyo me bhante hohi, āyasmato nissāya vacchāmi.’ (×3)\\
‘\emph{Ven. sir, may you be my teacher. I shall stay dependent on the ven. one.}’

The Ācariya:

‘Sādhu; lahu; opāyikaṃ; paṭirūpaṃ; pāsādikena sampādehi!’

‘\emph{It is good; …convenient; …suitable; …proper; … you should endeavour to
  conduct yourself in a good manner.}’

\suttaRef{Vin.I.60–61}

The bhikkhu:

‘Sādhu bhante. Ajja-t-agge-dāni thero mayhaṃ bhāro, Aham-pi therassa bhāro.’ (×3)

‘\emph{It is good, ven. sir. From this day onwards the Thera will be my burden and I
shall be the burden of the Thera.}’

\suttaRef{Sp.V.977}

At the end, bow three times and sit papiap. The Ajahn may offer advice and
encouragement in the practice.

\section{Kappiya-karaṇa (The making allowable)}

For fruit or vegetables that can grow again, bhikkhu:

‘Kappiyaṃ karohi’ ‘\emph{Make it allowable.}’

The lay-person, while ‘marking’ (cutting or tearing) the fruit, etc., responds:

‘Kappiyaṃ bhante.’ ‘\emph{It is allowable, ven. sir.}’ \suttaRef{Sp.IV.767–768}

\section{Entering Town after Midday}

Leave can be taken in one's own language, or in Pali:

‘Vikāle gāmappavesanaṃ āpucchāmi.’\\
‘\emph{I take leave to go to the town at the ‘wrong time’.}’ \suttaRef{cf. Kv.140}

\section{Saṅghadāna-apalokana (Sharing Saṅghadāna)}

After \emph{saṅghadāna} is offered, a bhikkhu, other than the \emph{Thera},
kneels and recites:

‘Yagghe bhante saṅgho jānātu.\\
Ayaṃ paṭhama bhāgo \emph{therassa} pāpuṇāti.\\
Avasesā bhāgā amhākañc’eva pāpuṇanti.\\
Bhikkhūnañca sāmaṇerānaṃ gahaṭṭhānaṃ\\
Te yathāsukhaṃ paribhuñjantu.’ (×3)

‘therassa’ → ‘mahātherassa’

\margintodo{review the English text}

‘\emph{May the Saṅgha hear me. The first portion (of this offering) goes to the
  Elders. The remainder is for the rest of us here.}’

The Sangha responds: ‘Sādhu.’ \suttaRef{cf. Sp.VII.1405–1409}

\section{Paṃsukūla-cīvara (Taking Forest-cloth)}

‘Imaṃ paṃsukūla-cīvaraṃ assāmikaṃ mayhaṃ pāpuṇāti.’\\
‘\emph{This rag-robe, which is ownerless, has reached me.}’

\section{Desanā}

\subsection{Requesting permission}

\textbf{(a)} To speak on Vinaya

\instr{(After bowing three times, with hands joined in añjali, addressing the senior bhikkhu)}

‘Okāsaṃ me bhante thero detu vinaya-kathaṃ kathetuṃ.’

‘Namo… (×3); Buddhaṃ Dhammaṃ Saṅghaṃ namassāmi.’

‘Vinayo sāsanassa āyūti karotu me āyasmā okāsaṃ ahan-taṃ vattukāmo.’

‘\emph{Ven. sir, please give permission to speak on Vinaya… Vinaya is the life
  of the religion. I ask for permission from the ven. one: I wish to speak about
  the Vinaya.}’

Reply: ‘Karomi āyasmato okāsaṃ.’\\
‘\emph{I give you the opportunity, ven. sir.}’ \suttaRef{cf. Vin.I.113}

\clearpage

\subsection{To speak on Dhamma}

\instr{(After bowing three times, with hands joined in añjali, addressing the senior bhikkhu)}

‘Okāsaṃ me bhante thero detu dhamma-kathaṃ kathetuṃ.’

‘Namo… (×3); Buddhaṃ Dhammaṃ Saṅghaṃ namassāmi.’

‘Apārutā tesaṃ amatassa dvārā ye sotavantā pamuñcantu saddhaṃ.’

‘\emph{Ven. sir, please give permission to speak on Dhamma… Open are the doors
  to the Deathless. May all those who have ears release their faith.}’

\section{Añjali}

Chanting and making formal requests is done with the hands in añjali.
This is a gesture of respect, made by placing the palms together
directly in front of the chest, with the fingers aligned and pointing
upwards.

\section{Requesting a Dhamma Talk}

\begin{instruction}
  After bowing three times, with hands joined in añjali,\\
  recite the following:
\end{instruction}

Brahmā ca lokādhipatī sahampati\\
Katañjalī anadhivaraṃ ayācatha

Santīdha sattāpparajakkha-jātikā\\
Desetu dhammaṃ anukampimaṃ pajaṃ

\begin{instruction}
  Bow three times again
\end{instruction}

\begin{english}
The Brahmā god Sahampati, Lord of the world,\\
With palms joined in reverence, requested a favour:

`Beings are here with but little dust in their eyes,\\
Pray, teach the Dhamma out of compassion for them.' \suttaRef{BV. v1}
\end{english}

\section{Acknowledging the Teaching}

One person:

Handa mayaṃ dhammakathāya sādhukāraṃ dadāmase\\
\emph{Now let us express our approval of this Dhamma Teaching.}

Response:

Sādhu, sādhu, sādhu, anumodāmi\\
\emph{It is well, I appreciate it.}

\subsection{After the talk on Vinaya or Dhamma}

‘Ayaṃ dhammā- / vinayā- / dhammavinayākathā sādh'āyasmantehi saṃrakkhetabbāti.’

‘\emph{This talk on Dhamma / Vinaya / Dhammavinaya should be well-preserved by you, ven. sirs.}’

The senior bhikkhu:

‘Handa mayaṃ ovādā dhammā/ vinayā- / dhammavinayā- kathāya sādhukāraṃ dadāmase.’

‘\emph{Now let us make the act of acknowledging this Dhamma / Vinaya / Dhammavinaya talk.}’

The listeners:

‘Sādhu. Sādhu. Sādhu. Anumodāmi.’

\subsection{Acknowledging the Teaching}

‘Handa mayaṃ \emph{dhamma-kathāya}/ovādakathāya sādhu-kāraṃ dadāmase.’

‘\emph{Now let us express our approval of this Dhamma Teaching.}’

If an exhortation:

‘dhamma-kathāya’ → ‘ovāda-kathāya’

Response:

‘Sādhu, Sādhu, Sādhu. Anumodāmi.’\\
‘\emph{It is well, I appreciate it.}’

\clearpage

\section{Requesting Paritta Chanting}

\begin{instruction}
  After bowing three times, with hands joined in añjali,\\
  recite the following
\end{instruction}

Vipatti-paṭibāhāya sabba-sampatti-siddhiyā\\
Sabbadukkha-vināsāya\\
Parittaṃ brūtha maṅgalaṃ

Vipatti-paṭibāhāya sabba-sampatti-siddhiyā\\
Sabbabhaya-vināsāya\\
Parittaṃ brūtha maṅgalaṃ

Vipatti-paṭibāhāya sabba-sampatti-siddhiyā\\
Sabbaroga-vināsāya\\
Parittaṃ brūtha maṅgalaṃ

\begin{instruction}
  Bow three times
\end{instruction}

\begin{english}
For warding off misfortune, for the arising of good fortune,\\
For the dispelling of all dukkha,\\
May you chant a blessing and protection.

For warding off misfortune, for the arising of good fortune,\\
For the dispelling of all fear,\\
May you chant a blessing and protection.

For warding off misfortune, for the arising of good fortune,\\
For the dispelling of all sickness,\\
May you chant a blessing and protection.
\end{english}

\clearpage

\section[Three Refuges \& the Five Precepts]{Requesting the Three Refuges\newline \& the Five Precepts (Thai Tradition)}

\label{three-refuges}

\instr{After bowing three times, with hands joined in añjali,\\
  recite the appropriate request.}

\enlargethispage{\baselineskip}

\prul{For a group from a monk}

\begin{twochants}
Mayaṃ bhante tisaraṇena saha & pañca sīlāni yācāma\\
Dutiyampi mayaṃ bhante tisaraṇena saha & pañca sīlāni yācāma\\
Tatiyampi mayaṃ bhante tisaraṇena saha & pañca sīlāni yācāma\\
\end{twochants}

\prul{For oneself from a monk}

\begin{twochants}
Ahaṃ bhante tisaraṇena saha & pañca sīlāni yācāmi\\
Dutiyampi ahaṃ bhante tisaraṇena saha & pañca sīlāni yācāmi\\
Tatiyampi ahaṃ bhante tisaraṇena saha & pañca sīlāni yācāmi
\end{twochants}

\prul{For a group from a nun}

\begin{twochants}
Mayaṃ ayye tisaraṇena saha & pañca sīlāni yācāma\\
Dutiyampi mayaṃ ayye tisaraṇena saha & pañca sīlāni yācāma\\
Tatiyampi mayaṃ ayye tisaraṇena saha & pañca sīlāni yācāma\\
\end{twochants}

\prul{For oneself from a nun}

\begin{twochants}
Ahaṃ ayye tisaraṇena saha & pañca sīlāni yācāmi\\
Dutiyampi ahaṃ ayye tisaraṇena saha & pañca sīlāni yācāmi\\
Tatiyampi ahaṃ ayye tisaraṇena saha & pañca sīlāni yācāmi\\
\end{twochants}

\prul{For a group from a layperson}

\begin{twochants}
Mayaṃ mitta tisaraṇena saha & pañca sīlāni yācāma\\
Dutiyampi mayaṃ mitta tisaraṇena saha & pañca sīlāni yācāma\\
Tatiyampi mayaṃ mitta tisaraṇena saha & pañca sīlāni yācāma\\
\end{twochants}

\prul{For oneself from a layperson}

\begin{twochants}
Ahaṃ mitta tisaraṇena saha & pañca sīlāni yācāmi\\
Dutiyampi ahaṃ mitta tisaraṇena saha & pañca sīlāni yācāmi\\
Tatiyampi ahaṃ mitta tisaraṇena saha & pañca sīlāni yācāmi\\
\end{twochants}

\begin{english}
  We/I, Venerable Sir/Sister/Friend,\\
  request the Three Refuges and the Five Precepts.

  For the second time,\\
  we/I, Venerable Sir/Sister/Friend,\\
  request the Three Refuges and the Five Precepts.

  For the third time,\\
  we/I, Venerable Sir/Sister/Friend,\\
  request the Three Refuges and the Five Precepts.
\end{english}

\instr{Repeat, after the leader has chanted ‘Namo tassa’ three times.}

Namo tassa bhagavato arahato sammāsambuddhassa (×3)

\begin{english}
  Homage to the Blessed, Noble, and Perfectly Enlightened One.
\end{english}

Buddhaṃ saraṇaṃ gacchāmi\\
Dhammaṃ saraṇaṃ gacchāmi\\
Saṅghaṃ saraṇaṃ gacchāmi

\begin{english}
  To the Buddha I go for refuge.\\
  To the Dhamma I go for refuge.\\
  To the Saṅgha I go for refuge.
\end{english}

\clearpage

Dutiyampi buddhaṃ saraṇaṃ gacchāmi\\
Dutiyampi dhammaṃ saraṇaṃ gacchāmi\\
Dutiyampi saṅghaṃ saraṇaṃ gacchāmi

\begin{english}
  For the second time\ldots
\end{english}

Tatiyampi buddhaṃ saraṇaṃ gacchāmi\\
Tatiyampi dhammaṃ saraṇaṃ gacchāmi\\
Tatiyampi saṅghaṃ saraṇaṃ gacchāmi

\begin{english}
  For the third time\ldots
\end{english}

\instr{Leader:}

[Tisaraṇa-gamanaṃ niṭṭhitaṃ]\\
\emph{This completes the going to the Three Refuges.}

\instr{Response:}

Āma bhante / Āma ayye / Āma mitta\\
\emph{Yes, Venerable Sir / Sister / Friend.}

\instr{Repeat each precept after the leader}

\enlargethispage{-\baselineskip}

{\raggedright

\begin{packedenumerate}
  \item Pāṇātipātā veramaṇī sikkhāpadaṃ samādiyāmi\\
    \emph{I undertake the precept to refrain from taking the life of any living~creature.}
  \item Adinnādānā veramaṇī sikkhāpadaṃ samādiyāmi\\
    \emph{I undertake the precept to refrain from taking that which is not given.}
  \item Kāmesu micchācārā veramaṇī sikkhāpadaṃ samādiyāmi\\
    \emph{I undertake the precept to refrain from sexual misconduct.}
  \item Musāvādā veramaṇī sikkhāpadaṃ samādiyāmi\\
    \emph{I undertake the precept to refrain from lying.}
  \item Surāmeraya-majja-pamādaṭṭhānā veramaṇī sikkhāpadaṃ samādiyāmi\\
    \emph{I undertake the precept to refrain from consuming intoxicating drink and drugs which lead to carelessness.}
\end{packedenumerate}

}

\instr{Leader:}

[Imāni pañca sikkhāpadāni\\
Sīlena sugatiṃ yanti\\
Sīlena bhogasampadā\\
Sīlena nibbutiṃ yanti\\
Tasmā sīlaṃ visodhaye]

\begin{english}
  These are the Five Precepts;\\
  virtue is the source of happiness,\\
  virtue is the source of true wealth,\\
  virtue is the source of peacefulness ---\\
  Therefore let virtue be purified.
\end{english}

\instr{Response:}

Sādhu, sādhu, sādhu

\instr{Bow three times}

\section[Three Refuges \& the Eight Precepts]{Requesting the Three Refuges\newline \& the Eight Precepts (Thai Tradition)}

\instr{After bowing three times, with hands joined in añjali,\\
  recite the appropriate request.}

\enlargethispage{\baselineskip}

\prul{For a group from a monk}

\begin{twochants}
Mayaṃ bhante tisaraṇena saha & aṭṭha sīlāni yācāma\\
Dutiyampi mayaṃ bhante tisaraṇena saha & aṭṭha sīlāni yācāma\\
Tatiyampi mayaṃ bhante tisaraṇena saha & aṭṭha sīlāni yācāma\\
\end{twochants}

\prul{For oneself from a monk}

\begin{twochants}
Ahaṃ bhante tisaraṇena saha & aṭṭha sīlāni yācāmi\\
Dutiyampi ahaṃ bhante tisaraṇena saha & aṭṭha sīlāni yācāmi\\
Tatiyampi ahaṃ bhante tisaraṇena saha & aṭṭha sīlāni yācāmi
\end{twochants}

\prul{For a group from a nun}

\begin{twochants}
Mayaṃ ayye tisaraṇena saha & aṭṭha sīlāni yācāma\\
Dutiyampi mayaṃ ayye tisaraṇena saha & aṭṭha sīlāni yācāma\\
Tatiyampi mayaṃ ayye tisaraṇena saha & aṭṭha sīlāni yācāma\\
\end{twochants}

\prul{For oneself from a nun}

\begin{twochants}
Ahaṃ ayye tisaraṇena saha & aṭṭha sīlāni yācāmi\\
Dutiyampi ahaṃ ayye tisaraṇena saha & aṭṭha sīlāni yācāmi\\
Tatiyampi ahaṃ ayye tisaraṇena saha & aṭṭha sīlāni yācāmi\\
\end{twochants}

\prul{For a group from a layperson}

\begin{twochants}
Mayaṃ mitta tisaraṇena saha & aṭṭha sīlāni yācāma\\
Dutiyampi mayaṃ mitta tisaraṇena saha & aṭṭha sīlāni yācāma\\
Tatiyampi mayaṃ mitta tisaraṇena saha & aṭṭha sīlāni yācāma\\
\end{twochants}

\prul{For oneself from a layperson}

\begin{twochants}
Ahaṃ mitta tisaraṇena saha & aṭṭha sīlāni yācāmi\\
Dutiyampi ahaṃ mitta tisaraṇena saha & aṭṭha sīlāni yācāmi\\
Tatiyampi ahaṃ mitta tisaraṇena saha & aṭṭha sīlāni yācāmi\\
\end{twochants}

\begin{english}
  We/I, Venerable Sir/Sister/Friend,\\
  request the Three Refuges and the Eight Precepts.

  For the second time,\\
  We/I, Venerable Sir/Sister/Friend,\\
  request the Three Refuges and the Eight Precepts.

  For the third time,\\
  We/I, Venerable Sir/Sister/Friend,\\
  request the Three Refuges and the Eight Precepts.
\end{english}

\instr{Repeat, after the leader has chanted ‘Namo tassa’ three times.}

Namo tassa bhagavato arahato sammāsambuddhassa (×3)
\begin{english}
  Homage to the Blessed, Noble, and Perfectly Enlightened One.
\end{english}

Buddhaṃ saraṇaṃ gacchāmi\\
Dhammaṃ saraṇaṃ gacchāmi\\
Saṅghaṃ saraṇaṃ gacchāmi

\begin{english}
  To the Buddha I go for refuge.\\
  To the Dhamma I go for refuge.\\
  To the Saṅgha I go for refuge.
\end{english}

Dutiyampi buddhaṃ saraṇaṃ gacchāmi\\
Dutiyampi dhammaṃ saraṇaṃ gacchāmi\\
Dutiyampi saṅghaṃ saraṇaṃ gacchāmi

\begin{english}
  For the second time\ldots
\end{english}

Tatiyampi buddhaṃ saraṇaṃ gacchāmi\\
Tatiyampi dhammaṃ saraṇaṃ gacchāmi\\
Tatiyampi saṅghaṃ saraṇaṃ gacchāmi

\begin{english}
  For the third time\ldots
\end{english}

\instr{Leader:}

[Tisaraṇa-gamanaṃ niṭṭhitaṃ]\\
\emph{This completes the going to the Three Refuges.}

\instr{Response:}

Āma bhante / Āma ayye / Āma mitta\\
\emph{Yes, Venerable Sir / Sister / Friend.}

\enlargethispage{-\baselineskip}

\instr{Repeat each precept after the leader}

{\raggedright

\begin{packedenumerate}
  \item Pāṇātipātā veramaṇī sikkhāpadaṃ samādiyāmi\\
  \emph{I undertake the precept to refrain from taking the life of any living~creature.}
  \item Adinnādānā veramaṇī sikkhāpadaṃ samādiyāmi\\
  \emph{I undertake the precept to refrain from taking that which is not given.}
  \item Abrahmacariyā veramaṇī sikkhāpadaṃ samādiyāmi\\
  \emph{I undertake the precept to refrain from any intentional sexual activity.}
  \item Musāvādā veramaṇī sikkhāpadaṃ samādiyāmi\\
  \emph{I undertake the precept to refrain from lying.}
  \item Surāmeraya-majja-pamādaṭṭhānā veramaṇī sikkhāpadaṃ samādiyāmi\\
  \emph{I undertake the precept to refrain from consuming intoxicating drink and drugs which lead to carelessness.}
  \item Vikālabhojanā veramaṇī sikkhāpadaṃ samādiyāmi.\\
  \emph{I undertake the precept to refrain from eating at inappropriate times.}
  \item Nacca-gīta-vādita-visūkadassanā mālā-gandha-vilepana-dhāraṇa-maṇḍana-vibhūsanaṭṭhānā veramaṇī sikkhāpadaṃ samādiyāmi.\\
  \emph{I undertake the precept to refrain from entertainment, beautification, and~adornment.}
  \item Uccāsayana-mahāsayanā veramaṇī sikkhāpadaṃ samādiyāmi.\\
  \emph{I undertake the precept to refrain from lying on a high or luxurious sleeping place.}
\end{packedenumerate}

}

\suttaRef{cf. A.IV.248–250}

\instr{Leader:}

[Imāni aṭṭha sikkhāpadāni samādiyāmi]

\instr{Response:}

Imāni aṭṭha sikkhāpadāni samādiyāmi (×3)

\begin{english}
  I undertake these Eight Precepts.
\end{english}

\instr{Leader:}

[Imāni aṭṭha sikkhāpadāni\\
Sīlena sugatiṃ yanti\\
Sīlena bhogasampadā\\
Sīlena nibbutiṃ yanti\\
Tasmā sīlaṃ visodhaye]

\begin{english}
  These are the Eight Precepts;\\
  virtue is the source of happiness,\\
  virtue is the source of true wealth,\\
  virtue is the source of peacefulness ---\\
  Therefore let virtue be purified.
\end{english}

\instr{Response:}

Sādhu, sādhu, sādhu.

\instr{Bow three times}

(NOTE: addition from the old uposatha section)

Alternatively, the laypeople may chant:

‘Imaṃ aṭṭh'aṅga-samannāgataṃ\\
buddhapaññattaṃ uposathaṃ, imañ-ca rattiṃ\\
imañca divasaṃ, samma-deva abhirakkhituṃ samādiyāmi.’

\emph{Bhk}: ‘Imāni aṭṭha sikkhā-padāni,\\
ajj'ekaṃ rattin-divaṃ, uposatha (sīla)\\
vasena sādhukaṃ (katvā appamādena) rakkhitabbāni.’

\emph{Laypeople}: ‘Āma bhante.’

\emph{Bhk}:
‘Sīlena sugatiṃ yanti,\\
Sīlena bhoga-sampadā,\\
Sīlena nibbutiṃ yanti,\\
Tasmā sīlaṃ visodhaye.’

\subsection{Asking Forgiveness To The Triple Gem}

\instr{(Men Chant)}\\\relax
Ahaṃ buddhañ ca dhammañ ca saṅghañ ca saraṇaṃ gato\\
upāsakattaṃ desesiṃ bhikkhu-saṅghassa sammukhā.

\instr{(Women Chant)}\\\relax
Ahaṃ buddhañ ca dhammañ ca saṅghañ ca saraṇaṃ gatā\\
upāsikattaṃ desesiṃ bhikkhu-saṅghassa sammukhā.

Etaṃ me saraṇaṃ khemaṃ,\\
etaṃ saraṇam uttamaṃ\\
etaṃ saraṇam āgamma sabba-dukkhā pamuccaye.\\
Yathā-balaṃ careyyāhaṃ sammā-sambuddha-sāsanaṃ

\sidepar{m.}%
dukkha-nissaraṇass' eva bhāgī assam anāgate.

\sidepar{w.}%
dukkha-nissaraṇass' eva bhāginissaṃ anāgate.

Kāyena vācāya va cetasā vā\\
buddhe kukammaṃ pakataṃ mayā yaṃ\\
buddho paṭigghaṅhātu accayantaṃ\\
kālantare saṃvarituṃ va buddhe

Kāyena vācāya va cetasā vā\\
dhamme kukammaṃ pakataṃ mayā yaṃ\\
dhammo paṭigghaṅhātu accayantaṃ\\
kālantare saṃvarituṃ va dhamme

Kāyena vācāya va cetasā vā\\
saṅghe kukammaṃ pakataṃ mayā yaṃ\\
saṅgho paṭigghaṅhātu accayantaṃ\\
kālantare saṃvarituṃ va sanghe

\subsection{Taking Leave}

Having undertaken the Eight Precepts, layfollowers may stay overnight. The next
morning they will take their leave from the senior monk:

\instr{Laypeople:}

Handadāni mayaṃ bhante\\
āpucchāma bahukiccā bahukaraṇīyā

\instr{Senior monk:}

‘Yassa dāni tumhe kālaṃ maññatha.’\\
‘\emph{Please do what is appropriate at this time.}’

\section{Eight Precepts (Sri Lankan Tradition)}

With hands in \emph{añjali}, the laypeople recite the following request:

‘Sādhu! Sādhu! Sādhu! Okāsa ahaṃ bhante ti-saraṇena saddhiṃ aṭṭh'aṅga sīlaṃ
dhammaṃ yācāmi, anuggahaṃ katvā sīlaṃ detha me bhante. Dutiyam-pi okāsa… detha
me bhante. Tatiyam-pi okāsa… detha me bhante.’

\emph{Bhk}: ‘Yaṃ ahaṃ vadāmi taṃ vadetha.’

\emph{Laypeople}: ‘Āma, bhante.’

\emph{Bhk}: ‘Namo…’ (×3)

\emph{Laypeople}: repeat.

\emph{Bhk}:\\
‘Buddhaṃ saraṇaṃ gacchāmi.\\
Dhammaṃ saraṇaṃ gacchāmi.\\
Saṅghaṃ saraṇaṃ gacchāmi.\\
Dutiyam-pi Buddhaṃ saraṇaṃ gacchāmi.\\
Dutiyam-pi Dhammaṃ saraṇaṃ gacchāmi.\\
Dutiyam-pi Saṅghaṃ saraṇaṃ gacchāmi.\\
Tatiyam-pi Buddhaṃ saraṇaṃ gacchāmi.\\
Tatiyam-pi Dhammaṃ saraṇaṃ gacchāmi.\\
Tatiyam-pi Saṅghaṃ saraṇaṃ gacchāmi.’

\emph{Laypeople}: repeat line by line.

\emph{Bhk}: ‘Saraṇagamanaṃ sampuṇṇaṃ.’

\emph{Laypeople}: ‘Āma, bhante.’

Then the laypeople repeat after the bhikku line by line:

‘Pāṇātipātā veramaṇī sikkhā-padaṃ samādiyāmi.\\
Adinnādānā veramaṇī sikkhā-padaṃ samādiyāmi.\\
Abrahma-cariyā veramaṇī sikkhā-padaṃ samādiyāmi.\\
Musāvādā veramaṇī sikkhā-padaṃ samādiyāmi.\\
Surā-meraya-majja-pamādaṭṭhānā veramaṇī sikkhā-padaṃ samādiyāmi.\\
Vikāla-bhojanā veramaṇī sikkhā-padaṃ samādiyāmi.\\
Nacca-gīta vādita visūka-dassana mālāgandha vilepana dhāraṇa maṇḍana
vibhūsanaṭṭhānā veramaṇī sikkhā-padaṃ samādiyāmi.\\
Uccā-sayana mahā-sayanā veramaṇī sikkhā-padaṃ samādiyāmi.’

\suttaRef{cf. A.IV.248–250}

\emph{Bhk}: ‘Imaṃ aṭṭh'aṅga-sīlaṃ samādiyāmi.’

\emph{Laypeople}: ‘Imaṃ aṭṭh'aṅga-sīlaṃ samādiyāmi.’ (×3)

\emph{Bhk}: ‘Ti-saraṇena saddhiṃ aṭṭh'aṅga-sīlaṃ dhammaṃ sādhukaṃ surakkhitaṃ
katvā appamādena sampādetha.’

\emph{Laypeople}: ‘Āma, bhante.’

\emph{Bhk}:\\
‘Sīlena sugatiṃ yanti,\\
Sīlena bhoga-sampadā,\\
Sīlena nibbutiṃ yanti,\\
Tasmā sīlaṃ visodhaye.’

\section{Five Precepts (Sri Lankan Tradition)}

With hands in \emph{añjali}, the laypeople recite the following request:

‘Sādhu! Sādhu! Sādhu! Okāsa ahaṃ bhante tisaraṇena saddhiṃ pañca-sīlaṃ dhammaṃ
yācāmi, anuggahaṃ katvā sīlaṃ detha me bhante. Dutiyam-pi okāsa… Tatiyam-pi
okāsa…’

\emph{Bhikkhu}: ‘Yaṃ ahaṃ vadāmi taṃ vadetha.’

\emph{Laypeople}: ‘Āma, bhante.’

\emph{Bhk}: ‘Namo…’ (×3)

\emph{Laypeople}: repeat.

\emph{Bhk}: ‘Saraṇagamanaṃ sampuṇṇaṃ.’

\emph{Laypeople}: ‘Āma, bhante.’

Then the bhikkhu recites, with the laypeople repeating line by line:

‘Pāṇātipātā veramaṇī sikkhā-padaṃ samādiyāmi.\\
Adinnādānā veramaṇī sikkhā-padaṃ samādiyāmi.\\
Kāmesu micchā-cārā veramaṇī sikkhā-padaṃ samādiyāmi.\\
Musā-vādā veramaṇī sikkhā-padaṃ samādiyāmi.\\
Surā-meraya-majja-pamādaṭṭhānā veramaṇī sikkhā-padaṃ samādiyāmi.’

\suttaRef{cf. A.IV.248–250}

\emph{Bhk}:

‘Tisaraṇena saddhiṃ pañcasīlaṃ dhammaṃ sādhukaṃ surakkhitaṃ katvā appamādena
sampādetha.’

\emph{Laypeople}: ‘Āma, bhante.’

\emph{Bhk}:

‘Sīlena sugatiṃ yanti\\
Sīlena bhoga-sampadā,\\
Sīlena nibbutiṃ yanti,\\
Tasmā sīlaṃ visodhaye.’

\section{Disrobing}

After the bhikkhus who are to witness the disrobing have assembled, the bhikkhu
who will disrobe should first confess his offences. Then, wearing all his three
robes, with his \emph{saṅghāti} on his left shoulder:

Bow three times.

‘Namo tassa bhagavato arahato sammā-sambuddhassa’ (×3)

Optionally, one may chant \emph{Recollection After Using the Requisites}
(p.\pageref{recollection-after-using}).

Bow three times.

Chant in Pali and in his own language:

‘Sikkhaṃ paccakkhāmi. Gihī'ti maṃ dhāretha.’\\
I give up the training. May you regard me as a layman.

He may state this once, three times, or as many times as he needs to assure
himself that he is now a layman and no longer a bhikkhu. If two or more are
disrobing, they should state this passage separately.

The former bhikkhu then withdraws to change into lay clothes. When he returns,
he may request the \emph{Three Refuges and Five Precepts}.

