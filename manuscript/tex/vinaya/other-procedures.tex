\chapter{Other Procedures}

\section{Nissaya (Dependence)}

The bhikkhu:

‘Ācariyo me bhante hohi, āyasmato nissāya vacchāmi.’ (×3)\\
‘\emph{Ven. sir, may you be my teacher. I shall stay dependent on the ven. one.}’

The Ācariya:

‘Sādhu; lahu; opāyikaṃ; paṭirūpaṃ; pāsādikena sampādehi!’

‘\emph{It is good; …convenient; …suitable; …proper; … you should endeavour to
  conduct yourself in a good manner.}’

\suttaRef{Vin.I.60–61}

The bhikkhu:

‘Sādhu bhante. Ajja-t-agge-dāni thero mayhaṃ bhāro, Aham-pi therassa bhāro.’ (×3)

‘\emph{It is good, ven. sir. From this day onwards the Thera will be my burden and I
shall be the burden of the Thera.}’

\suttaRef{Sp.V.977}

\section{Kappiya-karaṇa (The making allowable)}

For fruit or vegetables that can grow again, bhikkhu:

‘Kappiyaṃ karohi’\\
‘\emph{Make it allowable.}’

The lay-person, while ‘marking’ (cutting or tearing) the fruit, etc., responds:

‘Kappiyaṃ bhante.’\\
‘\emph{It is allowable, ven. sir.}’

\suttaRef{Sp.IV.767–768}

\section{Entering Town after Midday}

Leave can be taken in one's own language, or in Pali:

‘Vikāle gāmappavesanaṃ āpucchāmi.’\\
‘\emph{I take leave to go to the town at the ‘wrong time’.}’

\suttaRef{cf. Kv.140}

\section{Saṅghadāna-apalokana}

(Sharing Saṅghadāna)

After \emph{saṅghadāna} is offered, a bhikkhu, other than the \emph{Thera},
kneels and recites:

‘Yagghe bhante saṅgho jānātu. Ayaṃ paṭhamabhāgo \emph{therassa} pāpuṇāti,
Avasesā bhāgā avasesānaṃ bhikkhusāmaṇerānaṃ pāpuṇantu, Yathāsukhaṃ
paribhuñjantu.’ (×3)

‘therassa’ → ‘mahātherassa’

Or:

‘Yagghe bhante… Avasesā bhāgā amhākaṃ pāpuṇanti.’ (×3)

‘\emph{May the Saṅgha hear me. The first portion (of this offering) goes to the
Elders. The remainder is for the rest of us here.}’

The Sangha responds: ‘Sādhu.’

\suttaRef{Thai; cf. Sp.VII.1405–1409}

\section{Paṃsukūla-cīvara (Taking Forest-cloth)}

‘Imaṃ paṃsukūla-cīvaraṃ assāmikaṃ mayhaṃ pāpuṇāti.’\\
‘\emph{This rag-robe, which is ownerless, has reached me.}’

\suttaRef{Thai}

\section{Desanā}

\subsection{Requesting permission}

\textbf{(a)} To speak on Vinaya

( ⇓×3)

Addressing the senior bhikkhu:

‘Okāsaṃ me bhante thero detu vinaya-kathaṃ kathetuṃ.’

‘Namo… (×3); Buddhaṃ Dhammaṃ Saṅghaṃ namassāmi.’

‘Vinayo sāsanassa āyūti karotu me āyasmā okāsaṃ ahan-taṃ vattukāmo.’

‘\emph{Ven. sir, please give permission to speak on Vinaya… Vinaya is the life
  of the religion. I ask for permission from the ven. one: I wish to speak about
  the Vinaya.}’

Reply: ‘Karomi āyasmato okāsaṃ.’

‘\emph{I give you the opportunity, ven. sir.}’

\suttaRef{Thai; cf. Vin.I.113}

\subsection{To speak on Dhamma}

( ⇓×3)

Addressing the senior bhikkhu:

‘Okāsaṃ me bhante thero detu dhamma-kathaṃ kathetuṃ.’

‘Namo… (×3); Buddhaṃ Dhammaṃ Saṅghaṃ namassāmi.’

‘Apārutā tesaṃ amatassa dvārā ye sotavantā pamuñcantu saddhaṃ.’

‘\emph{Ven. sir, please give permission to speak on Dhamma… Open are the doors
  to the Deathless. May all those who have ears release their faith.}’

\suttaRef{Thai}

\section{Añjali}

Chanting and making formal requests is done with the hands in añjali.
This is a gesture of respect, made by placing the palms together
directly in front of the chest, with the fingers aligned and pointing
upwards.

\section{Requesting a Dhamma Talk}

\begin{instruction}
  After bowing three times, with hands joined in añjali,\\
  recite the following:
\end{instruction}

Brahmā ca lokādhipatī sahampati\\
Katañjalī anadhivaraṃ ayācatha

Santīdha sattāpparajakkha-jātikā\\
Desetu dhammaṃ anukampimaṃ pajaṃ

\begin{instruction}
  Bow three times again
\end{instruction}

\begin{english}
The Brahmā god Sahampati, Lord of the world,\\
With palms joined in reverence, requested a favour:

`Beings are here with but little dust in their eyes,\\
Pray, teach the Dhamma out of compassion for them.'
\end{english}

\suttaRef{BV. v1}

\section{Acknowledging the Teaching}

\begin{tabular}{@{} ll @{}}
One person: & Handa mayaṃ dhammakathāya sādhukāraṃ dadāmase \\
& \hspace*{1em}\tr{Now let us express our approval of this Dhamma Teaching.} \\
Response: & Sādhu, sādhu, sādhu, anumodāmi \\
& \hspace*{1em}\tr{It is well, I appreciate it.} \\
\end{tabular}

\subsection{After the talk on Vinaya or Dhamma}

‘Ayaṃ dhammā- / vinayā- / dhammavinayākathā sādh'āyasmantehi saṃrakkhetabbāti.’

‘\emph{This talk on Dhamma / Vinaya / Dhammavinaya should be well-preserved by you, ven. sirs.}’

The senior bhikkhu:

‘Handa mayaṃ ovādā dhammā/ vinayā- / dhammavinayā- kathāya sādhukāraṃ dadāmase.’

‘\emph{Now let us make the act of acknowledging this Dhamma / Vinaya / Dhammavinaya talk.}’

The listeners:

‘Sādhu. Sādhu. Sādhu. Anumodāmi.’

\suttaRef{Thai}

\subsection{Acknowledging the Teaching}

‘Handa mayaṃ \emph{dhamma-kathāya}/ovādakathāya sādhu-kāraṃ dadāmase.’

‘\emph{Now let us express our approval of this Dhamma Teaching.}’

If an exhortation:

‘dhamma-kathāya’ → ‘ovāda-kathāya’

Response:

‘Sādhu, Sādhu, Sādhu. Anumodāmi.’\\
‘\emph{It is well, I appreciate it.}’

\section{Requesting Paritta Chanting}

\begin{instruction}
  After bowing three times, with hands joined in añjali,\\
  recite the following
\end{instruction}

Vipatti-paṭibāhāya sabba-sampatti-siddhiyā\\
Sabbadukkha-vināsāya\\
Parittaṃ brūtha maṅgalaṃ

Vipatti-paṭibāhāya sabba-sampatti-siddhiyā\\
Sabbabhaya-vināsāya\\
Parittaṃ brūtha maṅgalaṃ

Vipatti-paṭibāhāya sabba-sampatti-siddhiyā\\
Sabbaroga-vināsāya\\
Parittaṃ brūtha maṅgalaṃ

\begin{instruction}
  Bow three times
\end{instruction}

\begin{english}
For warding off misfortune, for the arising of good fortune,\\
For the dispelling of all dukkha,\\
May you chant a blessing and protection.

For warding off misfortune, for the arising of good fortune,\\
For the dispelling of all fear,\\
May you chant a blessing and protection.

For warding off misfortune, for the arising of good fortune,\\
For the dispelling of all sickness,\\
May you chant a blessing and protection.
\end{english}

\section{Requesting the Three Refuges\newline \& the Five Precepts}

\label{three-refuges}

\subsection{Thai Tradition}

\begin{instruction}
  After bowing three times, with hands joined in añjali,\\
  recite the appropriate request.
\end{instruction}

\subsection{For a group from a monk}

\begin{twochants}
Mayaṃ bhante tisaraṇena saha & pañca sīlāni yācāma\\
Dutiyampi mayaṃ bhante tisaraṇena saha & pañca sīlāni yācāma\\
Tatiyampi mayaṃ bhante tisaraṇena saha & pañca sīlāni yācāma\\
\end{twochants}

\subsection{For oneself from a monk}

\begin{twochants}
Ahaṃ bhante tisaraṇena saha & pañca sīlāni yācāmi\\
Dutiyampi ahaṃ bhante tisaraṇena saha & pañca sīlāni yācāmi\\
Tatiyampi ahaṃ bhante tisaraṇena saha & pañca sīlāni yācāmi
\end{twochants}

\subsection{For a group from a nun}

\begin{twochants}
Mayaṃ ayye tisaraṇena saha & pañca sīlāni yācāma\\
Dutiyampi mayaṃ ayye tisaraṇena saha & pañca sīlāni yācāma\\
Tatiyampi mayaṃ ayye tisaraṇena saha & pañca sīlāni yācāma\\
\end{twochants}

\subsection{For oneself from a nun}

\begin{twochants}
Ahaṃ ayye tisaraṇena saha & pañca sīlāni yācāmi\\
Dutiyampi ahaṃ ayye tisaraṇena saha & pañca sīlāni yācāmi\\
Tatiyampi ahaṃ ayye tisaraṇena saha & pañca sīlāni yācāmi\\
\end{twochants}

\subsection{For a group from a layperson}

\begin{twochants}
Mayaṃ mitta tisaraṇena saha & pañca sīlāni yācāma\\
Dutiyampi mayaṃ mitta tisaraṇena saha & pañca sīlāni yācāma\\
Tatiyampi mayaṃ mitta tisaraṇena saha & pañca sīlāni yācāma\\
\end{twochants}

\subsection{For oneself from a layperson}

\begin{twochants}
Ahaṃ mitta tisaraṇena saha & pañca sīlāni yācāmi\\
Dutiyampi ahaṃ mitta tisaraṇena saha & pañca sīlāni yācāmi\\
Tatiyampi ahaṃ mitta tisaraṇena saha & pañca sīlāni yācāmi\\
\end{twochants}

\subsection{Translation}

\begin{english}
  We/I, Venerable Sir/Sister/Friend,\\
  request the Three Refuges and the Five Precepts.

  For the second time,\\
  we/I, Venerable Sir/Sister/Friend,\\
  request the Three Refuges and the Five Precepts.

  For the third time,\\
  we/I, Venerable Sir/Sister/Friend,\\
  request the Three Refuges and the Five Precepts.
\end{english}

\subsection{Taking the Three Refuges}

\begin{instruction}
  Repeat, after the leader has chanted the first three lines
\end{instruction}

Namo tassa bhagavato arahato sammāsambuddhassa\\
Namo tassa bhagavato arahato sammāsambuddhassa\\
Namo tassa bhagavato arahato sammāsambuddhassa

\begin{english}
  Homage to the Blessed, Noble, and Perfectly Enlightened One.\\
  Homage to the Blessed, Noble, and Perfectly Enlightened One.\\
  Homage to the Blessed, Noble, and Perfectly Enlightened One.
\end{english}

Buddhaṃ saraṇaṃ gacchāmi\\
Dhammaṃ saraṇaṃ gacchāmi\\
Saṅghaṃ saraṇaṃ gacchāmi

\begin{english}
  To the Buddha I go for refuge.\\
  To the Dhamma I go for refuge.\\
  To the Saṅgha I go for refuge.
\end{english}

Dutiyampi buddhaṃ saraṇaṃ gacchāmi\\
Dutiyampi dhammaṃ saraṇaṃ gacchāmi\\
Dutiyampi saṅghaṃ saraṇaṃ gacchāmi

\begin{english}
  For the second time, to the Buddha I go for refuge.\\
  For the second time, to the Dhamma I go for refuge.\\
  For the second time, to the Saṅgha I go for refuge.
\end{english}

Tatiyampi buddhaṃ saraṇaṃ gacchāmi\\
Tatiyampi dhammaṃ saraṇaṃ gacchāmi\\
Tatiyampi saṅghaṃ saraṇaṃ gacchāmi

\begin{english}
  For the third time, to the Buddha I go for refuge.\\
  For the third time, to the Dhamma I go for refuge.\\
  For the third time, to the Saṅgha I go for refuge.
\end{english}

\begin{instruction}
  Leader:
\end{instruction}

[Tisaraṇa-gamanaṃ niṭṭhitaṃ]

\begin{english}
  This completes the going to the Three Refuges.
\end{english}

\begin{instruction}
  Response:
\end{instruction}

Āma bhante / Āma ayye / Āma mitta

\begin{english}
  Yes, Venerable Sir/Sister/Friend.
\end{english}

\subsection{The Five Precepts}

\begin{instruction}
  Repeat each precept after the leader
\end{instruction}

\begin{precept}
  \setcounter{enumi}{0}
  \item Pāṇātipātā veramaṇī sikkhāpadaṃ samādiyāmi
\end{precept}

\begin{english}
  I undertake the precept to refrain from taking the life of any living~creature.
\end{english}

\begin{precept}
  \setcounter{enumi}{1}
  \item Adinnādānā veramaṇī sikkhāpadaṃ samādiyāmi
\end{precept}

\begin{english}
  I undertake the precept to refrain from taking that which is not given.
\end{english}

\begin{precept}
  \setcounter{enumi}{2}
  \item Kāmesu micchācārā veramaṇī sikkhāpadaṃ samādiyāmi
\end{precept}

\begin{english}
  I undertake the precept to refrain from sexual misconduct.
\end{english}

\begin{precept}
  \setcounter{enumi}{3}
  \item Musāvādā veramaṇī sikkhāpadaṃ samādiyāmi
\end{precept}

\begin{english}
  I undertake the precept to refrain from lying.
\end{english}

\begin{precept}
  \setcounter{enumi}{4}
  \item Surāmeraya-majja-pamādaṭṭhānā veramaṇī sikkhāpadaṃ samādiyāmi
\end{precept}

\begin{english}
  I undertake the precept to refrain from consuming intoxicating drink and drugs which lead to carelessness.
\end{english}

\begin{instruction}
  Leader:
\end{instruction}

[Imāni pañca sikkhāpadāni\\
Sīlena sugatiṃ yanti\\
Sīlena bhogasampadā\\
Sīlena nibbutiṃ yanti\\
Tasmā sīlaṃ visodhaye]

\begin{english}
  These are the Five Precepts;\\
  virtue is the source of happiness,\\
  virtue is the source of true wealth,\\
  virtue is the source of peacefulness ---\\
  Therefore let virtue be purified.
\end{english}

\begin{instruction}
  Response:
\end{instruction}

Sādhu, sādhu, sādhu

\begin{instruction}
  Bow three times
\end{instruction}

\section{Requesting the Three Refuges\newline \& the Eight Precepts}

\begin{instruction}
  After bowing three times, with hands joined in añjali,\\
  recite the appropriate request.
\end{instruction}

\subsection{For a group from a monk}

\begin{twochants}
Mayaṃ bhante tisaraṇena saha & aṭṭha sīlāni yācāma\\
Dutiyampi mayaṃ bhante tisaraṇena saha & aṭṭha sīlāni yācāma\\
Tatiyampi mayaṃ bhante tisaraṇena saha & aṭṭha sīlāni yācāma\\
\end{twochants}

\subsection{For oneself from a monk}

\begin{twochants}
Ahaṃ bhante tisaraṇena saha & aṭṭha sīlāni yācāmi\\
Dutiyampi ahaṃ bhante tisaraṇena saha & aṭṭha sīlāni yācāmi\\
Tatiyampi ahaṃ bhante tisaraṇena saha & aṭṭha sīlāni yācāmi
\end{twochants}

\subsection{For a group from a nun}

\begin{twochants}
Mayaṃ ayye tisaraṇena saha & aṭṭha sīlāni yācāma\\
Dutiyampi mayaṃ ayye tisaraṇena saha & aṭṭha sīlāni yācāma\\
Tatiyampi mayaṃ ayye tisaraṇena saha & aṭṭha sīlāni yācāma\\
\end{twochants}

\subsection{For oneself from a nun}

\begin{twochants}
Ahaṃ ayye tisaraṇena saha & aṭṭha sīlāni yācāmi\\
Dutiyampi ahaṃ ayye tisaraṇena saha & aṭṭha sīlāni yācāmi\\
Tatiyampi ahaṃ ayye tisaraṇena saha & aṭṭha sīlāni yācāmi\\
\end{twochants}

\subsection{For a group from a layperson}

\begin{twochants}
Mayaṃ mitta tisaraṇena saha & aṭṭha sīlāni yācāma\\
Dutiyampi mayaṃ mitta tisaraṇena saha & aṭṭha sīlāni yācāma\\
Tatiyampi mayaṃ mitta tisaraṇena saha & aṭṭha sīlāni yācāma\\
\end{twochants}

\subsection{For oneself from a layperson}

\begin{twochants}
Ahaṃ mitta tisaraṇena saha & aṭṭha sīlāni yācāmi\\
Dutiyampi ahaṃ mitta tisaraṇena saha & aṭṭha sīlāni yācāmi\\
Tatiyampi ahaṃ mitta tisaraṇena saha & aṭṭha sīlāni yācāmi\\
\end{twochants}

\subsection{Translation}

\begin{english}
  We/I, Venerable Sir/Sister/Friend,\\
  request the Three Refuges and the Eight Precepts.

  For the second time,\\
  We/I, Venerable Sir/Sister/Friend,\\
  request the Three Refuges and the Eight Precepts.

  For the third time,\\
  We/I, Venerable Sir/Sister/Friend,\\
  request the Three Refuges and the Eight Precepts.
\end{english}

\subsection{Taking the Three Refuges}%{{{1

\begin{instruction}
  Repeat, after the leader has chanted the first three lines
\end{instruction}

Namo tassa bhagavato arahato sammāsambuddhassa\\
Namo tassa bhagavato arahato sammāsambuddhassa\\
Namo tassa bhagavato arahato sammāsambuddhassa

\begin{english}
  Homage to the Blessed, Noble, and Perfectly Enlightened One.\\
  Homage to the Blessed, Noble, and Perfectly Enlightened One.\\
  Homage to the Blessed, Noble, and Perfectly Enlightened One.
\end{english}

Buddhaṃ saraṇaṃ gacchāmi\\
Dhammaṃ saraṇaṃ gacchāmi\\
Saṅghaṃ saraṇaṃ gacchāmi

\begin{english}
  To the Buddha I go for refuge.\\
  To the Dhamma I go for refuge.\\
  To the Saṅgha I go for refuge.
\end{english}

Dutiyampi buddhaṃ saraṇaṃ gacchāmi\\
Dutiyampi dhammaṃ saraṇaṃ gacchāmi\\
Dutiyampi saṅghaṃ saraṇaṃ gacchāmi

\begin{english}
  For the second time, to the Buddha I go for refuge.\\
  For the second time, to the Dhamma I go for refuge.\\
  For the second time, to the Saṅgha I go for refuge.
\end{english}

Tatiyampi buddhaṃ saraṇaṃ gacchāmi\\
Tatiyampi dhammaṃ saraṇaṃ gacchāmi\\
Tatiyampi saṅghaṃ saraṇaṃ gacchāmi

\begin{english}
  For the third time, to the Buddha I go for refuge.\\
  For the third time, to the Dhamma I go for refuge.\\
  For the third time, to the Saṅgha I go for refuge.
\end{english}

\begin{instruction}
  Leader:
\end{instruction}

[Tisaraṇa-gamanaṃ niṭṭhitaṃ]

\begin{english}
  This completes the going to the Three Refuges.
\end{english}

\begin{instruction}
  Response:
\end{instruction}

Āma bhante / Āma ayye / Āma mitta

\begin{english}
  Yes, Venerable Sir/Sister/Friend.
\end{english}

\subsection{The Eight Precepts}%{{{1

\begin{instruction}
  Repeat each precept after the leader
\end{instruction}

\begin{precept}
  \setcounter{enumi}{0}
  \item Pāṇātipātā veramaṇī sikkhāpadaṃ samādiyāmi
\end{precept}

\begin{english}
  I undertake the precept to refrain from taking the life of any living~creature.
\end{english}

\begin{precept}
  \setcounter{enumi}{1}
  \item Adinnādānā veramaṇī sikkhāpadaṃ samādiyāmi
\end{precept}

\begin{english}
  I undertake the precept to refrain from taking that which is not given.
\end{english}

\begin{precept}
  \setcounter{enumi}{2}
  \item Abrahmacariyā veramaṇī sikkhāpadaṃ samādiyāmi
\end{precept}

\begin{english}
  I undertake the precept to refrain from any intentional sexual activity.
\end{english}

\begin{precept}
  \setcounter{enumi}{3}
  \item Musāvādā veramaṇī sikkhāpadaṃ samādiyāmi
\end{precept}

\begin{english}
  I undertake the precept to refrain from lying.
\end{english}

\begin{precept}
  \setcounter{enumi}{4}
  \item Surāmeraya-majja-pamādaṭṭhānā veramaṇī sikkhāpadaṃ samādiyāmi
\end{precept}

\begin{english}
  I undertake the precept to refrain from consuming intoxicating drink and drugs which lead to carelessness.
\end{english}

\begin{precept}
  \setcounter{enumi}{5}
  \item Vikālabhojanā veramaṇī sikkhāpadaṃ samādiyāmi.
\end{precept}

\begin{english}
  I undertake the precept to refrain from eating at inappropriate times.
\end{english}

\begin{precept}
  \setcounter{enumi}{6}
  \item Nacca-gīta-vādita-visūkadassanā mālā-gandha-vilepana-dhāraṇa-maṇḍana-vibhūsanaṭṭhānā veramaṇī sikkhāpadaṃ samādiyāmi.
\end{precept}

\begin{english}
  I undertake the precept to refrain from entertainment, beautification, and~adornment.
\end{english}

\begin{precept}
  \setcounter{enumi}{7}
  \item Uccāsayana-mahāsayanā veramaṇī sikkhāpadaṃ samādiyāmi.
\end{precept}

\begin{english}
  I undertake the precept to refrain from lying on a high or luxurious sleeping place.
\end{english}

\begin{instruction}
  Leader:
\end{instruction}

[Imāni aṭṭha sikkhāpadāni samādiyāmi]

\begin{instruction}
  Response:
\end{instruction}

Imāni aṭṭha sikkhāpadāni samādiyāmi\\
Imāni aṭṭha sikkhāpadāni samādiyāmi\\
Imāni aṭṭha sikkhāpadāni samādiyāmi

\begin{english}
  I undertake these Eight Precepts.\\
  I undertake these Eight Precepts.\\
  I undertake these Eight Precepts.
\end{english}

\begin{instruction}
  Leader:
\end{instruction}

[Imāni aṭṭha sikkhāpadāni\\
Sīlena sugatiṃ yanti\\
Sīlena bhogasampadā\\
Sīlena nibbutiṃ yanti\\
Tasmā sīlaṃ visodhaye]

\begin{english}
  These are the Eight Precepts;\\
  virtue is the source of happiness,\\
  virtue is the source of true wealth,\\
  virtue is the source of peacefulness ---\\
  Therefore let virtue be purified.
\end{english}

\begin{instruction}
  Response:
\end{instruction}

Sādhu, sādhu, sādhu.

\begin{instruction}
  Bow three times
\end{instruction}

\section{Five Precepts (Sri Lankan Tradition)}

With hands in \emph{añjali}, the laypeople recite the following request:

‘Sādhu! Sādhu! Sādhu! Okāsa ahaṃ bhante tisaraṇena saddhiṃ pañca-sīlaṃ dhammaṃ
yācāmi, anuggahaṃ katvā sīlaṃ detha me bhante. Dutiyam-pi okāsa… Tatiyam-pi
okāsa…’

\emph{Bhikkhu}: ‘Yaṃ ahaṃ vadāmi taṃ vadetha.’

\emph{Laypeople}: ‘Āma, bhante.’

\emph{Bhk}: ‘Namo…’ (×3)

\emph{Laypeople}: repeat.

\emph{Bhk}: ‘Saraṇagamanaṃ sampuṇṇaṃ.’

\emph{Laypeople}: ‘Āma, bhante.’

Then the bhikkhu recites, with the laypeople repeating line by line:

‘Pāṇātipātā veramaṇī sikkhā-padaṃ samādiyāmi.\\
Adinnādānā veramaṇī sikkhā-padaṃ samādiyāmi.\\
Kāmesu micchā-cārā veramaṇī sikkhā-padaṃ samādiyāmi.\\
Musā-vādā veramaṇī sikkhā-padaṃ samādiyāmi.\\
Surā-meraya-majja-pamādaṭṭhānā veramaṇī sikkhā-padaṃ samādiyāmi.’

\suttaRef{cf. A.IV.248–250}

\emph{Bhk}:

‘Tisaraṇena saddhiṃ pañcasīlaṃ dhammaṃ sādhukaṃ surakkhitaṃ katvā appamādena
sampādetha.’

\emph{Laypeople}: ‘Āma, bhante.’

\emph{Bhk}:

‘Sīlena sugatiṃ yanti\\
Sīlena bhoga-sampadā,\\
Sīlena nibbutiṃ yanti,\\
Tasmā sīlaṃ visodhaye.’

