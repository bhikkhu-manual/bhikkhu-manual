\chapter{Rains and Kathina}

\section{Khamāpana-kammaṃ (Asking for Forgiveness)}

The bhikkhus: ‘Namo…’ (×3)
‘*Ā yasmante* pamādena,
dvārattayena kataṃ,
sabbaṃ aparādhaṃ khamatu *no* bhante.’
(‘Forgive us, ven. sir, for all wrong-doing done
carelessly to the ven. one by way of the three doors.’)

The senior bhikkhu:
‘Ahaṃ khamāmi,
*tumhehi pi* me khamitabbaṃ.’
(‘I forgive you. You should also forgive me.’)

The bhikkhus: ‘*Khamāma* bhante.’
(‘We forgive you, ven. sir.’)

Then the bhikkhus may bow while the senior
bhikkhu gives his blessing:
‘Evaṃ hotu evaṃ hotu,
Yo ca pubbe pamajjitvā pacchā so nappamajjati,
So’maṃ lokaṃ pabhāseti abbhā mutto va
candimā,
[Dhp,v.172]
Yassa pāpaṃ kataṃ kammaṃ kusalena pithīyati,
So’maṃ lokaṃ pabhāseti abbhā mutto va
[Dhp,v.173]
candimā,

Abhivādana sīlissa niccaṃ vuḍḍhāpacāyino,
Cattāro dhammā vaḍḍhanti:
Āyu vaṇṇo sukhaṃ balaṃ.’ [Dhp,v.109]
At the end of the blessing the bhikkhus, while
still bowing, respond: ‘Sādhu bhante.’
♦ For senior bhikkhus use ‘Āyasmante’. For
mo re s en ior bh ik kh u s u s e ‘The re ’, ‘ Ma hā there’, ‘Ācariye’, or ‘Upajjhāye’, as appropriate.
• When one bhikkhu asks for forgiveness:
‘no’ →‘me’; ‘tumhehi pi’ →‘tayā pi’
‘Khamāma’ →‘Khamāmi’

\section{Vassāvāso (Rains-residence)}

The Rains begins the day after the full-moon
d a y of J u ly; if J u ly h a s tw o fu l l mo o n s, it
begins after the second full moon. During this
time bhikkhus must live in a kuṭi with a lockable door.

(i) Entering the Rains (Thai tradition)
The boundaries are specified, then all resident
bhikkhus:
‘Imasmiṃ āvāse
imaṃ te-māsaṃ vassaṃ upema.’ (×3)
(‘We enter the Rains in this monastery
for three months.’)

• If one bhikkhu at a time: ‘upema’ →‘upemi’

• Alternatively:

‘Imasmiṃ vihāre
imaṃ te-māsaṃ vassaṃ upemi.’ (×3)
(‘I enter the Rains in this kuṭi for three months.’)

• Alternatively:

‘Idha vassaṃ upemi.’ (×3)
(‘I enter the Rains here.’)

[cf. Sp,V,1067]

(ii) Entering the Rains (Sri Lanka)
‘Imasmiṃ vihāre
imaṃ te-māsaṃ vassaṃ upemi.
Idha vassaṃ upemi.’
(‘I enter the Rains in this kuṭi for three months.
I enter the Rains here.’)

(iii) Sattāha-karaṇīya (Seven-day leave):
Allowable reasons: to go to nurse an ill
bhikkhu or one’s parents, support a bhikkhu
in danger of disrobing, aid another monastery, uphold the faith of lay supporters, etc..
One may take leave using one’s own language,
or the Pali:
‘Sattāha-karaṇīyaṃ kiccaṃ me-v-atthi
tasmā mayā gantabbaṃ imasmiṃ
sattāh’abbhantare nivattissāmi.’
(‘I have an obligation which must be fulfilled within
seven days. Therefore I have to go. I shall return
within seven days.’)
[cf. Vin,I,139]

(iv) Rains privileges:
These last for one month following the
pavāraṇā-day. One may: go wandering without taking leave; go without taking the complete set of robes; go taking any robes that
have accrued; keep extra robes beyond ten
days; eat a ‘group meal’, and ‘substitute an
invitation to a meal’.

\section{Pavāraṇā (Inviting Admonition)}

(i) For five or more bhikkhus
A ft e r th e pr e li mi n a r y d u t ie s , on e b h i k k h u
chants the ñatti:
‘Suṇātu me *bhante* saṅgho.
Ajja pavāraṇā *paṇṇarasī*.
Yadi saṅghassa pattakallaṃ,
Saṅgho *te-vācikaṃ* pavāreyya.’
(‘Ven. sirs, may the Community listen to me. Today
is the Pavāraṇā on the fifteenth (day of the
fortnight). If the Community is ready, the
Community should invite with three statements.’)
[cf. Vin,I,159]

• When it is the 14th day:
‘paṇṇarasī’ →‘cātuddasī’
• If the announcing bhikkhu is the most senior:
‘bhante’ →‘āvuso’
• If each bhikkhu is to state his invitation twice:
‘te-vācikaṃ’ →‘dve-vācikaṃ’

• If each bhikkhu is to state his invitation once:
‘te-vācikaṃ’ →‘eka-vācikaṃ’
• If bhikkhus of equal rains are to invite in

unison:
‘Saṅgho te-vācikaṃ pavāreyya’
→ ‘Saṅgho samāna-vassikaṃ pavāreyya’
(‘The Community should invite
in the manner of equal Rains.’)

After the ñatti, if each bhikkhu is to invite
‘three times’, then, in order of Rains:
‘*Saṅgham-bhante* pavāremi.
Diṭṭhena vā sutena vā parisaṅkāya vā,
vadantu maṃ āyasmanto anukampaṃ upādāya.
Passanto paṭikkarissāmi.
Dutiyam-pi bhante saṅghaṃ pavāremi.
Diṭṭhena vā sutena vā parisaṅkāya vā,
vadantu maṃ āyasmanto anukampaṃ upādāya.
Passanto paṭikkarissāmi.
Tatiyam-pi bhante saṅghaṃ pavāremi
Diṭṭhena vā sutena vā parisaṅkāya vā,
vadantu maṃ āyasmanto anukampaṃ upādāya.
Passanto paṭikkarissāmi.’
(‘Ven. sirs, I invite admonition from the Sangha.
According to what has been seen, heard or suspected,
may the ven. ones instruct me out of compassion.
Seeing it, I shall make amends.
For a second time… For a third time….’)

• For the most senior bhikkhu:

‘Saṅgham-bhante’ →‘Saṅghaṃ āvuso’
‘Dutiyam-pi bhante’ → ‘Dutiyam-pi āvuso’
‘Tatiyam-pi bhante’ → ‘Tatiyam-pi āvuso’

(ii) For four or three bhikkhus
Preliminary duties, then ñatti:
‘Suṇantu me *āyasmanto*,
Ajja pavāraṇā paṇṇarasī,
Yad’āyasmantānaṃ pattakallaṃ,
Mayaṃ aññamaññaṃ pavāreyyāma.’
(‘Sirs, may you listen to me. Today is the pavāraṇā
on the 15th (day of the fortnight). If there is
complete preparedness of the ven. ones, we should
pavāraṇā to each other.’)
[cf. Vin,I,162]

• If there are three bhikkhus:
‘āyasmanto’ →‘āyasmantā’

Then each bhikkhu in order of Rains:
‘Ahaṃ bhante āyasmante pavāremi.
Diṭṭhena vā sutena vā parisaṅkāya vā,
Vadantu maṃ āyasmanto anukampaṃ upādāya,
Passanto paṭikkarissāmi.
Dutiyam-pi bhante āyasmante pavāremi
Diṭṭhena vā sutena vā parisaṅkāya vā,
Vadantu maṃ āyasmanto anukampaṃ upādāya,
Passanto paṭikkarissāmi.
Tatiyam-pi bhante āyasmante pavāremi
Diṭṭhena vā sutena vā parisaṅkāya vā,
Vadantu maṃ āyasmanto anukampaṃ upādāya,
Passanto paṭikkarissāmi.’

• For the most senior bhikkhu:
‘bhante’ →‘āvuso’
• If there are three bhikkhus:
‘āyasmanto’ →‘āyasmantā’

(iii) For two bhikkhus
Preliminary duties, but no ñatti, then each
bhikkhu in order of Rains:
‘Ahaṃ bhante āyasmantaṃ pavāremi.
Diṭṭhena vā sutena vā parisaṅkāya vā,
Vadatu maṃ āyasmā anukampaṃ upādāya,
Passanto paṭikkarissāmi.
Dutiyam-pi bhante āyasmantaṃ pavāremi.
Diṭṭhena vā sutena vā parisaṅkāya vā,
Vadatu maṃ āyasmā anukampaṃ upādāya,
Passanto paṭikkarissāmi.
Tatiyam-pi bhante āyasmantaṃ pavāremi.
Diṭṭhena vā sutena vā parisaṅkāya vā,
Vadatu maṃ āyasmā anukampaṃ upādāya,
Passanto paṭikkarissāmi.’
[cf. Vin,I,163]

• For the senior bhikkhu: ‘bhante’ →‘āvuso’

(iv) For one bhikkhu
Preliminary duties, then:
‘Ajja me pavāraṇā.’
(‘Today is my pavāraṇā.’)

[Vin,I,163]

(v) Pavāraṇā by a sick bhikkhu
‘Pavāraṇaṃ dammi,
Pavāraṇaṃ me hara,
Mam’atthāya pavārehi.’
(‘I give my pavāraṇā. May you convey pavāraṇā for
me. May you pavāraṇā on my behalf.’)
[Vin,I,161]

• If the sick bhikkhu is the junior one:
‘hara’ → ‘haratha’
‘pavārehi’ →‘pavāretha’

The pavāraṇā of the sick bhikkhu (e.g. ‘Uttaro’)
is conveyed in his place in the order of Rains:
‘Āyasmā bhante ‘uttaro’ gilāno saṅghaṃ pavāreti,
Diṭṭhena vā sutena vā parisaṅkāya vā,
Vadantu taṃ āyasmanto anukampaṃ upādāya,
Passanto paṭikkarissati.
Dutiyam-pi bhante āyasmā ‘uttaro’ gilāno…
Passanto paṭikkarissati.
Tatiyam-pi bhante āyasmā ‘uttaro’ gilāno…
Passanto paṭikkarissati.’
(‘Ven. sirs, ven. ‘Uttaro’ who is sick makes pavāraṇā
to the Saṅgha. With what you have seen, heard and
suspected, may all of you instruct him out of
compassion. Seeing it, he will make amends.’)
[Sp,V,1075]

• If the conveying bhikkhu is senior to the sick

bhikkhu:

‘Āyasmā bhante ‘uttaro’’
→‘‘Uttaro’ bhante bhikkhu’

\section{Kaṭhina}

(i) Of fering the Kaṭhina (Thai Tradition)
Preliminary consultation of Sangha:
The first bhikkhu describes the merits obtained
in acknowledging the making of the kaṭhina
robe, and then asks the Sangha whether or not
it desires to do so. The bhikkhus respond by
saying in unison:
‘Ākaṅkhāma, bhante’
(We desire to do so, ven. sir.)

The second bhikkhu describes qualities of one
worthy of the kaṭhina-robe, and the bhikkhus
respond by remaining silent.
The third bhikkhu nominates the worthy
recipient, and the assembly responds:
‘Ruccati bhante’.
(It is pleasing, ven. sir.)

T he fou r th bh ik k hu ma k es th e f or ma l p r oposal, and the assembly responds:
‘Sādhu bhante’.
(It is well, ven. sir.)

• Bhikkhus senior to the speaker omit

‘bhante’.
Then two bhikkhus chant the formal motion
[ But cf . Vin,I,254]
and announcement.

(ii) Spreading the Kaṭhina
After the kaṭhina-robe has been sewed and

dyed, and the old robe relinquished, the new
robe is marked and determined, and then the
recipient chants one of the following:
‘Namo….’ (×3)
‘Imāya saṅghāṭiyā kaṭhinaṃ attharāmi.’
‘Iminā uttarāsaṅgena kaṭhinaṃ attharāmi.’
‘Iminā antaravāsakena kaṭhinaṃ attharāmi.’
(‘By means of this outer robe / upper robe / lower
[Sp,V,1109; Pv,XIV,4]
robe I spread the Kaṭhina’)

(iii) Kaṭhina Anumodanā
The recipient of the Kaṭhina:
‘Atthataṃ bhante saṅghassa kaṭhinaṃ,
Dhammiko kaṭhinatthāro, anumodatha.’ (×3)
(‘Ven. sirs, the spreading of the kaṭhina is in
accordance with the Dhamma. Please approve of it.’)

• If the recipient is senior to all the other
bhikkhus:
‘bhante’ →‘āvuso’

The rest of the Sangha, chanting together:
‘Atthataṃ bhante saṅghassa kaṭhinaṃ,
Dhammiko kaṭhinatthāro, anumodāma.’ (×3)
(‘Ven. sirs, the spreading of the kaṭhina is in
accordance with the Dhamma. We approve of it.’)
[Sp,V,1109; Pv,XIV,4]

• Bhikkhus senior to the recipient omit ‘bhante’.
• If approving one by one:
‘anumodāma’ →‘anumodāmi’
• for bhikkhus senior to the recipient
‘bhante’ →‘āvuso’.

♦ For a bhikkhu who completes the Kaṭhina
ceremony, the Rains privileges extend for a
further four months.
[Vin,III,261]

