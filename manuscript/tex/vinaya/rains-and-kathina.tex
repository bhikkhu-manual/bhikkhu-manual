\chapter{Rains and Kathina}

\section{Khamāpana-kammaṃ (Asking for Forgiveness)}
\label{asking-forgiveness}

\subsection{Setup}

Prepare an offering tray with two candles, incense, some flowers, and optionally
other gifts. Prepare a seat and water for the Ācariya if appropriate for the
occation. Wear your triple robe.

\subsection{Asking for Forgiveness}

All dependents kneel on toes before the Ācariya, the most senior dependent (SD)
in front, with the offering tray to his side.

\instr{(Bow three times, and start chanting in a bowed posture.)}

SD bhikkhu may prompt the chanting, then all dependents chanting together.

\emph{SD:} Na--\\
\emph{All:} ‘Namo tassa...’ (×3)

SD picks up the and holds the tray, still in bowed posture.

\emph{SD:} Ā--\\
\emph{All:} ‘\emph{Āyasmante} pamādena, dvārattayena kataṃ,\\ sabbaṃ aparādhaṃ khamatu \emph{no} bhante.’

(‘\emph{Forgive us, ven. sir, for all wrong-doing done carelessly to the ven. one by way of the three doors.}’)

SD offers the tray to the Ācariya.

The Ācariya:

‘Ahaṃ khamāmi, \emph{tumhehi pi} me khamitabbaṃ.’\\
‘\emph{I forgive you. You should also forgive me.}’

The bhikkhus: ‘\emph{Khamāma} bhante.’\\
‘\emph{We forgive you, ven. sir.}’

Then the bhikkhus may bow while the Ācariya gives his blessing.

% NOTE: not everyone gives the blessing like this.
%
% ‘Evaṃ hotu evaṃ hotu, Yo ca pubbe pamajjitvā pacchā so nappamajjati, So'maṃ
% lokaṃ pabhāseti abbhā mutto va candimā,\\
% \mbox{}\suttaRef{Dhp 172}
% 
% Yassa pāpaṃ kataṃ kammaṃ kusalena pithīyati, So'maṃ lokaṃ pabhāseti abbhā mutto
% va candimā, \suttaRef{Dhp 173}
% 
% Abhivādana sīlissa niccaṃ vuḍḍhāpacāyino, Cattāro dhammā vaḍḍhanti: Āyu vaṇṇo
% sukhaṃ balaṃ.’ \suttaRef{Dhp 109}

At the end of the blessing the bhikkhus, while still bowing, respond:

\emph{All:} ‘Sādhu bhante.’

For senior bhikkhus use ‘\emph{Āyasmante}’. For Ajahns use ‘\emph{There}’,
‘\emph{Mahāthere}’, ‘\emph{Ācariye}’, ‘\emph{Upajjhāye}’, as appropriate.

When entering Rains, asking for forgiveness is followed by taking dependence
(\emph{nissaya}), see p.\pageref{nissaya}.

When one bhikkhu asks for forgiveness:

\begin{tabular}{@{}lll@{}}
‘no’ & → & ‘me’\\
‘tumhehi pi’ & → & ‘tayā pi’\\
‘khamāma’ & → & ‘khamāmi’\\
\end{tabular}

\section{Vassāvāso (Rains-residence)}

The Rains begins the day after the full-moon day of July; if July has two full
moons, it begins after the second full moon. During this time bhikkhus must live
in a \emph{kuṭi} with a lockable door.

\subsection{Entering the Rains (Thai tradition)}

The boundaries are specified, then all resident bhikkhus:

‘Imasmiṃ āvāse imaṃ te-māsaṃ vassaṃ upema.’ (×3)\\
‘\emph{We enter the Rains in this monastery for three months.}’

If one bhikkhu at a time: ‘upema’ → ‘upemi’

Alternatively:

‘Imasmiṃ vihāre imaṃ te-māsaṃ vassaṃ upemi.’ (×3)\\
‘\emph{I enter the Rains in this kuṭi for three months.}’

Alternatively:

‘Idha vassaṃ upemi.’ (×3)\\
‘\emph{I enter the Rains here.}’ \suttaRef{cf. Sp.V.1067}

\subsection{Sattāha-karaṇīya (Seven-day leave)}

Allowable reasons: to go to nurse an ill bhikkhu or one's parents, support a
bhikkhu in danger of disrobing, aid another monastery, uphold the faith of lay
supporters, etc.

One may take leave using one's own language, or the Pali:

‘Sattāha-karaṇīyaṃ kiccaṃ me-v-atthi tasmā mayā gantabbaṃ imasmiṃ
sattāh'abbhantare nivattissāmi.’

‘\emph{I have an obligation which must be fulfilled within seven days. Therefore
  I have to go. I shall return within seven days.}’\\
\mbox{}\suttaRef{cf. Vin.I.139}

\subsection{Rains privileges}

These last for one month following the pavāraṇā-day. One may: go wandering
without taking leave; go without taking the complete set of robes; go taking any
robes that have accrued; keep extra robes beyond ten days; eat a ‘group meal’,
and ‘substitute an invitation to a meal’.

\section{Pavāraṇā (Inviting Admonition)}

\subsection{For five or more bhikkhus}

After the preliminary duties, one bhikkhu chants the \emph{ñatti}:

‘Suṇātu me \emph{bhante} saṅgho.\\
Ajja pavāraṇā \emph{paṇṇarasī}.\\
Yadi saṅghassa pattakallaṃ,\\
Saṅgho \emph{te-vācikaṃ} pavāreyya.’

‘\emph{Ven. sirs, may the Community listen to me. Today is the Pavāraṇā on the
  fifteenth (day of the fortnight). If the Community is ready, the Community
  should invite with three statements.}’

\suttaRef{cf. Vin.I.159}

When it is the 14th day:\\
‘paṇṇarasī’ → ‘cātuddasī’

If the announcing bhikkhu is the most senior:\\
‘bhante’ → ‘āvuso’

If each bhikkhu is to state his invitation twice:\\
‘te-vācikaṃ’ → ‘dve-vācikaṃ’

If each bhikkhu is to state his invitation once:\\
‘te-vācikaṃ’ → ‘eka-vācikaṃ’

If bhikkhus of equal rains are to invite in unison:

‘Saṅgho te-vācikaṃ pavāreyya’ → ‘Saṅgho samāna-vassikaṃ pavāreyya’

‘\emph{The Community should invite in the manner of equal Rains.}’

After the \emph{ñatti}, if each bhikkhu is to invite ‘three times’, then, in
order of Rains:

‘\emph{Saṅgham-bhante} pavāremi. Diṭṭhena vā sutena vā parisaṅkāya vā, vadantu
maṃ āyasmanto anukampaṃ upādāya. Passanto paṭikkarissāmi.

Dutiyam-pi bhante saṅghaṃ pavāremi. Diṭṭhena vā sutena vā parisaṅkāya vā,
vadantu maṃ āyasmanto anukampaṃ upādāya. Passanto paṭikkarissāmi.

Tatiyam-pi bhante saṅghaṃ pavāremi Diṭṭhena vā sutena vā parisaṅkāya vā, vadantu
maṃ āyasmanto anukampaṃ upādāya. Passanto paṭikkarissāmi.’

‘\emph{Ven. sirs, I invite admonition from the Sangha. According to what has
  been seen, heard or suspected, may the ven. ones instruct me out of
  compassion. Seeing it, I shall make amends. For a second time… For a third
  time….}’

For the most senior bhikkhu:

‘Saṅgham-bhante’ → ‘Saṅghaṃ āvuso’\\
‘Dutiyam-pi bhante’ → ‘Dutiyam-pi āvuso’\\
‘Tatiyam-pi bhante’ → ‘Tatiyam-pi āvuso’

\subsection{For four or three bhikkhus}

Preliminary duties, then \emph{ñatti}:

‘Suṇantu me \emph{āyasmanto}, ajja pavāraṇā paṇṇarasī, yad'āyasmantānaṃ
pattakallaṃ, mayaṃ aññamaññaṃ pavāreyyāma.’

‘\emph{Sirs, may you listen to me. Today is the pavāraṇā on the 15th (day of the
fortnight). If there is complete preparedness of the ven. ones, we should
pavāraṇā to each other.}’

\suttaRef{cf. Vin.I.162}

If there are three bhikkhus:\\
‘āyasmanto’ → ‘āyasmantā’

Then each bhikkhu in order of Rains:

‘Ahaṃ bhante āyasmante pavāremi. Diṭṭhena vā sutena vā parisaṅkāya vā, Vadantu
maṃ āyasmanto anukampaṃ upādāya, Passanto paṭikkarissāmi.

Dutiyam-pi bhante āyasmante pavāremi. Diṭṭhena vā sutena vā parisaṅkāya vā,
Vadantu maṃ āyasmanto anukampaṃ upādāya, Passanto paṭikkarissāmi.

Tatiyam-pi bhante āyasmante pavāremi. Diṭṭhena vā sutena vā parisaṅkāya vā,
Vadantu maṃ āyasmanto anukampaṃ upādāya, Passanto paṭikkarissāmi.’

For the most senior bhikkhu:\\
‘bhante’ → ‘āvuso’

If there are three bhikkhus:\\
‘āyasmanto’ → ‘āyasmantā’

\subsection{For two bhikkhus}

Preliminary duties, but no \emph{ñatti}, then each bhikkhu in order of Rains:

‘Ahaṃ bhante āyasmantaṃ pavāremi. Diṭṭhena vā sutena vā parisaṅkāya vā, Vadatu
maṃ āyasmā anukampaṃ upādāya, Passanto paṭikkarissāmi.

Dutiyam-pi bhante āyasmantaṃ pavāremi. Diṭṭhena vā sutena vā parisaṅkāya vā,
Vadatu maṃ āyasmā anukampaṃ upādāya, Passanto paṭikkarissāmi.

Tatiyam-pi bhante āyasmantaṃ pavāremi. Diṭṭhena vā sutena vā parisaṅkāya vā,
Vadatu maṃ āyasmā anukampaṃ upādāya, Passanto paṭikkarissāmi.’

For the senior bhikkhu: ‘bhante’ → ‘āvuso’

\suttaRef{cf. Vin.I.163}

\subsection{For one bhikkhu}

Preliminary duties, then:\\
‘Ajja me pavāraṇā.’\\
‘\emph{Today is my pavāraṇā.}’ \suttaRef{Vin.I.163}

\subsection{Pavāraṇā by a sick bhikkhu}

‘Pavāraṇaṃ dammi, pavāraṇaṃ me hara, mam'atthāya pavārehi.’

‘\emph{I give my pavāraṇā. May you convey pavāraṇā for me. May you pavāraṇā on
  my behalf.}’

\suttaRef{Vin.I.161}

If the sick bhikkhu is the junior one:

\begin{tabular}{@{}lll@{}}
‘hara’ & → & ‘haratha’\\
‘pavārehi’ & → & ‘pavāretha’\\
\end{tabular}

The \emph{pavāraṇā} of the sick bhikkhu (e.g. ‘Uttaro’) is conveyed in his place
in the order of Rains:

‘Āyasmā bhante ‘uttaro’ gilāno saṅghaṃ pavāreti, diṭṭhena vā sutena vā
parisaṅkāya vā, vadantu taṃ āyasmanto anukampaṃ upādāya, passanto
paṭikkarissati.

Dutiyam-pi bhante āyasmā ‘uttaro’ gilāno… passanto paṭikkarissati.

Tatiyam-pi bhante āyasmā ‘uttaro’ gilāno… passanto paṭikkarissati.’

‘\emph{Ven. sirs, ven. ‘Uttaro’ who is sick makes pavāraṇā to the Saṅgha. With
  what you have seen, heard and suspected, may all of you instruct him out of
  compassion. Seeing it, he will make amends.}’

If the conveying bhikkhu is senior to the sick bhikkhu:

‘Āyasmā bhante ‘uttaro’’ → ‘‘Uttaro’ bhante bhikkhu’

\suttaRef{Sp.V.1075}

\section{Kaṭhina}

\subsection{Offering the Kaṭhina (Thai Tradition)}

Preliminary consultation of Sangha:

The first bhikkhu describes the merits obtained in acknowledging the making of
the \emph{kaṭhina} robe, and then asks the Sangha whether or not it desires to
do so. The bhikkhus respond by saying in unison:

‘Ākaṅkhāma, bhante.’\\
‘\emph{We desire to do so, ven. sir.}’

The second bhikkhu describes qualities of one worthy of the kaṭhina-robe, and
the bhikkhus respond by remaining silent.

The third bhikkhu nominates the worthy recipient, and the assembly responds:

‘Ruccati bhante’.\\
‘\emph{It is pleasing, ven. sir.}’

The fourth bhikkhu makes the formal proposal, and the assembly responds:

‘Sādhu bhante’.\\
‘\emph{It is well, ven. sir.}’

Bhikkhus senior to the speaker omit ‘\emph{bhante}’.

Then two bhikkhus chant the formal motion and announcement.

\suttaRef{But cf. Vin.I.254}

\subsection{Procedure to Give the Kaṭhina-cloth}

The bhikkhus meet in the Dhamma Hall.

After bowing to the shrine, chant the `Dedication of Offerings' (\emph{Yo so
  bhagavā}\ldots), and `Preliminary Homage' (\emph{Namo tassa}).

The chanting bhikkhu announces the motion and decision to give the
\emph{Kaṭhina-cloth} to a particular bhikkhu (sec.\ref{kathina-sanghakamma}).

The bhikkhu receiving the robe, in front of everyone relinquishes the robe he
will replace, usually a \emph{sabong}.

Then he bindus the robe he has received in front of everyone. He leaves the room
with one or two bhikkhus, and changes into the new robe. He returns to the
gathered bhikkhus, determines the new robe and determines the \emph{Kaṭhina} by
chanting \emph{Spreading the Kaṭhina} (sec.\ref{spreading-the-kathina}).

Together, the other bhikkhus chant their anumodanā (sec.\ref{kathina-anumodana}).

\subsection{Kaṭhina Saṅghakamma}
\label{kathina-sanghakamma}

\instr{In the following, ‘Amaro Bhikkhu’ is the receiving senior bhikkhu.}

Suṇātu me bhante saṅgho. Idaṃ saṅghassa kaṭhina-dussaṃ uppannaṃ. Yadi saṅghassa
pattakallaṃ, saṅgho imaṃ kaṭhina-dussaṃ āyasmato \emph{Amarassa} dadeyya,
kaṭhinaṃ attharituṃ. Esā ñatti.

Suṇātu me bhante saṅgho. Idaṃ saṅghassa kaṭhina-dussaṃ uppannaṃ. Saṅgho imaṃ
kaṭhina-dussaṃ āyasmato \emph{Amarassa} deti, kaṭhinaṃ attharituṃ.
Yass'āyasmato khamati, imassa kaṭhina-dussassa āyasmato \emph{Amarassa} dānaṃ,
kaṭhinaṃ attharituṃ, so tuṇh'assa. Yassa nakkhamati, so bhāseyya.

Dinnaṃ idaṃ saṅghena kaṭhina-dussaṃ āyasmato \emph{Amarassa}, kaṭhinaṃ
attharituṃ. Khamati saṅghassa, tasmā tuṇhī. Evam-etaṃ dhārayāmi.

\suttaRef{Mv.VII.1.4}

\begin{english}

Venerable sirs, may the Community listen to me. This kaṭhina-cloth has arisen
for the Community. If the Community is ready, it should give this kaṭhina-cloth
to Venerable \emph{Amaro} to spread the kaṭhina. This is the motion.

Venerable sirs, may the Community listen to me. This kaṭhina-cloth has arisen
for the Community. The Community is giving this kaṭhina-cloth to Venerable
\emph{Amaro} to spread the kaṭhina. He to whom the giving of this kaṭhina-cloth
to Venerable \emph{Amaro} to spread the kaṭhina is agreeable should remain
silent. He to whom it is not agreeable should speak.

This kaṭhina-cloth is given by the Community to Venerable \emph{Amaro} to
spread the kaṭhina. This is agreeable to the Community, therefore it is silent.
Thus do I hold it.

\end{english}

\subsection{Spreading the Kaṭhina}
\label{spreading-the-kathina}

After the kaṭhina-robe has been sewed and dyed, and the old robe relinquished,
the new robe is marked and determined, and then the recipient chants \emph{one}
of the following:

‘Namo….’ (×3)\\
‘Imāya saṅghāṭiyā kaṭhinaṃ attharāmi.’\\
‘Iminā uttarāsaṅgena kaṭhinaṃ attharāmi.’\\
‘Iminā antaravāsakena kaṭhinaṃ attharāmi.’

‘\emph{By means of this outer robe / upper robe / lower robe I spread the Kaṭhina.}’

\suttaRef{Sp.V.1109; Pv.XIV.4}

\subsection{Kaṭhina Anumodanā}
\label{kathina-anumodana}

The recipient of the Kaṭhina:

‘Atthataṃ bhante saṅghassa kaṭhinaṃ, Dhammiko kaṭhinatthāro, anumodatha.’ (×3)

‘\emph{Ven.\ sirs, the spreading of the kaṭhina is in accordance with the Dhamma.
  Please approve of it.}’

If the recipient is senior to all the other bhikkhus:\\
‘bhante’ → ‘āvuso’

The rest of the Sangha, chanting together:

‘Atthataṃ bhante saṅghassa kaṭhinaṃ, Dhammiko kaṭhinatthāro, anumodāma.’ (×3)

‘\emph{Ven. sirs, the spreading of the kaṭhina is in accordance with the Dhamma.
  We approve of it.}’

\suttaRef{Sp.V.1109; Pv.XIV.4}

Bhikkhus senior to the recipient omit ‘\emph{bhante}’.

If approving one by one:\\
‘anumodāma’ → ‘anumodāmi’

For bhikkhus senior to the recipient:\\
‘bhante’ → ‘āvuso’.

For a bhikkhu who completes the \emph{Kaṭhina} ceremony, the Rains privileges
extend for a further four months.

\suttaRef{Vin.III.261}

