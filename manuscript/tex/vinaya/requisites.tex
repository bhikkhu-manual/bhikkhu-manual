\chapter{Requisites}

\section{Bindu (Marking)}

Before use, a new robe must be marked with three dots, \mbox{blue,} \mbox{green,} black or
brown in colour, saying, either out loud or mentally:

‘Imaṃ bindukappaṃ karomi.’ (×3)\\
‘\emph{I make this properly marked.}’ \suttaRef{cf. Vin.IV.120}

\section{Adhiṭṭhāna (Determining)}

‘Imaṃ \emph{saṅghāṭiṃ} adhiṭṭhāmi.’\\
‘\emph{I determine this outer robe.}’

For ‘\emph{saṅghāṭiṃ}’ substitute item as appropriate:

\begin{packeditemize}

\item uttarā-saṅgaṃ (upper robe)
\item antara-vāsakaṃ (lower robe)
\item pattaṃ (alms bowl)
\item nisīdanaṃ (sitting-cloth)
\item kaṇḍu-paṭicchādiṃ (skin-eruption covering cloth)
\item vassika-sāṭikaṃ (rains cloth)
\item paccattharaṇaṃ (sleeping cloth)
\item mukha-puñchana-colaṃ (handkerchief)
\item parikkhāra-colaṃ (small requisite)

\end{packeditemize}

The first three articles must be properly marked \emph{before} being determined
for use. Only one of each of these items may be determined at any one time.

The rains cloth may be used only during the four months of the Rains.

There is no limit to the number of articles which may be determined in each of
the last three categories above, e.g.:

‘Imāni \emph{paccattharaṇāni} adhiṭṭhāmi.’\\
‘\emph{I determine these sleeping cloths.}’

Substitute ‘\emph{mukhapuñchana-colāni}’ (handkerchiefs) or
‘\emph{parikkhāra-colāni}’ (small requisites) as appropriate.

Articles are determined either by touching the article and mentally reciting the
relevant Pali passage, or by uttering the Pali passage without touching the
article. In the latter case, if the article is beyond forearm's length:

\ifhandbookedition
\enlargethispage{\baselineskip}
\fi

\begin{tabular}{@{}lll@{}}
‘imaṃ’ (this) & → & ‘etaṃ’ (that)\\
‘imāni’ (these) & → & ‘etāni’ (those)\\
\end{tabular}

\suttaRef{Sp.III.643-644}

\section{Paccuddharaṇa (Relinquishing)}

When an outer robe, upper robe, lower robe, alms bowl or sitting-cloth is to be
replaced, the article already determined must first be relinquished from use:

‘Imaṃ saṅghāṭiṃ paccuddharāmi.’\\
‘\emph{I relinquish this outer robe.}’ \suttaRef{Sp.III.643}

Substitute the appropriate item for ‘\emph{saṅghāṭiṃ}’.

Apart from relinquishing from use, a determined article ceases to be determined
if it is given to another, is stolen, is taken on trust by a friend, or has a
large visible hole in it.

\section{Vikappana (Sharing Ownership)}

There are varied practices about sharing ownership. Here are the most common
ways.

\subsection{Generally Addressing the Recipient}
\label{general-address}

In the presence of the receiving bhikkhu, and with the article within forearm's length:

‘Imaṃ cīvaraṃ tuyhaṃ vikappemi.’\\
‘\emph{I share this robe with you.}’

‘Imāni cīvarāni tuyhaṃ vikappemi.’\\
‘\emph{\ldots{} these robes \ldots{}}’

‘Imaṃ pattaṃ tuyhaṃ vikappemi.’\\
‘\emph{\ldots{} this bowl \ldots{}}’

‘Ime patte tuyhaṃ vikappemi.’\\
‘\emph{\ldots{} these bowls \ldots{}}’

When the receiving bhikkhu is the senior:\\
‘tuyhaṃ’ → ‘āyasmato’

When it is shared with more than one bhikkhu:\\
‘tuyhaṃ’ → ‘tumhākaṃ’

When the article is beyond forearm's length:

‘imaṃ’ → ‘etaṃ’;\\
‘imāni’ → ‘etāni’;\\
‘ime’ → ‘ete’ \suttaRef{Vin.IV.122}

\subsection{Addressing the Recipient by Name}

In the presence of the receiving bhikkhu (who is named, e.g., ‘\emph{Uttaro}’),
and with the article within forearm's length, one says to another bhikkhu:

‘Imaṃ cīvaraṃ \emph{uttarassa} bhikkhuno vikappemi.’\\
‘\emph{I share this robe with Uttaro Bhikkhu.}’

When the receiving bhikkhu is the senior:\\
‘\emph{uttarassa} bhikkhuno’ → ‘āyasmato \emph{uttarassa}’

If it is shared with a novice:\\
‘\emph{uttarassa} bhikkhuno’ → ‘\emph{uttarassa} sāmaṇerassa’

To share a bowl: ‘cīvaraṃ’ → ‘pattaṃ’

If more than one article is to be shared substitute the plural form as in sec. \ref{general-address} above.

When the item is beyond forearm's length substitute as in sec. \ref{general-address} above.

\suttaRef{Vin.IV.122}

\subsection{Receiving Bhikkhu is Absent}
\label{receiving-bhikkhu-absent}

In the absence of the receiving bhikkhus, say to a witness:

‘Imaṃ cīvaraṃ vikappanatthāya tuyhaṃ dammi.’\\
‘\emph{I give this robe to you for the purpose of sharing.}’

The witness should then ask the original owner the names of two bhikkhus or
novices who are his friends or acquaintances:

‘Ko te mitto vā sandiṭṭho vā.’\\
‘\emph{Who is your friend or acquaintance?}’

\ifhandbookedition
\clearpage
\fi

After the original owner tells their names, e.g.,

‘\emph{Uttaro} bhikkhu ca \emph{tisso} sāmaṇero ca.’\\
‘\emph{Bhikkhu Uttaro and Sāmaṇera Tisso.}’

The witness then says:

‘Ahaṃ tesaṃ dammi.’ ‘\emph{I give it to them.}’

or

‘Ahaṃ \emph{uttarassa} bhikkhuno ca \emph{tissassa} sāmaṇerassa dammi.’\\
‘\emph{I give it to Bhikkhu Uttaro and Sāmaṇera Tisso.}’

\suttaRef{Vin.IV.122}

To share a bowl: ‘cīvaraṃ’ → ‘pattaṃ’

If more than one article is to be shared substitute the plural form as in sec.\ref{general-address} above.

When the item is beyond forearm's length substitute as in sec.\ref{general-address} above.

\section[Vikappana-paccuddharaṇa (Relinquishing)]{Vikappana-paccuddharaṇa (Relinquishing Shared Ownership)}

Before actually using the shared article, the other bhikkhu must relinquish his
share.

If the other bhikkhu is senior, and the article is within forearm's length:

‘Imaṃ cīvaraṃ mayhaṃ santakaṃ paribhuñja vā visajjehi vā yathāpaccayaṃ vā karohi.’\\
‘\emph{This robe of mine: you may use it, give it away, or do as you wish with it.}’

\suttaRef{cf. Kv.122}

When more than one robe is being relinquished:

\begin{tabular}{@{}lll@{}}
‘imaṃ cīvaraṃ’ & → & ‘imāni cīvarāni’\\
‘santakaṃ’ & → & ‘santakāni’\\
\end{tabular}

When the second owner is junior:

\begin{tabular}{@{}lll@{}}
‘paribhuñja’ & → & ‘paribhuñjatha’\\
‘visajjehi’ & → & ‘visajjetha’\\
‘karohi’ & → & ‘karotha’\\
\end{tabular}

If the articles are beyond forearm's length, change case accordingly:

\begin{tabular}{@{}lll@{}}
  ‘imaṃ’ (this) & → & ‘etaṃ’ (that)\\
  ‘imāni’ (these) & → & ‘etāni’ (those)\\
\end{tabular}

To rescind the shared ownership in the case when the receiving bhikkhu is absent
(sec. \ref{receiving-bhikkhu-absent}), the witness says:

‘Tesaṃ santakaṃ paribhuñja vā vissajjehi vā yathāpaccayaṃ vā karohi.’\\
‘\emph{Use what is theirs, give it away or do as you like with it.}’

To rescind the shared ownership of a bowl:\\
‘cīvaraṃ’ → ‘pattaṃ’

and alter according to sec. \ref{general-address} above.

The practice of some communities when sharing ownership of a bowl is that
permission is not required before using it. However, if the first owner wishes
to determine a shared bowl, the second owner should relinquish it first.

