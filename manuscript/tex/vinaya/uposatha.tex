\chapter{Uposatha}

\section{Pārisuddhi-uposatha (Purity Uposatha)}

(i) Pārisuddhi Before Sangha
Declaring one's purity before the Sangha:
‘Parisuddho ahaṃ bhante,
parisuddho'ti maṃ saṅgho dhāretu.’
(‘I, ven. sirs, am quite pure
May the Saṅgha hold me to be pure.’)
[cf. Vin,I,129]

(ii) Pārisuddhi for 3 Bhikkhus
The Pātimokkha requires at least four
bhikk hus. If there a re only thre e bhikk hus
then, after the preliminary duties and the general confession, one bhikkhu chants the ñatti:
‘Suṇantu me bhante āyasmantā ajj'uposatho
paṇṇaraso, yad'āyasmantānaṃ pattakallaṃ,
mayaṃ aññamaññaṃ pārisuddhi uposathaṃ
kareyyāma.’
(‘Let the ven. ones listen to me. Today is an Uposatha
day, which is a fifteenth (day of the fortnight) one. If
it seems right to the ven. ones let us carry out the
Observance with one another by way of entire purity.’)

• When it is the 14th day:
‘paṇṇaraso’ →‘cātuddaso’
• If the announcing bhikkhu is the most senior:
‘bhante’ →‘āvuso’
[Vin,I,124]

Then, starting with the senior bhikkhu:
‘Parisuddho ahaṃ āvuso,
parisuddho'ti maṃ dhāretha.’ (×3)
(‘I, friends, am quite pure.
Understand that I am quite pure.’)

For each of the two junior bhikkhus:
‘āvuso’ →‘bhante’
[Vin,I,124]

(iii) Pārisuddhi for 2 Bhikkhus
Omit the ñatti. The senior bhikkhu declares
purity first:
‘Parisuddho ahaṃ āvuso,
parisuddho'ti maṃ dhārehi.’ (×3)
• For the junior:
‘āvuso’ →‘bhante’
‘dhārehi’ →‘dhāretha’
[Vin,I,124–125]

(iv) Adhiṭṭhānuposatha (For a lone bhikkhu)
For a bhikkhu staying alone on the Uposatha
da y . Aft er th e P r e li m i na r y d ut ie s, h e th e n
determines:
‘Ajja me uposatho.’
(‘Today is an Uposatha day for me.’)
[Vin,I,125]

\section{Sick Bhikkhus}

(i) Pārisuddhi
(a) The sick bhikkhu makes general
confession, then:
‘Pārisuddhiṃ dammi,
pārisuddhiṃ me hara,
pārisuddhiṃ me ārocehi.’
(‘I give my purity. Please convey purity for me
(and) declare purity for me.’)

• If the sick bhikkhu is the junior:
‘hara’ →‘haratha’; ‘ārocehi’ →‘ārocetha’
[Vin,I,120]

(b) The sick bhikkhu's (e.g. ‘Uttaro's’)
purity is conveyed after the Pātimokkha:
‘Āyasmā bhante ‘uttaro’ bhikkhu gilāno,
parisuddho'ti paṭijāni,
parisuddho'ti taṃ saṅgho dhāretu.’
(‘Ven. sirs, ‘Uttaro Bhikkhu’ who is sick acknowedges
that he is pure. May the Saṅgha hold him to be pure.’)

• If the bhikkhu conveying purity is senior to

the sick bhikkhu:
‘Āyasmā bhante ‘uttaro’’
→‘‘Uttaro’ bhante bhikkhu’
[Thai; cf. Vin,I,121]

(ii) Sending Consent (Chanda)
(a) The sick bhikkhu sends his consent to
the saṅghākamma:
‘Chandaṃ dammi,
chandaṃ me hara,
chandaṃ me ārocehi.’
(‘I offer my consent. May you convey my consent (to
the Saṅgha). May you declare my consent to them.’)

• If the sick bhikkhu is the junior:
‘hara’ →‘haratha’; ‘ārocehi’ →‘ārocetha’
[Vin,I,121]

(b) Informing the Sangha of the sick
bhikkhu's consent:
‘Ā yasmā bhante ‘uttaro’
mayhaṃ chandaṃ adāsi,
tassa chando mayā āhaṭo,
sādhu bhante saṅgho dhāretu.’
(‘Ven. sirs, ‘Uttaro Bhikkhu’ has given his consent to
me. I have conveyed his consent. It is well, ven. sirs, if
the Saṅgha holds it to be so.’)

• If the bhikkhu conveying consent is senior to

the sick bhikkhu:
‘Āyasmā bhante ‘uttaro’’
→‘‘Uttaro’ bhante bhikkhu’
[Thai; cf. Vin,I,122]

(iii) Pārisuddhi + Chanda
When both purity and consent are conveyed to
the Sangha:

‘‘Uttaro’ bhante bhikkhu gilāno mayhaṃ
chandañca pārisuddhiñca adāsi,
tassa chando ca pārisuddhi ca mayā āhaṭā,
sādhu bhante saṅgho dhāretu.’
(‘Ven. sirs, ‘Uttaro Bhikkhu’ is sick. He has given his
consent and purity to me. I have conveyed his consent
and purity. It is well, ven. sirs, if the Sangha holds it
[cf.Vin,I,122]
to be so.’)

\section{Uposatha-day for Sāmaṇeras and Lay-followers}

(i) Ten Precepts for Sāmaṇeras (Thai tradition)
‘Pāṇātipātā veramaṇī.
Adinnādānā veramaṇī.
Abrahma-cariyā veramaṇī.
Musā-vādā veramaṇī.
Surā-meraya-majja-pamādaṭṭhānā veramaṇī.
Vikāla-bhojanā veramaṇī.
Nacca-gīta vādita visūka-dassanā veramaṇī.
Mālā-gandha vilepana dhāraṇa maṇḍana
vibhūsanaṭṭhānā veramaṇī.
Uccā-sayana mahā-sayanā veramaṇī.
Jātarūpa-rajata paṭiggahaṇā veramaṇī.’
(‘I undertake the precept to refrain from:
—destroying living beings.
—taking that which is not given.
—any kind of intentional sexual behaviour.
—false speech.

—intoxicating drinks and drugs that lead to
carelessness.
—eating at wrong times.
—dancing, singing, music and going to
entertainments.
—perfumes, beautification and adornment.
—lying on a high or luxurious sleeping place.
—accepting gold or silver.’)

After the tenth precept, the bhikkhu:
‘Imāni dasa sikkhā-padāni samādiyāmi.’
The sāmaṇera repeats this three times.
[cf. Vin,I,83–84]

(ii) Eight Precepts (Thai Tradition)
After bowing three times, with hands in
añjali, the laypeople recite the following
request:
‘Ma yaṃ bhante ti-saraṇena saha
aṭṭha sīlāni yācāma,
Dutiyam-pi mayaṃ bhante…
Tatiyam-pi mayaṃ bhante…’
(‘We, ven. sir, request the 3 Refuges and the 5
Precepts. For the second time… For the third time…’)

• As an individual, or one on behalf of a group:

‘Mayaṃ’ → ‘Ahaṃ’; ‘yācāma’ → ‘yācāmi’
Bhikkhu:
‘Namo…’ (×3)
Laypeople repeat.
Bhk: ‘Buddhaṃ saraṇaṃ gacchāmi.

Dhammaṃ saraṇaṃ gacchāmi.
Saṅghaṃ saraṇaṃ gacchāmi.
Dutiyam-pi Buddhaṃ saraṇaṃ gacchāmi.
Dutiyam-pi Dhammaṃ saraṇaṃ gacchāmi.
Dutiyam-pi Saṅghaṃ saraṇaṃ gacchāmi.
Tatiyam-pi Buddhaṃ saraṇaṃ gacchāmi.
Tatiyam-pi Dhammaṃ saraṇaṃ gacchāmi.
Tatiyam-pi Saṅghaṃ saraṇaṃ gacchāmi.’
Laypeople repeat line by line.
Bhk: ‘Ti-saraṇa-gamanaṃ niṭṭhitaṃ.’
(‘This completes the going to the 3 Refuges.’)

Laypeople:

‘Āma bhante.’
(‘Yes, ven. sir.’)

Then the bhikkhu recites, with the laypeople
repeating line by line:
‘Pāṇātipātā veramaṇī sikkhā-padaṃ samādiyāmi.
Adinnādānā veramaṇī sikkhā-padaṃ samādiyāmi.
Abrahma-cariyā veramaṇī sikkhā-padaṃ
samādiyāmi.
Musāvādā veramaṇī sikkhā-padaṃ samādiyāmi.
Surā-meraya-majja-pamādaṭṭhānā veramaṇī
sikkhā-padaṃ samādiyāmi.
Vikāla-bhojanā veramaṇī sikkhā-padaṃ
samādiyāmi.
Nacca-gīta vādita visūka-dassana mālāgandha vilepana dhāraṇa maṇḍana
vibhūsanaṭṭhānā veramaṇī sikkhā-padaṃ
samādiyāmi.

Uccā-sayana mahā-sayanā veramaṇī
sikkhā-padaṃ samādiyāmi.’
[cf.A,IV,248–250]
(‘I undertake the precept to refrain from:
—destroying living beings.
—taking that which is not given.
—any kind of intentional sexual behaviour.
—false speech.
—intoxicating drinks and drugs that lead to
carelessness.
—eating at wrong times.
—dancing, singing, music and going to
entertainments.
—perfumes, beautification and adornment.
—lying on a high or luxurious sleeping place.
—accepting gold or silver.’)

Bhk: ‘Imāni aṭṭha sikkhā-padāni samādiyāmi’
Laypeople:
‘Imāni aṭṭha sikkhā-padāni samādiyāmi’ (×3)
(‘I undertake the Eight Precepts.’)

The bhikkhu then chants:
‘Imāni aṭṭha sikkhā-padāni
Sīlena sugatiṃ yanti,
Sīlena bhoga-sampadā,
Sīlena nibbutiṃ yanti,
Tasmā sīlaṃ visodhaye.’
(‘These Eight Precepts
Have morality as a vehicle for happiness,
Have morality as a vehicle for good fortune,
Have morality as a vehicle for liberation,
Let morality therefore be purified.’)

The Laypeople may respond with:
‘Sādhu, sādhu, sādhu!’
• Alternatively, the laypeople may chant:
‘Imaṃ aṭṭh'aṅga-samannāgataṃ buddhapaññattaṃ uposathaṃ, imañ-ca rattiṃ imañca divasaṃ, samma-deva abhirakkhituṃ
samādiyāmi.’
Bhk:
‘Imāni aṭṭha sikkhā-padāni,
ajj'ekaṃ rattin-divaṃ,
uposatha (sīla) vasena sādhukaṃ (katvā
appamādena) rakkhitabbāni.’
Laypeople:
‘Āma bhante.’
Bhk:
‘Sīlena sugatiṃ yanti,
Sīlena bhoga-sampadā,
Sīlena nibbutiṃ yanti,
Tasmā sīlaṃ visodhaye.’
• Having undertaken the Eight Precepts, layfollowers may stay overnight. The next morning they will take their leave from the bhikkhu,
who responds:
‘Yassa dāni tumhe kālaṃ maññatha.’
(‘Please do what is appropriate at this time.’)

♦ See below page 47 for the Five Precepts.

(iii) Eight Precepts (Sri Lankan Tradition)
With hands in añjali, the laypeople recite the
following request:
‘Sādhu! Sādhu! Sādhu!

Okāsa ahaṃ bhante ti-saraṇena saddhiṃ
aṭṭh'aṅga sīlaṃ dhammaṃ yācāmi,
anuggahaṃ katvā sīlaṃ detha me bhante.
Dutiyam-pi okāsa… detha me bhante.
Tatiyam-pi okāsa… detha me bhante.’
Bhk:
‘Yaṃ ahaṃ vadāmi taṃ vadetha.’
Laypeople: ‘Āma, bhante.’
Bhk:
‘Namo…’ (×3)
Laypeople repeat.
Bhk: ‘Buddhaṃ saraṇaṃ gacchāmi.
Dhammaṃ saraṇaṃ gacchāmi.
Saṅghaṃ saraṇaṃ gacchāmi.
Dutiyam-pi Buddhaṃ saraṇaṃ gacchāmi.
Dutiyam-pi Dhammaṃ saraṇaṃ gacchāmi.
Dutiyam-pi Saṅghaṃ saraṇaṃ gacchāmi.
Tatiyam-pi Buddhaṃ saraṇaṃ gacchāmi.
Tatiyam-pi Dhammaṃ saraṇaṃ gacchāmi.
Tatiyam-pi Saṅghaṃ saraṇaṃ gacchāmi.’
Laypeople repeat line by line.
Bhk:
‘Saraṇagamanaṃ sampuṇṇaṃ.’
Laypeople: ‘Āma, bhante.’
Then the bhikkhu recites, with the laypeople
repeating line by line:
‘Pāṇātipātā veramaṇī sikkhā-padaṃ samādiyāmi.
Adinnādānā veramaṇī sikkhā-padaṃ samādiyāmi.
Abrahma-cariyā veramaṇī sikkhā-padaṃ
samādiyāmi.
Musāvādā veramaṇī sikkhā-padaṃ samādiyāmi.

Surā-meraya-majja-pamādaṭṭhānā veramaṇī
sikkhā-padaṃ samādiyāmi.
Vikāla-bhojanā veramaṇī sikkhā-padaṃ
samādiyāmi.
Nacca-gīta vādita visūka-dassana mālāgandha vilepana dhāraṇa maṇḍana
vibhūsanaṭṭhānā veramaṇī sikkhā-padaṃ
samādiyāmi.
Uccā-sayana mahā-sayanā veramaṇī
[cf. A,IV,248–250]
sikkhā-padaṃ samādiyāmi.’
(Translation see previous section.)

Bhk: ‘Imaṃ aṭṭh'aṅga-sīlaṃ samādiyāmi.’
Laypeople: ‘Imaṃ aṭṭh'aṅga-sīlaṃ
samādiyāmi.’ (×3)
Bhk: ‘Ti-saraṇena saddhiṃ aṭṭh'aṅga-sīlaṃ
dhammaṃ sādhukaṃ surakkhitaṃ katvā
appamādena sampādetha.’
Laypeople: ‘Āma, bhante.’
Bhk:

‘Sīlena sugatiṃ yanti,
Sīlena bhoga-sampadā,
Sīlena nibbutiṃ yanti,
Tasmā sīlaṃ visodhaye.’
(Translation see previous section.)

♦ See below page 49 for the Five Precepts.

