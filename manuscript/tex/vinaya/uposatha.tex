\chapter{Uposatha}

\section{Pārisuddhi-uposatha (Purity Uposatha)}

\subsection{Pārisuddhi Before Sangha}

Declaring one's purity before the Sangha:

‘Parisuddho ahaṃ bhante, parisuddho'ti maṃ saṅgho dhāretu.’

‘\emph{I, ven. sirs, am quite pure (of offences). May the Saṅgha hold me to be pure.}’

\suttaRef{Vin.I.120–129}

\subsection{Pārisuddhi for Three Bhikkhus}

The Pātimokkha requires at least four bhikkhus. If there are only three bhikkhus then, after the preliminary duties and the general confession, one bhikkhu
chants the \emph{ñatti}:

‘Suṇantu me bhante āyasmantā ajj'uposatho paṇṇaraso, yad'āyasmantānaṃ
pattakallaṃ, mayaṃ aññamaññaṃ pārisuddhi uposathaṃ kareyyāma.’

‘\emph{Let the ven. ones listen to me. Today is an Observance day, which is a
  fifteenth (day of the fortnight). If it seems right to the ven. ones let
  us carry out the Observance with one another by way of entire purity.}’

When it is the 14th day:\\
‘paṇṇaraso’ → ‘cātuddaso’

If the announcing bhikkhu is the most senior:\\
‘bhante’ → ‘āvuso’

Then, starting with the senior bhikkhu:

‘Parisuddho ahaṃ āvuso, parisuddho'ti maṃ dhāretha.’ (×3)\\
‘\emph{I, friends, am quite pure. Understand that I am quite pure.}’

For each of the two junior bhikkhus:\\
‘āvuso’ → ‘bhante’

\subsection{Pārisuddhi for Two Bhikkhus}

Omit the \emph{ñatti}. The senior bhikkhu declares purity first:

‘Parisuddho ahaṃ āvuso, parisuddho'ti maṃ dhārehi.’ (×3)

For the junior:\\
‘āvuso’ → ‘bhante’\\
‘dhārehi’ → ‘dhāretha’

\subsection{Adhiṭṭhānuposatha (For a lone bhikkhu)}

For a bhikkhu staying alone on the Uposatha day. After the preliminary duties,
he then determines:

‘Ajja me uposatho.’\\
‘\emph{Today is an Observance day for me.}’

\section{Sick Bhikkhus}

\subsection{Pārisuddhi}

\textbf{(a)} The sick bhikkhu makes general confession, then:

‘Pārisuddhiṃ dammi, pārisuddhiṃ me hara, pārisuddhiṃ me ārocehi.’

‘\emph{I give my purity. Please convey purity for me (and) declare purity for me.}’

If the sick bhikkhu is the junior:\\
‘hara’ → ‘haratha’\\
‘ārocehi’ → ‘ārocetha’

\clearpage

\textbf{(b)} The sick bhikkhu's (e.g. Uttaro's) purity is conveyed after the
Pātimokkha:

‘Āyasmā bhante ‘uttaro’ bhikkhu gilāno, parisuddho'ti paṭijāni, parisuddho'ti taṃ saṅgho dhāretu.’\\
‘\emph{Ven. sirs, ‘Uttaro Bhikkhu’ who is sick acknowedges that he is pure. May
  the Saṅgha hold him to be pure.}’

If the bhikkhu conveying purity is senior to the sick bhikkhu:

‘Āyasmā bhante \emph{uttaro}’ → ‘\emph{Uttaro} bhante bhikkhu’

\subsection{Sending Consent (Chanda)}

\textbf{(a)} The sick bhikkhu sends his consent to the \emph{saṅghakamma}:

‘Chandaṃ dammi, chandaṃ me hara, chandaṃ me ārocehi.’\\
‘\emph{I offer my consent. May you convey my consent (to the Saṅgha). May you
  declare my consent to them.}’

If the sick bhikkhu is the junior:\\
‘hara’ → ‘haratha’\\
‘ārocehi’ → ‘ārocetha’

\textbf{(b)} Informing the Sangha of the sick bhikkhu's consent:

‘Āyasmā bhante ‘uttaro’ mayhaṃ chandaṃ adāsi, tassa chando mayā āhaṭo, sādhu bhante saṅgho dhāretu.’\\
‘\emph{Ven. sirs, ‘Uttaro Bhikkhu’ has given his consent to me. I have conveyed
  his consent. It is well, ven. sirs, if the Saṅgha holds it to be so.}’

If the bhikkhu conveying consent is senior to the sick bhikkhu:

‘Āyasmā bhante \emph{uttaro}’ → ‘\emph{Uttaro} bhante bhikkhu’

\subsection{Pārisuddhi + Chanda}

When both purity and consent are conveyed to the Sangha:

‘\emph{Uttaro} bhante bhikkhu gilāno mayhaṃ chandañca pārisuddhiñca adāsi, tassa
chando ca pārisuddhi ca mayā āhaṭā, sādhu bhante saṅgho dhāretu.’

‘\emph{Ven. sirs, ‘Uttaro Bhikkhu’ is sick. He has given his consent and purity
  to me. I have conveyed his consent and purity. It is well, ven. sirs, if the
  Sangha holds it to be so.}’

