\chapter{Uposatha}

\section{Pārisuddhi-uposatha (Purity Uposatha)}

\subsection{Pārisuddhi Before Sangha}

Declaring one's purity before the Sangha:

‘Parisuddho ahaṃ bhante, parisuddho'ti maṃ saṅgho dhāretu.’\\
‘\emph{I, ven. sirs, am quite pure May the Saṅgha hold me to be pure.}’

\suttaRef{cf. Vin.I.129}

\subsection{Pārisuddhi for 3 Bhikkhus}

The Pātimokkha requires at least four bhikkhus. If there a re only thre e bhikk
hus then, after the preliminary duties and the general confession, one bhikkhu
chants the \emph{ñatti}:

‘Suṇantu me bhante āyasmantā ajj'uposatho paṇṇaraso, yad'āyasmantānaṃ
pattakallaṃ, mayaṃ aññamaññaṃ pārisuddhi uposathaṃ kareyyāma.’

‘\emph{Let the ven. ones listen to me. Today is an Uposatha day, which is a
  fifteenth (day of the fortnight) one. If it seems right to the ven. ones let
  us carry out the Observance with one another by way of entire purity.}’

When it is the 14th day:\\
‘paṇṇaraso’ → ‘cātuddaso’

If the announcing bhikkhu is the most senior:\\
‘bhante’ → ‘āvuso’ \suttaRef{Vin.I.124}

Then, starting with the senior bhikkhu:

‘Parisuddho ahaṃ āvuso, parisuddho'ti maṃ dhāretha.’ (×3)\\
‘\emph{I, friends, am quite pure. Understand that I am quite pure.}’

For each of the two junior bhikkhus:\\
‘āvuso’ → ‘bhante’ \suttaRef{Vin.I.124}

\subsection{Pārisuddhi for 2 Bhikkhus}

Omit the \emph{ñatti}. The senior bhikkhu declares purity first:

‘Parisuddho ahaṃ āvuso, parisuddho'ti maṃ dhārehi.’ (×3)

For the junior:

‘āvuso’ → ‘bhante’\\
‘dhārehi’ → ‘dhāretha’

\suttaRef{Vin.I.124–125}

\subsection{Adhiṭṭhānuposatha (For a lone bhikkhu)}

For a bhikkhu staying alone on the Uposatha day. After the Preliminary duties,
he then determines:

‘Ajja me uposatho.’\\
‘\emph{Today is an Uposatha day for me.}’ \suttaRef{Vin.I.125}

\section{Sick Bhikkhus}

\subsection{Pārisuddhi}

\textbf{(a)} The sick bhikkhu makes general confession, then:

‘Pārisuddhiṃ dammi, pārisuddhiṃ me hara, pārisuddhiṃ me ārocehi.’\\
‘\emph{I give my purity. Please convey purity for me (and) declare purity for me.}’

If the sick bhikkhu is the junior:

‘hara’ → ‘haratha’ ; ‘ārocehi’ → ‘ārocetha’

\suttaRef{Vin.I.120}

\textbf{(b)} The sick bhikkhu's (e.g. ‘Uttaro's’) purity is conveyed after the
Pātimokkha:

‘Āyasmā bhante ‘uttaro’ bhikkhu gilāno, parisuddho'ti paṭijāni, parisuddho'ti taṃ saṅgho dhāretu.’\\
‘\emph{Ven. sirs, ‘Uttaro Bhikkhu’ who is sick acknowedges that he is pure. May
  the Saṅgha hold him to be pure.}’

If the bhikkhu conveying purity is senior to the sick bhikkhu:

‘Āyasmā bhante \emph{uttaro}’ → ‘\emph{Uttaro} bhante bhikkhu’

\suttaRef{Thai; cf. Vin,I,121}

\subsection{Sending Consent (Chanda)}

\textbf{(a)} The sick bhikkhu sends his consent to the \emph{saṅghākamma}:

‘Chandaṃ dammi, chandaṃ me hara, chandaṃ me ārocehi.’\\
‘\emph{I offer my consent. May you convey my consent (to the Saṅgha). May you
  declare my consent to them.}’

If the sick bhikkhu is the junior:

‘hara’ → ‘haratha’ ; ‘ārocehi’ → ‘ārocetha’

\suttaRef{Vin.I.121}

\textbf{(b)} Informing the Sangha of the sick bhikkhu's consent:

‘Āyasmā bhante ‘uttaro’ mayhaṃ chandaṃ adāsi, tassa chando mayā āhaṭo, sādhu bhante saṅgho dhāretu.’\\
‘\emph{Ven. sirs, ‘Uttaro Bhikkhu’ has given his consent to me. I have conveyed
  his consent. It is well, ven. sirs, if the Saṅgha holds it to be so.}’

If the bhikkhu conveying consent is senior to the sick bhikkhu:

‘Āyasmā bhante \emph{uttaro}’ → ‘\emph{Uttaro} bhante bhikkhu’

\suttaRef{Thai; cf. Vin.I.122}

\subsection{Pārisuddhi + Chanda}

When both purity and consent are conveyed to the Sangha:

‘\emph{Uttaro} bhante bhikkhu gilāno mayhaṃ chandañca pārisuddhiñca adāsi, tassa
chando ca pārisuddhi ca mayā āhaṭā, sādhu bhante saṅgho dhāretu.’

‘\emph{Ven. sirs, ‘Uttaro Bhikkhu’ is sick. He has given his consent and purity
  to me. I have conveyed his consent and purity. It is well, ven. sirs, if the
  Sangha holds it to be so.}’

\suttaRef{cf. Vin.I.122}

\section{Uposatha-day for Sāmaṇeras and Lay-followers}

\subsection{Ten Precepts for Sāmaṇeras (Thai tradition)}

‘Pāṇātipātā veramaṇī.\\
Adinnādānā veramaṇī.\\
Abrahma-cariyā veramaṇī.\\
Musā-vādā veramaṇī.\\
Surā-meraya-majja-pamādaṭṭhānā veramaṇī.\\
Vikāla-bhojanā veramaṇī.\\
Nacca-gīta vādita visūka-dassanā veramaṇī.\\
Mālā-gandha vilepana dhāraṇa maṇḍana vibhūsanaṭṭhānā veramaṇī.\\
Uccā-sayana mahā-sayanā veramaṇī.\\
Jātarūpa-rajata paṭiggahaṇā veramaṇī.’

{\itshape

‘I undertake the precept to refrain from:

\begin{packeditemize}

\item destroying living beings.
\item taking that which is not given.
\item any kind of intentional sexual behaviour.
\item false speech.
\item intoxicating drinks and drugs that lead to carelessness.
\item eating at wrong times.
\item dancing, singing, music and going to entertainments.
\item perfumes, beautification and adornment.
\item lying on a high or luxurious sleeping place.
\item accepting gold or silver.’

\end{packeditemize}

}

After the tenth precept, the bhikkhu:

‘Imāni dasa sikkhā-padāni samādiyāmi.’

The sāmaṇera repeats this three times.

\suttaRef{cf. Vin.I.83–84}

\subsection{Eight Precepts (Thai Tradition)}

After bowing three times, with hands in \emph{añjali}, the laypeople recite the
following request:

‘\emph{Mayaṃ} bhante ti-saraṇena saha aṭṭha sīlāni \emph{yācāma}.\\
Dutiyam-pi mayaṃ bhante…\\
Tatiyam-pi mayaṃ bhante…’

‘\emph{We, ven. sir, request the 3 Refuges and the 5 Precepts.\\
  For the second time… For the third time…}’

As an individual, or one on behalf of a group:

‘Mayaṃ’ → ‘Ahaṃ’ ; ‘yācāma’ → ‘yācāmi’

\emph{Bhikkhu}: ‘Namo…’ (×3)\\
\emph{Laypeople}: repeat.

\emph{Bhk}: ‘Buddhaṃ saraṇaṃ gacchāmi.\\
Dhammaṃ saraṇaṃ gacchāmi.\\
Saṅghaṃ saraṇaṃ gacchāmi.\\
Dutiyam-pi Buddhaṃ saraṇaṃ gacchāmi.\\
Dutiyam-pi Dhammaṃ saraṇaṃ gacchāmi.\\
Dutiyam-pi Saṅghaṃ saraṇaṃ gacchāmi.\\
Tatiyam-pi Buddhaṃ saraṇaṃ gacchāmi.\\
Tatiyam-pi Dhammaṃ saraṇaṃ gacchāmi.\\
Tatiyam-pi Saṅghaṃ saraṇaṃ gacchāmi.’

\emph{Laypeople}: repeat line by line.

\emph{Bhk}: ‘Ti-saraṇa-gamanaṃ niṭṭhitaṃ.’\\
‘\emph{This completes the going to the 3 Refuges.}’

\emph{Laypeople}: ‘Āma bhante.’\\
‘\emph{Yes, ven. sir.}’

Then the bhikkhu recites, with the laypeople repeating line by line:

‘Pāṇātipātā veramaṇī sikkhā-padaṃ samādiyāmi.\\
Adinnādānā veramaṇī sikkhā-padaṃ samādiyāmi.\\
Abrahma-cariyā veramaṇī sikkhā-padaṃ samādiyāmi.\\
Musāvādā veramaṇī sikkhā-padaṃ samādiyāmi.\\
Surā-meraya-majja-pamādaṭṭhānā veramaṇī sikkhā-padaṃ samādiyāmi.\\
Vikāla-bhojanā veramaṇī sikkhā-padaṃ samādiyāmi.\\
Nacca-gīta vādita visūka-dassana mālāgandha vilepana dhāraṇa maṇḍana
vibhūsanaṭṭhānā veramaṇī sikkhā-padaṃ samādiyāmi.\\
Uccā-sayana mahā-sayanā veramaṇī sikkhā-padaṃ samādiyāmi.’

\suttaRef{cf. A.IV.248–250}

{\itshape

‘I undertake the precept to refrain from:

\begin{packeditemize}

\item destroying living beings.
\item taking that which is not given.
\item any kind of intentional sexual behaviour.
\item false speech.
\item intoxicating drinks and drugs that lead to carelessness.
\item eating at wrong times.
\item dancing, singing, music and going to entertainments.
\item perfumes, beautification and adornment.
\item lying on a high or luxurious sleeping place.
\item accepting gold or silver.’

\end{packeditemize}

}

\emph{Bhk}: ‘Imāni aṭṭha sikkhā-padāni samādiyāmi’

\emph{Laypeople}: ‘Imāni aṭṭha sikkhā-padāni samādiyāmi’ (×3)\\
‘\emph{I undertake the Eight Precepts.}’

The bhikkhu then chants:

‘Imāni aṭṭha sikkhā-padāni\\
Sīlena sugatiṃ yanti,\\
Sīlena bhoga-sampadā,\\
Sīlena nibbutiṃ yanti,\\
Tasmā sīlaṃ visodhaye.’

‘\emph{These Eight Precepts\\
Have morality as a vehicle for happiness,\\
Have morality as a vehicle for good fortune,\\
Have morality as a vehicle for liberation,\\
Let morality therefore be purified.}’

The Laypeople may respond with:

‘Sādhu, sādhu, sādhu!’

Alternatively, the laypeople may chant:

‘Imaṃ aṭṭh'aṅga-samannāgataṃ buddhapaññattaṃ uposathaṃ, imañ-ca rattiṃ imañca
divasaṃ, samma-deva abhirakkhituṃ samādiyāmi.’

\emph{Bhk}: ‘Imāni aṭṭha sikkhā-padāni, ajj'ekaṃ rattin-divaṃ, uposatha (sīla)
vasena sādhukaṃ (katvā appamādena) rakkhitabbāni.’

\emph{Laypeople}: ‘Āma bhante.’

\emph{Bhk}:

‘Sīlena sugatiṃ yanti,\\
Sīlena bhoga-sampadā,\\
Sīlena nibbutiṃ yanti,\\
Tasmā sīlaṃ visodhaye.’

Having undertaken the Eight Precepts, layfollowers may stay overnight. The next
morning they will take their leave from the bhikkhu, who responds:

‘Yassa dāni tumhe kālaṃ maññatha.’\\
‘\emph{Please do what is appropriate at this time.}’

% See below page 47 for the Five Precepts.

\subsection{Eight Precepts (Sri Lankan Tradition)}

With hands in \emph{añjali}, the laypeople recite the following request:

‘Sādhu! Sādhu! Sādhu! Okāsa ahaṃ bhante ti-saraṇena saddhiṃ aṭṭh'aṅga sīlaṃ
dhammaṃ yācāmi, anuggahaṃ katvā sīlaṃ detha me bhante. Dutiyam-pi okāsa… detha
me bhante. Tatiyam-pi okāsa… detha me bhante.’

\emph{Bhk}: ‘Yaṃ ahaṃ vadāmi taṃ vadetha.’

\emph{Laypeople}: ‘Āma, bhante.’

\emph{Bhk}: ‘Namo…’ (×3)

\emph{Laypeople}: repeat.

\emph{Bhk}:

‘Buddhaṃ saraṇaṃ gacchāmi.\\
Dhammaṃ saraṇaṃ gacchāmi.\\
Saṅghaṃ saraṇaṃ gacchāmi.\\
Dutiyam-pi Buddhaṃ saraṇaṃ gacchāmi.\\
Dutiyam-pi Dhammaṃ saraṇaṃ gacchāmi.\\
Dutiyam-pi Saṅghaṃ saraṇaṃ gacchāmi.\\
Tatiyam-pi Buddhaṃ saraṇaṃ gacchāmi.\\
Tatiyam-pi Dhammaṃ saraṇaṃ gacchāmi.\\
Tatiyam-pi Saṅghaṃ saraṇaṃ gacchāmi.’

\emph{Laypeople}: repeat line by line.

\emph{Bhk}: ‘Saraṇagamanaṃ sampuṇṇaṃ.’

\emph{Laypeople}: ‘Āma, bhante.’

Then the bhikkhu recites, with the laypeople repeating line by line:

‘Pāṇātipātā veramaṇī sikkhā-padaṃ samādiyāmi.\\
Adinnādānā veramaṇī sikkhā-padaṃ samādiyāmi.\\
Abrahma-cariyā veramaṇī sikkhā-padaṃ samādiyāmi.\\
Musāvādā veramaṇī sikkhā-padaṃ samādiyāmi.\\
Surā-meraya-majja-pamādaṭṭhānā veramaṇī sikkhā-padaṃ samādiyāmi.\\
Vikāla-bhojanā veramaṇī sikkhā-padaṃ samādiyāmi.\\
Nacca-gīta vādita visūka-dassana mālāgandha vilepana dhāraṇa maṇḍana
vibhūsanaṭṭhānā veramaṇī sikkhā-padaṃ samādiyāmi.\\
Uccā-sayana mahā-sayanā veramaṇī sikkhā-padaṃ samādiyāmi.’

\suttaRef{cf. A.IV.248–250}

(Translation see previous section.)

\emph{Bhk}: ‘Imaṃ aṭṭh'aṅga-sīlaṃ samādiyāmi.’

\emph{Laypeople}: ‘Imaṃ aṭṭh'aṅga-sīlaṃ samādiyāmi.’ (×3)

\emph{Bhk}: ‘Ti-saraṇena saddhiṃ aṭṭh'aṅga-sīlaṃ dhammaṃ sādhukaṃ surakkhitaṃ
katvā appamādena sampādetha.’

\emph{Laypeople}: ‘Āma, bhante.’

\emph{Bhk}:

‘Sīlena sugatiṃ yanti,\\
Sīlena bhoga-sampadā,\\
Sīlena nibbutiṃ yanti,\\
Tasmā sīlaṃ visodhaye.’

(Translation see previous section.)

% See below page 49 for the Five Precepts.

