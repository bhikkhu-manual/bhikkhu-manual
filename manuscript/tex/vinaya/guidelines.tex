\chapter{Guidelines}

\section[Establishing of the Pāṭimokkha]{The Ten Reasons for the Establishing of the Pāṭimokkha}

\begin{packedenumerate}

\item `For the excellence of the Sangha;
\item for the wellbeing of the Sangha;
\item for the control of ill-controlled bhikkhus;
\item for the comfort of wellbehaved bhikkhus;
\item for the restraint of the \emph{āsavā} in this present state;
\item for protection against the \emph{āsavā} in a future state;
\item to give confidence to those of little faith;
\item to increase the confidence of the faithful;
\item to establish the True Dhamma;
\item to support the Vinaya.'

\end{packedenumerate}

\suttaRef{Vin.III.20; A.V.70}

\section{The Four Great Standards (Mahāpadesa)}

`Whatever things are not prohibited as unallowable but agree with things that
are unallowable, being opposed to things that are allowable — such things are
unsuitable.

`Whatever things are not prohibited as unallowable but agree with things that
are allowable, being opposed to things that are unallowable — such things are
suitable.

`Whatever things are not permitted as allowable but agree with things that are
unallowable, being opposed to things that are allowable — such things are
unsuitable.

`Whatever things are not permitted as allowable but agree with things that are
allowable, being opposed to things that are unallowable — such things are
suitable.'

\suttaRef{Vin.I.250}

\ifhandbookedition
\vspace*{-\baselineskip}
\fi

\section{Upholding the Principles}

`If there is some obstacle to [the practice of the training rules], due to time
and place, the rules should be upheld indirectly and not given up entirely, for
otherwise there will be no principles (for discipline). A community without
principles for discipline cannot last long\ldots{}'

\suttaRef{Entrance to the Vinaya, I.230}

