\chapter{Evening Chanting}

\section*{Dedication of Offerings}

\firstline{Yo so bhagavā arahaṁ sammāsambuddho}

[Yo so] bhagavā arahaṁ sammāsambuddho\\
Svākkhāto yena bhagavatā dhammo\\
Supaṭipanno yassa bhagavato sāvakasaṅgho\\
Tam-mayaṁ bhagavantaṁ sadhammaṁ sasaṅghaṁ\\
Imehi sakkārehi yathārahaṁ āropitehi abhipūjayāma\\
Sādhu no bhante bhagavā sucira-parinibbutopi\\
Pacchimā-janatānukampa-mānasā\\
Ime sakkāre duggata-paṇṇākāra-bhūte paṭiggaṇhātu\\
Amhākaṁ dīgharattaṁ hitāya sukhāya\\
Arahaṁ sammāsambuddho bhagavā\\
Buddhaṁ bhagavantaṁ abhivādemi

[Svākkhāto] bhagavatā dhammo\\
Dhammaṁ namassāmi

[Supaṭipanno] bhagavato sāvakasaṅgho\\
Saṅghaṁ namāmi

\section*{Dedication of Offerings (English)}

[To the Blessed One,] the Lord, who fully attained\\
\vin perfect enlightenment,\\
To the Teaching, which he expounded so well,\\
And to the Blessed One's disciples who have practised well,\\
To these --- the Buddha, the Dhamma, and the Saṅgha ---\\
We render with offerings our rightful homage.\\
It is well for us that the Blessed One, having attained liberation,\\
Still had compassion for later generations.\\
May these simple offerings be accepted\\
For our long-lasting benefit and for the happiness it gives us.\\
The Lord, the Perfectly Enlightened and Blessed One ---\\
I render homage to the Buddha, the Blessed One.

[The Teaching,] so completely explained by him ---\\
I bow to the Dhamma.

[The Blessed One's disciples,] who have practised well ---\\
I bow to the Saṅgha.

\section*{Preliminary Homage}

\begin{leader}
  [Handa mayaṁ buddhassa bhagavato pubbabhāga-namakāraṁ karomase]
\end{leader}

Namo tassa bhagavato arahato sammāsambuddhassa (×3)

\section*{Preliminary Homage (English)}

\begin{leader}
  [Now let us pay preliminary homage to the Buddha.]
\end{leader}

Homage to the Blessed, Noble, and Perfectly Enlightened One. (×3)

\clearpage

\section*{Recollection of the Buddha}

\begin{leader}
  [Handa mayaṁ buddhānussatinayaṁ karomase]
\end{leader}

Taṁ kho pana bhagavantaṁ evaṁ kalyāṇo\\
\vin kittisaddo abbhuggato\\
Itipi so bhagavā arahaṁ sammāsambuddho\\
Vijjācaraṇa-sampanno sugato lokavidū\\
Anuttaro purisadamma-sārathi satthā deva-manussānaṁ\\
\vin buddho bhagavā'ti

\section*{Recollection of the Buddha (English)}

\begin{leader}
  [Now let us chant the recollection of the Buddha.]
\end{leader}

A good word of the Blessed One's reputation has spread as follows:\\
He, the Blessed One, is indeed the Pure One,\\
\vin the Perfectly Enlightened One;\\
He is impeccable in conduct and understanding,\\
\vin the Accomplished One, the Knower of the Worlds;\\
He trains perfectly those who wish to be trained;\\
\vin he is Teacher of gods and humans; he is Awake and Holy.

\section*{Supreme Praise of the Buddha}

\begin{leader}
  [Handa mayaṁ buddhābhigītiṁ karomase]
\end{leader}

Buddh'vārahanta-varatādiguṇābhiyutto\\
Suddhābhiñāṇa-karuṇāhi samāgatatto\\
Bodhesi yo sujanataṁ kamalaṁ va sūro\\
Vandām'ahaṁ tam-araṇaṁ sirasā jinendaṁ\\
Buddho yo sabba-pāṇīnaṁ saraṇaṁ khemam-uttamaṁ\\
Paṭhamānussatiṭṭhānaṁ vandāmi taṁ siren'ahaṁ\\
Buddhassāh'asmi dāso/dāsī va buddho me sāmi-kissaro\\
Buddho dukkhassa ghātā ca vidhātā ca hitassa me\\
Buddhass'āhaṁ niyyādemi sarīrañ-jīvitañ-cidaṁ\\
Vandanto'haṁ/Vandantī'haṁ carissāmi\\
\vin buddhass'eva subodhitaṁ\\
Natthi me saraṇaṁ aññaṁ buddho me saraṇaṁ varaṁ\\
Etena sacca-vajjena vaḍḍheyyaṁ satthu-sāsane\\
Buddhaṁ me vandamānena/vandamānāya\\
\vin yaṁ puññaṁ pasutaṁ idha\\
Sabbepi antarāyā me māhesuṁ tassa tejasā

\instr{(Bowing)}

Kāyena vācāya va cetasā vā\\
Buddhe kukammaṁ pakataṁ mayā yaṁ\\
Buddho paṭiggaṇhātu accayantaṁ\\
Kālantare saṁvarituṁ va buddhe

\section*{Supreme Praise of the Buddha (English)}

\begin{leader}
  [Now let us chant the supreme praise of the Buddha.]
\end{leader}

The Buddha, the truly worthy one, endowed with\\
\vin such excellent qualities,\\
Whose being is composed of purity, transcendental wisdom,\\
\vin and compassion,\\
Who has enlightened the wise like the sun awakening the lotus ---\\
I bow my head to that peaceful chief of conquerors.\\
The Buddha, who is the safe, secure refuge of all beings ---\\
As the First Object of Recollection,\\\vin I venerate him with bowed head.\\
I am indeed the Buddha's servant,\\\vin the Buddha is my Lord and Guide.\\
The Buddha is sorrow's destroyer, who bestows blessings on me.\\
To the Buddha I dedicate this body and life,\\
And in devotion I will walk the Buddha's Path of Awakening.\\
For me there is no other refuge, the Buddha is my excellent refuge.\\
By the utterance of this Truth, may I grow in the Master's Way.\\
By my devotion to the Buddha, and the blessing of this practice ---\\
By its power, may all obstacles be overcome.

\instr{(Bowing)}

By body, speech, or mind,\\
For whatever wrong action I have committed towards the Buddha,\\
May my acknowledgement of fault be accepted,\\
That in future there may be restraint regarding the Buddha.

\section*{Recollection of the Dhamma}

\begin{leader}
  [Handa mayaṁ dhammānussatinayaṁ karomase]
\end{leader}

Svākkhāto bhagavatā dhammo\\
Sandiṭṭhiko akāliko ehipassiko\\
Opanayiko paccattaṁ veditabbo viññūhī'ti

\clearpage

\section*{Recollection of the Dhamma (English)}

\begin{leader}
  [Now let us chant the recollection of the Dhamma.]
\end{leader}

The Dhamma is well expounded by the Blessed One,\\
Apparent here and now, timeless, encouraging investigation,\\
Leading inwards, to be experienced individually by the wise.

\section*{Supreme Praise of the Dhamma}

\begin{leader}
  [Handa mayaṁ dhammābhigītiṁ karomase]
\end{leader}

Svākkhātat'ādiguṇa-yoga-vasena seyyo\\
Yo magga-pāka-pariyatti-vimokkha-bhedo\\
Dhammo kuloka-patanā tada-dhāri-dhārī\\
Vandām'ahaṁ tama-haraṁ vara-dhammam-etaṁ\\
Dhammo yo sabba-pāṇīnaṁ saraṇaṁ khemam-uttamaṁ\\
Dutiyānussatiṭṭhānaṁ vandāmi taṁ siren'ahaṁ\\
Dhammassāh'asmi dāso/dāsī va dhammo me sāmi-kissaro\\
Dhammo dukkhassa ghātā ca vidhātā ca hitassa me\\
Dhammass'āhaṁ niyyādemi sarīrañ-jīvitañ-cidaṁ\\
Vandantohaṁ/Vandantīhaṁ carissāmi\\
\vin dhammass'eva sudhammataṁ\\
Natthi me saraṇaṁ aññaṁ dhammo me saraṇaṁ varaṁ\\
Etena sacca-vajjena vaḍḍheyyaṁ satthu-sāsane\\
Dhammaṁ me vandamānena/vandamānāya\\
\vin yaṁ puññaṁ pasutaṁ idha\\
Sabbepi antarāyā me māhesuṁ tassa tejasā

\clearpage

\instr{(Bowing)}

Kāyena vācāya va cetasā vā\\
Dhamme kukammaṁ pakataṁ mayā yaṁ\\
Dhammo paṭiggaṇhātu accayantaṁ\\
Kālantare saṁvarituṁ va dhamme

\section*{Supreme Praise of the Dhamma (English)}

\begin{leader}
  [Now let us chant the supreme praise of the Dhamma.]
\end{leader}

It is excellent because it is `well expounded,'\\
And it can be divided into Path and Fruit, Learning and Liberation.\\
The Dhamma holds those who uphold it from falling into delusion.\\
I revere the excellent Teaching, that which removes darkness ---\\
The Dhamma, which is the supreme, secure refuge of all beings ---\\
As the Second Object of Recollection,\\\vin I venerate it with bowed head.\\
I am indeed the Dhamma's servant,\\\vin the Dhamma is my Lord and Guide.\\
The Dhamma is sorrow's destroyer, and it bestows blessings on me.\\
To the Dhamma I dedicate this body and life,\\
And in devotion I will walk this excellent way of Truth.\\
For me there is no other refuge,\\\vin the Dhamma is my excellent refuge.\\
By the utterance of this Truth, may I grow in the Master's Way.\\
By my devotion to the Dhamma, and the blessing of this practice ---\\
By its power, may all obstacles be overcome.

\clearpage

\instr{(Bowing)}

By body, speech, or mind,\\
For whatever wrong action I have committed\\\vin towards the Dhamma,\\
May my acknowledgement of fault be accepted,\\
That in future there may be restraint regarding the Dhamma.

\section*{Recollection of the Saṅgha}

\begin{leader}
  [Handa mayaṁ saṅghānussatinayaṁ karomase]
\end{leader}

Supaṭipanno bhagavato sāvakasaṅgho\\
Ujupaṭipanno bhagavato sāvakasaṅgho\\
Ñāyapaṭipanno bhagavato sāvakasaṅgho\\
Sāmīcipaṭipanno bhagavato sāvakasaṅgho\\
Yadidaṁ cattāri purisayugāni aṭṭha purisapuggalā\\
Esa bhagavato sāvakasaṅgho\\
Āhuneyyo pāhuneyyo dakkhiṇeyyo añjali-karaṇīyo\\
Anuttaraṁ puññakkhettaṁ lokassā'ti

\section*{Recollection of the Saṅgha (English)}

\begin{leader}
  [Now let us chant the recollection of the Saṅgha.]
\end{leader}

They are the Blessed One's disciples, who have practised well,\\
Who have practised directly,\\
Who have practised insightfully,\\
Those who practise with integrity ---\\
That is the four pairs, the eight kinds of noble beings ---\\
These are the Blessed One's disciples.\\
Such ones are worthy of gifts, worthy of hospitality,\\
\vin worthy of offerings, worthy of respect;\\
They give occasion for incomparable goodness\\\vin to arise in the world.

\section*{Supreme Praise of the Saṅgha}

\begin{leader}
  [Handa mayaṁ saṅghābhigītiṁ karomase]
\end{leader}

Saddhammajo supaṭipatti-guṇādiyutto\\
Yo'ṭṭhabbidho ariyapuggala-saṅgha-seṭṭho\\
Sīlādidhamma-pavarāsaya-kāya-citto\\
Vandām'ahaṁ tam-ariyāna-gaṇaṁ susuddhaṁ\\
Saṅgho yo sabba-pāṇīnaṁ saraṇaṁ khemam-uttamaṁ\\
Tatiyānussatiṭṭhānaṁ vandāmi taṁ siren'ahaṁ\\
Saṅghass'āhasmi dāso/dāsī va saṅgho me sāmi-kissaro\\
Saṅgho dukkhassa ghātā ca vidhātā ca hitassa me\\
Saṅghass'āhaṁ niyyādemi sarīrañ-jīvitañ-cidaṁ\\
Vandanto'haṁ/Vandantī'haṁ carissāmi\\
\vin saṅghassopaṭipannataṁ\\
Natthi me saraṇaṁ aññaṁ saṅgho me saraṇaṁ varaṁ\\
Etena sacca-vajjena vaḍḍheyyaṁ satthu-sāsane\\
Saṅghaṁ me vandamānena/vandamānāya\\
\vin yaṁ puññaṁ pasutaṁ idha\\
Sabbepi antarāyā me māhesuṁ tassa tejasā

\instr{(Bowing)}

Kāyena vācāya va cetasā vā\\
Saṅghe kukammaṁ pakataṁ mayā yaṁ\\
Saṅgho paṭiggaṇhātu accayantaṁ\\
Kālantare saṁvarituṁ va saṅghe

\section*{Supreme Praise of the Saṅgha (English)}

\begin{leader}
  [Now let us chant the supreme praise of the Saṅgha.]
\end{leader}

Born of the Dhamma, that Saṅgha which has practised well,\\
The field of the Saṅgha formed of eight kinds of noble beings,\\
Guided in body and mind by excellent morality and virtue.\\
I revere that assembly of noble beings perfected in purity.\\
The Saṅgha, which is the supreme, secure refuge of all beings ---\\
As the Third Object of Recollection, I venerate it with bowed head.

I am indeed the Saṅgha's servant, the Saṅgha is my Lord and Guide.\\
The Saṅgha is sorrow's destroyer and it bestows blessings on me.\\
To the Saṅgha I dedicate this body and life,\\
And in devotion I will walk the well-practised way of the Saṅgha.\\
For me there is no other refuge, the Saṅgha is my excellent refuge.\\
By the utterance of this Truth, may I grow in the Master's Way.\\
By my devotion to the Saṅgha, and the blessing of this practice ---\\
By its power, may all obstacles be overcome.

\instr{(Bowing)}

By body, speech, or mind,\\
For whatever wrong action I have committed towards the Saṅgha,\\
May my acknowledgement of fault be accepted,\\
That in future there may be restraint regarding the Saṅgha.

\section*{Closing Homage}

[Arahaṁ] sammāsambuddho bhagavā\\
Buddhaṁ bhagavantaṁ abhivādemi

[Svākkhāto] bhagavatā dhammo\\
Dhammaṁ namassāmi

[Supaṭipanno] bhagavato sāvakasaṅgho\\
Saṅghaṁ namāmi

\section*{Closing Homage (English)}

[The Lord,] the Perfectly Enlightened and Blessed One ---\\
I render homage to the Buddha, the Blessed One.

[The Teaching,] so completely explained by him ---\\
I bow to the Dhamma.

[The Blessed One's disciples,] who have practised well ---\\
I bow to the Saṅgha.

