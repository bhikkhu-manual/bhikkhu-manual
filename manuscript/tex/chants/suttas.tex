\chapter{Suttas}

\section{Dhammacakkappavattana Sutta}

\firstline{Anuttaraṃ abhisambodhiṃ sambujjhitvā tathāgato}

\begin{leader}
\soloinstr{Solo introduction}

\begin{solotwochants}
Anuttaraṃ abhisambodhiṃ & sambujjhitvā tathāgato\\
Pathamaṃ yaṃ adesesi & dhammacakkaṃ anuttaraṃ\\
Sammadeva pavattento & loke appativattiyaṃ\\
Yatthākkhātā ubho antā & paṭipatti ca majjhimā\\
Catūsvāriyasaccesu & visuddhaṃ ñāṇadassanaṃ\\
Desitaṃ dhammarājena & sammāsambodhikittanaṃ\\
Nāmena vissutaṃ suttaṃ & dhammacakkappavattanaṃ\\
Veyyākaraṇapāthena & saṅgītantam bhaṇāma se\\
\end{solotwochants}
\end{leader}

[Evaṃ me sutaṃ]

Ekaṃ samayaṃ bhagavā bārāṇasiyaṃ viharati isipatane migadāye. Tatra kho
bhagavā pañcavaggiye bhikkhū āmantesi:

Dve'me, bhikkhave, antā pabbajitena na sevitabbā: yo cāyaṃ kāmesu
kāma-sukh'allikānuyogo; hīno, gammo, pothujjaniko, anariyo,
anattha-sañhito; yo cāyaṃ atta-kilamathānuyogo; dukkho, anariyo,
anattha-sañhito.

Ete te, bhikkhave, ubho ante anupagamma majjhimā paṭipadā tathāgatena
abhisambuddhā cakkhukaraṇī, ñāṇakaraṇī, upasamāya, abhiññāya,
sambodhāya, nibbānāya saṃvattati.

Katamā ca sā, bhikkhave, majjhimā paṭipadā tathāgatena abhisambuddhā
cakkhukaraṇī ñāṇakaraṇī, upasamāya, abhiññāya, sambodhāya, nibbānāya
saṃvattati.

Ayam-eva ariyo aṭṭhaṅgiko maggo seyyathīdaṃ:

Sammā-diṭṭhi, sammā-saṅkappo, sammā-vācā, sammā-kammanto, sammā-ājīvo,
sammā-vāyāmo, sammā-sati, sammā-samādhi.

Ayaṃ kho sā, bhikkhave, majjhimā paṭipadā tathāgatena abhisambuddhā
cakkhukaraṇī ñāṇakaraṇī, upasamāya, abhiññāya, sambodhāya, nibbānāya
saṃvattati.

Idaṃ kho pana, bhikkhave, dukkhaṃ ariya-saccaṃ:

Jātipi dukkhā, jarāpi dukkhā, maranampi dukkhaṃ,
soka-parideva-dukkha-domanass'upāyāsāpi dukkhā, appiyehi sampayogo
dukkho, piyehi vippayogo dukkho, yamp'icchaṃ na labhati tampi dukkhaṃ,
saṅkhittena pañcupādānakkhandā dukkhā.

Idaṃ kho pana, bhikkhave, dukkha-samudayo ariya-saccaṃ:

Yā'yaṃ taṇhā ponobbhavikā nandi-rāga-sahagatā tatra-tatrābhinandinī
seyyathīdaṃ: kāma-taṇhā, bhava-taṇhā, vibhava-taṇhā.

Idaṃ kho pana, bhikkhave, dukkha-nirodho ariya-saccaṃ:

Yo tassā yeva taṇhāya asesa-virāga-nirodho, cāgo, paṭinissaggo, mutti,
anālayo.

Idaṃ kho pana, bhikkhave, dukkha-nirodha-gāminī paṭipadā ariya-saccaṃ:

Ayam-eva ariyo aṭṭhaṅgiko maggo seyyathīdam: sammā-diṭṭhi,
sammā-saṅkappo, sammā-vācā, sammā-kammanto, sammā-ājīvo, sammā-vāyāmo,
sammā-sati, sammā-samādhi.

[Idaṃ dukkhaṃ] ariya-saccan'ti me bhikkhave, pubbe ananussutesu dhammesu
cakkhuṃ udapādi, ñāṇaṃ udapādi, paññā udapādi, vijjā udapādi, āloko
udapādi.

Taṃ kho pan'idaṃ dukkhaṃ ariya-saccaṃ pariññeyyan'ti me bhikkhave, pubbe
ananussutesu dhammesu cakkhuṃ udapādi, ñāṇaṃ udapādi, paññā udapādi,
vijjā udapādi, āloko udapādi.

Taṃ kho pan'idaṃ dukkhaṃ ariya-saccaṃ pariññātan'ti me bhikkhave, pubbe
ananussutesu dhammesu cakkhuṃ udapādi, ñāṇaṃ udapādi, paññā udapādi,
vijjā udapādi, āloko udapādi.

Idaṃ dukkha-samudayo ariya-saccan'ti me bhikkhave, pubbe ananussutesu
dhammesu cakkhuṃ udapādi, ñāṇaṃ udapādi, paññā udapādi, vijjā udapādi,
āloko udapādi.

Taṃ kho pan'idaṃ dukkhasamudayo ariyasaccaṃ pahātabban'ti me bhikkhave,
pubbe ananussutesu dhammesu cakkhuṃ udapādi, ñāṇaṃ udapādi, paññā
udapādi, vijjā udapādi, āloko udapādi.

Taṃ kho pan'idaṃ dukkha-samudayo ariya-saccaṃ pahīnan'ti me bhikkhave, pubbe
ananussutesu dhammesu cakkhuṃ udapādi, ñāṇaṃ udapādi, paññā udapādi,
vijjā udapādi, āloko udapādi.

Idaṃ dukkha-nirodho ariya-saccan'ti me bhikkhave, pubbe ananussutesu
dhammesu cakkhuṃ udapādi, ñāṇaṃ udapādi, paññā udapādi, vijjā udapādi,
āloko udapādi.

Taṃ kho pan'idaṃ dukkha-nirodho ariya-saccaṃ sacchikātabban'ti me bhikkhave,
pubbe ananussutesu dhammesu cakkhuṃ udapādi, ñāṇaṃ udapādi, paññā
udapādi, vijjā, udapādi āloko udapādi.

Taṃ kho pan'idaṃ dukkha-nirodho ariya-saccaṃ sacchikatan'ti me bhikkhave,
pubbe ananussutesu dhammesu cakkhuṃ udapādi, ñāṇaṃ udapādi, paññā
udapādi, vijjā udapādi, āloko udapādi.

Idaṃ dukkha-nirodha-gāminī paṭipadā ariya-saccan'ti me bhikkhave, pubbe
ananussutesu dhammesu cakkhuṃ udapādi, ñāṇaṃ udapādi, paññā udapādi,
vijjā udapādi, āloko udapādi.

Taṃ kho pan'idaṃ dukkha-nirodha-gāminī paṭipadā ariya-saccaṃ bhāvetabban'ti
me bhikkhave, pubbe ananussutesu dhammesu cakkhuṃ udapādi, ñāṇaṃ
udapādi, paññā udapādi, vijjā udapādi, āloko udapādi.

Taṃ kho pan'idaṃ dukkha-nirodha-gāminī paṭipadā ariya-saccaṃ bhāvitan'ti me
bhikkhave, pubbe ananussutesu dhammesu cakkhuṃ udapādi, ñāṇaṃ udapādi,
paññā udapādi, vijjā udapādi, āloko udapādi.

[Yāva kīvañca me bhikkhave,] imesu catūsu ariya-saccesu evan-ti-parivaṭṭaṃ
dvādas'ākāraṃ yathā-bhūtaṃ ñāṇa-dassanaṃ na suvisuddhaṃ ahosi, n'eva tāv'āhaṃ
bhikkhave, sadevake loke samārake sabrahmake sassamaṇa-brāhmaṇiyā pajāya
sadeva-manussāya anuttaraṃ sammā-sambodhiṃ abhisambuddho paccaññāsiṃ.

Yato ca kho me bhikkhave, imesu catūsu ariya-saccesu evan-ti-parivaṭṭaṃ
dvādas'ākāraṃ yathā-bhūtaṃ ñāṇa-dassanaṃ suvisuddham ahosi, ath'āham
bhikkhave, sadevake loke samārake sabrahmake sassamaṇa-brāhmaṇiyā pajāya
sadeva-manussāya anuttaraṃ sammā-sambodhiṃ abhisambuddho paccaññāsiṃ.

Ñāṇañca pana me dassanaṃ udapādi, akuppā me vimutti ayam-antimā jāti,
natthi dāni punabbhavo'ti.

Idam-avoca bhagavā. Attamanā pañcavaggiyā bhikkhū bhagavato bhāsitaṃ
abhinanduṃ.

Imasmiñca pana veyyākaraṇasmiṃ bhaññamāne āyasmato koṇḍaññassa virajaṃ
vītamalaṃ dhammacakkhuṃ udapādi: yaṃ kiñci samudaya-dhammaṃ sabban-taṃ
nirodha-dhamman'ti.

[Pavattite ca bhagavatā] dhammacakke bhummā devā saddamanussāvesuṃ:

Etaṃ bhagavatā bārāṇasiyaṃ isipatane migadāye anuttaraṃ dhammacakkaṃ
pavattitaṃ appaṭivattiyaṃ samaṇena vā brāhmaṇena vā devena vā mārena vā
brahmunā vā kenaci vā lokasmin'ti.

Bhummānaṃ devānaṃ saddaṃ sutvā, cātummahārājikā devā
saddamanussāvesuṃ\ldots

Cātummahārājikānaṃ devānaṃ saddaṃ sutvā, tāvatiṃsā devā
saddamanussāvesuṃ\ldots

Tāvatiṃsānaṃ devānaṃ saddaṃ sutvā, yāmā devā saddamanussāvesuṃ\ldots

Yāmānaṃ devānaṃ saddaṃ sutvā, tusitā devā saddamanussāvesuṃ\ldots

Tusitānaṃ devānaṃ saddaṃ sutvā, nimmānaratī devā saddamanussāvesum\ldots

Nimmānaratīnaṃ devānaṃ saddaṃ sutvā, paranimmitavasavattī devā
saddamanussāvesuṃ\ldots

Paranimmitavasavattīnaṃ devānaṃ saddaṃ sutvā, brahmakāyikā devā
saddamanussāvesuṃ:

Etaṃ bhagavatā bārāṇasiyaṃ isipatane migadāye anuttaraṃ dhammacakkaṃ
pavattitaṃ appaṭivattiyaṃ samaṇena vā brāhmaṇena vā devena vā mārena vā
brahmunā vā kenaci vā lokasmin'ti.

Iti'ha tena khaṇena, tena muhuttena, yāva brahmalokā saddo abbhuggacchi.
Ayañca dasa-sahassī lokadhātu saṅkampi sampakampi sampavedhi, appamāṇo ca
oḷāro obhāso loke pāturahosi atikkammeva devānaṃ devānubhāvaṃ.

Atha kho bhagavā udānaṃ udānesi:

Aññāsi vata bho koṇḍañño, aññāsi vata bho koṇḍañño ti. Iti hidaṃ āyasmato
koṇḍaññassa aññā-koṇḍañño tveva nāmaṃ ahosī ti.

Dhammacakkappavattana-suttaṃ niṭṭhitaṃ.

\suttaRef{SN 56.11; Vin.I.10f}

\clearpage

\section{Anatta-lakkhaṇa Sutta}

\firstline{Yantaṃ sattehi dukkhena ñeyyaṃ anattalakkhaṇaṃ}

\begin{leader}
\soloinstr{Solo introduction}

{\setlength{\tabcolsep}{0.9em}
\begin{solotwochants}
Yantaṃ sattehi dukkhena & ñeyyaṃ anattalakkhaṇaṃ\\
Attavādattasaññāṇaṃ  & sammadeva vimocanaṃ\\
Sambuddho taṃ pakāsesi & diṭṭhasaccāna yoginaṃ\\
Uttariṃ paṭivedhāya & bhāvetuṃ ñāṇamuttamaṃ\\
Yantesaṃ diṭṭhadhammānam & ñāṇenupaparikkhataṃ\\
Sabbāsavehi cittāni & vimucciṃsu asesato\\
Tathā ñāṇānussārena & sāsanaṃ kātumicchataṃ\\
Sādhūnaṃ atthasiddhatthaṃ & taṃ suttantaṃ bhaṇāma se\\
\end{solotwochants}
}
\end{leader}

[Evaṃ me sutaṃ]

Ekaṃ samayaṃ bhagavā bārāṇasiyaṃ viharati isipatane migadāye. Tatra kho
bhagavā pañcavaggiye bhikkhū āmantesi:

Rūpaṃ bhikkhave anattā, rūpañca hidaṃ bhikkhave attā abhavissa, nayidaṃ
rūpaṃ ābādhāya saṃvatteyya, labbhetha ca rūpe, evaṃ me rūpaṃ hotu, evaṃ
me rūpaṃ mā ahosī ti.

Yasmā ca kho bhikkhave rūpaṃ anattā, tasmā rūpaṃ ābādhāya saṃvattati, na ca
labbhati rūpe, evaṃ me rūpaṃ hotu, evaṃ me rūpaṃ mā ahosī ti.

Vedanā anattā, vedanā ca hidaṃ bhikkhave attā abhavissa, nayidaṃ vedanā
ābādhāya saṃvatteyya, labbhetha ca vedanāya, evaṃ me vedanā hotu, evaṃ
me vedanā mā ahosī ti.

Yasmā ca kho bhikkhave vedanā anattā, tasmā vedanā ābādhāya saṃvattati, na ca
labbhati vedanāya, evaṃ me vedanā hotu, evaṃ me vedanā mā ahosī ti.

Saññā anattā, saññā ca hidaṃ bhikkhave attā abhavissa, nayidaṃ saññā
ābādhāya saṃvatteyya, labbhetha ca saññāya, evaṃ me saññā hotu, evaṃ me
saññā mā ahosī ti.

Yasmā ca kho bhikkhave saññā anattā, tasmā, saññā ābādhāya saṃvattati,
na ca labbhati saññāya, evaṃ me saññā hotu, evaṃ me saññā mā ahosī ti.

Saṅkhārā anattā, saṅkhārā ca hidaṃ bhikkhave attā abhavissaṃsu, nayidaṃ
saṅkhārā ābādhāya saṃvatteyyuṃ, labbhetha ca saṅkhāresu, evaṃ me
saṅkhārā hontu, evaṃ me saṅkhārā mā ahesun ti.

Yasmā ca kho bhikkhave saṅkhārā anattā, tasmā saṅkhārā ābādhāya
saṃvattanti, na ca labbhati saṅkhāresu, evaṃ me saṅkhārā hontu, evaṃ me
saṅkhārā mā ahesun ti.

Viññāṇaṃ anattā, viññāṇañca hidaṃ bhikkhave attā abhavissa, nayidaṃ
viññānam ābādhāya saṃvatteyya, labbhetha ca viññāne evaṃ me viññāṇaṃ
hotu, evaṃ me viññāṇaṃ mā ahosī ti.

Yasmā ca kho bhikkhave viññāṇaṃ anattā, tasmā viññāṇaṃ ābādhāya
saṃvattati, na ca labbhati viññāne, evaṃ me viññāṇaṃ hotu, evaṃ me
viññāṇaṃ mā ahosī ti.

[Taṃ kiṃ maññatha bhikkhave,] rūpam niccaṃ vā aniccaṃ vā ti.

Aniccaṃ bhante.

Yam panāniccaṃ, dukkhaṃ vā taṃ sukhaṃ vā ti.

Dukkhaṃ bhante.

Yam panāniccaṃ dukkhaṃ viparināma-dhammaṃ, kallaṃ nu taṃ samanupassituṃ,
etaṃ mama, esoham'asmi, eso me attā ti.

No hetaṃ bhante.

Taṃ kiṃ maññatha bhikkhave, vedanā niccā vā aniccā vā ti.

Aniccā bhante.

Yam panāniccaṃ, dukkhaṃ vā taṃ sukhaṃ vā ti.

Dukkhaṃ bhante.

Yam panāniccaṃ dukkhaṃ viparināma-dhammaṃ, kallaṃ nu taṃ samanupassituṃ,
etaṃ mama, esoham'asmi, eso me attā ti.

No hetaṃ bhante.

Taṃ kiṃ maññatha bhikkhave, saññā niccā vā aniccā vā ti.

Aniccā bhante.

Yam panāniccaṃ, dukkhaṃ vā taṃ sukhaṃ vā ti.

Dukkhaṃ bhante.

Yam panāniccaṃ dukkhaṃ viparināma-dhammaṃ, kallaṃ nu taṃ samanupassituṃ,
etaṃ mama, esoham'asmi, eso me attā ti.

No hetaṃ bhante.

Taṃ kiṃ maññatha bhikkhave, saṅkhārā niccā vā aniccā vā ti.

Aniccā bhante.

Yam panāniccaṃ, dukkhaṃ vā taṃ sukhaṃ vā ti.

Dukkhaṃ bhante.

Yam panāniccaṃ dukkhaṃ viparināma-dhammaṃ, kallaṃ nu taṃ samanupassituṃ,
etaṃ mama, esoham'asmi, eso me attā ti.

No hetaṃ bhante.

Taṃ kiṃ maññatha bhikkhave, viññāṇaṃ niccaṃ vā aniccaṃ vā ti.

Aniccaṃ bhante.

Yam panāniccaṃ, dukkhaṃ vā taṃ sukhaṃ vā ti.

Dukkhaṃ bhante.

Yam panāniccaṃ dukkhaṃ viparināma-dhammaṃ, kallaṃ nu taṃ samanupassituṃ
etaṃ mama, esoham'asmi, eso me attā ti.

No hetaṃ bhante.

[Tasmā tiha bhikkhave] yaṃ kiñci rūpaṃ atītānāgata-paccuppannaṃ ajjhattaṃ
vā bahiddhā vā oḷārikaṃ vā sukhumaṃ vā hīnaṃ vā paṇītaṃ vā yandūre
santike vā, sabbaṃ rūpaṃ netaṃ mama, nesoham'asmi, na me so attā ti,
evametaṃ yathābhūtaṃ sammappaññāya daṭṭhabbaṃ.

Yā kāci vedanā atītānāgata-paccuppannā ajjhattā vā bahiddhā vā oḷārikā
vā sukhumā vā hīnā vā paṇītā vā yā dūre santike vā, sabbā vedanā netaṃ
mama, nesoham'asmi, na me so attā ti, evametaṃ yathābhūtaṃ sammappaññāya
daṭṭhabbaṃ.

Yā kāci saññā atītānāgata-paccuppannā ajjhattā vā bahiddhā vā oḷārikā vā
sukhumā vā hīnā vā paṇītā vā yā dūre santike vā, sabbā saññā netaṃ mama,
nesoham'asmi, na me so attā ti, evametaṃ yathābhūtaṃ sammappaññāya
daṭṭhabbaṃ.

Ye keci saṅkhārā atītānāgata-paccuppannā ajjhattā vā bahiddhā vā oḷārikā
vā sukhumā vā hīnā vā paṇītā vā ye dūre santike vā, sabbe saṅkhārā netaṃ
mama, nesoham'asmi, na me so attā ti, evametaṃ yathābhūtaṃ sammappaññāya
daṭṭhabbaṃ.

Yaṃ kiñci viññāṇaṃ atītānāgata-paccuppannaṃ ajjhattaṃ vā bahiddhā vā
oḷārikaṃ vā sukhumaṃ vā hīnaṃ vā paṇītaṃ vā yandūre santike vā, sabbaṃ
viññāṇaṃ netaṃ mama, nesoham'asmi, na me so attā ti, evametaṃ yathābhūtaṃ
sammappaññāya daṭṭhabbaṃ.

[Evaṃ passaṃ bhikkhave] sutvā ariyasāvako rūpasmim pi nibbindati, vedanāya
pi nibbindati, saññāya pi nibbindati, saṅkhāresu pi nibbindati,
viññāṇasmim pi nibbindati, nibbindaṃ virajjati, virāgā vimuccati,
vimuttasmiṃ vimuttam iti ñāṇaṃ hoti, khīṇā jāti, vusitaṃ brahmacariyaṃ,
kataṃ karaṇīyaṃ, nāparaṃ itthattāyā ti pajānātī ti.

[Idam-avoca bhagavā.] Attamanā pañcavaggiyā bhikkhū bhagavato bhāsitaṃ
abhinanduṃ. Imasmiñca pana veyyākaraṇasmiṃ bhaññamāne pañcavaggiyānaṃ
bhikkhūnaṃ anupādāya āsavehi cittāni vimucciṃsū ti.

Anattalakkhaṇa-suttaṃ niṭṭhitaṃ.

\suttaRef{SN 22.59; Vin.I.13f}

\section{Āditta-pariyāya Sutta}

\firstline{Veneyyadamanopāye sabbaso pāramiṃ gato}

\begin{leader}
\soloinstr{Solo introduction}

\begin{solotwochants}
Veneyyadamanopāye  & sabbaso pāramiṃ gato\\
Amoghavacano buddho & abhiññāyānusāsako\\
Ciṇṇānurūpato cāpi & dhammena vinayaṃ pajaṃ\\
Ciṇṇāggipāricariyānaṃ & sambojjhārahayoginaṃ\\
Yamādittapariyāyaṃ & desayanto manoharaṃ\\
Te sotāro vimocesi & asekkhāya vimuttiyā\\
Tathevopaparikkhāya & viññūṇaṃ sotumicchataṃ\\
Dukkhatālakkhaṇopāyaṃ & taṃ suttantaṃ bhaṇāma se\\
\end{solotwochants}
\end{leader}

[Evaṃ me sutaṃ]

Ekaṃ samayaṃ bhagavā gayāyaṃ viharati gayāsīse saddhiṃ bhikkhu-sahassena.
Tatra kho bhagavā bhikkhū āmantesi:

Sabbaṃ bhikkhave ādittaṃ. Kiñca bhikkhave sabbaṃ ādittaṃ.

Cakkhuṃ bhikkhave ādittaṃ, rūpā ādittā, cakkhuviññāṇaṃ ādittaṃ,
cakkhusamphasso āditto, yampidaṃ cakkhusamphassapaccayā uppajjati
vedayitaṃ sukhaṃ vā dukkhaṃ vā adukkhamasukhaṃ vā tam pi ādittaṃ. Kena
ādittaṃ. Ādittaṃ rāgagginā dosagginā mohagginā, ādittaṃ jātiyā
jarāmaraṇena sokehi paridevehi dukkhehi domanassehi upāyāsehi ādittan'ti
vadāmi.

Sotaṃ ādittaṃ, saddā ādittā, sotaviññāṇaṃ ādittaṃ, sotasamphasso āditto,
yampidaṃ sotasamphassapaccayā uppajjati vedayitaṃ sukhaṃ vā dukkhaṃ vā
adukkhamasukhaṃ vā tam pi ādittaṃ. Kena ādittaṃ. Ādittaṃ rāgagginā
dosagginā mohagginā, ādittaṃ jātiyā jarāmaraṇena sokehi paridevehi
dukkhehi domanassehi upāyāsehi ādittan'ti vadāmi.

Ghānaṃ ādittaṃ, gandhā ādittā, ghānaviññāṇaṃ ādittaṃ, ghānasamphasso
āditto, yampidaṃ ghānasamphassapaccayā uppajjati vedayitaṃ sukhaṃ vā
dukkhaṃ vā adukkhamasukhaṃ vā tam pi ādittaṃ. Kena ādittaṃ. Ādittaṃ
rāgagginā dosagginā mohagginā, ādittaṃ jātiyā jarāmaraṇena sokehi
paridevehi dukkhehi domanassehi upāyāsehi ādittan'ti vadāmi.

Jivhā ādittā, rasā ādittā, jivhāviññāṇam ādittaṃ, jivhāsamphasso āditto,
yampidaṃ jivhāsamphassapaccayā uppajjati vedayitaṃ sukhaṃ vā dukkhaṃ vā
adukkhamasukhaṃ vā tam pi ādittaṃ. Kena ādittaṃ. Ādittaṃ rāgagginā
dosagginā mohagginā, ādittaṃ jātiyā jarāmaraṇena sokehi paridevehi
dukkhehi domanassehi upāyāsehi ādittan'ti vadāmi.

Kāyo āditto, phoṭṭhabbā ādittā, kāyaviññāṇaṃ ādittaṃ, kāyasamphasso
āditto, yampidaṃ kāyasamphassapaccayā uppajjati vedayitaṃ sukhaṃ vā
dukkhaṃ vā adukkhamasukhaṃ vā tam pi ādittaṃ. Kena ādittaṃ. Ādittaṃ
rāgagginā dosagginā mohagginā, ādittaṃ jātiyā jarāmaraṇena sokehi
paridevehi dukkhehi domanassehi upāyāsehi ādittan'ti vadāmi.

Mano āditto, dhammā ādittā, manoviññāṇaṃ ādittaṃ, manosamphasso āditto,
yampidaṃ manosamphassapaccayā uppajjati vedayitaṃ sukhaṃ vā dukkhaṃ vā
adukkhamasukhaṃ vā tam pi ādittaṃ. Kena ādittaṃ. Ādittaṃ rāgagginā
dosagginā mohagginā, ādittaṃ jātiyā jarāmaraṇena sokehi paridevehi
dukkhehi domanassehi upāyāsehi ādittan'ti vadāmi.

[Evaṃ passaṃ bhikkhave] sutvā ariyasāvako cakkhusmiṃ pi nibbindati,
rūpesu pi nibbindati, cakkhuviññāṇe pi nibbindati, cakkhusamphassepi
nibbindati, yampidaṃ cakkhusamphassapaccayā uppajjati vedayitaṃ sukhaṃ
vā dukkhaṃ vā adukkhamasukhaṃ vā tasmiṃ pi nibbindati.

Sotasmiṃ pi nibbindati, saddesu pi nibbindati, sotaviññāṇe pi
nibbindati, sotasamphassepi nibbindati, yampidaṃ sotasamphassapaccayā
uppajjati vedayitaṃ sukhaṃ vā dukkhaṃ vā adukkhamasukhaṃ vā tasmiṃ pi
nibbindati.

Ghānasmiṃ pi nibbindati, gandhesu pi nibbindati, ghānaviññāṇe pi
nibbindati, ghānasamphassepi nibbindati, yampidaṃ ghānasamphassapaccayā
uppajjati vedayitaṃ sukhaṃ vā dukkhaṃ vā adukkhamasukhaṃ vā tasmiṃ pi
nibbindati.

Jivhāya pi nibbindati, rasesu pi nibbindati, jivhāviññāṇe pi nibbindati,
jivhāsamphassepi nibbindati, yampidaṃ jivhāsamphassapaccayā uppajjati
vedayitaṃ sukhaṃ vā dukkhaṃ vā adukkhamasukhaṃ vā tasmiṃ pi nibbindati.

Kāyasmiṃ pi nibbindati, phoṭṭhabbesu pi nibbindati, kāyaviññāṇe pi
nibbindati, kāyasamphassepi nibbindati, yampidaṃ kāyasamphassapaccayā
uppajjati vedayitaṃ sukhaṃ vā dukkhaṃ vā adukkhamasukhaṃ vā tasmiṃ pi
nibbindati.

Manasmiṃ pi nibbindati, dhammesu pi nibbindati, manoviññāṇe pi
nibbindati, manosamphassepi nibbindati, yampidaṃ manosamphassapaccayā
uppajjati vedayitaṃ sukhaṃ vā dukkhaṃ vā adukkhamasukhaṃ vā tasmiṃ pi
nibbindati.

Nibbindaṃ virajjati, virāgā vimuccati, vimuttasmiṃ, vimuttam iti ñāṇaṃ
hoti, khīṇā jāti, vusitaṃ brahmacariyaṃ, kataṃ karaṇīyaṃ, nāparaṃ
itthattāyā ti pajānātī ti.

[Idam-avoca bhagavā.] Attamanā te bhikkhū bhagavato bhāsitaṃ abhinanduṃ.
Imasmiñca pana veyyākaraṇasmiṃ bhaññamāne tassa bhikkhu-sahassassa
anupādāya āsavehi cittāni vimucciṃsū ti.

Ādittapariyāya-suttaṃ niṭṭhitaṃ.

\suttaRef{SN 35.28; Vin.I.34}

\section{Dhaj'agga Sutta}

% TODO review text? compare with suttacentral

[Evam-me sutaṃ.] Ekaṃ samayaṃ Bhagavā, Sāvatthiyaṃ viharati, Jeta-vane
Anāthapiṇḍikassa ārāme. Tatra kho Bhagavā bhikkhū āmantesi: “bhikkhavo-ti”.
“Bhadante-ti,” te bhikkhū Bhagavato paccassosuṃ. Bhagavā etad avoca:

“Bhūta-pubbaṃ bhikkhave devāsura-saṅgāmo samupabbūḷho ahosi. Atha kho bhikkhave
Sakko devānamindo deve tāva-tiṃse āmantesi: ‘Sace mārisā devānaṃ saṅgāma-gatānaṃ
uppajjeyya bhayaṃ vā chambhitattaṃ vā lomahaṃso vā, mameva tasmiṃ samaye
dhaj’aggaṃ ullokeyyātha. Mamaṃ hi vo dhaj’aggaṃ ullokayataṃ yaṃ bhavissati
bhayaṃ vā chambhitattaṃ vā loma-haṃso vā, so pahīyissati.’

‘No ce me dhaj’aggaṃ ullokeyyātha, atha Pajāpatissa deva-rājassa dhaj’aggaṃ
ullokeyyātha. Pajāpatissa hi vo deva-rājassa dhaj’aggaṃ ullokayataṃ yaṃ
bhavissati bhayaṃ vā chambhitattaṃ vā loma-haṃso vā, so pahīyissati’.

‘No ce Pajāpatissa deva-rājassa dhaj’aggaṃ ullokeyyātha, atha Varuṇassa
deva-rājassa dhaj’aggaṃ ullokeyyātha. Varuṇassa hi vo deva-rājassa dha’jaggaṃ
ullokayataṃ yaṃ bhavissati bhayaṃ vā chambhitattaṃ vā lomahaṃso vā, so
pahīyissati’.

‘No ce Varuṇassa deva-rājassa dhaj’aggaṃ ullokeyyātha, atha Īsānassa
deva-rājassa dhaj’aggaṃ ullokeyyātha. Īsānassa hi vo devarājassa dhaj’aggaṃ
ullokayataṃ yaṃ bhavissati bhayaṃ vā chambhitattaṃ vā loma-haṃso vā, so
pahīyissatī-ti.’

“Taṃ kho pana bhikkhave Sakkassa vā devānam indassa dhaj’aggaṃ ullokayataṃ,
Pajāpatissa vā deva-rājassa dhaj’aggaṃ ullokayataṃ, Varuṇassa vā deva-rājassa
dhaj’aggaṃ ullokayataṃ, Īsānassa vā devarājassa dhaj’aggaṃ ullokayataṃ yaṃ
bhavissati bhayaṃ vā chambhitattaṃ vā loma-haṃso vā, so pahīyethāpi no’pi
pahīyetha.

“Taṃ kissa hetu? Sakko hi, bhikkhave, devānam indo avītarāgo avītadoso avītamoho
bhīru chambhī utrāsī palāyī-ti.

“Ahañ-ca kho, bhikkhave, evaṃ vadāmi: Sace tumhākaṃ, bhikkhave, arañña-gatānaṃ
vā rukkha-mūla-gatānaṃ vā suññāgāra-gatānaṃ vā uppajjeyya bhayaṃ vā
chambhitattaṃ vā loma-haṃso vā, mam eva tasmiṃ samaye anussareyyātha:

‘Iti pi so bhagavā arahaṃ sammā-sambuddho, vijjā-caraṇa-sampanno sugato
loka-vidū, anuttaro purisa-damma-sārathi satthā devamanussānaṃ Buddho
Bhagavā-ti. Mamaṃ hi vo bhikkhave anussarataṃ, yaṃ bhavissati bhayaṃ vā
chambhitattaṃ vā loma-haṃso vā, so pahīyissati.

“No ce maṃ anussareyyātha, atha dhammaṃ anussareyyātha:

‘Svākkhāto Bhagavatā dhammo, sandiṭṭhiko akāliko ehi-passiko, opanayiko
paccattaṃ veditabbo viññūhī-ti. Dhammaṃ hi vo bhikkhave anussarataṃ, yaṃ
bhavissati bhayaṃ vā chambhitattaṃ vā loma-haṃso vā, so pahīyissati.

“No ce dhammaṃ anussareyyātha, atha saṅghaṃ anussareyyātha:

‘Supaṭipanno Bhagavato sāvaka-saṅgho, uju-paṭipanno Bhagavato sāvaka-saṅgho,
ñāya-paṭipanno Bhagavato sāvaka-saṅgho, sāmīci-paṭipanno Bhagavato
sāvaka-saṅgho, yad-idaṃ cattāri purisa-yugāni aṭṭha purisapuggalā, esa Bhagavato
sāvaka-saṅgho, āhuneyyo pāhuneyyo dakkhiṇeyyo añjalikaraṇīyo, anuttaraṃ
puññakkhettaṃ lokassā-ti. Saṅghaṃ hi vo bhikkhave anussarataṃ yaṃ bhavissati
bhayaṃ vā chambhitattaṃ vā lomahaṃso vā, so pahīyissati.

“Taṃ kissa hetu? Tathāgato hi bhikkhave arahaṃ sammā-sambuddho, vītarāgo
vītadoso vītamoho, abhīru acchambhī anutrāsī apalāyīti.”

Idam avoca Bhagavā. Idaṃ vatvā sugato athāparaṃ etad avoca satthā:

“Araññe rukkha-mūle vā,\\
Suññ’āgāre va bhikkhavo;\\
Anussaretha Sambuddhaṃ,\\
Bhayaṃ tumhāka no siyā.\\
No ce Buddhaṃ sareyyātha,\\
Loka-jeṭṭhaṃ narāsabhaṃ;\\
Atha dhammaṃ sareyyātha,\\
Niyyānikaṃ sudesitaṃ.\\
No ce dhammaṃ sareyyātha,\\
Niyyānikaṃ sudesitaṃ;\\
Atha saṅghaṃ sareyyātha,\\
Puññakkhettaṃ anuttaraṃ.\\
Evaṃ-Buddhaṃ sarantānaṃ,\\
Dhammaṃ saṅghañ-ca bhikkhavo;\\
Bhayaṃ vā chambhitattaṃ vā,\\
Loma-haṃso na hessatī-ti.”

Dhaj’agga Suttaṃ Niṭṭhitaṃ.

\suttaRef{SN 11.3}

\section{Girimānanda-suttaṃ}

% TODO review text? compare with suttacentral

Evaṃ me sutaṃ: Ekaṃ samayaṃ Bhagavā Sāvatthiyaṃ viharati Jeta-vane
Anāthapiṇḍikassa ārāme. Tena kho pana samayena āyasmā Girimānando ābādhiko hoti
dukkhito bāḷha-gilāno. Atha kho āyasmā Ānando yena Bhagavā ten’upasaṅkami;
upasaṅkamitvā Bhagavantaṃ abhivādetvā ekam-antaṃ nisīdi. Ekam-antaṃ nisinno kho
āyasmā Ānando Bhagavantaṃ etad-avoca:

“Āyasmā, Bhante, Girimānando ābādhiko hoti dukkhito bāḷha-gilāno. Sādhu Bhante
Bhagavā yen’āyasmā Girimānando ten’upasaṅkamatu anukampaṃ upādāyā-ti.”

“Sace kho tvaṃ Ānanda Girimānandassa bhikkhuno dasa saññā bhāseyyāsi, ṭhānaṃ kho
pan’etaṃ vijjati yaṃ Girimānandassa bhikkhuno dasa saññā sutvā so ābādho ṭhānaso
paṭipassambheyya.

“Katamā dasa? Anicca-saññā, anatta-saññā, asubha-saññā, ādīnava-saññā,
pahāna-saññā, virāga-saññā, nirodha-saññā, sabba-loke anabhirata-saññā,
sabba-saṅkhāresu aniccāsaññā, ānāpānassati.

“Katamā c’Ānanda anicca-saññā? Idh’Ānanda, bhikkhu arañña-gato vā
rukkhamūla-gato vā suññāgāra-gato vā iti paṭisañcikkhati: ‘rūpaṃ aniccaṃ, vedanā
aniccā, saññā aniccā, saṅkhārā aniccā, viññāṇaṃ aniccan-ti. Iti imesu pañcasu
upādānakkhandhesu aniccānupassī viharati. Ayaṃ vuccat’Ānanda anicca-saññā.

“Katamā c’Ānanda anatta-saññā? Idh’Ānanda, bhikkhu arañña-gato vā
rukkhamūla-gato vā suññāgāra-gato vā iti paṭisañcikkhati: ‘cakkhuṃ anattā, rūpā
anattā, sotaṃ anattā, saddā anattā, ghānaṃ anattā, gandhā anattā, jivhā anattā,
rasā anattā, kāyo anattā, phoṭṭhabbā anattā, mano anattā, dhammā anattā-ti. Iti
imesu chasu ajjhattikabāhiresu āyatanesu anattānupassī viharati. Ayaṃ
vuccat’Ānanda, anatta-saññā.

“Katamā c’Ānanda, asubha-saññā? Idh’Ānanda, bhikkhu imam-eva kāyaṃ uddhaṃ
pāda-talā adho kesa-matthakā taca-pariyantaṃ pūraṃ nānāppakārassa asucino
paccavekkhati: ‘Atthi imasmiṃ kāye: kesā, lomā, nakhā, dantā, taco, maṃsaṃ,
nhāru, aṭṭhi, aṭṭhi-miñjaṃ, vakkaṃ, hadayaṃ, yakanaṃ, kilomakaṃ, pihakaṃ,
papphāsaṃ, antaṃ, anta-guṇaṃ, udariyaṃ, karīsaṃ, pittaṃ, semhaṃ, pubbo, lohitaṃ,
sedo, medo, assu, vasā, kheḷo, siṅghāṇikā, lasikā, muttan-ti.’ Iti imasmiṃ kāye
asubhānupassī viharati. Ayaṃ vuccat’Ānanda asubha-saññā.

“Katamā c’Ānanda ādīnava-saññā? Idh’Ānanda, bhikkhu arañña-gato vā
rukkhamūla-gato vā suññāgāra-gato vā iti paṭisañcikkhati: ‘Bahu-dukkho kho ayaṃ
kāyo bahu-ādīnavo. Iti imasmiṃ kāye vividhā ābādhā uppajjanti, seyyathīdaṃ:
cakkhu-rogo, sota-rogo, ghāna-rogo, jivhā-rogo, kāya-rogo, sīsa-rogo,
kaṇṇa-rogo, mukha-rogo, dantarogo, oṭṭha-rogo, kāso, sāso, pināso, ḍāho, jaro,
kucchi-rogo, mucchā, pakkhandikā, sūlā, visūcikā, kuṭṭhaṃ, gaṇḍo, kilāso, soso,
apamāro, daddu, kaṇḍu, kacchu, nakhasā, vitacchikā, lohitaṃ, pittaṃ, madhu-meho,
aṃsā, piḷakā, bhagandalā, pitta-samuṭṭhānā ābādhā, semha-samuṭṭhānā ābādhā,
vātasamuṭṭhānā ābādhā, sannipātikā ābādhā, utupariṇāma-jā ābādhā,
visama-parihāra-jā ābādhā, opakkamikā ābādhā, kamma-vipāka-jā ābādhā, sītaṃ,
uṇhaṃ, jighacchā, pipāsā, uccāro, passāvo-ti.’ Iti imasmiṃ kāye ādīnavānupassī
viharati. Ayaṃ vuccat’Ānanda ādīnava-saññā.

“Katamā c’Ānanda pahāna-saññā? Idh’Ānanda, bhikkhu uppannaṃ kāmavitakkaṃ
nādhivāseti, pajahati, vinodeti, byantīkaroti, anabhāvaṃ gameti. Uppannaṃ
byāpāda-vitakkaṃ nādhivāseti, pajahati, vinodeti, byantīkaroti, anabhāvaṃ
gameti. Uppannaṃ vihiṃsā-vitakkaṃ nādhivāseti, pajahati, vinodeti, byantīkaroti,
anabhāvaṃ gameti. Uppann’uppanne pāpake akusale dhamme nādhivāseti, pajahati,
vinodeti, byantīkaroti, anabhāvaṃ gameti. Ayaṃ vuccat’Ānanda pahāna-saññā.

“Katamā c’Ānanda, virāga-saññā? Idh’Ānanda, bhikkhu arañña-gato vā
rukkhamūla-gato vā suññāgāra-gato vā iti paṭisañcikkhati: ‘Etaṃ santaṃ, etaṃ
paṇītaṃ, yad-idaṃ sabba-saṅkhāra-samatho sabbūpadhippaṭinissaggo taṇhākkhayo
virāgo nibbānan-ti.’ Ayaṃ vuccat’Ānanda virāgasaññā.

“Katamā c’Ānanda, nirodha-saññā? Idh’Ānanda, bhikkhu arañña-gato vā
rukkhamūla-gato vā suññāgāra-gato vā iti paṭisañcikkhati: ‘Etaṃ santaṃ, etaṃ
paṇītaṃ, yad-idaṃ sabba-saṅkhāra-samatho sabbūpadhippaṭinissaggo taṇhākkhayo
nirodho nibbānan-ti.’ Ayaṃ vuccat’Ānanda nirodhasaññā.

“Katamā c’Ānanda, sabba-loke anabhiratasaññā? Idh’Ānanda, bhikkhu ye loke
upādānā cetaso adhiṭṭhānābhinivesānusayā, te pajahanto viharati anupādiyanto.
Ayaṃ vuccat’Ānanda sabba-loke anabhirata-saññā.

“Katamā c’Ānanda sabba-saṅkhāresu aniccāsaññā? Idh’Ānanda bhikkhu
sabba-saṅkhāresu aṭṭīyati, harāyati, jigucchati. Ayaṃ vuccat’ Ānanda,
sabba-saṅkhāresu aniccā-saññā.

“Katamā c’Ānanda ānāpānassati?
Idh’Ānanda, bhikkhu arañña-gato vā rukkhamūla-gato vā suññāgāra-gato vā nisīdati,
pallaṅkaṃ ābhujitvā, ujuṃ kāyaṃ paṇidhāya,
parimukhaṃ satiṃ upaṭṭhapetvā. So sato’va
assasati sato’va passasati.

Dīghaṃ vā assasanto: ‘Dīghaṃ assasāmī-ti’ pajānāti. Dīghaṃ vā passasanto:
‘Dīghaṃ passasāmī-ti’ pajānāti. Rassaṃ vā assasanto: ‘Rassaṃ assasāmī-ti’
pajānāti. Rassaṃ vā passasanto: ‘Rassaṃ passasāmī-ti’ pajānāti.
‘Sabba-kāyapaṭisaṃvedī assasissāmī-ti’ sikkhati. ‘Sabbakāya-paṭisaṃvedī
passasissāmī-ti’ sikkhati. ‘Passambhayaṃ kāya-saṅkhāraṃ assasissāmī-ti’
sikkhati. ‘Passambhayaṃ kāya-saṅkhāraṃ passasissāmī-ti’ sikkhati.

‘Pīti-paṭisaṃvedī assasissāmī-ti’ sikkhati. ‘Pīti-paṭisaṃvedī passasissāmī-ti’
sikkhati. ‘Sukha-paṭisaṃvedī assasissāmī-ti’ sikkhati. ‘Sukha-paṭisaṃvedī
passasissāmī-ti’ sikkhati. ‘Citta-saṅkhāra-paṭisaṃvedī assasissāmī-ti’ sikkhati.
‘Citta-saṅkhāra-paṭisaṃvedī passasissāmī-ti’ sikkhati. ‘Passambhayaṃ
cittasaṅkhāraṃ assasissāmī-ti’ sikkhati. ‘Passambhayaṃ citta-saṅkhāraṃ
passasissāmīti’ sikkhati.

‘Citta-paṭisaṃvedī assasissāmī-ti’ sikkhati. ‘Citta-paṭisaṃvedī passasissāmī-ti’
sikkhati. ‘Abhippamodayaṃ cittaṃ assasissāmī-ti’ sikkhati. ‘Abhippamodayaṃ
cittaṃ passasissāmī-ti’ sikkhati. ‘Samādahaṃ cittaṃ assasissāmī-ti’ sikkhati.
‘Samādahaṃ cittaṃ passasissāmī-ti’ sikkhati. ‘Vimocayaṃ cittaṃ assasissāmī-ti’
sikkhati. ‘Vimocayaṃ cittaṃ passasissāmī-ti’ sikkhati.

‘Aniccānupassī assasissāmī-ti’ sikkhati. ‘Aniccānupassī passasissāmī-ti’
sikkhati. ‘Virāgānupassī assasissāmī-ti’ sikkhati. ‘Virāgānupassī
passasissāmī-ti’ sikkhati. ‘Nirodhānupassī assasissāmī-ti’ sikkhati.
‘Nirodhānupassī passasissāmī-ti’ sikkhati. ‘Paṭinissaggānupassī assasissāmī-ti’
sikkhati. ‘Paṭinissaggānupassī passasissāmī-ti’ sikkhati. Ayaṃ vuccat’ Ānanda,
ānāpānassati.

“Sace kho tvaṃ, Ānanda, Girimānandassa bhikkhuno imā dasa saññā bhāseyyāsi,
ṭhānaṃ kho pan’etaṃ vijjati yaṃ Girimānandassa bhikkhuno imā dasa saññā sutvā so
ābādho ṭhānaso paṭippassambheyyā-ti.”

Atha kho āyasmā Ānando Bhagavato santike imā dasa saññā uggahetvā yen’āyasmā
Girimānando ten’upasaṅkami; upasaṅkamitvā āyasmato Girimānandassa imā dasa saññā
abhāsi.

Atha kho āyasmato Girimānandassa dasa saññā sutvā so ābādho ṭhānaso
paṭippassambhi, vuṭṭhāhi c’āyasmā Girimānando tamhā ābādhā. Tathā pahīno ca
pan’āyasmato Girimānandassa so ābādho ahosī-ti.”

Girimānanda Suttaṃ Niṭṭhitaṃ.

\suttaRef{AN 10.60}

\section{Cātur-appamaññā-pāṭho}

\firstline{Mettā-saha-gatena cetasā}

% TODO review text? compare with suttacentral
% TODO separate as a reflection chant?

Atthi kho tena Bhagavatā jānatā passatā arahatā sammā-sambuddhena, catasso
appamaññāyo sammad-akkhātā.

“Idha bhikkhu mettā-saha-gatena cetasā ekaṃ disaṃ pharitvā viharati, Tathā
dutiyaṃ tathā tatiyaṃ tathā catutthaṃ, Iti uddham-adho tiriyaṃ sabbadhi
sabbattatāya sabbāvantaṃ lokaṃ. Mettā-saha-gatena cetasā vipulena mahaggatena
appamāṇena averena abyāpajjhena pharitvā viharati.

Karuṇā-saha-gatena cetasā ekaṃ disaṃ pharitvā viharati, Tathā dutiyaṃ tathā
tatiyaṃ tathā catutthaṃ, Iti uddham-adho tiriyaṃ sabbadhi sabbattatāya
sabbāvantaṃ lokaṃ. Karuṇā-saha-gatena cetasā vipulena mahaggatena appamāṇena
averena abyāpajjhena pharitvā viharati.

Muditā-sahagatena cetasā ekaṃ disaṃ pharitvā viharati, Tathā dutiyaṃ tathā
tatiyaṃ tathā catutthaṃ, Iti uddham-adho tiriyaṃ sabbadhi sabbattatāya
sabbāvantaṃ lokaṃ. Muditā-saha-gatena cetasā vipulena mahaggatena appamāṇena
averena abyāpajjhena pharitvā viharati.

Upekkhā-saha-gatena cetasā ekaṃ disaṃ pharitvā viharati, Tathā dutiyaṃ tathā
tatiyaṃ tathā catutthaṃ, Iti uddham-adho tiriyaṃ sabbadhi sabbattatāya
sabbāvantaṃ lokaṃ. Upekkhā-saha-gatena cetasā vipulena mahaggatena appamāṇena
averena abyāpajjhena pharitvā viharati.”

Imā kho tena Bhagavatā jānatā passatā arahatā sammā-sambuddhena, catasso
appamaññāyo sammad-akkhātā-ti.

\suttaRef{cf. A.I.192}

\section{Dasa-dhamma-sutta-pāṭho}

\firstline{Dasa ime bhikkhave dhammā pabbajitena}

% TODO review text? compare with suttacentral

Evaṃ me sutaṃ. Ekaṃ samayaṃ Bhagavā, Sāvatthiyaṃ viharati, Jeta-vane
Anāthapiṇḍikassa ārāme. Tatra kho Bhagavā bhikkhū āmantesi: “bhikkhavo-ti”.
“Bhadante-ti” te bhikkhū Bhagavato paccassosuṃ, Bhagavā etad-avoca:

[Handa mayaṃ pabbajita-abhiṇhapaccavekkhaṇa-pāṭhaṃ bhaṇāmase:]

“Dasa ime bhikkhave dhammā pabbajitena abhiṇhaṃ paccavekkhitabbā. Katame dasa?

Vevaṇṇiy’amhi ajjhūpagato-ti, pabbajitena abhiṇhaṃ paccavekkhitabbaṃ.

Para-paṭibaddhā me jīvikā-ti, pabbajitena abhiṇhaṃ paccavekkhitabbaṃ.

Añño me ākappo karaṇīyo-ti, pabbajitena abhiṇhaṃ paccavekkhitabbaṃ.

Kacci nu kho me attā sīlato na upavadatī-ti, pabbajitena abhiṇhaṃ paccavekkhitabbaṃ.

Kacci nu kho maṃ anuvicca viññū sabrahmacārī sīlato na upavadantī-ti, pabbajitena abhiṇhaṃ paccavekkhitabbaṃ.

Sabbehi me piyehi manāpehi nānā-bhāvo vinā-bhāvo-ti, pabbajitena abhiṇhaṃ paccavekkhitabbaṃ.

Kammassako’mhi kamma-dāyādo kamma-yoni kamma-bandhu kamma-paṭisaraṇo, yaṃ kammaṃ karissāmi kalyāṇaṃ vā pāpakaṃ vā tassa dāyādo bhavissāmī-ti, pabbajitena abhiṇhaṃ paccavekkhitabbaṃ.

Katham-bhūtassa me rattin-divā vītipatantī-ti, pabbajitena abhiṇhaṃ paccavekkhitabbaṃ.

Kacci nu kho’haṃ suññāgāre abhiramāmī-ti, pabbajitena abhiṇhaṃ paccavekkhitabbaṃ.

Atthi nu kho me uttari-manussa-dhammā alam-ariya-ñāṇa-dassana-viseso adhigato, so’haṃ pacchime kāle sabrahma-cārīhi puṭṭho na maṅku bhavissāmī-ti, pabbajitena abhiṇhaṃ paccavekkhitabbaṃ.

Ime kho bhikkhave dasa dhammā pabbajitena abhiṇhaṃ paccavekkhitabbā-ti.”

Idam avoca Bhagavā. Attamanā te bhikkhū Bhagavato bhāsitaṃ abhinandun-ti.

\suttaRef{A.I.87f}

