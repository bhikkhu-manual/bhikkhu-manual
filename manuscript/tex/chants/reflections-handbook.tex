\chapter{Reflections}

% TODO review sections
% Gavesako: Where the Pali is followed by an English translation in this section, we can just use the English title.

\section{Reflection on the Four Requisites}

\begin{leader}
  [Handa mayaṃ taṅkhaṇika-\\ paccavekkhaṇa-pāṭhaṃ bhaṇāmase]
\end{leader}

\firstline{Paṭisaṅkhā yoniso cīvaraṃ paṭisevāmi}

[Paṭisaṅkhā] yoniso cīvaraṃ paṭisevāmi,\\
yāvadeva sītassa paṭighātāya, uṇhassa paṭighātāya,\\
ḍaṃsa-makasa-vātātapa-siriṃsapa-samphassānaṃ\\
paṭighātāya, yāvadeva hirikopina-paṭicchādanatthaṃ

\begin{english}
  Wisely reflecting, I use the robe: only to ward off cold, to ward off heat, to
  ward off the touch of flies, mosquitoes, wind, burning and creeping things,
  only for the sake of modesty.
\end{english}

[Paṭisaṅkhā] yoniso piṇḍapātaṃ paṭisevāmi, neva davāya, na madāya, na maṇḍanāya,
na vibhūsanāya, yāvadeva imassa kāyassa ṭhitiyā, yāpanāya, vihiṃsūparatiyā,
brahmacariyānuggahāya, iti purāṇañca vedanaṃ paṭihaṅkhāmi, navañca vedanaṃ na
uppādessāmi, yātrā ca me bhavissati anavajjatā ca phāsuvihāro cā'ti

\begin{english}
  Wisely reflecting, I use almsfood: not for fun, not for pleasure, not for
  fattening, not for beautification, only for the maintenance and nourishment of
  this body, for keeping it healthy, for helping with the Holy Life; thinking
  thus, `I will allay hunger without overeating, so that I may continue to live
  blamelessly and at ease.'
\end{english}

[Paṭisaṅkhā] yoniso senāsanaṃ paṭisevāmi,\\
yāvadeva sītassa paṭighātāya, uṇhassa paṭighātāya,\\
ḍaṃsa-makasa-vātātapa-siriṃsapa-samphassānaṃ\\
paṭighātāya, yāvadeva utuparissaya vinodanaṃ paṭisallānārāmatthaṃ

\begin{english}
  Wisely reflecting, I use the lodging: only to ward off cold, to ward off heat,
  to ward off the touch of flies, mosquitoes, wind, burning and creeping things,
  only to remove the danger from weather, and for living in seclusion.
\end{english}

[Paṭisaṅkhā] yoniso gilāna-paccaya-bhesajja-parikkhāraṃ paṭisevāmi, yāvadeva
uppannānaṃ veyyābādhikānaṃ vedanānaṃ paṭighātāya, abyāpajjha-paramatāyā'ti

\begin{english}
  Wisely reflecting, I use supports for the sick and medicinal requisites: only
  to ward off painful feelings that have arisen, for the maximum freedom from
  disease.
\end{english}

\suttaRef{M.I.10}

\section{Five Subjects for Frequent Recollection}

\begin{leader}
  [Handa mayaṃ abhiṇha-paccavekkhaṇa-pāṭhaṃ bhaṇāmase]
\end{leader}

\firstline{Jarā-dhammomhi jaraṃ anatīto}

\instr{(Men Chant)}

[Jarā-dhammomhi] jaraṃ anatīto\\
Byādhi-dhammomhi byādhiṃ anatīto\\
Maraṇa-dhammomhi maraṇaṃ anatīto

Sabbehi me piyehi manāpehi nānābhāvo vinābhāvo

Kammassakomhi kammadāyādo kammayoni kammabandhu kammapaṭisaraṇo\\
Yaṃ kammaṃ karissāmi, kalyāṇaṃ vā pāpakaṃ vā, tassa dāyādo bhavissāmi

Evaṃ amhehi abhiṇhaṃ paccavekkhitabbaṃ

\instr{(Women Chant)}

[Jarā-dhammāmhi] jaraṃ anatītā\\
Byādhi-dhammāmhi byādhiṃ anatītā\\
Maraṇa-dhammāmhi maraṇaṃ anatītā

Sabbehi me piyehi manāpehi nānābhāvo vinābhāvo

Kammassakāmhi kammadāyādā kammayoni kammabandhu kammapaṭisaraṇā\\
Yaṃ kammaṃ karissāmi, kalyāṇaṃ vā pāpakaṃ vā, tassa dāyādā bhavissāmi

Evaṃ amhehi abhiṇhaṃ paccavekkhitabbaṃ

\suttaRef{A.III.71}

% TODO add English
% Gavesako: Add English version here "I am of the nature to..."

\section{Ten Subjects for Frequent Recollection}

\firstline{Dasa ime bhikkhave}

\begin{leader}
  [Handa mayaṃ pabbajita\hyp{}abhiṇha-\\ paccavekkhaṇa\hyp{}pāṭhaṃ bhaṇāmase]
\end{leader}

[Dasa ime bhikkhave] dhammā pabbajitena abhiṇhaṃ paccavekkhitabbā, katame dasa

\begin{english}
  Bhikkhus, there are ten dhammas which should be reflected upon again and again by one who has gone forth. \prul{What} are these ten?
\end{english}

Vevaṇṇiyamhi ajjhūpagato'ti pabbajitena abhiṇhaṃ paccavekkhitabbaṃ

\begin{english}
  `I am no longer living according to worldly aims and values.'\\
  This should be reflected upon again and again\\
  by one who has gone forth.
\end{english}

Parapaṭibaddhā me jīvikā'ti pabbajitena abhiṇhaṃ paccavekkhitabbaṃ

\begin{english}
  `My very life is sustained through the gifts of others.'\\
  This should be reflected upon again and again\\
  by one who has gone forth.
\end{english}

Añño me ākappo karaṇīyo'ti pabbajitena abhiṇhaṃ paccavekkhitabbaṃ

\begin{english}
  `I should strive to abandon my former habits.'\\
  This should be reflected upon again and again\\
  by one who has gone forth.
\end{english}

Kacci nu kho me attā sīlato na upavadatī'ti pabbajitena abhiṇhaṃ paccavekkhitabbaṃ

\begin{english}
  `Does regret over my conduct arise in my mind?'\\
  This should be reflected upon again and again\\
  by one who has gone forth.
\end{english}

Kacci nu kho maṃ anuvicca viññū sabrahmacārī sīlato na upavadantī'ti pabbajitena abhiṇhaṃ paccavekkhitabbaṃ

\begin{english}
  `Could my spiritual companions find fault with my conduct?'\\
  This should be reflected upon again and again\\
  by one who has gone forth.
\end{english}

Sabbehi me piyehi manāpehi nānābhāvo vinābhāvo'ti pabbajitena abhiṇhaṃ paccavekkhitabbaṃ

\begin{english}
  `All that is mine, beloved and pleasing, will become otherwise, will become separated from me.'\\
  This should be reflected upon again and again\\
  by one who has gone forth.
\end{english}

Kammassakomhi kammadāyādo kammayoni kammabandhu kammapaṭisaraṇo, yaṃ kammaṃ karissāmi, kalyāṇaṃ vā pāpakaṃ vā, tassa dāyādo bhavissāmī'ti pabbajitena abhiṇhaṃ paccavekkhitabbaṃ

\begin{english}
  `I am the owner of my kamma, heir to my kamma,\\
  born of my kamma, related to my kamma,\\
  abide supported by my kamma; whatever kamma I shall do,\\
  for good or for ill, of \prul{that} I will be the heir.'\\
  This should be reflected upon again and again\\
  by one who has gone forth.
\end{english}

`Kathambhūtassa me rattindivā vītipatantī'ti pabbajitena abhiṇhaṃ paccavekkhitabbaṃ

\begin{english}
  `The days and nights are relentlessly passing; how well am I spending my time?'\\
  This should be reflected upon again and again\\
  by one who has gone forth.
\end{english}

Kacci nu kho'haṃ suññāgāre abhiramāmī'ti pabbajitena abhiṇhaṃ paccavekkhitabbaṃ

\begin{english}
  `Do I delight in solitude or not?'\\
  This should be reflected upon again and again\\
  by one who has gone forth.
\end{english}

Atthi nu kho me uttari-manussa-dhammā alamariya-ñāṇa-dassana-viseso adhigato, so'haṃ pacchime kāle sabrahmacārīhi puṭṭho na maṅku bhavissāmī'ti pabbajitena abhiṇhaṃ paccavekkhitabbaṃ

\begin{english}
  `Has my practice borne fruit with freedom or insight so that at the end of my life I need not feel ashamed when questioned by my spiritual companions?'\\
  This should be reflected upon again and again\\
  by one who has gone forth.
\end{english}

Ime kho bhikkhave dasa dhammā pabbajitena abhiṇhaṃ paccavekkhitabbā'ti

\begin{english}
  Bhikkhus, these are the ten dhammas to be reflected upon again and again by one who has gone forth.
\end{english}

\suttaRef{A.V,87}

\section{Suffusion With the Divine Abidings}

\firstline{Mettā-sahagatena}

\begin{leader}
  [Handa mayaṃ caturappamaññā-obhāsanaṃ karomase]
\end{leader}

[Mettā-sahagatena] cetasā ekaṃ disaṃ pharitvā viharati\\
Tathā dutiyaṃ tathā tatiyaṃ tathā catutthaṃ\\
Iti uddhamadho tiriyaṃ sabbadhi sabbattatāya\\
Sabbāvantaṃ lokaṃ mettā-sahagatena cetasā\\
Vipulena mahaggatena appamāṇena averena abyāpajjhena\\
\vin pharitvā viharati

Karuṇā-sahagatena cetasā ekaṃ disaṃ pharitvā viharati\\
Tathā dutiyaṃ tathā tatiyaṃ tathā catutthaṃ\\
Iti uddhamadho tiriyaṃ sabbadhi sabbattatāya\\
Sabbāvantaṃ lokaṃ karuṇā-sahagatena cetasā\\
Vipulena mahaggatena appamāṇena averena abyāpajjhena\\
\vin pharitvā viharati

Muditā-sahagatena cetasā ekaṃ disaṃ pharitvā viharati\\
Tathā dutiyaṃ tathā tatiyaṃ tathā catutthaṃ\\
Iti uddhamadho tiriyaṃ sabbadhi sabbattatāya\\
Sabbāvantaṃ lokaṃ muditā-sahagatena cetasā\\
Vipulena mahaggatena appamāṇena averena abyāpajjhena\\
\vin pharitvā viharati

Upekkhā-sahagatena cetasā ekaṃ disaṃ pharitvā viharati\\
Tathā dutiyaṃ tathā tatiyaṃ tathā catutthaṃ\\
Iti uddhamadho tiriyaṃ sabbadhi sabbattatāya\\
Sabbāvantaṃ lokaṃ upekkhā-sahagatena cetasā\\
Vipulena mahaggatena appamāṇena averena abyāpajjhena\\
\vin pharitvā viharatī'ti

\suttaRef{D.I.251}

% TODO Gavesako: Add "I will abide..." in English here

\section{Dedication of Merit to the Devas and Others}

% Devatādi-patti-dāna-gāthā
% Gavesako: I have only done this chant at Dhammayut monasteries. Is it necessary?

\begin{leader}
  [Handa mayaṃ patti-dāna-gāthāyo bhaṇāmase]
\end{leader}

\firstline{Yā devatā santi vihāra-vāsinī}

\enlargethispage{\baselineskip}

Yā devatā santi vihāra-vāsinī\\
Thūpe ghare bodhi-ghare tahiṃ tahiṃ\\
Tā dhamma-dānena bhavantu pūjitā\\
Sotthiṃ karonte'dha vihāra-maṇḍale\\
Therā ca majjhā navakā ca bhikkhavo\\
Sārāmikā dāna-patī upāsakā\\
Gāmā ca desā nigamā ca issarā\\
Sappāṇa-bhūtā sukhitā bhavantu te\\
Jalābu-jā ye pi ca aṇḍa-sambhavā\\
Saṃseda-jātā atha-v-opapātikā\\
Niyyānikaṃ dhamma-varaṃ paṭicca te\\
Sabbe pi dukkhassa karontu saṅkhayaṃ.\\
Ṭhātu ciraṃ sataṃ dhammo\\
Dhamma-dharā ca puggalā\\
Saṅgho hotu samaggo va\\
Atthāya ca hitāya ca\\
Amhe rakkhatu saddhammo\\
Sabbe pi dhamma-cārino\\
Vuḍḍhiṃ sampāpuṇeyyāma\\
Dhamme ariyappavedite.

\subsection{Pasannā hontu sabbe pi}

% Gavesako: I have only done this chant at Dhammayut monasteries. Is it necessary?

\firstline{Pasannā hontu sabbe pi}

Pasannā hontu sabbe pi\\
Pāṇino Buddha-sāsane.\\
Sammā-dhāraṃ pavecchanto\\
Kāle devo pavassatu.\\
Vuḍḍhi-bhāvāya sattānaṃ\\
Samiddhaṃ netu medaniṃ.\\
Mātā-pitā ca atra-jaṃ\\
Niccaṃ rakkhanti puttakaṃ.\\
Evaṃ dhammena rājāno\\
Pajaṃ rakkhantu sabbadā.

% FIXME needs an English title

\section{Recollection After Using the Requisites}
\label{recollection-after-using}

\begin{leader}
  [Handa mayaṃ atīta-paccavekkhaṇa-pāṭhaṃ bhaṇāmase]
\end{leader}

% FIXME \todo{very similar to the other reflection on the requisites, review punctuation}
% Gavesako: This is always done in Thailand during evening chanting.

\firstline{Ajja mayā apaccavekkhitvā yaṃ cīvaraṃ}

Ajja mayā apaccavekkhitvā yaṃ cīvaraṃ paribhuttaṃ,\\
taṃ yāvadeva sītassa paṭighātāya, uṇhassa paṭighātāya,\\
ḍaṃsa-makasa-vātātapa-siriṃsapa-samphassānaṃ\\
paṭighātāya, yāvadeva hirikopina paṭicchādan'atthaṃ.

Ajja mayā apaccavekkhitvā yo piṇḍapāto paribhutto, so n'eva davāya, na madāya, na
maṇḍanāya, na vibhūsanāya, yāvad-eva imassa kāyassa ṭhitiyā, yāpanāya,
vihiṃsūparatiyā, brahmacariyānuggahāya, iti purāṇañca vedanaṃ paṭihaṅkhāmi,
navañca vedanaṃ na uppādessāmi, yātrā ca me bhavissati anavajjatā ca
phāsuvihāro cā'ti.

Ajja mayā apaccavekkhitvā yaṃ senāsanaṃ paribhuttaṃ, taṃ yāvadeva sītassa
paṭighātāya, uṇhassa paṭighātāya, ḍaṃsa-makasa-vātātapa-siriṃsapa-samphassānaṃ
paṭighātāya, yāvadeva utuparissaya vinodanaṃ paṭisallānārāmatthaṃ.

Ajja mayā apaccavekkhitvā yo gilāna-paccayabhesajja-\\ parikkhāro paribhutto, so
yāvadeva uppannānaṃ veyyābādhikānaṃ vedanānaṃ paṭighātāya,\\
abyāpajjha-paramatāyā'ti. \suttaRef{cf. M.I.10}

% Gavesako: Add English version here

\section[Reflection on the Off-Putting Qualities]{Reflection on the Off-Putting Qualities of the Requisites}

\begin{leader}
  [Handa mayaṃ dhātu-paṭikūla-\\ paccavekkhaṇa-pāṭhaṃ bhaṇāmase]
\end{leader}

\firstline{Yathā paccayaṃ pavattamānaṃ dhātu-mattam}

[Yathā paccayaṃ] pavattamānaṃ dhātu-mattam-ev'etaṃ\\
Yad idaṃ cīvaraṃ tad upabhuñjako ca puggalo\\
Dhātu-mattako, nissatto, nijjīvo, suñño\\
Sabbāni pana imāni cīvarāni ajigucchanīyāni\\
Imaṃ pūti-kāyaṃ patvā, ativiya jigucchanīyāni jāyanti\\
Yathā paccayaṃ pavattamānaṃ dhātu-mattam-ev'etaṃ\\
Yad idaṃ piṇḍapāto tad upabhuñjako ca puggalo\\
Dhātu-mattako, nissatto, nijjīvo, suñño\\
Sabbo panāyaṃ piṇḍapāto ajigucchanīyo\\
Imaṃ pūti-kāyaṃ patvā, ativiya jigucchanīyo jāyati\\
Yathā paccayaṃ pavattamānaṃ dhātu-mattam-ev'etaṃ\\
Yad idaṃ senāsanaṃ tad upabhuñjako ca puggalo\\
Dhātu-mattako, nissatto, nijjīvo, suñño\\
Sabbāni pana imāni senāsanāni ajigucchanīyāni\\
Imaṃ pūti-kāyaṃ patvā, ativiya jigucchanīyāni jāyanti\\
Yathā paccayaṃ pavattamānaṃ dhātu-mattam-ev'etaṃ\\
Yad idaṃ gilāna-paccaya-bhesajja-parikkhāro tad upabhuñjako ca puggalo\\
Dhātu-mattako, nissatto, nijjīvo, suñño\\
Sabbo panāyaṃ\\
gilāna-paccaya-bhesajja-parikkhāro ajigucchanīyo\\
Imaṃ pūti-kāyaṃ patvā, ativiya jigucchanīyo jāyati

% TODO Gavesako: Add English version here. This seems to be a modern compilation somewhat based on the MN 28 sub-commentary.

\section{Reflection on Universal Well-Being}

% Mettāpharaṇaṃ

\begin{leader}
  [Handa mayam mettāpharaṇaṃ karomase]
\end{leader}

\firstline{Ahaṃ sukhito homi niddukkho homi}

[Ahaṃ sukhito homi] niddukkho homi, avero homi, abyāpajjho homi, anīgho homi,
sukhī attānaṃ pariharāmi

Sabbe sattā sukhitā hontu, sabbe sattā averā hontu, sabbe sattā abyāpajjhā
hontu, sabbe sattā anīghā hontu, sabbe sattā sukhī attānaṃ pariharantu

Sabbe sattā sabbadukkhā pamuccantu

Sabbe sattā laddha-sampattito mā vigacchantu

Sabbe sattā kammassakā kammadāyādā kammayonī kammabandhū kammapaṭisaraṇā,
yaṃ kammaṃ karissanti, kalyāṇaṃ vā pāpakaṃ vā, tassa dāyādā bhavissanti

\suttaRef{M.I.288; A.V.88}

\section{Reflection on Universal Well-Being (English)}

% Gavesako: Is this second title necessary here? Maybe just have one title for both Pali and English version following it.

\begin{leader}
  [Now let us chant the reflections on universal well-being]
\end{leader}

\firstline{May I abide in well-being}

[May I abide in well-being,]\\
In freedom from affliction,\\
In freedom from hostility,\\
In freedom from ill-will,\\
In freedom from anxiety,\\
And may I maintain well-being in myself.

May everyone abide in well-being,\\
In freedom from hostility,\\
In freedom from ill-will,\\
In freedom from anxiety, and may they\\
Maintain well-being in themselves.

May all beings be released from all suffering.

And may they not be parted from the good fortune they have attained.

When they act upon intention,\\
All beings are the owners of their action and inherit its results.\\
Their future is born from such action,\\
companion to such action,\\
And its results will be their home.

All actions with intention,\\
Be they skilful or harmful ---\\
Of such acts they will be the heirs.

\suttaRef{M.I.288; A.V.88}

\section[The Unconditioned]{Reflection on the Unconditioned}

\begin{leader}
  [Handa mayaṃ nibbāna-sutta-pāṭhaṃ bhaṇāmase]
\end{leader}

\firstline{Atthi bhikkhave ajātaṃ abhūtaṃ akataṃ}

Atthi bhikkhave ajātaṃ abhūtaṃ akataṃ asaṅkhataṃ

\begin{english}
  There is an Unborn, Unoriginated, Uncreated and Unformed.
\end{english}

No cetaṃ bhikkhave abhavissa ajātaṃ abhūtaṃ akataṃ asaṅkhataṃ

\begin{english}
  If there was not this Unborn, this Unoriginated, this Uncreated, this~Unformed,
\end{english}

Na yidaṃ jātassa bhūtassa katassa saṅkhatassa nissaraṇaṃ paññāyetha

\begin{english}
  Freedom from the world of the born, the originated, the created, the formed would not be possible.
\end{english}

Yasmā ca kho bhikkhave atthi ajātaṃ abhūtaṃ akataṃ asaṅkhataṃ

\begin{english}
  But since there is an Unborn, Unoriginated, Uncreated and Unformed,
\end{english}

Tasmā jātassa bhūtassa katassa saṅkhatassa nissaraṇaṃ paññāyati

\begin{english}
  Therefore is freedom possible from the world of the born, the originated, the created and the formed.
\end{english}

\suttaRef{Ud.8.3}

\section{Reflection on the Thirty-Two Parts}

\begin{leader}
  [Handa mayaṃ dvattiṃsākāra-pāṭhaṃ bhaṇāmase]
\end{leader}

\firstline{Ayaṃ kho me kāyo uddhaṃ pādatalā}

[Ayaṃ kho] me kāyo uddhaṃ pādatalā adho kesamatthakā\\
tacapariyanto pūro nānappakārassa asucino

Atthi imasmiṃ kāye

kesā, lomā, nakhā, dantā, taco, maṃsaṃ, nahārū, aṭṭhī, aṭṭhimiñjaṃ, vakkaṃ, hadayaṃ, yakanaṃ, kilomakaṃ, pihakaṃ, papphāsaṃ, antaṃ, antaguṇaṃ, udariyaṃ, karīsaṃ, pittaṃ, semhaṃ, pubbo, lohitaṃ, sedo, medo, assu, vasā, kheḷo, siṅghāṇikā, lasikā, muttaṃ, matthaluṅgan'ti

Evam-ayaṃ me kāyo uddhaṃ pādatalā adho kesamatthakā\\
tacapariyanto pūro nānappakārassa asucino\\
\suttaRef{cf. M.I.57}

\section{Verses on the Sharing of Merit}

% Sabba-patti-dāna-gāthā

\firstline{Puññass'idāni katassa yān'aññāni katāni me}

\begin{leader}
  [Handa mayaṃ sabba-patti-dāna-gāthāyo bhaṇāmase]
\end{leader}

% FIXME \todo{Title is the same as with ‘Yā devatā\ldots’ above}
% Gavesako: This chant is done very often in our tradition.

\begin{twochants}
Puññass'idāni katassa & yān'aññāni katāni me\\
Tesañca bhāgino hontu & sattānantāppamāṇakā\\
Ye piyā guṇavantā ca & mayhaṃ mātā-pitādayo\\
Diṭṭhā me cāpyadiṭṭhā vā & aññe majjhatta-verino\\
Sattā tiṭṭhanti lokasmiṃ & te bhummā catu-yonikā\\
Pañc'eka-catu-vokārā & saṃsarantā bhavābhave\\
Ñātaṃ ye patti-dānam-me & anumodantu te sayaṃ\\
Ye c'imaṃ nappajānanti & devā tesaṃ nivedayuṃ\\
Mayā dinnāna-puññānaṃ & anumodana-hetunā\\
Sabbe sattā sadā hontu & averā sukha-jīvino\\
Khemappadañca pappontu & tesāsā sijjhataṃ subhā\\
\end{twochants}

% TODO Gavesako: Add English version here.

\subsection{Yan-dāni me kataṃ puññaṃ}

% Gavesako: I never heard this chant done anywhere. Is it necessary?

\firstline{Yan-dāni me kataṃ puññaṃ}

\begin{twochants}
Yan-dāni me kataṃ puññaṃ & tenānen'uddisena ca,\\
Khippaṃ sacchikareyyāhaṃ & dhamme lok'uttare nava.\\
Sace tāva abhabbo'haṃ & saṃsāre pana saṃsaraṃ,\\
Niyato bodhi-satto va & sambuddhena viyākato.\\
Nāṭṭhārasa pi abhabba & ṭhānāni pāpuṇeyy'ahaṃ.\\
Manussattañ-ca liṅgañ-ca & pabbajjañ-c'upasampadaṃ.\\
Labhitvā pesalo sīlī & dhāreyyaṃ satthu sāsanaṃ,\\
Sukhā-paṭipado khippābhiñño & sacchikareyyahaṃ.\\
Arahatta-phalaṃ aggaṃ & vijj'ādi-guṇ'alaṅ-kataṃ,\\
Yadi n'uppajjati Buddho & kammaṃ paripūrañ-ca me,\\
Evaṃ sante labheyyāhaṃ & pacceka-bodhim-uttaman-ti.
\end{twochants}

\section{Verses of Sharing and Aspiration}

% TODO Pali title
% Uddissanādhiṭṭhāna-gāthā

\begin{leader}
  [Handa mayaṃ uddissanādhiṭṭhāna-gāthāyo bhaṇāmase]
\end{leader}

\firstline{Iminā puññakammena upajjhāyā guṇuttarā}

[Iminā puññakammena] upajjhāyā guṇuttarā\\
Ācariyūpakārā ca mātāpitā ca ñātakā\\
Suriyo candimā rājā guṇavantā narāpi ca\\
Brahma-mārā ca indā ca lokapālā ca devatā\\
Yamo mittā manussā ca majjhattā verikāpi ca\\
Sabbe sattā sukhī hontu puññāni pakatāni me\\
Sukhañca tividhaṃ dentu khippaṃ pāpetha vomataṃ\\
Iminā puññakammena iminā uddissena ca\\
Khipp'āhaṃ sulabhe ceva taṇhūpādāna-chedanaṃ\\
Ye santāne hīnā dhammā yāva nibbānato mamaṃ\\
Nassantu sabbadā yeva yattha jāto bhave bhave\\
Ujucittaṃ satipaññā sallekho viriyamhinā\\
Mārā labhantu nokāsaṃ kātuñca viriyesu me\\
Buddhādhipavaro nātho dhammo nātho varuttamo\\
Nātho paccekabuddho ca saṅgho nāthottaro mamaṃ\\
Tesottamānubhāvena mārokāsaṃ labhantu mā

% Gavesako: Note that in the Wat Pah Pong chanting book there is an additional line at the end of this chant (Dasapuññānubhāvena mārokāsaṃ labhantu mā). I wonder how they chant it at WPN? The version we use is actually from the Dhammayut chanting book. This can be explained by the rather muddled way of putting our original chanting book together from various sources and not having a lot of contact with Thailand at that time. Ajahn Ariyesako who edited the first version of the Bhikkhu Manual was a Dhammayut monk.

\section{Verses of Sharing and Aspiration (English)}

\begin{leader}
  [Now let us chant the verses of sharing and aspiration]
\end{leader}

\firstline{Through the goodness that arises from my practice}

Through the goodness that arises from my practice,\\
May my spiritual teachers and guides of great virtue,\\
My mother, my father, and my relatives,\\
The Sun and the Moon, and all virtuous leaders of the world,\\
May the highest gods and evil forces,\\
Celestial beings, guardian spirits of the Earth,\\\vin and the Lord of Death,\\
May those who are friendly, indifferent, or hostile,\\
May all beings receive the blessings of my life,\\
May they soon attain the threefold bliss\\\vin and realize the Deathless.\\
Through the goodness that arises from my practice,\\
And through this act of sharing,\\
May all desires and attachments quickly cease\\
And all harmful states of mind.\\
Until I realize Nibbāna,\\
In every kind of birth, may I have an upright mind,\\
With mindfulness and wisdom, austerity and vigour.\\
May the forces of delusion not take hold\\\vin nor weaken my resolve.

The Buddha is my excellent refuge,\\
Unsurpassed is the protection of the Dhamma,\\
The Solitary Buddha is my noble guide,\\
The Saṅgha is my supreme support.\\
Through the supreme power of all these,\\
May darkness and delusion be dispelled.

\section{Sabbe sattā sadā hontu}

% FIXME English title. Perhaps "Sharing of Blessings (short version)"?

\firstline{Sabbe sattā sadā hontu}

\begin{twochants}
Sabbe sattā sadā hontu & averā sukha-jīvino\\
Kataṃ puñña-phalaṃ mayhaṃ & sabbe bhāgī bhavantu te
\end{twochants}

% TODO Gavesako: Add English version. This short chant is sometimes done as a simple dedication after doing some meritorious act. 
% May all beings always live happily, free from animosity.
% May all share in the blessings springing from the good I have done.

\clearpage

\section{Ti-loka-vijaya-rāja-patti-dāna-gāthā}

\firstline{Yaṅ kiñci kusalaṃ kammaṃ}

Yaṅ kiñci kusalaṃ kammaṃ\\
\vin kattabbaṃ kiriyaṃ mama\\
Kāyena vācā manasā\\
\vin ti-dase sugataṃ kataṃ\\
Ye sattā saññino atthi\\
\vin ye ca sattā asaññino\\
Kataṃ puñña-phalaṃ mayhaṃ\\
\vin sabbe bhāgī bhavantu te\\
Ye taṃ kataṃ suviditaṃ\\
\vin dinnaṃ puñña-phalaṃ mayā\\
Ye ca tattha na jānanti\\
\vin devā gantvā nivedayuṃ\\
Sabbe lokamhi ye sattā\\
\vin jīvant'āhāra-hetukā\\
Manuññaṃ bhojanaṃ sabbe\\
\vin labhantu mama cetasā.

\suttaRef{Apadāna 4}

% Gavesako: This chant is only done by Mr. Tan Nam and seems to be pupular with Cambodians. Is it necessary?

