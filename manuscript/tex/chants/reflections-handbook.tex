\chapter{Reflections}

\section{Reflection on the Four Requisites}

\begin{leader}
  [Handa mayaṁ taṅkhaṇika-\\ paccavekkhaṇa-pāṭhaṁ bhaṇāmase]
\end{leader}

\firstline{Paṭisaṅkhā yoniso cīvaraṁ paṭisevāmi}

[Paṭisaṅkhā] yoniso cīvaraṁ paṭisevāmi,\\
yāvadeva sītassa paṭighātāya, uṇhassa paṭighātāya,\\
ḍaṁsa-makasa-vātātapa-siriṁsapa-samphassānaṁ\\
paṭighātāya, yāvadeva hirikopina-paṭicchādanatthaṁ

\begin{english}
  Wisely reflecting, I use the robe: only to ward off cold, to ward off heat, to
  ward off the touch of flies, mosquitoes, wind, burning and creeping things,
  only for the sake of modesty.
\end{english}

[Paṭisaṅkhā] yoniso piṇḍapātaṁ paṭisevāmi, neva davāya, na madāya, na maṇḍanāya,
na vibhūsanāya, yāvadeva imassa kāyassa ṭhitiyā, yāpanāya, vihiṁsūparatiyā,
brahmacariyānuggahāya, iti purāṇañca vedanaṁ paṭihaṅkhāmi, navañca vedanaṁ na
uppādessāmi, yātrā ca me bhavissati anavajjatā ca phāsuvihāro cā'ti

\begin{english}
  Wisely reflecting, I use almsfood: not for fun, not for pleasure, not for
  fattening, not for beautification, only for the maintenance and nourishment of
  this body, for keeping it healthy, for helping with the Holy Life; thinking
  thus, `I~will allay hunger without overeating, so that I may continue to live
  blamelessly and at ease.'
\end{english}

[Paṭisaṅkhā] yoniso senāsanaṁ paṭisevāmi,\\
yāvadeva sītassa paṭighātāya, uṇhassa paṭighātāya,\\
ḍaṁsa-makasa-vātātapa-siriṁsapa-samphassānaṁ\\
paṭighātāya, yāvadeva utuparissaya vinodanaṁ paṭisallānārāmatthaṁ

\begin{english}
  Wisely reflecting, I use the lodging: only to ward off cold, to ward off heat,
  to ward off the touch of flies, mosquitoes, wind, burning and creeping things,
  only to remove the danger from weather, and for living in seclusion.
\end{english}

[Paṭisaṅkhā] yoniso gilāna-paccaya-bhesajja-parikkhāraṁ paṭisevāmi, yāvadeva
uppannānaṁ veyyābādhikānaṁ vedanānaṁ paṭighātāya, abyāpajjha-paramatāyā'ti

\begin{english}
  Wisely reflecting, I use supports for the sick and medicinal requisites: only
  to ward off painful feelings that have arisen, for the maximum freedom from
  disease.
\end{english}

\suttaRef{M.I.10}

\section{Five Subjects for Frequent Recollection}

\begin{leader}
  [Handa mayaṁ abhiṇha-paccavekkhaṇa-pāṭhaṁ bhaṇāmase]
\end{leader}

\firstline{Jarā-dhammomhi jaraṁ anatīto}

\instr{(Men Chant)}

[Jarā-dhammomhi] jaraṁ anatīto

\begin{english}
  I am of the nature to age, I have not gone beyond ageing.
\end{english}

Byādhi-dhammomhi byādhiṁ anatīto

\begin{english}
  I am of the nature to sicken, I have not gone beyond sickness.
\end{english}

Maraṇa-dhammomhi maraṇaṁ anatīto

\begin{english}
  I am of the nature to die, I have not gone beyond dying.
\end{english}

Sabbehi me piyehi manāpehi nānābhāvo vinābhāvo

\begin{english}
  All that is mine, beloved and pleasing,\\
  will become otherwise, will become separated from me.
\end{english}

Kammassakomhi kammadāyādo kammayoni kammabandhu kammapaṭisaraṇo\\
Yaṁ kammaṁ karissāmi, kalyāṇaṁ vā pāpakaṁ vā, tassa dāyādo bhavissāmi

\begin{english}
  I am the owner of my kamma, heir to my kamma, born of my kamma, related to my
  kamma, abide supported by my kamma. Whatever kamma I shall do, for good or for
  ill, of that I will be the heir.
\end{english}

Evaṁ amhehi abhiṇhaṁ paccavekkhitabbaṁ

\begin{english}
  Thus we should frequently recollect.
\end{english}

\instr{(Women Chant)}

[Jarā-dhammāmhi] jaraṁ anatītā

\begin{english}
  I am of the nature to age, I have not gone beyond ageing.
\end{english}

Byādhi-dhammāmhi byādhiṁ anatītā

\begin{english}
  I am of the nature to sicken, I have not gone beyond sickness.
\end{english}

Maraṇa-dhammāmhi maraṇaṁ anatītā

\begin{english}
  I am of the nature to die, I have not gone beyond dying.
\end{english}

Sabbehi me piyehi manāpehi nānābhāvo vinābhāvo

\begin{english}
  All that is mine, beloved and pleasing,\\
  will become otherwise, will become separated from me.
\end{english}

Kammassakāmhi kammadāyādā kammayoni kammabandhu kammapaṭisaraṇā\\
Yaṁ kammaṁ karissāmi, kalyāṇaṁ vā pāpakaṁ vā, tassa dāyādā bhavissāmi

\begin{english}
  I am the owner of my kamma, heir to my kamma, born of my kamma, related to my
  kamma, abide supported by my kamma. Whatever kamma I shall do, for good or for
  ill, of that I will be the heir.
\end{english}

Evaṁ amhehi abhiṇhaṁ paccavekkhitabbaṁ

\begin{english}
  Thus we should frequently recollect.\\
  \suttaRef{A.III.71}
\end{english}

\section{Ten Subjects for Frequent Recollection}

\firstline{Dasa ime bhikkhave}

\begin{leader}
  [Handa mayaṁ pabbajita\hyp{}abhiṇha-\\ paccavekkhaṇa\hyp{}pāṭhaṁ bhaṇāmase]
\end{leader}

[Dasa ime bhikkhave] dhammā pabbajitena abhiṇhaṁ paccavekkhitabbā, katame dasa

\begin{english}
  Bhikkhus, there are ten dhammas which should be reflected upon, again and again, by one who has gone forth. What are these ten?
\end{english}

Vevaṇṇiyamhi ajjhūpagato'ti pabbajitena abhiṇhaṁ paccavekkhitabbaṁ

\begin{english}
  `I am no longer living according to worldly aims and values.'\\
  This should be reflected upon, again and again,\\
  by one who has gone forth.
\end{english}

Parapaṭibaddhā me jīvikā'ti pabbajitena abhiṇhaṁ paccavekkhitabbaṁ

\begin{english}
  `My very life is sustained through the gifts of others.'\\
  This should be reflected upon, again and again,\\
  by one who has gone forth.
\end{english}

Añño me ākappo karaṇīyo'ti pabbajitena abhiṇhaṁ paccavekkhitabbaṁ

\begin{english}
  `I should strive to abandon my former habits.'\\
  This should be reflected upon, again and again,\\
  by one who has gone forth.
\end{english}

Kacci nu kho me attā sīlato na upavadatī'ti pabbajitena abhiṇhaṁ paccavekkhitabbaṁ

\begin{english}
  `Does regret over my conduct arise in my mind?'\\
  This should be reflected upon, again and again,\\
  by one who has gone forth.
\end{english}

Kacci nu kho maṁ anuvicca viññū sabrahmacārī sīlato na upavadantī'ti pabbajitena abhiṇhaṁ paccavekkhitabbaṁ

\begin{english}
  `Could my spiritual companions find fault with my conduct?'\\
  This should be reflected upon, again and again,\\
  by one who has gone forth.
\end{english}

Sabbehi me piyehi manāpehi nānābhāvo vinābhāvo'ti pabbajitena abhiṇhaṁ paccavekkhitabbaṁ

\begin{english}
  `All that is mine, beloved and pleasing, will become otherwise, will become separated from me.'\\
  This should be reflected upon, again and again,\\
  by one who has gone forth.
\end{english}

Kammassakomhi kammadāyādo kammayoni kammabandhu kammapaṭisaraṇo, yaṁ kammaṁ karissāmi, kalyāṇaṁ vā pāpakaṁ vā, tassa dāyādo bhavissāmī'ti pabbajitena abhiṇhaṁ paccavekkhitabbaṁ

\begin{english}
  `I am the owner of my kamma, heir to my kamma,\\
  born of my kamma, related to my kamma,\\
  abide supported by my kamma; whatever kamma I shall do,\\
  for good or for ill, of that I will be the heir.'\\
  This should be reflected upon, again and again,\\
  by one who has gone forth.
\end{english}

`Kathambhūtassa me rattindivā vītipatantī'ti pabbajitena abhiṇhaṁ paccavekkhitabbaṁ

\begin{english}
  `The days and nights are relentlessly passing; how well am I spending my time?'\\
  This should be reflected upon, again and again,\\
  by one who has gone forth.
\end{english}

Kacci nu kho'haṁ suññāgāre abhiramāmī'ti pabbajitena abhiṇhaṁ paccavekkhitabbaṁ

\begin{english}
  `Do I delight in solitude or not?'\\
  This should be reflected upon, again and again,\\
  by one who has gone forth.
\end{english}

Atthi nu kho me uttari-manussa-dhammā alamariya-ñāṇa-dassana-viseso adhigato, so'haṁ pacchime kāle sabrahmacārīhi puṭṭho na maṅku bhavissāmī'ti pabbajitena abhiṇhaṁ paccavekkhitabbaṁ

\begin{english}
  `Has my practice borne fruit with freedom or insight so that at the end of my life I need not feel ashamed when questioned by my spiritual companions?'\\
  This should be reflected upon, again and again,\\
  by one who has gone forth.
\end{english}

Ime kho bhikkhave dasa dhammā pabbajitena abhiṇhaṁ paccavekkhitabbā'ti

\begin{english}
  Bhikkhus, these are the ten dhammas to be reflected upon, again and again, by one who has gone forth.
\end{english}

\suttaRef{A.V.87}

\section{Caturappamaññā-obhāsana}

\firstline{Mettā-sahagatena}

\begin{leader}
  [Handa mayaṁ caturappamaññā-obhāsanaṁ karomase]
\end{leader}

[Mettā-sahagatena] cetasā ekaṁ disaṁ pharitvā viharati\\
Tathā dutiyaṁ tathā tatiyaṁ tathā catutthaṁ\\
Iti uddhamadho tiriyaṁ sabbadhi sabbattatāya\\
Sabbāvantaṁ lokaṁ mettā-sahagatena cetasā\\
Vipulena mahaggatena appamāṇena averena\\
abyāpajjhena pharitvā viharati

Karuṇā-sahagatena cetasā ekaṁ disaṁ pharitvā viharati\\
Tathā dutiyaṁ tathā tatiyaṁ tathā catutthaṁ\\
Iti uddhamadho tiriyaṁ sabbadhi sabbattatāya\\
Sabbāvantaṁ lokaṁ karuṇā-sahagatena cetasā\\
Vipulena mahaggatena appamāṇena averena\\
abyāpajjhena pharitvā viharati

Muditā-sahagatena cetasā ekaṁ disaṁ pharitvā viharati\\
Tathā dutiyaṁ tathā tatiyaṁ tathā catutthaṁ\\
Iti uddhamadho tiriyaṁ sabbadhi sabbattatāya\\
Sabbāvantaṁ lokaṁ muditā-sahagatena cetasā\\
Vipulena mahaggatena appamāṇena averena\\
abyāpajjhena pharitvā viharati

\clearpage

Upekkhā-sahagatena cetasā ekaṁ disaṁ pharitvā viharati\\
Tathā dutiyaṁ tathā tatiyaṁ tathā catutthaṁ\\
Iti uddhamadho tiriyaṁ sabbadhi sabbattatāya\\
Sabbāvantaṁ lokaṁ upekkhā-sahagatena cetasā\\
Vipulena mahaggatena appamāṇena averena\\
abyāpajjhena pharitvā viharatī'ti \suttaRef{D.I.251}

\subsubsection{Suffusion With the Divine Abidings}

\firstline{I will abide}

\smallskip

\begin{leader}
  [Now let us make the Four Boundless Qualities\\ shine forth.]
\end{leader}

[I will abide] pervading one quarter\\
with a heart imbued with loving-kindness;\\
Likewise the second, likewise the third,\\ likewise the fourth;\\
So above and below, around and everywhere;\\ and to all as to myself.\\
I will abide pervading the all-encompassing\\
world with a heart imbued with loving-kindness;\\
abundant, exalted, immeasurable, without hostility,\\
and without ill-will.

I will abide pervading one quarter\\
with a heart imbued with compassion;\\
Likewise the second, likewise the third,\\ likewise the fourth;\\
So above and below, around and everywhere;\\ and to all as to myself.\\
I will abide pervading the all-encompassing\\
world with a heart imbued with compassion;\\
abundant, exalted, immeasurable, without hostility,\\
and without ill-will.

I will abide pervading one quarter\\
with a heart imbued with gladness;\\
Likewise the second, likewise the third,\\ likewise the fourth;\\
So above and below, around and everywhere;\\ and to all as to myself.\\
I will abide pervading the all-encompassing\\
world with a heart imbued with gladness;\\
abundant, exalted, immeasurable, without hostility,\\
and without ill-will.

I will abide pervading one quarter\\
with a heart imbued with equanimity;\\
Likewise the second, likewise the third,\\ likewise the fourth;\\
So above and below, around and everywhere;\\ and to all as to myself.\\
I will abide pervading the all-encompassing\\
world with a heart imbued with equanimity;\\
abundant, exalted, immeasurable, without hostility,\\
and without ill-will.

\section{Recollection After Using the Requisites}
\label{recollection-after-using}

\begin{leader}
  [Handa mayaṁ atīta-paccavekkhaṇa-pāṭhaṁ bhaṇāmase]
\end{leader}

\firstline{Ajja mayā apaccavekkhitvā yaṁ cīvaraṁ}

Ajja mayā apaccavekkhitvā yaṁ cīvaraṁ paribhuttaṁ, taṁ yāvadeva sītassa
paṭighātāya, uṇhassa paṭighātāya, ḍaṁsa-makasa-vātātapa-siriṁsapa-samphassānaṁ
paṭighātāya, yāvadeva hirikopina paṭicchādan'atthaṁ.

\begin{english}
  Whatever robe I used today without consideration, was only to ward off cold,
  to ward off heat, to ward off the touch of flies, mosquitoes, wind, burning
  and creeping things, only for the sake of modesty.
\end{english}

Ajja mayā apaccavekkhitvā yo piṇḍapāto paribhutto, so n'eva davāya, na madāya,
na maṇḍanāya, na vibhūsanāya, yāvad-eva imassa kāyassa ṭhitiyā, yāpanāya,
vihiṁsūparatiyā, brahmacariyānuggahāya, iti purāṇañca vedanaṁ paṭihaṅkhāmi,
navañca vedanaṁ na uppādessāmi, yātrā ca me bhavissati anavajjatā ca phāsuvihāro
cā'ti.

\begin{english}
  Whatever alms-food I used today without consideration, was not for fun, not
  for pleasure, not for fattening, not for beautification, only for the
  maintenance and nourishment of this body, for keeping it healthy, for helping
  with the Holy Life; thinking thus, `I will allay hunger without overeating, so
  that I may continue to live blamelessly and at ease.'
\end{english}

Ajja mayā apaccavekkhitvā yaṁ senāsanaṁ paribhuttaṁ, taṁ yāvadeva sītassa
paṭighātāya, uṇhassa paṭighātāya, ḍaṁsa-makasa-vātātapa-siriṁsapa-samphassānaṁ
paṭighātāya, yāvadeva utuparissaya vinodanaṁ paṭisallānārāmatthaṁ.

\begin{english}
  Whatever lodging I used today without consideration, was only to ward off
  cold, to ward off heat, to ward off the touch of flies, mosquitoes, wind,
  burning and creeping things, only to remove the danger from weather, and for
  living in seclusion.
\end{english}

Ajja mayā apaccavekkhitvā yo gilāna-paccayabhesajja-\\ parikkhāro paribhutto, so
yāvadeva uppannānaṁ veyyābādhikānaṁ vedanānaṁ paṭighātāya,
abyāpajjha-paramatāyā'ti.

\begin{english}
  Whatever medicinal requisite for supporting the sick I used today without
  consideration, was only to ward off painful feelings that have arisen, for the
  maximum freedom from disease.\\
  \suttaRef{M.I.10}
\end{english}

\section[Reflection on the Off-Putting Qualities]{Reflection on the Off-Putting Qualities of the Requisites}

% Pali title: Dhātu-paṭikūla-paccavekkhaṇa-pāṭho

% This seems to be a modern compilation somewhat based on the MN 28 sub-commentary.

\begin{leader}
  [Handa mayaṁ dhātu-paṭikūla-\\ paccavekkhaṇa-pāṭhaṁ bhaṇāmase]
\end{leader}

\firstline{Yathā paccayaṁ pavattamānaṁ dhātu-mattam}

[Yathā paccayaṁ] pavattamānaṁ dhātu-mattam-ev'etaṁ

\packedtrline{Composed of only elements according to causes and conditions}

Yad idaṁ cīvaraṁ tad upabhuñjako ca puggalo

\packedtrline{Are these robes and so is the person wearing them;}

Dhātu-mattako, nissatto, nijjīvo, suñño

\packedtrline{Merely elements, not a being, without a soul,\\ and empty of self.}

Sabbāni pana imāni cīvarāni ajigucchanīyāni

\packedtrline{None of these robes are innately repulsive}

Imaṁ pūti-kāyaṁ patvā, ativiya jigucchanīyāni jāyanti

\packedtrline{But touching this unclean body, they become disgusting~indeed.}

Yathā paccayaṁ pavattamānaṁ dhātu-mattam-ev'etaṁ

\packedtrline{Composed of only elements according to causes and conditions}

Yad idaṁ piṇḍapāto tad upabhuñjako ca puggalo

\packedtrline{Is this almsfood and so is the person eating it;}

Dhātu-mattako, nissatto, nijjīvo, suñño

\packedtrline{Merely elements, not a being, without a soul,\\ and empty of self.}

Sabbo panāyaṁ piṇḍapāto ajigucchanīyo

\packedtrline{None of this almsfood is innately repulsive}

Imaṁ pūti-kāyaṁ patvā, ativiya jigucchanīyo jāyati

\packedtrline{But touching this unclean body, it becomes disgusting~indeed.}

Yathā paccayaṁ pavattamānaṁ dhātu-mattam-ev'etaṁ

\packedtrline{Composed of only elements according to causes and conditions}

Yad idaṁ senāsanaṁ tad upabhuñjako ca puggalo

\packedtrline{Is this dwelling and so is the person using it;}

Dhātu-mattako, nissatto, nijjīvo, suñño

\packedtrline{Merely elements, not a being, without a soul,\\ and empty of self.}

Sabbāni pana imāni senāsanāni ajigucchanīyāni

\packedtrline{None of these dwellings are innately repulsive}

Imaṁ pūti-kāyaṁ patvā, ativiya jigucchanīyāni jāyanti

\packedtrline{But touching this unclean body, they become disgusting~indeed.}

Yathā paccayaṁ pavattamānaṁ dhātu-mattam-ev'etaṁ

\packedtrline{Composed of only elements according to causes and conditions}

Yad idaṁ gilāna-paccaya-bhesajja-parikkhāro tad upabhuñjako ca puggalo

\packedtrline{Is this medicinal requisite and so is the person that takes it;}

Dhātu-mattako, nissatto, nijjīvo, suñño

\packedtrline{Merely elements, not a being, without a soul,\\ and empty of self.}

Sabbo panāyaṁ gilāna-paccaya-bhesajja-parikkhāro ajigucchanīyo

\packedtrline{None of this medicinal requisite is innately repulsive}

Imaṁ pūti-kāyaṁ patvā, ativiya jigucchanīyo jāyati

\packedtrline{But touching this unclean body, it becomes disgusting~indeed.}

\section{Mettāpharaṇa}

\begin{leader}
  [Handa mayam mettāpharaṇaṁ karomase]
\end{leader}

\firstline{Ahaṁ sukhito homi niddukkho homi}

[Ahaṁ sukhito homi] niddukkho homi, avero homi, abyāpajjho homi, anīgho homi,
sukhī attānaṁ pariharāmi

Sabbe sattā sukhitā hontu, sabbe sattā averā hontu, sabbe sattā abyāpajjhā
hontu, sabbe sattā anīghā hontu, sabbe sattā sukhī attānaṁ pariharantu

Sabbe sattā sabbadukkhā pamuccantu

Sabbe sattā laddha-sampattito mā vigacchantu

Sabbe sattā kammassakā kammadāyādā kammayonī kammabandhū kammapaṭisaraṇā,
yaṁ kammaṁ karissanti, kalyāṇaṁ vā pāpakaṁ vā, tassa dāyādā bhavissanti

\suttaRef{M.I.288; A.V.88}

\subsubsection{Reflection on Universal Well-Being}

\enlargethispage{\baselineskip}

\begin{leader}
  [Now let us chant the reflections on universal well-being]
\end{leader}

\firstline{May I abide in well-being}

[May I abide in well-being,]\\
In freedom from affliction,\\
In freedom from hostility,\\
In freedom from ill-will,\\
In freedom from anxiety,\\
And may I maintain well-being in myself.

May everyone abide in well-being,\\
In freedom from hostility,\\
In freedom from ill-will,\\
In freedom from anxiety, and may they\\
Maintain well-being in themselves.

May all beings be released from all suffering.

And may they not be parted from the good fortune\\
they have attained.

When they act upon intention,\\
All beings are the owners of their action\\
and inherit its results.\\
Their future is born from such action,\\
companion to such action,\\
And its results will be their home.

All actions with intention,\\
Be they skilful or harmful --\\
Of such acts they will be the heirs. \suttaRef{M.I.288; A.V.88}

\section[The Unconditioned]{Reflection on the Unconditioned}

\begin{leader}
  [Handa mayaṁ nibbāna-sutta-pāṭhaṁ bhaṇāmase]
\end{leader}

\firstline{Atthi bhikkhave ajātaṁ abhūtaṁ akataṁ}

Atthi bhikkhave ajātaṁ abhūtaṁ akataṁ asaṅkhataṁ

\begin{english}
  There is an Unborn, Unoriginated, Uncreated and Unformed.
\end{english}

No cetaṁ bhikkhave abhavissa ajātaṁ abhūtaṁ akataṁ asaṅkhataṁ

\begin{english}
  If there was not this Unborn, this Unoriginated, this Uncreated, this~Unformed,
\end{english}

Na yidaṁ jātassa bhūtassa katassa saṅkhatassa nissaraṇaṁ paññāyetha

\begin{english}
  Freedom from the world of the born, the originated, the created, the formed would not be possible.
\end{english}

Yasmā ca kho bhikkhave atthi ajātaṁ abhūtaṁ akataṁ asaṅkhataṁ

\begin{english}
  But since there is an Unborn, Unoriginated, Uncreated and Unformed,
\end{english}

Tasmā jātassa bhūtassa katassa saṅkhatassa nissaraṇaṁ paññāyati

\begin{english}
  Therefore is freedom possible from the world of the born, the originated, the created and the formed. \suttaRef{Ud.8.3}
\end{english}

\section{Reflection on the Thirty-Two Parts}

\begin{leader}
  [Handa mayaṁ dvattiṁsākāra-pāṭhaṁ bhaṇāmase]
\end{leader}

\firstline{Ayaṁ kho me kāyo uddhaṁ pādatalā}

[Ayaṁ kho] me kāyo uddhaṁ pādatalā adho kesamatthakā\\
tacapariyanto pūro nānappakārassa asucino

\begin{english}
  This, which is my body, from the soles of the feet up, and down from the crown of the head, is a sealed bag of skin filled with unattractive things.
\end{english}

Atthi imasmiṁ kāye

\begin{english}
  In this body there are:
\end{english}

{\centering
\setArrayStretch{1}

\begin{tabular}{ r l }
kesā            & \tr{hair of the head} \\
lomā            & \tr{hair of the body} \\
nakhā           & \tr{nails} \\
dantā           & \tr{teeth} \\
taco            & \tr{skin} \\
maṁsaṁ          & \tr{flesh} \\
nahārū          & \tr{sinews} \\
\end{tabular}

\begin{tabular}{ r l }
aṭṭhī           & \tr{bones} \\
aṭṭhimiñjaṁ     & \tr{bone marrow} \\
vakkaṁ          & \tr{kidneys} \\
hadayaṁ         & \tr{heart} \\
yakanaṁ         & \tr{liver} \\
kilomakaṁ       & \tr{membranes} \\
pihakaṁ         & \tr{spleen} \\
papphāsaṁ       & \tr{lungs} \\
antaṁ           & \tr{bowels} \\
antaguṇaṁ       & \tr{entrails} \\
udariyaṁ        & \tr{undigested food} \\
karīsaṁ         & \tr{excrement} \\
pittaṁ          & \tr{bile} \\
semhaṁ          & \tr{phlegm} \\
pubbo           & \tr{pus} \\
lohitaṁ         & \tr{blood} \\
sedo            & \tr{sweat} \\
medo            & \tr{fat} \\
assu            & \tr{tears} \\
vasā            & \tr{grease} \\
kheḷo           & \tr{spittle} \\
siṅghāṇikā      & \tr{mucus} \\
lasikā          & \tr{oil of the joints} \\
muttaṁ          & \tr{urine} \\
matthaluṅgan'ti & \tr{brain} \\
\end{tabular}

\restoreArrayStretch
}

Evam-ayaṁ me kāyo uddhaṁ pādatalā adho kesamatthakā\\
tacapariyanto pūro nānappakārassa asucino

\begin{english}
  This, then, which is my body, from the soles of the feet up, and down from the crown of the head, is a sealed bag of skin filled with unattractive things. \suttaRef{M.I.57}
\end{english}

\section{Sabba-patti-dāna-gāthā}

\englishTitle{Verses on the Sharing of Merit}

\firstline{Puññass'idāni katassa yān'aññāni katāni me}

\begin{leader}
  [Handa mayaṁ sabba-patti-dāna-gāthāyo bhaṇāmase]
\end{leader}

Puññass'idāni katassa\\
Yān'aññāni katāni me\\
Tesañca bhāgino hontu\\
Sattānantāppamāṇakā

\begin{english}
  May whatever living beings,\\
  Without measure, without end,\\
  Partake of all the merit,\\
  From the good deeds I have done:
\end{english}

Ye piyā guṇavantā ca\\
Mayhaṁ mātā-pitādayo\\
Diṭṭhā me cāpyadiṭṭhā vā\\
Aññe majjhatta-verino

\begin{english}
  Those loved and full of goodness,\\
  My mother and my father dear,\\
  Beings seen by me and those unseen,\\
  Those neutral and averse,
\end{english}

Sattā tiṭṭhanti lokasmiṁ\\
Te bhummā catu-yonikā\\
Pañc'eka-catu-vokārā\\
Saṁsarantā bhavābhave

\begin{english}
  Beings established in the world,\\
  From the three planes and four grounds of birth,\\
  With five aggregates or one or four,\\
  Wand'ring on from realm to realm,
\end{english}

Ñātaṁ ye patti-dānam-me\\
Anumodantu te sayaṁ\\
Ye c'imaṁ nappajānanti\\
Devā tesaṁ nivedayuṁ

\enlargethispage{\baselineskip}

\begin{english}
  Those who know my act of dedication,\\
  May they all rejoice in it,\\
  And as for those yet unaware,\\
  May the devas let them know.
\end{english}

Mayā dinnāna-puññānaṁ anumodana-hetunā\\
Sabbe sattā sadā hontu\\
Averā sukha-jīvino\\
Khemappadañca pappontu\\
Tesāsā sijjhataṁ subhā

\begin{english}
  By rejoicing in my sharing,\\
  May all beings live at ease,\\
  In freedom from hostility,\\
  May their good wishes be fulfilled,\\
  And may they all reach safety.
\end{english}

\section{Uddissanādhiṭṭhāna-gāthā}

\enlargethispage{\baselineskip}

\begin{leader}
  [Handa mayaṁ uddissanādhiṭṭhāna-gāthāyo bhaṇāmase]
\end{leader}

\firstline{Iminā puññakammena upajjhāyā guṇuttarā}

[Iminā puññakammena] upajjhāyā guṇuttarā\\
Ācariyūpakārā ca mātāpitā ca ñātakā\\
Suriyo candimā rājā guṇavantā narāpi ca\\
Brahma-mārā ca indā ca lokapālā ca devatā\\
Yamo mittā manussā ca majjhattā verikāpi ca\\
Sabbe sattā sukhī hontu puññāni pakatāni me\\
Sukhañca tividhaṁ dentu khippaṁ pāpetha vomataṁ\\
Iminā puññakammena iminā uddissena ca\\
Khipp'āhaṁ sulabhe ceva taṇhūpādāna-chedanaṁ\\
Ye santāne hīnā dhammā yāva nibbānato mamaṁ\\
Nassantu sabbadā yeva yattha jāto bhave bhave\\
Ujucittaṁ satipaññā sallekho viriyamhinā

\clearpage

Mārā labhantu nokāsaṁ kātuñca viriyesu me\\
Buddhādhipavaro nātho dhammo nātho varuttamo\\
Nātho paccekabuddho ca saṅgho nāthottaro mamaṁ\\
Tesottamānubhāvena mārokāsaṁ labhantu mā\\\relax
[Dasapuññānubhāvena mārokāsaṁ labhantu mā]

\instr{(This chant is a short excerpt from a longer composition. Some
  monasteries include the last line in brackets.)}

% NOTE: This chant is an excerpt from a longer chant, composed by King Mongkut
% and translated by Ajahn Buddhadasa.
%
% It is a custom to include the additional line in brackets in Wat Pah Pong, and
% other monsteries following Thai chanting books such as the Wat Marp Jan book.

\subsubsection{Verses of Sharing and Aspiration}

\enlargethispage{\baselineskip}

\begin{leader}
  [Now let us chant the verses of sharing and aspiration]
\end{leader}

\firstline{Through the goodness that arises from my practice}

Through the goodness that arises from my practice,\\
May my spiritual teachers and guides of great virtue,\\
My mother, my father, and my relatives,\\
The Sun and the Moon, and all virtuous\\\vin leaders of the world,\\
May the highest gods and evil forces,\\
Celestial beings, guardian spirits of the Earth,\\\vin and the Lord of Death,\\
May those who are friendly, indifferent, or hostile,\\
May all beings receive the blessings of my life,\\
May they soon attain the threefold bliss\\\vin and realize the Deathless.\\
Through the goodness that arises from my practice,\\
And through this act of sharing,\\
May all cravings and attachments quickly cease\\
And all harmful states of mind.\\
Until I realize Nibbāna,\\
In every kind of birth, may I have an upright mind,\\
With mindfulness and wisdom, austerity and vigour.\\
May the forces of delusion not take hold\\\vin nor weaken my resolve.\\
The Buddha is my excellent refuge,\\
Unsurpassed is the protection of the Dhamma,\\
The Solitary Buddha is my noble guide,\\
The Saṅgha is my supreme support.\\
Through the supreme power of all these,\\
May darkness and delusion be dispelled.\\\relax
[By the power of the ten merits,\\
May Māra gain no opening.]

\section{Sabbe sattā sadā hontu}

\enlargethispage{\baselineskip}

\firstline{Sabbe sattā sadā hontu}

\begin{paritta}
Sabbe sattā sadā hontu\\
Averā sukha-jīvino\\
Kataṁ puñña-phalaṁ mayhaṁ\\
Sabbe bhāgī bhavantu te
\end{paritta}

\begin{english}
  May all beings always live happily, free from animosity.\\
  May all share in the blessings springing from the good I~have~done.
\end{english}

