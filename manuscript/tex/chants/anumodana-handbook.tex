\chapter{Anumodanā}

\section{Yathā vāri-vahā pūrā}

\firstline{Yathā vāri-vahā pūrā}

\begin{paritta}
  Yathā vāri-vahā pūrā\\\vin paripūrenti sāgaraṃ\\
  Evam-eva ito dinnaṃ\\\vin petānaṃ upakappati \suttaRef{Khp.VII.v8}

  Icchitaṃ patthitaṃ tumhaṃ\\\vin khippam-eva samijjhatu\\
  Sabbe pūrentu saṅkappā\\\vin cando paṇṇaraso yathā\\
  Maṇi jotiraso yathā \suttaRef{DhpA.I.198}

\firstline{Sabb'ītiyo vivajjantu sabba-rogo vinassatu}

  Sabb'ītiyo vivajjantu\\\vin sabba-rogo vinassatu\\
  Mā te bhavatv-antarāyo\\\vin sukhī dīgh'āyuko bhava\\
  Abhivādana-sīlissa\\\vin niccaṃ vuḍḍhāpacāyino\\
  Cattāro dhammā vaḍḍhanti\\\vin āyu vaṇṇo sukhaṃ balaṃ \suttaRef{Dhp 109}
\end{paritta}

\firstline{Bhavatu sabba-maṅgalaṃ}

Bhavatu sabba-maṅgalaṃ rakkhantu sabba-devatā\\
Sabba-buddhānubhāvena sadā sotthī bhavantu te

Bhavatu sabba-maṅgalaṃ rakkhantu sabba-devatā\\
Sabba-dhammānubhāvena sadā sotthī bhavantu te

Bhavatu sabba-maṅgalaṃ rakkhantu sabba-devatā\\
Sabba-saṅghānubhāvena sadā sotthī bhavantu te

\subsubsection{Just as Rivers}

Just as rivers full of water entirely fill up the sea, so will what's here been
given bring blessings to departed spirits.

\suttaRef{Khp.VII.v8}

May all your hopes and all your longings come true in no long time. May all your
wishes be fulfilled like on the fifteenth day the Moon or like a bright and
shining gem.

\suttaRef{DhpA.I.198}

May all misfortunes be avoided, may all illness be dispelled, may you never meet
with dangers, may you be happy and live long. For those who are respectful, who
always honour the elders, four are the qualities which will increase: Life,
beauty, happiness and strength.

\suttaRef{Dhp 109}

May every blessing come to be and all good spirits guard you well. Through the
power of all Buddhas \ldots\ Dhammas \ldots\ Saṅghas may you always be at ease.

\subsubsection{Sabba-roga-vinimutto}

\instr{(This shorter form is sometimes used instead of `Yathā\ldots')}

\firstline{Sabba-roga-vinimutto}

\bigskip

\begin{paritta}
  Sabba-roga-vinimutto\\\vin sabba-santāpa-vajjito\\
  Sabba-veram-atikkanto\\\vin nibbuto ca tuvam-bhava

  Sabb'ītiyo vivajjantu\\\vin sabba-rogo vinassatu\\
  Mā te bhavatv-antarāyo\\\vin sukhī dīgh'āyuko bhava\\
  Abhivādana-sīlissa\\\vin niccaṃ vuḍḍhāpacāyino\\
  Cattāro dhammā vaḍḍhanti\\\vin āyu vaṇṇo sukhaṃ balaṃ \suttaRef{Dhp 109}
\end{paritta}

\bigskip

\begin{english}
May you be freed from all disease, safe from all torment, beyond all animosity
and unbound.

May all misfortunes be avoided\ldots
\end{english}

\section{Bhojana-dānānumodanā}

\firstline{Āyu-do bala-do dhīro vaṇṇa-do paṭibhāṇa-do}

\begin{twochants}
  Āyu-do bala-do dhīro & vaṇṇa-do paṭibhāṇa-do\\
  Sukhassa dātā medhāvī & sukhaṃ so adhigacchati\\
  Āyuṃ datvā balaṃ vaṇṇaṃ & sukhañ-ca paṭibhāna-do\\
  Dīgh'āyu yasavā hoti & yattha yatthūpapajjatī'ti
\end{twochants}

\suttaRef{A.III.42}

\section{Aggappasāda-sutta-gāthā}

\firstline{Aggato ve pasannānaṃ}

\enlargethispage{\baselineskip}

\begin{paritta}
  Aggato ve pasannānaṃ\\\vin aggaṃ dhammaṃ vijānataṃ\\
  Agge Buddhe pasannānaṃ\\\vin dakkhiṇeyye anuttare\\
  Agge dhamme pasannānaṃ\\\vin virāgūpasame sukhe\\
  Agge saṅghe pasannānaṃ\\\vin puññakkhette anuttare\\
  Aggasmiṃ dānaṃ dadataṃ\\\vin aggaṃ puññaṃ pavaḍḍhati\\
  Aggaṃ āyu ca vaṇṇo ca\\\vin yaso kitti sukhaṃ balaṃ\\
  Aggassa dātā medhāvī\\\vin agga-dhamma-samāhito\\
  Deva-bhūto manusso vā\\\vin aggappatto pamodatī'ti \suttaRef{A.II.35; A.III.36}
\end{paritta}

\section{Adāsi-me ādi-gāthā}

% Alternative Pali title: Tiro-kuḍḍa-kaṇḍa

\firstline{Adāsi me akāsi me}

\begin{paritta}
Adāsi me akāsi me\\\vin ñāti-mittā sakhā ca me\\
Petānaṃ dakkhiṇaṃ dajjā\\\vin pubbe katam-anussaraṃ\\
Na hi ruṇṇaṃ vā soko vā\\\vin yā v'aññā paridevanā\\
Na taṃ petānam-atthāya\\\vin evaṃ tiṭṭhanti ñātayo

\firstline{Ayañ-ca kho dakkhiṇā dinnā}

Ayañ-ca kho dakkhiṇā dinnā\\\vin saṅghamhi supatiṭṭhitā\\
Dīgha-rattaṃ hitāy'assa\\\vin ṭhānaso upakappati\\
So ñāti-dhammo ca ayaṃ nidassito\\\vin petāna'pūjā ca katā uḷārā\\
Balañ-ca bhikkhūnam-anuppadinnaṃ\\\vin tumhehi puññaṃ pasutaṃ anappakan'ti.
\end{paritta}

% English source: Bodhivana Vol 2, p.182

\begin{english}
  ``He gave to me, he acted on my behalf, and he was my relative, companion,
  friend.'' Offerings should be given for the dead when one reflects thus on
  what was done in the past. For no weeping or sorrowing or any kind of
  lamentation benefit the dead whose relatives keep acting in that way.

  \bigskip

  But when this offering is given, well-placed in the Sangha, it works for their
  long-term benefit and they profit immediately. In this way the proper duty to
  relatives has been shown and great honour has been done to the dead and the
  monks have been given strength: You have acquried merit that is not small.
\end{english}

\suttaRef{Khp.VII.v10-13}

\section{Kāla-dāna-sutta-gāthā}

\firstline{Kāle dadanti sapaññā vadaññū vīta-maccharā}

\begin{paritta}
  Kāle dadanti sapaññā\\\vin vadaññū vīta-maccharā\\
  Kālena dinnaṃ ariyesu\\\vin uju-bhūtesu tādisu\\
  Vippasanna-manā tassa\\\vin vipulā hoti dakkhiṇā\\
  Ye tattha anumodanti\\\vin veyyāvaccaṃ karonti vā\\
  Na tena dakkhiṇā onā\\\vin te pi puññassa bhāgino\\
  Tasmā dade appaṭivāna-citto\\\vin yattha dinnaṃ mahapphalaṃ\\
  Puññāni para-lokasmiṃ\\\vin patiṭṭhā honti pāṇinan'ti \suttaRef{A.III.41}
\end{paritta}

\section{Ratanattay'ānubhāv'ādi-gāthā}

\firstline{Ratanattay'ānubhāvena ratanattaya-tejasā}

\begin{paritta}
  Ratanattay'ānubhāvena\\\vin ratanattaya-tejasā\\
  Dukkha-roga-bhayā verā\\\vin sokā sattu c'upaddavā\\
  Anekā antarāyā pi\\\vin vinassantu asesato\\
  Jaya-siddhi dhanaṃ lābhaṃ\\\vin sotthi bhāgyaṃ sukhaṃ balaṃ\\
  Siri āyu ca vaṇṇo ca\\\vin bhogaṃ vuḍḍhī ca yasavā\\
  Sata-vassā ca āyu ca\\\vin jīva-siddhī bhavantu te
\end{paritta}

\section{Culla-maṅgala-cakka-vāḷa}

\enlargethispage{\baselineskip}

\firstline{Sabba-buddh'ānubhāvena}

Sabba-buddh'ānubhāvena\\
sabba-dhamm'ānubhāvena\\
sabba-saṅgh'ānubhāvena

Buddha-ratanaṃ dhamma-ratanaṃ saṅgha-ratanaṃ

Tiṇṇaṃ ratanānaṃ ānubhāvena\\
Catur-āsīti-sahassa-dhammakkhandh'ānubhāvena\\
Piṭakattay'ānubhāvena\\
Jina-sāvak'ānubhāvena

Sabbe te rogā sabbe te bhayā sabbe te antarāyā sabbe te upaddavā sabbe te
dunnimittā sabbe te avamaṅgalā vinassantu

Āyu-vaḍḍhako dhana-vaḍḍhako siri-vaḍḍhako yasa-vaḍḍhako bala-vaḍḍhako
vaṇṇa-vaḍḍhako sukha-vaḍḍhako hotu sabbadā

Dukkha-roga-bhayā verā sokā sattu c'upaddavā\\
Anekā antarāyā pi vinassantu ca tejasā\\
Jaya-siddhi dhanaṃ lābhaṃ\\
Sotthi bhāgyaṃ sukhaṃ balaṃ\\
Siri āyu ca vaṇṇo ca bhogaṃ vuḍḍhī ca yasavā\\
Sata-vassā ca āyū ca jīva-siddhī bhavantu te

Bhavatu sabba-maṅgalaṃ\ldots{}

\section{Mahā-maṅgala-cakka-vāḷa}

\firstline{Siri-dhiti-mati-tejo-jayasiddhi}

Siri-dhiti-mati-tejo-jayasiddhi-mahiddhi-mahāguṇā-parimita-puññādhikarassa
sabbantarāya-nivāraṇa-samatthassa bhagavato arahato sammā-sambuddhassa

Dvattiṃsa-mahā-purisa-lakkhaṇānubhāvena\\
asītyānubyañjanānubhāvena\\
aṭṭhuttara-sata-maṅgalānubhāvena\\
chabbaṇṇa-raṃsiyānubhāvena ketumālānubhāvena\\
dasa-pāramitānubhāvena\\
dasa-upapāramitānubhāvena\\
dasa-paramattha-pāramitānubhāvena\\
sīla-samādhi-paññānubhāvena\\
buddhānubhāvena\\
dhammānubhāvena\\
saṅghānubhāvena\\
tejānubhāvena\\
iddhānubhāvena\\
balānubhāvena\\
ñeyya-dhammānubhāvena\\
caturāsīti-sahassa-dhamma-kkhandhānubhāvena\\
nava-lokuttara-dhammānubhāvena\\
aṭṭhaṅgika-maggānubhāvena\\
aṭṭha-samāpattiyānubhāvena\\
chaḷabhiññānubhāvena\\
catu-sacca-ñāṇānubhāvena\\
dasa-bala-ñāṇānubhāvena\\
sabbaññuta-ñāṇānubhāvena\\
mettā-karuṇā-muditā-upekkhānubhāvena\\
sabba-parittānubhāvena\\
ratanattaya-saraṇānubhāvena

\clearpage

tuyhaṃ sabba-roga-sok'upaddava-\\ dukkha-domanass'upāyāsā vinassantu\\
sabba-antarāyā pi vinassantu\\
sabba-saṅkappā tuyhaṃ samijjhantu\\
dīghāyukā tuyhaṃ hotu sata-vassa-jīvena\\
samaṅgiko hotu sabbadā

Ākāsa-pabbata-vana-bhūmi-gaṅgā-mahāsamuddā ārakkhakā
devatā sadā tumhe anurakkhantu

% Text source: Chomtong chanting book

% FIXME Gavesako: Is anybody ever going to use this? In northern Thailand they have many such local 'prayers' which are not used elsewhere. I suggest to leave it out.

\section{Vihāra-dāna-gāthā}

\begin{paritta}
  Sītaṃ uṇhaṃ paṭihanti\\\vin tato vāḷamigāni ca\\
  sariṃsape ca makase\\\vin sisire cāpi vuṭṭhiyo\\
  Tato vātātapo ghoro\\\vin sañjāto paṭihaññati\\
  Leṇatthañ ca sukhatthañ ca\\\vin jhāyituñ ca vipassituṃ\\
  Vihāradānaṃ saṅghassa\\\vin aggaṃ buddhehi vaṇṇitaṃ\\
  Tasmā hi paṇḍito poso\\\vin sampassaṃ attham attano\\
  Vihāre kāraye ramme\\\vin vāsayettha bahu-ssute\\
  Tesaṃ annañ ca pānañ ca\\\vin vattha-senāsanāni ca\\
  Dadeyya uju-bhūtesu\\\vin vippasannena cetasā\\
  Te tassa dhammaṃ desenti\\\vin sabbadukkhāpanūdanaṃ\\
  Yaṃ so dhammaṃ idh'aññāya\\\vin parinibbātayanāsavo'ti
\end{paritta}

% Source: Chomtong chanting book

% FIXME Gavesako: This is a chant specially used to acknowledge the offering of a Vihara only. Is it ever going to be used by anyone?

