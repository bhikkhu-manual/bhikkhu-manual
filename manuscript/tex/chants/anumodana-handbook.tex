\chapter{Anumodanā}

\section{Yathā vāri-vahā pūrā}

\firstline{Yathā vāri-vahā pūrā}

Yathā vāri-vahā pūrā, paripūrenti sāgaraṃ\\
Evam-eva ito dinnaṃ, petānaṃ upakappati

\suttaRef{Khp.VII.v8}

Icchitaṃ patthitaṃ tumhaṃ, khippam-eva samijjhatu\\
Sabbe pūrentu saṅkappā, cando paṇṇaraso yathā\\
Maṇi jotiraso yathā

\suttaRef{DhpA.I.198}

\firstline{Sabb'ītiyo vivajjantu sabba-rogo vinassatu}

Sabb'ītiyo vivajjantu, sabba-rogo vinassatu\\
Mā te bhavatv-antarāyo, sukhī dīgh'āyuko bhava\\
Abhivādana-sīlissa, niccaṃ vuḍḍhāpacāyino\\
Cattāro dhammā vaḍḍhanti, āyu vaṇṇo sukhaṃ balaṃ

\suttaRef{Dhp 109}

\firstline{Bhavatu sabba-maṅgalaṃ}

Bhavatu sabba-maṅgalaṃ rakkhantu sabba-devatā\\
Sabba-buddhānubhāvena sadā sotthī bhavantu te

Bhavatu sabba-maṅgalaṃ rakkhantu sabba-devatā\\
Sabba-dhammānubhāvena sadā sotthī bhavantu te

Bhavatu sabba-maṅgalaṃ rakkhantu sabba-devatā\\
Sabba-saṅghānubhāvena sadā sotthī bhavantu te

\subsubsection{Just as Rivers}

Just as rivers full of water entirely fill up the sea, so will what's here been
given bring blessings to departed spirits.

\suttaRef{Khp.VII.v8}

May all your hopes and all your longings come true in no long time. May all your
wishes be fulfilled like on the fifteenth day the Moon or like a bright and
shining gem.

\suttaRef{DhpA.I.198}

May all misfortunes be avoided, may all illness be dispelled, may you never meet
with dangers, may you be happy and live long. For those who are respectful, who
always honour the elders, four are the qualities which will increase: Life,
beauty, happiness and strength.

\suttaRef{Dhp 109}

May every blessing come to be and all good spirits guard you well. Through the
power of all Buddhas \ldots\ Dhammas \ldots\ Saṅghas may you always be at ease.

\section{Sabba-roga-vinimutto}

\instr{(This shorter form is sometimes used instead of `Yathā vāri-vahā pūrā'.)}

\firstline{Sabba-roga-vinimutto}

Sabba-roga-vinimutto, sabba-santāpa-vajjito\\
Sabba-veram-atikkanto, nibbuto ca tuvam-bhava

Sabb'ītiyo vivajjantu, sabba-rogo vinassatu\\
Mā te bhavatv-antarāyo, sukhī dīgh'āyuko bhava\\
Abhivādana-sīlissa, niccaṃ vuḍḍhāpacāyino\\
Cattāro dhammā vaḍḍhanti, āyu vaṇṇo sukhaṃ balaṃ

\suttaRef{Dhp 109}

May you be freed from all disease, safe from all torment, beyond all animosity
and unbound.

May all misfortunes be avoided\ldots

\section{Bhojana-dānānumodanā}

\firstline{Āyu-do bala-do dhīro vaṇṇa-do paṭibhāṇa-do}

\begin{twochants}
  Āyu-do bala-do dhīro & vaṇṇa-do paṭibhāṇa-do\\
  Sukhassa dātā medhāvī & sukhaṃ so adhigacchati\\
  Āyuṃ datvā balaṃ vaṇṇaṃ & sukhañ-ca paṭibhāna-do\\
  Dīgh'āyu yasavā hoti & yattha yatthūpapajjatī'ti
\end{twochants}

\suttaRef{A.III.42}

\section{Aggappasāda-sutta-gāthā}

\firstline{Aggato ve pasannānaṃ}

\begin{twochants}
  Aggato ve pasannānaṃ & aggaṃ dhammaṃ vijānataṃ\\
  Agge Buddhe pasannānaṃ & dakkhiṇeyye anuttare\\
  Agge dhamme pasannānaṃ & virāgūpasame sukhe\\
  Agge saṅghe pasannānaṃ & puññakkhette anuttare\\
  Aggasmiṃ dānaṃ dadataṃ & aggaṃ puññaṃ pavaḍḍhati\\
  Aggaṃ āyu ca vaṇṇo ca & yaso kitti sukhaṃ balaṃ\\
  Aggassa dātā medhāvī & agga-dhamma-samāhito\\
  Deva-bhūto manusso vā & aggappatto pamodatī'ti
\end{twochants}

\suttaRef{A.II.35; A.III.36}

\section{Adāsi-me ādi-gāthā (Tiro-kuḍḍa-kaṇḍa)}

\firstline{Adāsi me akāsi me}

\begin{twochants}
Adāsi me akāsi me & ñāti-mittā sakhā ca me\\
Petānaṃ dakkhiṇaṃ dajjā & pubbe katam-anussaraṃ\\
Na hi ruṇṇaṃ vā soko vā & yā v'aññā paridevanā\\
Na taṃ petānam-atthāya & evaṃ tiṭṭhanti ñātayo\\
\end{twochants}

\firstline{Ayañ-ca kho dakkhiṇā dinnā}

Ayañ-ca kho dakkhiṇā dinnā\\
Saṅghamhi supatiṭṭhitā\\
Dīgha-rattaṃ hitāy'assa\\
Ṭhānaso upakappati\\
So ñāti-dhammo ca ayaṃ nidassito\\
Petāna'pūjā ca katā uḷārā\\
Balañ-ca bhikkhūnam-anuppadinnaṃ\\
Tumhehi puññaṃ pasutaṃ anappakan'ti.

% English source: Bodhivana Vol 2, p.182

\begin{english}
  ``He gave to me, he acted on my behalf, and he was my relative, companion,
  friend.'' Offerings should be given for the dead when one reflects thus on
  what was done in the past. For no weeping or sorrowing or any kind of
  lamentation benefit the dead whose relatives keep acting in that way.

  But when this offering is given, well-placed in the Sangha, it works for their
  long-term benefit and they profit immediately. In this way the proper duty to
  relatives has been shown and great honour has been done to the dead and the
  monks have been given strength: You have acquried merit that is not small.
\end{english}

\suttaRef{Khp.VII.v10-13}

\section{Kāla-dāna-sutta-gāthā}

\firstline{Kāle dadanti sapaññā vadaññū vīta-maccharā}

\begin{twochants}
  Kāle dadanti sapaññā & vadaññū vīta-maccharā\\
  Kālena dinnaṃ ariyesu & uju-bhūtesu tādisu\\
  Vippasanna-manā tassa & vipulā hoti dakkhiṇā\\
  Ye tattha anumodanti & veyyāvaccaṃ karonti vā\\
  Na tena dakkhiṇā onā & te pi puññassa bhāgino\\
  Tasmā dade appaṭivāna-citto & yattha dinnaṃ mahapphalaṃ\\
  Puññāni para-lokasmiṃ & patiṭṭhā honti pāṇinan'ti
\end{twochants}

\suttaRef{A.III.41}

\section{Ratanattay'ānubhāv'ādi-gāthā}

\firstline{Ratanattay'ānubhāvena ratanattaya-tejasā}

\begin{twochants}
Ratanattay'ānubhāvena & ratanattaya-tejasā\\
Dukkha-roga-bhayā verā & sokā sattu c'upaddavā\\
Anekā antarāyā pi & vinassantu asesato\\
Jaya-siddhi dhanaṃ lābhaṃ & sotthi bhāgyaṃ sukhaṃ balaṃ\\
Siri āyu ca vaṇṇo ca & bhogaṃ vuḍḍhī ca yasavā\\
Sata-vassā ca āyu ca & jīva-siddhī bhavantu te
\end{twochants}

\section{Culla-maṅgala-cakka-vāḷa}

\firstline{Sabba-buddh'ānubhāvena}

Sabba-buddh'ānubhāvena sabba-dhamm'ānubhāvena sabba-saṅgh'ānubhāvena

Buddha-ratanaṃ dhamma-ratanaṃ saṅgha-ratanaṃ

Tiṇṇaṃ ratanānaṃ ānubhāvena\\
Catur-āsīti-sahassa-dhammakkhandh'ānubhāvena\\
Piṭakattay'ānubhāvena\\
Jina-sāvak'ānubhāvena

Sabbe te rogā sabbe te bhayā sabbe te antarāyā sabbe te upaddavā sabbe te
dunnimittā sabbe te avamaṅgalā vinassantu

Āyu-vaḍḍhako dhana-vaḍḍhako siri-vaḍḍhako yasa-vaḍḍhako bala-vaḍḍhako
vaṇṇa-vaḍḍhako sukha-vaḍḍhako hotu sabbadā

Dukkha-roga-bhayā verā sokā sattu c'upaddavā\\
Anekā antarāyā pi vinassantu ca tejasā\\
\mbox{Jaya-siddhi dhanaṃ lābhaṃ sotthi bhāgyaṃ sukhaṃ balaṃ}\\
Siri āyu ca vaṇṇo ca bhogaṃ vuḍḍhī ca yasavā\\
Sata-vassā ca āyū ca jīva-siddhī bhavantu te

Bhavatu sabba-maṅgalaṃ\ldots{}

\section{Mahā-maṅgala-cakka-vāḷa}

\firstline{Siri-dhiti-mati-tejo-jayasiddhi}

Siri-dhiti-mati-tejo-jayasiddhi-mahiddhi-mahāguṇā-parimita-puññādhikarassa
sabbantarāya-nivāraṇa-samatthassa bhagavato arahato sammā-sambuddhassa
dvattiṃsa-mahā-purisa-lakkhaṇānubhāvena

asītyānubyañjanānubhāvena\\
aṭṭhuttara-sata-maṅgalānubhāvena\\
chabbaṇṇa-raṃsiyānubhāvena ketumālānubhāvena\\
dasa-pāramitānubhāvena\\
dasa-upapāramitānubhāvena\\
dasa-paramattha-pāramitānubhāvena\\
sīla-samādhi-paññānubhāvena\\
buddhānubhāvena\\
dhammānubhāvena\\
saṅghānubhāvena\\
tejānubhāvena\\
iddhānubhāvena\\
balānubhāvena\\
ñeyya-dhammānubhāvena\\
caturāsīti-sahassa-dhamma-kkhandhānubhāvena\\
nava-lokuttara-dhammānubhāvena\\
aṭṭhaṅgika-maggānubhāvena\\
aṭṭha-samāpattiyānubhāvena\\
chaḷabhiññānubhāvena\\
catu-sacca-ñāṇānubhāvena\\
dasa-bala-ñāṇānubhāvena\\
sabbaññuta-ñāṇānubhāvena\\
mettā-karuṇā-muditā-upekkhānubhāvena\\
sabba-parittānubhāvena\\
ratanattaya-saraṇānubhāvena

tuyhaṃ sabba-roga-sok'upaddava-dukkha-domanass'upāyāsā vinassantu\\
sabba-antarāyā pi vinassantu\\
sabba-saṅkappā tuyhaṃ samijjhantu\\
dīghāyukā tuyhaṃ hotu sata-vassa-jīvena samaṅgiko hotu sabbadā

ākāsa-pabbata-vana-bhūmi-gaṅgā-mahāsamuddā ārakkhakā
devatā sadā tumhe anurakkhantu

% Text source: Chomtong chanting book
% Gavesako: Is anybody ever going to use this? In northern Thailand they have many such local 'prayers' which are not used elsewhere. I suggest to leave it out.

\section{Vihāra-dāna-gāthā}

\begin{twochants}
  Sītaṃ uṇhaṃ paṭihanti & tato vāḷamigāni ca\\
  sariṃsape ca makase & sisire cāpi vuṭṭhiyo\\
  Tato vātātapo ghoro & sañjāto paṭihaññati\\
  Leṇatthañ ca sukhatthañ ca & jhāyituñ ca vipassituṃ\\
  Vihāradānaṃ saṅghassa & aggaṃ buddhehi vaṇṇitaṃ\\
  Tasmā hi paṇḍito poso & sampassaṃ attham attano\\
  Vihāre kāraye ramme & vāsayettha bahu-ssute\\
  Tesaṃ annañ ca pānañ ca & vattha-senāsanāni ca\\
  Dadeyya uju-bhūtesu & vippasannena cetasā\\
  Te tassa dhammaṃ desenti & sabbadukkhāpanūdanaṃ\\
  Yaṃ so dhammaṃ idh'aññāya & parinibbātayanāsavo'ti
\end{twochants}

% Source: Chomtong chanting book
% Gavesako: This is a chant specially used to acknowledge the offering of a Vihara only. Is it ever going to be used by anyone?

