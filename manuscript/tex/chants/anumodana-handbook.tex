\chapter{Anumodanā}

\section{Yathā vāri-vahā pūrā}

\englishTitle{Just as Rivers}

\firstline{Yathā vāri-vahā pūrā}

Yathā vāri-vahā pūrā paripūrenti sāgaraṁ

\begin{cprenglish}
  Just as rivers full of water entirely fill up the sea,
\end{cprenglish}

Evam-eva ito dinnaṁ petānaṁ upakappati

\begin{cprenglish}
  So will what's here been given bring blessings to departed spirits.\\
  \suttaRef{Khp.VII.v8}
\end{cprenglish}

Icchitaṁ patthitaṁ tumhaṁ

\begin{cprenglish}
  May all your hopes and all your longings
\end{cprenglish}

Khippam-eva samijjhatu

\begin{cprenglish}
  Come true in no long time.
\end{cprenglish}

Sabbe pūrentu saṅkappā

\begin{cprenglish}
  May all your wishes be fulfilled
\end{cprenglish}

Cando paṇṇaraso yathā

\begin{cprenglish}
  Like on the fifteenth day the moon
\end{cprenglish}

Maṇi jotiraso yathā

\begin{cprenglish}
  Or like a bright and shining gem.\\
  \suttaRef{DhpA.I.198}
\end{cprenglish}

\firstline{Sabb'ītiyo vivajjantu sabba-rogo vinassatu}

Sabb'ītiyo vivajjantu

\begin{cprenglish}
  May all misfortunes be avoided,
\end{cprenglish}

Sabba-rogo vinassatu

\begin{cprenglish}
  May all illness be dispelled,
\end{cprenglish}

Mā te bhavatv-antarāyo

\begin{cprenglish}
  May you never meet with dangers,
\end{cprenglish}

Sukhī dīgh'āyuko bhava

\begin{cprenglish}
  May you be happy and live long.
\end{cprenglish}

Abhivādana-sīlissa\\
Niccaṁ vuḍḍhāpacāyino\\
Cattāro dhammā vaḍḍhanti\\
Āyu vaṇṇo sukhaṁ balaṁ

\begin{cprenglish}
  For those who are respectful,\\
  Who always honour the elders,\\
  Four are the qualities which will increase:\\
  Life, beauty, happiness and strength. \suttaRef{Dhp 109}
\end{cprenglish}

\firstline{Bhavatu sabba-maṅgalaṁ}

Bhavatu sabba-maṅgalaṁ

\begin{cprenglish}
  May every blessing come to be
\end{cprenglish}

Rakkhantu sabba-devatā

\begin{cprenglish}
  And all good spirits guard you well.
\end{cprenglish}

Sabba-buddhānubhāvena

\begin{cprenglish}
  Through the power of all Buddhas
\end{cprenglish}

Sadā sotthī bhavantu te

\begin{cprenglish}
  May you always be at ease.
\end{cprenglish}

Bhavatu sabba-maṅgalaṁ

\begin{cprenglish}
  May every blessing come to be
\end{cprenglish}

Rakkhantu sabba-devatā

\begin{cprenglish}
  And all good spirits guard you well.
\end{cprenglish}

Sabba-dhammānubhāvena

\begin{cprenglish}
  Through the power of all Dhammas
\end{cprenglish}

Sadā sotthī bhavantu te

\begin{cprenglish}
  May you always be at ease.
\end{cprenglish}

Bhavatu sabba-maṅgalaṁ

\begin{cprenglish}
  May every blessing come to be
\end{cprenglish}

Rakkhantu sabba-devatā

\begin{cprenglish}
  And all good spirits guard you well.
\end{cprenglish}

Sabba-saṅghānubhāvena

\begin{cprenglish}
  Through the power of all Saṅghas
\end{cprenglish}

Sadā sotthī bhavantu te

\begin{cprenglish}
  May you always be at ease.
\end{cprenglish}

\subsubsection{Sabba-roga-vinimutto}

\instr{(This shorter form is sometimes used instead of `Yathā\ldots')}

\firstline{Sabba-roga-vinimutto}

\bigskip

\enlargethispage{\baselineskip}

\begin{paritta}
  Sabba-roga-vinimutto\\\vin sabba-santāpa-vajjito\\
  Sabba-veram-atikkanto\\\vin nibbuto ca tuvam-bhava\\
  Sabb'ītiyo vivajjantu\\\vin sabba-rogo vinassatu\\
  Mā te bhavatv-antarāyo\\\vin sukhī dīgh'āyuko bhava\\
  Abhivādana-sīlissa\\\vin niccaṁ vuḍḍhāpacāyino\\
  Cattāro dhammā vaḍḍhanti\\\vin āyu vaṇṇo sukhaṁ balaṁ \suttaRef{Dhp 109}
\end{paritta}

\bigskip

\begin{english}
May you be freed from all disease, safe from all torment, beyond all animosity
and at peace.

May all misfortunes be avoided\ldots
\end{english}

\section{Bhojana-dānānumodanā}

\firstline{Āyu-do bala-do dhīro vaṇṇa-do paṭibhāṇa-do}

\begin{twochants}
  Āyu-do bala-do dhīro & vaṇṇa-do paṭibhāṇa-do\\
  Sukhassa dātā medhāvī & sukhaṁ so adhigacchati\\
  Āyuṁ datvā balaṁ vaṇṇaṁ & sukhañ-ca paṭibhāna-do\\
  Dīgh'āyu yasavā hoti & yattha yatthūpapajjatī'ti
\end{twochants}

\suttaRef{A.III.42}

\section{Aggappasāda-sutta-gāthā}

\firstline{Aggato ve pasannānaṁ}

\begin{paritta}
  Aggato ve pasannānaṁ\\\vin aggaṁ dhammaṁ vijānataṁ\\
  Agge Buddhe pasannānaṁ\\\vin dakkhiṇeyye anuttare\\
  Agge dhamme pasannānaṁ\\\vin virāgūpasame sukhe\\
  Agge saṅghe pasannānaṁ\\\vin puññakkhette anuttare\\
  Aggasmiṁ dānaṁ dadataṁ\\\vin aggaṁ puññaṁ pavaḍḍhati\\
  Aggaṁ āyu ca vaṇṇo ca\\\vin yaso kitti sukhaṁ balaṁ\\
  Aggassa dātā medhāvī\\\vin agga-dhamma-samāhito\\
  Deva-bhūto manusso vā\\\vin aggappatto pamodatī'ti \suttaRef{A.II.35; A.III.36}
\end{paritta}

\section{Adāsi-me ādi-gāthā}

% Alternative Pali title: Tiro-kuḍḍa-kaṇḍa

\firstline{Adāsi me akāsi me}

\enlargethispage{\baselineskip}

\begin{paritta}
Adāsi me akāsi me\\\vin ñāti-mittā sakhā ca me\\
Petānaṁ dakkhiṇaṁ dajjā\\\vin pubbe katam-anussaraṁ\\
Na hi ruṇṇaṁ vā soko vā\\\vin yā v'aññā paridevanā\\
Na taṁ petānam-atthāya\\\vin evaṁ tiṭṭhanti ñātayo

\firstline{Ayañ-ca kho dakkhiṇā dinnā}

Ayañ-ca kho dakkhiṇā dinnā\\\vin saṅghamhi supatiṭṭhitā\\
Dīgha-rattaṁ hitāy'assa\\\vin ṭhānaso upakappati\\
So ñāti-dhammo ca ayaṁ nidassito\\\vin petāna'pūjā ca katā uḷārā\\
Balañ-ca bhikkhūnam-anuppadinnaṁ\\\vin tumhehi puññaṁ pasutaṁ anappakan'ti.
\end{paritta}

% English source: Bodhivana Vol 2, p.182

\begin{english}
  \setlength{\parskip}{8pt}%
  ``He gave to me, he acted on my behalf, and he was my relative, companion,
  friend.'' Offerings should be given for the dead when one reflects thus on
  what was done in the past. For no weeping or sorrowing or any kind of
  lamentation benefit the dead whose relatives keep acting in that way.

  But when this offering is given, well-placed in the Sangha, it works for their
  long-term benefit and they profit immediately. In this way the proper duty to
  relatives has been shown and great honour has been done to the dead and the
  monks have been given strength: You have acquried merit that is not small.\\ \suttaRef{Khp.VII.v10-13}
\end{english}

\section{Kāla-dāna-sutta-gāthā}

\firstline{Kāle dadanti sapaññā vadaññū vīta-maccharā}

\begin{paritta}
  Kāle dadanti sapaññā\\\vin vadaññū vīta-maccharā\\
  Kālena dinnaṁ ariyesu\\\vin uju-bhūtesu tādisu\\
  Vippasanna-manā tassa\\\vin vipulā hoti dakkhiṇā\\
  Ye tattha anumodanti\\\vin veyyāvaccaṁ karonti vā\\
  Na tena dakkhiṇā onā\\\vin te pi puññassa bhāgino\\
  Tasmā dade appaṭivāna-citto\\\vin yattha dinnaṁ mahapphalaṁ\\
  Puññāni para-lokasmiṁ\\\vin patiṭṭhā honti pāṇinan'ti\\ \suttaRef{A.III.41}
\end{paritta}

\section{Ratanattay'ānubhāv'ādi-gāthā}

\firstline{Ratanattay'ānubhāvena ratanattaya-tejasā}

\begin{paritta}
  Ratanattay'ānubhāvena\\\vin ratanattaya-tejasā\\
  Dukkha-roga-bhayā verā\\\vin sokā sattu c'upaddavā\\
  Anekā antarāyā pi\\\vin vinassantu asesato\\
  Jaya-siddhi dhanaṁ lābhaṁ\\\vin sotthi bhāgyaṁ sukhaṁ balaṁ\\
  Siri āyu ca vaṇṇo ca\\\vin bhogaṁ vuḍḍhī ca yasavā\\
  Sata-vassā ca āyu ca\\\vin jīva-siddhī bhavantu te
\end{paritta}

\clearpage

\section{Culla-maṅgala-cakka-vāḷa}

\enlargethispage{\baselineskip}

\firstline{Sabba-buddh'ānubhāvena}

Sabba-buddh'ānubhāvena\\
sabba-dhamm'ānubhāvena\\
sabba-saṅgh'ānubhāvena

Buddha-ratanaṁ dhamma-ratanaṁ saṅgha-ratanaṁ

Tiṇṇaṁ ratanānaṁ ānubhāvena\\
Catur-āsīti-sahassa-dhammakkhandh'ānubhāvena\\
Piṭakattay'ānubhāvena\\
Jina-sāvak'ānubhāvena

Sabbe te rogā sabbe te bhayā sabbe te antarāyā sabbe te upaddavā sabbe te
dunnimittā sabbe te avamaṅgalā vinassantu

Āyu-vaḍḍhako dhana-vaḍḍhako siri-vaḍḍhako yasa-vaḍḍhako bala-vaḍḍhako
vaṇṇa-vaḍḍhako sukha-vaḍḍhako hotu sabbadā

Dukkha-roga-bhayā verā sokā sattu c'upaddavā\\
Anekā antarāyā pi vinassantu ca tejasā\\
Jaya-siddhi dhanaṁ lābhaṁ\\
Sotthi bhāgyaṁ sukhaṁ balaṁ\\
Siri āyu ca vaṇṇo ca bhogaṁ vuḍḍhī ca yasavā\\
Sata-vassā ca āyū ca jīva-siddhī bhavantu te

Bhavatu sabba-maṅgalaṁ\ldots{}

\section{Mahā-maṅgala-cakka-vāḷa}

\firstline{Siri-dhiti-mati-tejo-jayasiddhi}

Siri-dhiti-mati-tejo-jayasiddhi-mahiddhi-mahāguṇā-parimita-puññādhikarassa
sabbantarāya-nivāraṇa-samatthassa bhagavato arahato sammā-sambuddhassa

Dvattiṁsa-mahā-purisa-lakkhaṇānubhāvena\\
asītyānubyañjanānubhāvena\\
aṭṭhuttara-sata-maṅgalānubhāvena\\
chabbaṇṇa-raṁsiyānubhāvena ketumālānubhāvena\\
dasa-pāramitānubhāvena\\
dasa-upapāramitānubhāvena\\
dasa-paramattha-pāramitānubhāvena\\
sīla-samādhi-paññānubhāvena\\
buddhānubhāvena\\
dhammānubhāvena\\
saṅghānubhāvena\\
tejānubhāvena\\
iddhānubhāvena\\
balānubhāvena\\
ñeyya-dhammānubhāvena\\
caturāsīti-sahassa-dhamma-kkhandhānubhāvena\\
nava-lokuttara-dhammānubhāvena\\
aṭṭhaṅgika-maggānubhāvena\\
aṭṭha-samāpattiyānubhāvena\\
chaḷabhiññānubhāvena\\
catu-sacca-ñāṇānubhāvena\\
dasa-bala-ñāṇānubhāvena\\
sabbaññuta-ñāṇānubhāvena\\
mettā-karuṇā-muditā-upekkhānubhāvena\\
sabba-parittānubhāvena\\
ratanattaya-saraṇānubhāvena\\
tuyhaṁ sabba-roga-sok'upaddava-\\ dukkha-domanass'upāyāsā vinassantu\\
sabba-antarāyā pi vinassantu\\
sabba-saṅkappā tuyhaṁ samijjhantu\\
dīghāyukā tuyhaṁ hotu sata-vassa-jīvena\\
samaṅgiko hotu sabbadā

Ākāsa-pabbata-vana-bhūmi-gaṅgā-mahāsamuddā ārakkhakā
devatā sadā tumhe anurakkhantu

% Text source: Chomtong chanting book

\section{Vihāra-dāna-gāthā}

\begin{paritta}
  Sītaṁ uṇhaṁ paṭihanti\\\vin tato vāḷamigāni ca\\
  sariṁsape ca makase\\\vin sisire cāpi vuṭṭhiyo\\
  Tato vātātapo ghoro\\\vin sañjāto paṭihaññati\\
  Leṇatthañ ca sukhatthañ ca\\\vin jhāyituñ ca vipassituṁ\\
  Vihāradānaṁ saṅghassa\\\vin aggaṁ buddhehi vaṇṇitaṁ\\
  Tasmā hi paṇḍito poso\\\vin sampassaṁ attham attano\\
  Vihāre kāraye ramme\\\vin vāsayettha bahu-ssute\\
  Tesaṁ annañ ca pānañ ca\\\vin vattha-senāsanāni ca\\
  Dadeyya uju-bhūtesu\\\vin vippasannena cetasā\\
  Te tassa dhammaṁ desenti\\\vin sabbadukkhāpanūdanaṁ\\
  Yaṁ so dhammaṁ idh'aññāya\\\vin parinibbātayanāsavo'ti
\end{paritta}

% Source: Chomtong chanting book

% https://suttacentral.net/pli-tv-kd16/en/horner-brahmali

% https://www.dhammatalks.org/books/ChantingGuide/Section0089.html

\begin{english}
  \setlength{\parskip}{8pt}%
  They ward off cold and heat and beasts of prey from there\\
  And creeping things and gnats and rains in the wet season.\\
  When the dreaded hot wind arises, that is warded off.\\
  To meditate and obtain insight in a refuge and at ease:

  A dwelling-place is praised by the Awakened One\\
  \vin as chief gift to an Order.\\
  Therefore a wise man, looking to his own weal,\\
  Should have charming dwelling-places built\\
  So that those who have heard much can stay therein.

  To these food and drink, raiment and lodgings\\
  He should give, to the upright, with mind purified.\\
  (Then) these teach him Dhamma dispelling every ill;\\
  He, knowing that Dhamma,\\
  \vin here attains Nibbāna, free of taints.\\ \suttaRef{Vin.II.147}
\end{english}

% NOTE: The original translation: He, knowing that Dhamma, here attains Nibbāna, canker-less.
%
% Using as: 'free of taints'.

