\chapter{Reflections}

\section{Reflection on the Four Requisites}

\begin{leader}
  [Handa mayaṃ taṅkhaṇika-\\ paccavekkhaṇa-pāṭhaṃ bhaṇāmase]
\end{leader}

[Paṭisaṅkhā] yoniso cīvaraṃ paṭisevāmi,\\
yāvadeva sītassa paṭighātāya, uṇhassa paṭighātāya,\\
ḍaṃsa-makasa-vātātapa-siriṃsapa-samphassānaṃ\\
paṭighātāya, yāvadeva hirikopina-paṭicchādanatthaṃ

[Paṭisaṅkhā] yoniso piṇḍapātaṃ paṭisevāmi, neva davāya, na madāya, na maṇḍanāya,
na vibhūsanāya, yāvadeva imassa kāyassa ṭhitiyā, yāpanāya, vihiṃsūparatiyā,
brahmacariyānuggahāya, iti purāṇañca vedanaṃ paṭihaṅkhāmi, navañca vedanaṃ na
uppādessāmi, yātrā ca me bhavissati anavajjatā ca phāsuvihāro cā'ti

[Paṭisaṅkhā] yoniso senāsanaṃ paṭisevāmi,\\
yāvadeva sītassa paṭighātāya, uṇhassa paṭighātāya,\\
ḍaṃsa-makasa-vātātapa-siriṃsapa-samphassānaṃ\\
paṭighātāya, yāvadeva utuparissaya vinodanaṃ paṭisallānārāmatthaṃ

[Paṭisaṅkhā] yoniso gilāna-paccaya-bhesajja-parikkhāraṃ paṭisevāmi, yāvadeva
uppannānaṃ veyyābādhikānaṃ vedanānaṃ paṭighātāya, abyāpajjha-paramatāyā'ti\\
\suttaRef{M.I.10}

\vspace*{-\baselineskip}

\section{Five Subjects for Frequent Recollection}

\begin{leader}
  [Handa mayaṃ abhiṇha-paccavekkhaṇa-pāṭhaṃ bhaṇāmase]
\end{leader}

\instr{(Men Chant)}\\\relax
[Jarā-dhammomhi] jaraṃ anatīto

\instr{(Women Chant)}\\\relax
[Jarā-dhammāmhi] jaraṃ anatītā

\sidepar{m.}%
Byādhi-dhammomhi byādhiṃ anatīto

\sidepar{w.}%
Byādhi-dhammāmhi byādhiṃ anatītā

\sidepar{m.}%
Maraṇa-dhammomhi maraṇaṃ anatīto

\sidepar{w.}%
Maraṇa-dhammāmhi maraṇaṃ anatītā

Sabbehi me piyehi manāpehi nānābhāvo vinābhāvo

\sidepar{m.}%
Kammassakomhi kammadāyādo kammayoni kammabandhu kammapaṭisaraṇo\\
Yaṃ kammaṃ karissāmi, kalyāṇaṃ vā pāpakaṃ vā, tassa dāyādo bhavissāmi

\sidepar{w.}%
Kammassakāmhi kammadāyādā kammayoni kammabandhu kammapaṭisaraṇā\\
Yaṃ kammaṃ karissāmi, kalyāṇaṃ vā pāpakaṃ vā, tassa dāyādā bhavissāmi

Evaṃ amhehi abhiṇhaṃ paccavekkhitabbaṃ\\
\suttaRef{A.III.71f}

\vspace*{-\baselineskip}

\section{Patti-dāna-gāthā}

\begin{leader}
  [Handa mayaṃ patti-dāna-gāthāyo bhaṇāmase.]
\end{leader}

\enlargethispage{\baselineskip}

Yā devatā santi vihāra-vāsinī\\
Thūpe ghare bodhi-ghare tahiṃ tahiṃ\\
Tā dhamma-dānena bhavantu pūjitā\\
Sotthiṃ karonte'dha vihāra-maṇḍale\\
Therā ca majjhā navakā ca bhikkhavo\\
Sārāmikā dāna-patī upāsakā\\
Gāmā ca desā nigamā ca issarā\\
Sappāṇa-bhūtā sukhitā bhavantu te\\
Jalābu-jā ye pi ca aṇḍa-sambhavā\\
Saṃseda-jātā atha-v-opapātikā\\
Niyyānikaṃ dhamma-varaṃ paṭicca te\\
Sabbe pi dukkhassa karontu saṅkhayaṃ.\\
Ṭhātu ciraṃ sataṃ dhammo\\
Dhamma-dharā ca puggalā\\
Saṅgho hotu samaggo va\\
Atthāya ca hitāya ca\\
Amhe rakkhatu saddhammo\\
Sabbe pi dhamma-cārino\\
Vuḍḍhiṃ sampāpuṇeyyāma\\
Dhamme ariyappavedite.

\clearpage

\sidepar{\pointerMark}%
Pasannā hontu sabbe pi\\
Pāṇino Buddha-sāsane.\\
Sammā-dhāraṃ pavecchanto\\
Kāle devo pavassatu.\\
Vuḍḍhi-bhāvāya sattānaṃ\\
Samiddhaṃ netu medaniṃ.\\
Mātā-pitā ca atra-jaṃ\\
Niccaṃ rakkhanti puttakaṃ.\\
Evaṃ dhammena rājāno\\
Pajaṃ rakkhantu sabbadā.

\section{Atīta-paccavekkhaṇa-pāṭho}

% TODO who is using this? very similar to the other reflection on the requisites

% TODO review punctuation

\begin{leader}
  [Handa mayaṃ atīta-paccavekkhaṇa-pāṭhaṃ bhaṇāmase.]
\end{leader}

Ajja mayā apaccavekkhitvā yaṃ cīvaraṃ paribhuttaṃ,\\
taṃ yāvad-eva sītassa paṭighātāya, uṇhassa paṭighātāya,\\
ḍaṃsa-makasa-vātātapa-siriṃsapa-samphassānaṃ\\
paṭighātāya, yāvad-eva hiri-kopīna paṭicchādan'atthaṃ.

Ajja mayā apaccavekkhitvā yo piṇḍapāto paribhutto, so n'eva davāya na madāya na
maṇḍanāya na vibhūsanāya, yāvad-eva imassa kāyassa ṭhitiyā yāpanāya
vihiṃsūparatiyā brahma-cariyānuggahāya, iti purāṇañ-ca vedanaṃ paṭihaṅkhāmi,
navañ-ca vedanaṃ na uppādessāmi, yātrā ca me bhavissati anavajjatā ca
phāsuvihāro cā-ti.

Ajja mayā apaccavekkhitvā yaṃ senāsanaṃ paribhuttaṃ, taṃ yāvad-eva sītassa
paṭighātāya, uṇhassa paṭighātāya, ḍaṃsa-makasa-vātātapa-siriṃsapasamphassānaṃ
paṭighātāya, yāvad-eva utu-parissaya-vinodanaṃ paṭisallān'ārām'atthaṃ.

Ajja mayā apaccavekkhitvā yo gilāna-paccayabhesajja-\\ parikkhāro paribhutto, so
yāvad-eva uppannānaṃ veyyābādhikānaṃ Vedanānaṃ paṭighātāya,\\
abyāpajjha-paramatāyā-ti. \suttaRef{cf. M.I.10}

\section{Reflection on the Off-Putting Qualities of the Requisites}

\begin{leader}
  [Handa mayaṃ dhātu-paṭikūla-\\ paccavekkhaṇa-pāṭhaṃ bhaṇāmase]
\end{leader}

[Yathā paccayaṃ] pavattamānaṃ dhātu-mattam-ev'etaṃ\\
Yad idaṃ cīvaraṃ tad upabhuñjako ca puggalo\\
Dhātu-mattako, nissatto, nijjīvo, suñño\\
Sabbāni pana imāni cīvarāni ajigucchanīyāni\\
Imaṃ pūti-kāyaṃ patvā, ativiya jigucchanīyāni jāyanti\\
Yathā paccayaṃ pavattamānaṃ dhātu-mattam-ev'etaṃ\\
Yad idaṃ piṇḍapāto tad upabhuñjako ca puggalo\\
Dhātu-mattako, nissatto, nijjīvo, suñño\\
Sabbo panāyaṃ piṇḍapāto ajigucchanīyo\\
Imaṃ pūti-kāyaṃ patvā, ativiya jigucchanīyo jāyati\\
Yathā paccayaṃ pavattamānaṃ dhātu-mattam-ev'etaṃ\\
Yad idaṃ senāsanaṃ tad upabhuñjako ca puggalo\\
Dhātu-mattako, nissatto, nijjīvo, suñño\\
Sabbāni pana imāni senāsanāni ajigucchanīyāni\\
Imaṃ pūti-kāyaṃ patvā, ativiya jigucchanīyāni jāyanti\\
Yathā paccayaṃ pavattamānaṃ dhātu-mattam-ev'etaṃ\\
Yad idaṃ gilāna-paccaya-bhesajja-parikkhāro tad upabhuñjako ca puggalo\\
Dhātu-mattako, nissatto, nijjīvo, suñño\\
Sabbo panāyaṃ\\
gilāna-paccaya-bhesajja-parikkhāro ajigucchanīyo\\
Imaṃ pūti-kāyaṃ patvā, ativiya jigucchanīyo jāyati

\enlargethispage{\baselineskip}

\section{Reflection on Universal Well-Being}

\begin{leader}
  [Handa mayam mettāpharaṇaṃ karomase]
\end{leader}

[Ahaṃ sukhito homi] niddukkho homi, avero homi, abyāpajjho homi, anīgho homi,
sukhī attānaṃ pariharāmi

Sabbe sattā sukhitā hontu, sabbe sattā averā hontu, sabbe sattā abyāpajjhā
hontu, sabbe sattā anīghā hontu, sabbe sattā sukhī attānaṃ pariharantu. Sabbe
sattā sabbadukkhā pamuccantu, sabbe sattā\\
laddha-sampattito mā vigacchantu

Sabbe sattā kammassakā kammadāyādā kammayonī kammabandhū kammapaṭisaraṇā,
yaṃ kammaṃ karissanti, kalyāṇaṃ vā pāpakaṃ vā, tassa dāyādā bhavissanti \suttaRef{M.I.288}

\section{Reflection on the Thirty-Two Parts}

\begin{leader}
  [Handa mayaṃ dvattiṃsākāra-pāṭhaṃ bhaṇāmase]
\end{leader}

[Ayaṃ kho] me kāyo uddhaṃ pādatalā adho kesamatthakā\\
tacapariyanto pūro nānappakārassa asucino

Atthi imasmiṃ kāye

kesā, lomā, nakhā, dantā, taco, maṃsaṃ, nahārū, aṭṭhī, aṭṭhimiñjaṃ, vakkaṃ, hadayaṃ, yakanaṃ, kilomakaṃ, pihakaṃ, papphāsaṃ, antaṃ, antaguṇaṃ, udariyaṃ, karīsaṃ, pittaṃ, semhaṃ, pubbo, lohitaṃ, sedo, medo, assu, vasā, kheḷo, siṅghāṇikā, lasikā, muttaṃ, matthaluṅgan'ti 

Evam-ayaṃ me kāyo uddhaṃ pādatalā adho kesamatthakā\\
tacapariyanto pūro nānappakārassa asucino\\
\suttaRef{cf. M.I.57}

\vspace*{-\baselineskip}

\section{Patti-dāna-gāthā}

% TODO title is the same as with 'Yā devatā...' above

\begin{tabularx}{\linewidth}{@{}l l@{}}
Puññass'idāni katassa & yān'aññāni katāni me,\\
Tesañ-ca bhāgino hontu & sattānantāppamāṇaka.\\
Ye piyā guṇavantā ca & mayhaṃ mātā-pitā-dayo.\\
Diṭṭhā me cāpyadiṭṭhā vā & aññe majjhatta-verino;\\
Sattā tiṭṭhanti lokasmiṃ & te bhummā catu-yonikā.\\
Pañc'eka-catu-vokārā & saṃsarantā bhavābhave:\\
Ñātaṃ ye patti-dānam-me, & anumodantu te sayaṃ.\\
Ye c'imaṃ nappajānanti & devā tesaṃ nivedayuṃ.\\
Mayā dinnāna-puññānaṃ & anumodana-hetunā.\\
Sabbe sattā sadā hontu & averā sukha-jīvino.\\
Khemappadañ-ca pappontu & tesāsā sijjhataṃ subhā.\\
% \end{tabularx}
%
% FIXME add the mark
% 
% \sidepar{\pointerMark}
% 
% \begin{tabularx}{\linewidth}{@{}l l@{}}
Yan-dāni me kataṃ puññaṃ & tenānen'uddisena ca,\\
Khippaṃ sacchikareyyāhaṃ & dhamme lok'uttare nava.\\
Sace tāva abhabbo'haṃ & saṃsāre pana saṃsaraṃ,\\
Niyato bodhi-satto va & sambuddhena viyākato.\\
Nāṭṭhārasa pi abhabba & ṭhānāni pāpuṇeyy'ahaṃ.\\
Manussattañ-ca liṅgañ-ca & pabbajjañ-c'upasampadaṃ.\\
Labhitvā pesalo sīlī & dhāreyyaṃ satthu sāsanaṃ,\\
Sukhā-paṭipado khippābhiñño & sacchikareyyahaṃ.\\
Arahatta-phalaṃ aggaṃ & vijj'ādi-guṇ'alaṅ-kataṃ,\\
Yadi n'uppajjati Buddho & kammaṃ paripūrañ-ca me,\\
Evaṃ sante labheyyāhaṃ & pacceka-bodhim-uttaman-ti.
\end{tabularx}

\section{Verses of Sharing and Aspiration}

\begin{leader}
  [Handa mayaṃ uddissanādhiṭṭhāna-gāthāyo bhaṇāmase]
\end{leader}

[Iminā puññakammena] upajjhāyā guṇuttarā\\
Ācariyūpakārā ca mātāpitā ca ñātakā\\
Suriyo candimā rājā guṇavantā narāpi ca\\
Brahma-mārā ca indā ca lokapālā ca devatā\\
Yamo mittā manussā ca majjhattā verikāpi ca\\
Sabbe sattā sukhī hontu puññāni pakatāni me\\
Sukhañca tividhaṃ dentu khippaṃ pāpetha vomataṃ\\
Iminā puññakammena iminā uddissena ca\\
Khipp'āhaṃ sulabhe ceva taṇhūpādāna-chedanaṃ\\
Ye santāne hīnā dhammā yāva nibbānato mamaṃ\\
Nassantu sabbadā yeva yattha jāto bhave bhave\\
Ujucittaṃ satipaññā sallekho viriyamhinā\\
Mārā labhantu nokāsaṃ kātuñca viriyesu me\\
Buddhādhipavaro nātho dhammo nātho varuttamo\\
Nātho paccekabuddho ca saṅgho nāthottaro mamaṃ\\
Tesottamānubhāvena mārokāsaṃ labhantu mā

\section{Sabbe sattā sadā hontu}

Sabbe sattā sadā hontu\\
\vin averā sukha-jīvino\\
Kataṃ puñña-phalaṃ mayhaṃ\\
\vin sabbe bhāgī bhavantu te

\section{Ti-loka-vijaya-rāja-patti-dāna-gāthā}

Yaṅ kiñci kusalaṃ kammaṃ\\
\vin kattabbaṃ kiriyaṃ mama\\
Kāyena vācā manasā\\
\vin ti-dase sugataṃ kataṃ\\
Ye sattā saññino atthi\\
\vin ye ca sattā asaññino\\
Kataṃ puñña-phalaṃ mayhaṃ\\
\vin sabbe bhāgī bhavantu te\\
Ye taṃ kataṃ suviditaṃ\\
\vin dinnaṃ puñña-phalaṃ mayā\\
Ye ca tattha na jānanti\\
\vin devā gantvā nivedayuṃ\\
Sabbe lokamhi ye sattā\\
\vin jīvant'āhāra-hetukā\\
Manuññaṃ bhojanaṃ sabbe\\
\vin labhantu mama cetasā.

\suttaRef{Apadāna 4}

