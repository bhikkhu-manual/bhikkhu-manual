\chapter{Evening Reflections}

\chapter{Atīta-paccavekkhaṇa-pāṭho}

[Handa mayaṃ atīta-paccavekkhaṇa-pāṭhaṃ bhaṇāmase.]

Ajja mayā apaccavekkhitvā yaṃ cīvaraṃ
paribhuttaṃ,
Taṃ yāvad-eva sītassa paṭighātāya,

Uṇhassa paṭighātāya,
Ḍaṃsa-makasa-vātātapa-siriṃsapasamphassānaṃ paṭighātāya,
Yāvad-eva hiri-kopīna paṭicchādan’atthaṃ.
Ajja mayā apaccavekkhitvā yo piṇḍapāto
paribhutto,
So n’eva davāya na madāya na maṇḍanāya na
vibhūsanāya,
Yāvad-eva imassa kāyassa ṭhitiyā yāpanāya
vihiṃsūparatiyā brahma-cariyānuggahāya,
Iti purāṇañ-ca vedanaṃ paṭihaṅkhāmi,
Navañ-ca vedanaṃ na uppādessāmi,
Yātrā ca me bhavissati anavajjatā ca phāsuvihāro cā-ti.
Ajja mayā apaccavekkhitvā yaṃ senāsanaṃ
paribhuttaṃ,
Taṃ yāvad-eva sītassa paṭighātāya,
Uṇhassa paṭighātāya,
Ḍaṃsa-makasa-vātātapa-siriṃsapasamphassānaṃ paṭighātāya,
Yāvad-eva utu-parissaya-vinodanaṃ
paṭisallān’ārām’atthaṃ.
Ajja mayā apaccavekkhitvā yo gilāna-paccayabhesajja-parikkhāro paribhutto,
So yāvad-eva uppannānaṃ veyyābādhikānaṃ
vedanānaṃ paṭighātāya,
[cf. M.I.10]
Abyāpajjha-paramatāyā-ti.

\chapter{Reflection on the Off-Putting Qualities of the Requisites}

\begin{leader}
  [Handa mayaṃ dhātu-paṭikūla-paccavekkhaṇa-pāṭhaṃ bhaṇāmase]
\end{leader}

[Yathā paccayaṃ] pavattamānaṃ dhātu-mattam-ev'etaṃ\\
Yad idaṃ cīvaraṃ tad upabhuñjako ca puggalo\\
Dhātu-mattako\\
Nissatto\\
Nijjīvo\\
Suñño\\
Sabbāni pana imāni cīvarāni ajigucchanīyāni\\
Imaṃ pūti-kāyaṃ patvā\\
Ativiya jigucchanīyāni jāyanti\\
Yathā paccayaṃ pavattamānaṃ dhātu-mattam-ev'etaṃ\\
Yad idaṃ piṇḍapāto tad upabhuñjako ca puggalo\\
Dhātu-mattako\\
Nissatto\\
Nijjīvo\\
Suñño\\
Sabbo panāyaṃ piṇḍapāto ajigucchanīyo\\
Imaṃ pūti-kāyaṃ patvā\\
Ativiya jigucchanīyo jāyati\\
Yathā paccayaṃ pavattamānaṃ dhātu-mattam-ev'etaṃ\\
Yad idaṃ senāsanaṃ tad upabhuñjako ca puggalo\\
Dhātu-mattako\\
Nissatto\\
Nijjīvo\\
Suñño\\
Sabbāni pana imāni senāsanāni ajigucchanīyāni\\
Imaṃ pūti-kāyaṃ patvā\\
Ativiya jigucchanīyāni jāyanti\\
Yathā paccayaṃ pavattamānaṃ dhātu-mattam-ev'etaṃ\\
Yad idaṃ gilāna-paccaya-bhesajja-parikkhāro tad upabhuñjako ca puggalo\\
Dhātu-mattako\\
Nissatto\\
Nijjīvo\\
Suñño\\
Sabbo panāyaṃ gilāna-paccaya-bhesajja-parikkhāro ajigucchanīyo\\
Imaṃ pūti-kāyaṃ patvā\\
Ativiya jigucchanīyo jāyati

\chapter{Reflection on Universal Well-Being}

\begin{leader}
  [Handa mayam mettāpharaṇaṃ karomase]
\end{leader}

[Ahaṃ sukhito homi]\\
Niddukkho homi\\
Avero homi\\
Abyāpajjho homi\\
Anīgho homi\\
Sukhī attānaṃ pariharāmi

Sabbe sattā sukhitā hontu\\
Sabbe sattā averā hontu\\
Sabbe sattā abyāpajjhā hontu\\
Sabbe sattā anīghā hontu\\
Sabbe sattā sukhī attānaṃ pariharantu

Sabbe sattā sabbadukkhā pamuccantu

Sabbe sattā laddha-sampattito mā vigacchantu

Sabbe sattā kammassakā kammadāyādā kammayonī\\
\vin kammabandhū kammapaṭisaraṇā\\
Yaṃ kammaṃ karissanti\\
Kalyāṇaṃ vā pāpakaṃ vā\\
Tassa dāyādā bhavissanti

\chapter{Reflection on the Thirty-Two Parts}

\begin{leader}
  [Handa mayaṃ dvattiṃsākāra-pāṭhaṃ bhaṇāmase]
\end{leader}

[Ayaṃ kho] me kāyo uddhaṃ pādatalā adho kesamatthakā\\
tacapariyanto pūro nānappakārassa asucino

Atthi imasmiṃ kāye

kesā, lomā, nakhā, dantā, taco, maṃsaṃ, nahārū, aṭṭhī, aṭṭhimiñjaṃ, vakkaṃ, hadayaṃ, yakanaṃ, kilomakaṃ, pihakaṃ, papphāsaṃ, antaṃ, antaguṇaṃ, udariyaṃ, karīsaṃ, pittaṃ, semhaṃ, pubbo, lohitaṃ, sedo, medo, assu, vasā, kheḷo, siṅghāṇikā, lasikā, muttaṃ, matthaluṅgan'ti 

Evam-ayaṃ me kāyo uddhaṃ pādatalā adho kesamatthakā\\
tacapariyanto pūro nānappakārassa asucino

\chapter{Patti-dāna-gāthā}

Puññass’idāni katassa\\
Yān’aññāni katāni me,\\
Tesañ-ca bhāgino hontu\\
Sattānantāppamāṇaka.\\
Ye piyā guṇavantā ca\\
Mayhaṃ mātā-pitā-dayo.\\
Diṭṭhā me cāpyadiṭṭhā vā\\
Aññe majjhatta-verino;\\
Sattā tiṭṭhanti lokasmiṃ.\\
Te bhummā catu-yonikā.\\
Pañc’eka-catu-vokārā.\\
Saṃsarantā bhavābhave:\\
Ñātaṃ ye patti-dānam-me,\\
Anumodantu te sayaṃ.\\
Ye c’imaṃ nappajānanti\\
Devā tesaṃ nivedayuṃ.\\
Mayā dinnāna-puññānaṃ\\
Anumodana-hetunā.\\
Sabbe sattā sadā hontu\\
Averā sukha-jīvino.\\
Khemappadañ-ca pappontu.\\
Tesāsā sijjhataṃ subhā.

(♦) Yan-dāni me kataṃ puññaṃ\\
Tenānen’uddisena ca,\\
Khippaṃ sacchikareyyāhaṃ\\
Dhamme lok’uttare nava.\\
Sace tāva abhabbo’haṃ\\
Saṃsāre pana saṃsaraṃ,\\
Niyato bodhi-satto va\\
Sambuddhena viyākato.\\
Nāṭṭhārasa pi abhabbaṬhānāni pāpuṇeyy’ahaṃ.\\
Manussattañ-ca liṅgañ-ca\\
Pabbajjañ-c’upasampadaṃ.\\
Labhitvā pesalo sīlī\\
Dhāreyyaṃ satthu sāsanaṃ,\\
Sukhā-paṭipado khippābhiñño sacchikareyyahaṃ.\\
Arahatta-phalaṃ aggaṃ\\
Vijj’ādi-guṇ’alaṅ-kataṃ,\\
Yadi n’uppajjati Buddho\\
Kammaṃ paripūrañ-ca me,\\
Evaṃ sante labheyyāhaṃ\\
Pacceka-bodhim-uttaman-ti.

\chapter{Verses of Sharing and Aspiration}

\begin{leader}
  [Handa mayaṃ uddissanādhiṭṭhāna-gāthāyo bhaṇāmase]
\end{leader}

[Iminā puññakammena] upajjhāyā guṇuttarā\\
Ācariyūpakārā ca mātāpitā ca ñātakā\\
Suriyo candimā rājā guṇavantā narāpi ca\\
Brahma-mārā ca indā ca lokapālā ca devatā\\
Yamo mittā manussā ca majjhattā verikāpi ca\\
Sabbe sattā sukhī hontu puññāni pakatāni me\\
Sukhañca tividhaṃ dentu khippaṃ pāpetha vomataṃ\\
Iminā puññakammena iminā uddissena ca\\
Khipp'āhaṃ sulabhe ceva taṇhūpādāna-chedanaṃ\\
Ye santāne hīnā dhammā yāva nibbānato mamaṃ\\
Nassantu sabbadā yeva yattha jāto bhave bhave\\
Ujucittaṃ satipaññā sallekho viriyamhinā\\
Mārā labhantu nokāsaṃ kātuñca viriyesu me\\
Buddhādhipavaro nātho dhammo nātho varuttamo\\
Nātho paccekabuddho ca saṅgho nāthottaro mamaṃ\\
Tesottamānubhāvena mārokāsaṃ labhantu mā

\chapter{Sabbe sattā sadā hontu}

Sabbe sattā sadā hontu\\
Averā sukha-jīvino.\\
Kataṃ puñña-phalaṃ mayhaṃ\\
Sabbe bhāgī bhavantu te.

\chapter{Ti-loka-vijaya-rāja-patti-dāna-gāthā}

Yaṅ kiñci kusalaṃ kammaṃ\\
Kattabbaṃ kiriyaṃ mama\\
Kāyena vācā manasā\\
Ti-dase sugataṃ kataṃ\\
Ye sattā saññino atthi\\
Ye ca sattā asaññino\\
Kataṃ puñña-phalaṃ mayhaṃ\\
Sabbe bhāgī bhavantu te\\
Ye taṃ kataṃ suviditaṃ\\
Dinnaṃ puñña-phalaṃ mayā\\
Ye ca tattha na jānanti\\
Devā gantvā nivedayuṃ\\
Sabbe lokamhi ye sattā\\
Jīvant’āhāra-hetukā\\
Manuññaṃ bhojanaṃ sabbe\\
Labhantu mama cetasā.

