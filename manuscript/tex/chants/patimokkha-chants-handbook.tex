\chapter{Pāṭimokkha Chants}

\section{Ovāda-pāṭimokkha-gāthā}

\begin{leader}
  [Handa mayaṃ ovāda-pāṭimokkha-gāthāyo bhaṇāmase]
\end{leader}

\firstline{Sabba-pāpassa akaraṇaṃ}
\firstline{Khantī paramaṃ tapo tītikkhā}

Sabba-pāpassa akaraṇaṃ\\
Kusalassūpasampadā\\
Sacitta-pariyodapanaṃ\\
Etaṃ buddhāna sāsanaṃ\\
Khantī paramaṃ tapo tītikkhā\\
Nibbānaṃ paramaṃ vadanti buddhā\\
Na hi pabbajito parūpaghātī\\
Samaṇo hoti paraṃ viheṭhayanto\\
Anūpavādo anūpaghāto\\
Pāṭimokkhe ca saṃvaro\\
Mattaññutā ca bhattasmiṃ\\
Pantañca sayan'āsanaṃ\\
Adhicitte ca āyogo\\
Etaṃ buddhāna sāsanaṃ

\suttaRef{Dhp 183-185}

\subsubsection{Verses on the Training Code}

Not doing any evil;\\
To be committed to the good;\\
To purify one's mind:\\
These are the teachings of all Buddhas.

Patient endurance is the highest practice,\\
\vin burning out defilements;\\
The Buddhas say Nibbāna is supreme.\\
Not a renunciant is one who injures others;\\
Whoever troubles others can't be called a monk.\\

Not to insult and not to injure;\\
To live restrained by training rules;\\
Knowing one's measure at the meal;\\
Retreating to a lonely place;\\
Devotion to the higher mind:\\
These are the teachings of all Buddhas.

\section{Sacca-kiriyā-gāthā}

\begin{leader}
  [Handa mayaṃ sacca-kiriyā-gāthāyo bhaṇāmase]
\end{leader}

\firstline{Natthi me saraṇaṃ aññaṃ}

Natthi me saraṇaṃ aññaṃ buddho me saraṇaṃ varaṃ\\
Etena sacca-vajjena sotthi me hotu sabbadā

Natthi me saraṇaṃ aññaṃ dhammo me saraṇaṃ varaṃ\\
Etena sacca-vajjena sotthi me hotu sabbadā

Natthi me saraṇaṃ aññaṃ saṅgho me saraṇaṃ varaṃ\\
Etena sacca-vajjena sotthi me hotu sabbadā

% English source: Mahamakut Patimokkha, p.138

\begin{english}
  For me there is no other Refuge, the Buddha \ldots\ Dhamma \ldots\ Sangha is
  my excellent refuge. By the utterance of this Truth, may there be blessings
  for me.
\end{english}

\section{Sīl'uddesa-pāṭho}

\begin{leader}
  [Handa mayaṃ sīl'uddesa-pāṭhaṃ bhaṇāmase]
\end{leader}

\firstline{Bhāsitam idaṃ tena bhagavatā jānatā passatā}

Bhāsitam idaṃ tena bhagavatā jānatā passatā\\
arahatā sammā-sambuddhena\\
Sampanna-sīlā bhikkhave viharatha\\
sampanna-pāṭimokkhā\\
Pāṭimokkha-saṃvara-saṃvutā viharatha\\
ācāra-gocara-sampannā\\
Aṇu-mattesu vajjesu bhaya-dassāvī\\
samādāya sikkhatha sikkhāpadesū'ti

% English source: Mahamakut Patimokkha, p.138

\begin{english}
  This has been said by the Lord, One-who-knows, One-who-sees, the Arahant, the
  Perfect Buddha enlightened by himself: `Bhikkhus, be perfect in moral
  conduct. Be perfect in the Pāṭimokkha. Dwell restrained in accordance with the
  the Pāṭimokkha. Be perfect in conduct and resort, seeing danger even in the
  slightest faults. Train yourselves by undertaking rightly the rules of training.'
\end{english}

Tasmā-tih'amhehi sikkhitabbaṃ\\
Sampanna-sīlā viharissāma sampanna-pāṭimokkhā\\
Pāṭimokkha-saṃvara-saṃvutā viharissāma\\
ācāra-gocara-sampannā\\
Aṇu-mattesu vajjesu bhaya-dassāvī\\
samādāya sikkhissāma sikkhāpadesū'ti\\
Evañ hi no sikkhitabbaṃ

\begin{english}
  Therefore we should train ourselves thus: `We will be perfect in the
  Pāṭimokkha. We will dwell restrained in accordance with the Pāṭimokkha. We
  will be perfect in conduct and resort, seeing danger even in the slightest
  faults.' Thus indeed we should train ourselves.
\end{english}

\suttaRef{D.I.63; D.III.266f}

\section{Tāyana-gāthā}

\begin{leader}
  [Handa mayaṃ tāyana-gāthāyo bhaṇāmase]
\end{leader}

\firstline{Chinda sotaṃ parakkamma}

Chinda sotaṃ parakkamma\\
Kāme panūda brāhmaṇa\\
Nappahāya muni kāme\\
N'ekattam-upapajjati

Kayirā ce kayirāthenaṃ\\
Daḷham-enaṃ parakkame\\
Sithilo hi paribbājo\\
Bhiyyo ākirate rajaṃ

Akataṃ dukkaṭaṃ seyyo\\
Pacchā tappati dukkaṭaṃ\\
Katañca sukataṃ seyyo\\
Yaṃ katvā nānutappati

Kuso yathā duggahito\\
Hattham-evānukantati\\
Sāmaññaṃ dupparāmaṭṭhaṃ\\
Nirayāyūpakaḍḍhati

Yaṃ kiñci sithilaṃ kammaṃ\\
Saṅkiliṭṭhañca yaṃ vataṃ\\
Saṅkassaraṃ brahma-cariyaṃ\\
Na taṃ hoti mahapphalan'ti

\suttaRef{S.I.49f}

\subsubsection{The Verses of Tāyana}

Exert yourself and cut the stream.\\
Discard sense pleasures, brahmin;\\
Not letting sensual pleasures go,\\
A sage will not reach unity.

Vigorously, with all one's strength,\\
It should be done, what should be done;\\
A lax monastic life stirs up\\
The dust of passions all the more.

Better is not to do bad deeds\\
That afterwards would bring remorse;\\
It's rather good deeds one should do\\
Which having done one won't regret.

As Kusa-grass, when wrongly grasped,\\
Will only cut into one's hand\\
So does the monk's life wrongly led\\
Indeed drag one to hellish states.

Whatever deed that's slackly done,\\
Whatever vow corruptly kept,\\
The Holy Life led in doubtful ways ---\\
All these will never bear great fruit.

\section{Sāmaṇera-sikkhā}

\firstline{Anuññāsi kho bhagavā sāmaṇerānaṃ dasa}

Anuññāsi kho bhagavā\\
Sāmaṇerānaṃ dasa sikkhā-padāni

\begin{english}
  Ten novice training rules\\
  were established by the Blessed One.
\end{english}

Tesu ca sāmaṇerehi sikkhituṃ

\begin{english}
  They are the things in which a novice should train:
\end{english}

Pāṇātipātā veramaṇī

\begin{english}
  Abstaining from killing living beings
\end{english}

Adinn'ādānā veramaṇī

\begin{english}
  Abstaining from taking what is not given
\end{english}

Abrahma-cariyā veramaṇī

\begin{english}
  Abstaining from unchastity
\end{english}

Musā-vādā veramaṇī

\begin{english}
  Abstaining from false speech
\end{english}

Surā-meraya-majja-pamādaṭṭhānā veramaṇī

\begin{english}
  Abstaining from intoxicants that dull the mind
\end{english}

Vikāla-bhojanā veramaṇī

\begin{english}
  Abstaining from eating at the wrong time
\end{english}

Nacca-gīta-vādita-visūka-dassanā veramaṇī

\begin{english}
  Abstaining from dancing, singing, music and watching shows
\end{english}

Mālā-gandha-vilepana-dhāraṇa-\\
\vin maṇḍana-vibhūsanaṭṭhānā veramaṇī

\begin{english}
  Abstaining from perfumes, beautification and adornment
\end{english}

Uccā-sayana-mahā-sayanā veramaṇī

\begin{english}
  Abstaining from lying on high or luxurious beds
\end{english}

Jāta-rūpa-rajata-paṭiggahaṇā veramaṇī'ti.

\begin{english}
  Abstaining from using gold, silver or money.
\end{english}

\suttaRef{Vin.I.83f}

Anuññāsi kho Bhagavā\\
Dasahi aṅgehi samannāgataṃ sāmaṇeraṃ nāsetuṃ

\begin{english}
  Ten grounds for a novice to be dismissed\\
  were established by the Blessed One.
\end{english}

Katamehi dasahi

\begin{english}
  What are these ten?
\end{english}

Pāṇātipātī hoti

\begin{english}
  He is a killer of living beings
\end{english}

Adinn'ādāyī hoti

\begin{english}
  He is a taker of what is not given
\end{english}

Abrahma-cārī hoti

\begin{english}
  He is a practicioner of unchastity
\end{english}

Musā-vādī hoti

\begin{english}
  He is a speaker of falsity
\end{english}

Majja-pāyī hoti

\begin{english}
  He is a consumer of intoxicants
\end{english}

Buddhassa avaṇṇaṃ bhāsati

\begin{english}
  He speaks in dispraise of the Buddha
\end{english}

Dhammassa avaṇṇaṃ bhāsati

\begin{english}
  He speaks in dispraise of the Dhamma
\end{english}

Saṅghassa avaṇṇaṃ bhāsati

\begin{english}
  He speaks in dispraise of the Saṅgha
\end{english}

Micchā-diṭṭhiko hoti

\begin{english}
  He is a holder of wrong views
\end{english}

Bhikkhunī-dūsako hoti

\begin{english}
  He has corrupted a nun
\end{english}

Anuññāsi kho Bhagavā\\
Imehi dasahi aṅgehi samannāgataṃ sāmaṇeraṃ nāsetun'ti.

\begin{english}
  These are the ten grounds for a novice to be dismissed\\
  which were established by the Blessed One.
\end{english}

\suttaRef{Vin.I.85}

Anuññāsi kho Bhagavā\\
Pañcahi aṅgehi samannāgatassa sāmaṇerassa daṇḍa-kammaṃ kātuṃ

\begin{english}
  Five grounds for a novice to be punished\\
  were established by the Blessed One.
\end{english}

Katamehi pañcahi

\begin{english}
  What are these five?
\end{english}

Bhikkhūnaṃ alābhāya parisakkati

\begin{english}
  He strives for the loss of the Bhikkhus
\end{english}

Bhikkhūnaṃ anatthāya parisakkati

\begin{english}
  He strives for the non-benefit of the Bhikkhus
\end{english}

Bhikkhūnaṃ anāvāsāya parisakkati

\begin{english}
  He strives for the non-residence of the Bhikkhus
\end{english}

Bhikkhū akkosati paribhāsati

\begin{english}
  He insults or abuses the Bhikkhus
\end{english}

Bhikkhū bhikkhūhi bhedeti

\begin{english}
  He causes a split between the Bhikkhus
\end{english}

Anuññāsi kho Bhagavā\\
Imehi pañcahi aṅgehi samannāgatassa\\
sāmaṇerassa daṇḍa-kammaṃ kātun'ti

\begin{english}
  These are the ten grounds for a novice to be punished\\
  that were established by the Blessed One.
\end{english}

\suttaRef{Vin.I.84}

