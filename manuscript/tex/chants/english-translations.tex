\chapter{English Translations}

\section{Reflection on the Four Requisites}

Wisely reflecting, I use the robe: only to ward off cold, to ward off heat, to ward off the touch of flies, mosquitoes, wind, burning and creeping things, only for the sake of modesty.

Wisely reflecting, I use almsfood: not for fun, not for pleasure, not for fattening, not for beautification, only for the maintenance and nourishment of this body, for keeping it healthy, for helping with the Holy Life; thinking thus, `I will allay hunger without overeating, so that I may continue to live blamelessly and at ease.'

Wisely reflecting, I use the lodging: only to ward off cold, to ward off heat, to ward off the touch of flies, mosquitoes, wind, burning and creeping things, only to remove the danger from weather, and for living in seclusion.

Wisely reflecting, I use supports for the sick and medicinal requisites: only to ward off painful feelings that have arisen, for the maximum freedom from disease.

\suttaRef{M.I.10}

\section{Reflection on Universal Well-Being}

\begin{leader}
[Now let us chant the reflections on universal well-being.]
\end{leader}

[May I abide in well-being,]\\
In freedom from affliction,\\
In freedom from hostility,\\
In freedom from ill-will,\\
In freedom from anxiety,\\
And may I maintain well-being in myself.

May everyone abide in well-being,\\
In freedom from hostility,\\
In freedom from ill-will,\\
In freedom from anxiety, and may they\\
Maintain well-being in themselves.

May all beings be released from all suffering.

And may they not be parted from the good fortune they have attained.

When they act upon intention,\\
All beings are the owners of their action and inherit its results.\\
Their future is born from such action, companion to such action,\\
And its results will be their home.

All actions with intention,\\
Be they skilful or harmful ---\\
Of such acts they will be the heirs.

\suttaRef{cf. M.I.288; A.V.88}

\section{Verses of Sharing and Aspiration}

\begin{leader}
[Now let us chant the verses of sharing and aspiration.]
\end{leader}

Through the goodness that arises from my practice,\\
May my spiritual teachers and guides of great virtue,\\
My mother, my father, and my relatives,\\
The Sun and the Moon, and all virtuous leaders of the world,\\
May the highest gods and evil forces,\\
Celestial beings, guardian spirits of the Earth, and the Lord of Death,\\
May those who are friendly, indifferent, or hostile,\\
May all beings receive the blessings of my life,\\
May they soon attain the threefold bliss and realize the Deathless.\\
Through the goodness that arises from my practice,\\
And through this act of sharing,\\
May all desires and attachments quickly cease\\
And all harmful states of mind.\\
Until I realize Nibbāna,\\
In every kind of birth, may I have an upright mind,\\
With mindfulness and wisdom, austerity and vigour.\\
May the forces of delusion not take hold nor weaken my resolve.\\
The Buddha is my excellent refuge,\\
Unsurpassed is the protection of the Dhamma,\\
The Solitary Buddha is my noble guide,\\
The Saṅgha is my supreme support.\\
Through the supreme power of all these,\\
May darkness and delusion be dispelled.

\section{The Buddha's Words on Loving-Kindness}

\begin{leader}
  [Now let us chant the Buddha's words on loving-kindness.]
\end{leader}

[This is what should be done]\\
By one who is skilled in goodness\\
And who knows the path of peace:\\
Let them be able and upright,\\
Straightforward and gentle in speech,

Humble and not conceited,\\
Contented and easily satisfied,\\
Unburdened with duties and frugal in their ways.\\
Peaceful and calm, and wise and skilful,\\
Not proud and demanding in nature.

Let them not do the slightest thing\\
That the wise would later reprove,\\
Wishing: In gladness and in safety,\\
May all beings be at ease.

Whatever living beings there may be,\\
Whether they are weak or strong, omitting none,\\
The great or the mighty, medium, short, or small,

The seen and the unseen,\\
Those living near and far away,\\
Those born and to be born,\\
May all beings be at ease.

Let none deceive another\\
Or despise any being in any state.\\
Let none through anger or ill-will\\
Wish harm upon another.

Even as a mother protects with her life\\
Her child, her only child,\\
So with a boundless heart\\
Should one cherish all living beings,\\
Radiating kindness over the entire world:

Spreading upwards to the skies\\
And downwards to the depths,\\
Outwards and unbounded,\\
Freed from hatred and ill-will.

Whether standing or walking, seated, \\
Or lying down --- free from drowsiness ---\\
One should sustain this recollection.\\
This is said to be the sublime abiding.

By not holding to fixed views,\\
The pure-hearted one, having clarity of vision,\\
Being freed from all sense-desires,\\
Is not born again into this world.

\suttaRef{Sn.vv143–152}

\section{Ten Subjects for Frequent Recollection by One Who Has Gone Forth}

Bhikkhus, there are ten dhammas which should be reflected upon again and again by one who has gone forth. What are these ten?

`I am no longer living according to worldly aims and values.'\\
This should be reflected upon again and again\\
by one who has gone forth.

`My very life is sustained through the gifts of others.'\\
This should be reflected upon again and again\\
by one who has gone forth.

`I should strive to abandon my former habits.'\\
This should be reflected upon again and again\\
by one who has gone forth.

`Does regret over my conduct arise in my mind?'\\
This should be reflected upon again and again\\
by one who has gone forth.

`Could my spiritual companions find fault with my conduct?'\\
This should be reflected upon again and again\\
by one who has gone forth.

`All that is mine, beloved and pleasing, will become otherwise, will become separated from me.'\\
This should be reflected upon again and again\\
by one who has gone forth.

`I am the owner of my kamma, heir to my kamma, born of my kamma,\\
related to my kamma, abide supported by my kamma; whatever kamma I shall do, for good or for ill, of that I will be the heir.'\\
This should be reflected upon again and again\\
by one who has gone forth.

`The days and nights are relentlessly passing; how well am I spending\\ my time?'\\
This should be reflected upon again and again\\
by one who has gone forth.

`Do I delight in solitude or not?'\\
This should be reflected upon again and again\\
by one who has gone forth.

`Has my practice borne fruit with freedom or insight so that at the end of my life I need not feel ashamed when questioned by my spiritual companions?'\\
This should be reflected upon again and again\\
by one who has gone forth.

Bhikkhus, these are the ten dhammas to be reflected upon again and again by one who has gone forth.

\suttaRef{A.I.87f}

