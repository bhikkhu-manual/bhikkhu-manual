\section{Āditta-pariyāya-sutta}

\firstline{Veneyyadamanopāye sabbaso pāramiṁ gato}

\soloinstr{Solo introduction}

\ifhandbookedition
\begingroup
\fontsize{9}{13.5}\selectfont
\setlength{\tabcolsep}{0.9em}
\fi

\begin{solotwochants}
Veneyyadamanopāye  & sabbaso pāramiṁ gato\\
Amoghavacano buddho & abhiññāyānusāsako\\
Ciṇṇānurūpato cāpi & dhammena vinayaṁ pajaṁ\\
Ciṇṇāggipāricariyānaṁ & sambojjhārahayoginaṁ\\
Yamādittapariyāyaṁ & desayanto manoharaṁ\\
Te sotāro vimocesi & asekkhāya vimuttiyā\\
Tathevopaparikkhāya & viññūṇaṁ sotumicchataṁ\\
Dukkhatālakkhaṇopāyaṁ & taṁ suttantaṁ bhaṇāma se\\
\end{solotwochants}

\ifhandbookedition
\endgroup
\fi

[Evaṁ me sutaṁ]

Ekaṁ samayaṁ bhagavā gayāyaṁ viharati gayāsīse saddhiṁ bhikkhu-sahassena.
Tatra kho bhagavā bhikkhū āmantesi:

Sabbaṁ bhikkhave ādittaṁ. Kiñca bhikkhave sabbaṁ ādittaṁ.

Cakkhuṁ bhikkhave ādittaṁ, rūpā ādittā, cakkhuviññāṇaṁ ādittaṁ,
cakkhusamphasso āditto, yampidaṁ cakkhusamphassapaccayā uppajjati
vedayitaṁ sukhaṁ vā dukkhaṁ vā adukkhamasukhaṁ vā tam pi ādittaṁ. Kena
ādittaṁ. Ādittaṁ rāgagginā dosagginā mohagginā, ādittaṁ jātiyā
jarāmaraṇena sokehi paridevehi dukkhehi domanassehi upāyāsehi ādittan'ti
vadāmi.

Sotaṁ ādittaṁ, saddā ādittā, sotaviññāṇaṁ ādittaṁ, sotasamphasso āditto,
yampidaṁ sotasamphassapaccayā uppajjati vedayitaṁ sukhaṁ vā dukkhaṁ vā
adukkhamasukhaṁ vā tam pi ādittaṁ. Kena ādittaṁ. Ādittaṁ rāgagginā
dosagginā mohagginā, ādittaṁ jātiyā jarāmaraṇena sokehi paridevehi
dukkhehi domanassehi upāyāsehi ādittan'ti vadāmi.

Ghānaṁ ādittaṁ, gandhā ādittā, ghānaviññāṇaṁ ādittaṁ, ghānasamphasso
āditto, yampidaṁ ghānasamphassapaccayā uppajjati vedayitaṁ sukhaṁ vā
dukkhaṁ vā adukkhamasukhaṁ vā tam pi ādittaṁ. Kena ādittaṁ. Ādittaṁ
rāgagginā dosagginā mohagginā, ādittaṁ jātiyā jarāmaraṇena sokehi
paridevehi dukkhehi domanassehi upāyāsehi ādittan'ti vadāmi.

Jivhā ādittā, rasā ādittā, jivhāviññāṇam ādittaṁ, jivhāsamphasso āditto,
yampidaṁ jivhāsamphassapaccayā uppajjati vedayitaṁ sukhaṁ vā dukkhaṁ vā
adukkhamasukhaṁ vā tam pi ādittaṁ. Kena ādittaṁ. Ādittaṁ rāgagginā
dosagginā mohagginā, ādittaṁ jātiyā jarāmaraṇena sokehi paridevehi
dukkhehi domanassehi upāyāsehi ādittan'ti vadāmi.

Kāyo āditto, phoṭṭhabbā ādittā, kāyaviññāṇaṁ ādittaṁ, kāyasamphasso
āditto, yampidaṁ kāyasamphassapaccayā uppajjati vedayitaṁ sukhaṁ vā
dukkhaṁ vā adukkhamasukhaṁ vā tam pi ādittaṁ. Kena ādittaṁ. Ādittaṁ
rāgagginā dosagginā mohagginā, ādittaṁ jātiyā jarāmaraṇena sokehi
paridevehi dukkhehi domanassehi upāyāsehi ādittan'ti vadāmi.

Mano āditto, dhammā ādittā, manoviññāṇaṁ ādittaṁ, manosamphasso āditto,
yampidaṁ manosamphassapaccayā uppajjati vedayitaṁ sukhaṁ vā dukkhaṁ vā
adukkhamasukhaṁ vā tam pi ādittaṁ. Kena ādittaṁ. Ādittaṁ rāgagginā
dosagginā mohagginā, ādittaṁ jātiyā jarāmaraṇena sokehi paridevehi
dukkhehi domanassehi upāyāsehi ādittan'ti vadāmi.

[Evaṁ passaṁ bhikkhave] sutvā ariyasāvako cakkhusmiṁ pi nibbindati,
rūpesu pi nibbindati, cakkhuviññāṇe pi nibbindati, cakkhusamphasse pi
nibbindati, yampidaṁ cakkhusamphassapaccayā uppajjati vedayitaṁ sukhaṁ
vā dukkhaṁ vā adukkhamasukhaṁ vā tasmiṁ pi nibbindati.

\ifhandbookedition
\enlargethispage{\baselineskip}
\fi

Sotasmiṁ pi nibbindati, saddesu pi nibbindati, sotaviññāṇe pi
nibbindati, sotasamphasse pi nibbindati, yampidaṁ sotasamphassapaccayā
uppajjati vedayitaṁ sukhaṁ vā dukkhaṁ vā adukkhamasukhaṁ vā tasmiṁ pi
nibbindati.

Ghānasmiṁ pi nibbindati, gandhesu pi nibbindati, ghānaviññāṇe pi
nibbindati, ghānasamphasse pi nibbindati, yampidaṁ ghānasamphassapaccayā
uppajjati vedayitaṁ sukhaṁ vā dukkhaṁ vā adukkhamasukhaṁ vā tasmiṁ pi
nibbindati.

Jivhāya pi nibbindati, rasesu pi nibbindati, jivhāviññāṇe pi nibbindati,
jivhāsamphasse pi nibbindati, yampidaṁ jivhāsamphassapaccayā uppajjati
vedayitaṁ sukhaṁ vā dukkhaṁ vā adukkhamasukhaṁ vā tasmiṁ pi nibbindati.

Kāyasmiṁ pi nibbindati, phoṭṭhabbesu pi nibbindati, kāyaviññāṇe pi
nibbindati, kāyasamphasse pi nibbindati, yampidaṁ kāyasamphassapaccayā
uppajjati vedayitaṁ sukhaṁ vā dukkhaṁ vā adukkhamasukhaṁ vā tasmiṁ pi
nibbindati.

Manasmiṁ pi nibbindati, dhammesu pi nibbindati, manoviññāṇe pi
nibbindati, manosamphassepi nibbindati, yampidaṁ manosamphassapaccayā
uppajjati vedayitaṁ sukhaṁ vā dukkhaṁ vā adukkhamasukhaṁ vā tasmiṁ pi
nibbindati.

Nibbindaṁ virajjati, virāgā vimuccati, vimuttasmiṁ vimuttam iti ñāṇaṁ
hoti, khīṇā jāti, vusitaṁ brahmacariyaṁ, kataṁ karaṇīyaṁ, nāparaṁ
itthattāyā ti pajānātī ti.

[Idam-avoca bhagavā.] Attamanā te bhikkhū bhagavato bhāsitaṁ abhinanduṁ.
Imasmiñca pana veyyākaraṇasmiṁ bhaññamāne tassa bhikkhu-sahassassa
anupādāya āsavehi cittāni vimucciṁsū ti.

Ādittapariyāya-suttaṁ niṭṭhitaṁ.

\suttaRef{S.IV.19; Vin.I.34}

