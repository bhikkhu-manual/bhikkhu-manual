\section{Anatta-lakkhaṇa-sutta}

\firstline{Yantaṁ sattehi dukkhena ñeyyaṁ anattalakkhaṇaṁ}

\soloinstr{Solo introduction}

\ifhandbookedition
\begingroup
\fontsize{9}{13.5}\selectfont
\setlength{\tabcolsep}{0.9em}
\fi

\begin{solotwochants}
Yantaṁ sattehi dukkhena & ñeyyaṁ anattalakkhaṇaṁ\\
Attavādattasaññāṇaṁ  & sammadeva vimocanaṁ\\
Sambuddho taṁ pakāsesi & diṭṭhasaccāna yoginaṁ\\
Uttariṁ paṭivedhāya & bhāvetuṁ ñāṇamuttamaṁ\\
Yantesaṁ diṭṭhadhammānam & ñāṇenupaparikkhataṁ\\
Sabbāsavehi cittāni & vimucciṁsu asesato\\
Tathā ñāṇānussārena & sāsanaṁ kātumicchataṁ\\
Sādhūnaṁ atthasiddhatthaṁ & taṁ suttantaṁ bhaṇāma se\\
\end{solotwochants}

\ifhandbookedition
\endgroup
\fi

[Evaṁ me sutaṁ]

Ekaṁ samayaṁ bhagavā bārāṇasiyaṁ viharati isipatane migadāye. Tatra kho
bhagavā pañcavaggiye bhikkhū āmantesi:

Rūpaṁ bhikkhave anattā, rūpañca hidaṁ bhikkhave attā abhavissa, nayidaṁ
rūpaṁ ābādhāya saṁvatteyya, labbhetha ca rūpe, evaṁ me rūpaṁ hotu, evaṁ
me rūpaṁ mā ahosī'ti.

Yasmā ca kho bhikkhave rūpaṁ anattā, tasmā rūpaṁ ābādhāya saṁvattati, na ca
labbhati rūpe, evaṁ me rūpaṁ hotu, evaṁ me rūpaṁ mā ahosī'ti.

Vedanā anattā, vedanā ca hidaṁ bhikkhave attā abhavissa, nayidaṁ vedanā
ābādhāya saṁvatteyya, labbhetha ca vedanāya, evaṁ me vedanā hotu, evaṁ
me vedanā mā ahosī'ti.

Yasmā ca kho bhikkhave vedanā anattā, tasmā vedanā ābādhāya saṁvattati, na ca
labbhati vedanāya, evaṁ me vedanā hotu, evaṁ me vedanā mā ahosī'ti.

Saññā anattā, saññā ca hidaṁ bhikkhave attā abhavissa, nayidaṁ saññā
ābādhāya saṁvatteyya, labbhetha ca saññāya, evaṁ me saññā hotu, evaṁ me
saññā mā ahosī'ti.

Yasmā ca kho bhikkhave saññā anattā, tasmā saññā ābādhāya saṁvattati,
na ca labbhati saññāya, evaṁ me saññā hotu, evaṁ me saññā mā ahosī'ti.

Saṅkhārā anattā, saṅkhārā ca hidaṁ bhikkhave attā abhavissaṁsu, nayidaṁ
saṅkhārā ābādhāya saṁvatteyyuṁ, labbhetha ca saṅkhāresu, evaṁ me
saṅkhārā hontu, evaṁ me saṅkhārā mā ahesun'ti.

Yasmā ca kho bhikkhave saṅkhārā anattā, tasmā saṅkhārā ābādhāya
saṁvattanti, na ca labbhati saṅkhāresu, evaṁ me saṅkhārā hontu, evaṁ me
saṅkhārā mā ahesun'ti.

Viññāṇaṁ anattā, viññāṇañca hidaṁ bhikkhave attā abhavissa, nayidaṁ
viññāṇaṁ ābādhāya saṁvatteyya, labbhetha ca viññāṇe evaṁ me viññāṇaṁ
hotu, evaṁ me viññāṇaṁ mā ahosī'ti.

Yasmā ca kho bhikkhave viññāṇaṁ anattā, tasmā viññāṇaṁ ābādhāya
saṁvattati, na ca labbhati viññāṇe, evaṁ me viññāṇaṁ hotu, evaṁ me
viññāṇaṁ mā ahosī'ti.

[Taṁ kiṁ maññatha bhikkhave] rūpam niccaṁ vā aniccaṁ vā'ti.
Aniccaṁ bhante.
Yam panāniccaṁ, dukkhaṁ vā taṁ sukhaṁ vā'ti.
Dukkhaṁ bhante.

Yam panāniccaṁ dukkhaṁ viparināma-dhammaṁ, kallaṁ nu taṁ samanupassituṁ,
etaṁ mama, esoham'asmi, eso me attā'ti.
No hetaṁ bhante.

Taṁ kiṁ maññatha bhikkhave, vedanā niccā vā aniccā vā'ti.
Aniccā bhante.
Yam panāniccaṁ, dukkhaṁ vā taṁ sukhaṁ vā'ti.
Dukkhaṁ bhante.

Yam panāniccaṁ dukkhaṁ viparināma-dhammaṁ, kallaṁ nu taṁ samanupassituṁ,
etaṁ mama, esoham'asmi, eso me attā'ti.
No hetaṁ bhante.

Taṁ kiṁ maññatha bhikkhave, saññā niccā vā aniccā vā'ti.
Aniccā bhante.
Yam panāniccaṁ, dukkhaṁ vā taṁ sukhaṁ vā'ti.
Dukkhaṁ bhante.

Yam panāniccaṁ dukkhaṁ viparināma-dhammaṁ, kallaṁ nu taṁ samanupassituṁ,
etaṁ mama, esoham'asmi, eso me attā'ti.
No hetaṁ bhante.

Taṁ kiṁ maññatha bhikkhave, saṅkhārā niccā vā aniccā vā'ti.
Aniccā bhante.
Yam panāniccaṁ, dukkhaṁ vā taṁ sukhaṁ vā'ti.
Dukkhaṁ bhante.

Yam panāniccaṁ dukkhaṁ viparināma-dhammaṁ, kallaṁ nu taṁ samanupassituṁ,
etaṁ mama, esoham'asmi, eso me attā'ti.
No hetaṁ bhante.

Taṁ kiṁ maññatha bhikkhave, viññāṇaṁ niccaṁ vā aniccaṁ vā'ti.
Aniccaṁ bhante.
Yam panāniccaṁ, dukkhaṁ vā taṁ sukhaṁ vā'ti.
Dukkhaṁ bhante.

Yam panāniccaṁ dukkhaṁ viparināma-dhammaṁ, kallaṁ nu taṁ samanupassituṁ
etaṁ mama, esoham'asmi, eso me attā'ti.
No hetaṁ bhante.

[Tasmā tiha bhikkhave] yaṁ kiñci rūpaṁ atītānāgata-paccuppannaṁ ajjhattaṁ
vā bahiddhā vā oḷārikaṁ vā sukhumaṁ vā hīnaṁ vā paṇītaṁ vā yandūre
santike vā, sabbaṁ rūpaṁ netaṁ mama, nesoham'asmi, na me so attā'ti,
evametaṁ yathābhūtaṁ sammappaññāya daṭṭhabbaṁ.

Yā kāci vedanā atītānāgata-paccuppannā ajjhattā vā bahiddhā vā oḷārikā
vā sukhumā vā hīnā vā paṇītā vā yā dūre santike vā, sabbā vedanā netaṁ
mama, nesoham'asmi, na me so attā'ti, evametaṁ yathābhūtaṁ sammappaññāya
daṭṭhabbaṁ.

Yā kāci saññā atītānāgata-paccuppannā ajjhattā vā bahiddhā vā oḷārikā vā
sukhumā vā hīnā vā paṇītā vā yā dūre santike vā, sabbā saññā netaṁ mama,
nesoham'asmi, na me so attā'ti, evametaṁ yathābhūtaṁ sammappaññāya
daṭṭhabbaṁ.

Ye keci saṅkhārā atītānāgata-paccuppannā ajjhattā vā bahiddhā vā oḷārikā
vā sukhumā vā hīnā vā paṇītā vā ye dūre santike vā, sabbe saṅkhārā netaṁ
mama, nesoham'asmi, na me so attā'ti, evametaṁ yathābhūtaṁ sammappaññāya
daṭṭhabbaṁ.

Yaṁ kiñci viññāṇaṁ atītānāgata-paccuppannaṁ ajjhattaṁ vā bahiddhā vā
oḷārikaṁ vā sukhumaṁ vā hīnaṁ vā paṇītaṁ vā yandūre santike vā, sabbaṁ
viññāṇaṁ netaṁ mama, nesoham'asmi, na me so attā'ti, evametaṁ yathābhūtaṁ
sammappaññāya daṭṭhabbaṁ.

[Evaṁ passaṁ bhikkhave] sutvā ariyasāvako rūpasmim pi nibbindati, vedanāya
pi nibbindati, saññāya pi nibbindati, saṅkhāresu pi nibbindati,
viññāṇasmim pi nibbindati, nibbindaṁ virajjati, virāgā vimuccati,
vimuttasmiṁ vimuttam iti ñāṇaṁ hoti, khīṇā jāti, vusitaṁ brahmacariyaṁ,
kataṁ karaṇīyaṁ, nāparaṁ itthattāyā'ti pajānātī'ti.

[Idam-avoca bhagavā.] Attamanā pañcavaggiyā bhikkhū bhagavato bhāsitaṁ
abhinanduṁ. Imasmiñca pana veyyākaraṇasmiṁ bhaññamāne pañcavaggiyānaṁ
bhikkhūnaṁ anupādāya āsavehi cittāni vimucciṁsū'ti.

Anattalakkhaṇa-suttaṁ niṭṭhitaṁ.

\suttaRef{S.III.66; Vin.I.13f}

