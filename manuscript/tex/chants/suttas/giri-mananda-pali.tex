\section{Girimānanda-suttaṃ}

Evaṃ me sutaṃ: Ekaṃ samayaṃ Bhagavā Sāvatthiyaṃ viharati Jeta-vane
Anāthapiṇḍikassa ārāme. Tena kho pana samayena āyasmā Girimānando ābādhiko hoti
dukkhito bāḷha-gilāno. Atha kho āyasmā Ānando yena Bhagavā ten’upasaṅkami;
upasaṅkamitvā Bhagavantaṃ abhivādetvā ekam-antaṃ nisīdi. Ekam-antaṃ nisinno kho
āyasmā Ānando Bhagavantaṃ etad-avoca:

“Āyasmā, Bhante, Girimānando ābādhiko hoti dukkhito bāḷha-gilāno. Sādhu Bhante
Bhagavā yen’āyasmā Girimānando ten’upasaṅkamatu anukampaṃ upādāyā-ti.”

“Sace kho tvaṃ Ānanda Girimānandassa bhikkhuno dasa saññā bhāseyyāsi, ṭhānaṃ kho
pan’etaṃ vijjati yaṃ Girimānandassa bhikkhuno dasa saññā sutvā so ābādho ṭhānaso
paṭipassambheyya.

“Katamā dasa? Anicca-saññā, anatta-saññā, asubha-saññā, ādīnava-saññā,
pahāna-saññā, virāga-saññā, nirodha-saññā, sabba-loke anabhirata-saññā,
sabba-saṅkhāresu aniccāsaññā, ānāpānassati.

“Katamā c’Ānanda anicca-saññā? Idh’Ānanda, bhikkhu arañña-gato vā
rukkhamūla-gato vā suññāgāra-gato vā iti paṭisañcikkhati: ‘rūpaṃ aniccaṃ, vedanā
aniccā, saññā aniccā, saṅkhārā aniccā, viññāṇaṃ aniccan-ti. Iti imesu pañcasu
upādānakkhandhesu aniccānupassī viharati. Ayaṃ vuccat’Ānanda anicca-saññā.

“Katamā c’Ānanda anatta-saññā? Idh’Ānanda, bhikkhu arañña-gato vā
rukkhamūla-gato vā suññāgāra-gato vā iti paṭisañcikkhati: ‘cakkhuṃ anattā, rūpā
anattā, sotaṃ anattā, saddā anattā, ghānaṃ anattā, gandhā anattā, jivhā anattā,
rasā anattā, kāyo anattā, phoṭṭhabbā anattā, mano anattā, dhammā anattā-ti. Iti
imesu chasu ajjhattikabāhiresu āyatanesu anattānupassī viharati. Ayaṃ
vuccat’Ānanda, anatta-saññā.

“Katamā c’Ānanda, asubha-saññā? Idh’Ānanda, bhikkhu imam-eva kāyaṃ uddhaṃ
pāda-talā adho kesa-matthakā taca-pariyantaṃ pūraṃ nānāppakārassa asucino
paccavekkhati: ‘Atthi imasmiṃ kāye: kesā, lomā, nakhā, dantā, taco, maṃsaṃ,
nhāru, aṭṭhi, aṭṭhi-miñjaṃ, vakkaṃ, hadayaṃ, yakanaṃ, kilomakaṃ, pihakaṃ,
papphāsaṃ, antaṃ, anta-guṇaṃ, udariyaṃ, karīsaṃ, pittaṃ, semhaṃ, pubbo, lohitaṃ,
sedo, medo, assu, vasā, kheḷo, siṅghāṇikā, lasikā, muttan-ti.’ Iti imasmiṃ kāye
asubhānupassī viharati. Ayaṃ vuccat’Ānanda asubha-saññā.

“Katamā c’Ānanda ādīnava-saññā? Idh’Ānanda, bhikkhu arañña-gato vā
rukkhamūla-gato vā suññāgāra-gato vā iti paṭisañcikkhati: ‘Bahu-dukkho kho ayaṃ
kāyo bahu-ādīnavo. Iti imasmiṃ kāye vividhā ābādhā uppajjanti, seyyathīdaṃ:
cakkhu-rogo, sota-rogo, ghāna-rogo, jivhā-rogo, kāya-rogo, sīsa-rogo,
kaṇṇa-rogo, mukha-rogo, dantarogo, oṭṭha-rogo, kāso, sāso, pināso, ḍāho, jaro,
kucchi-rogo, mucchā, pakkhandikā, sūlā, visūcikā, kuṭṭhaṃ, gaṇḍo, kilāso, soso,
apamāro, daddu, kaṇḍu, kacchu, nakhasā, vitacchikā, lohitaṃ, pittaṃ, madhu-meho,
aṃsā, piḷakā, bhagandalā, pitta-samuṭṭhānā ābādhā, semha-samuṭṭhānā ābādhā,
vātasamuṭṭhānā ābādhā, sannipātikā ābādhā, utupariṇāma-jā ābādhā,
visama-parihāra-jā ābādhā, opakkamikā ābādhā, kamma-vipāka-jā ābādhā, sītaṃ,
uṇhaṃ, jighacchā, pipāsā, uccāro, passāvo-ti.’ Iti imasmiṃ kāye ādīnavānupassī
viharati. Ayaṃ vuccat’Ānanda ādīnava-saññā.

“Katamā c’Ānanda pahāna-saññā? Idh’Ānanda, bhikkhu uppannaṃ kāmavitakkaṃ
nādhivāseti, pajahati, vinodeti, byantīkaroti, anabhāvaṃ gameti. Uppannaṃ
byāpāda-vitakkaṃ nādhivāseti, pajahati, vinodeti, byantīkaroti, anabhāvaṃ
gameti. Uppannaṃ vihiṃsā-vitakkaṃ nādhivāseti, pajahati, vinodeti, byantīkaroti,
anabhāvaṃ gameti. Uppann’uppanne pāpake akusale dhamme nādhivāseti, pajahati,
vinodeti, byantīkaroti, anabhāvaṃ gameti. Ayaṃ vuccat’Ānanda pahāna-saññā.

“Katamā c’Ānanda, virāga-saññā? Idh’Ānanda, bhikkhu arañña-gato vā
rukkhamūla-gato vā suññāgāra-gato vā iti paṭisañcikkhati: ‘Etaṃ santaṃ, etaṃ
paṇītaṃ, yad-idaṃ sabba-saṅkhāra-samatho sabbūpadhippaṭinissaggo taṇhākkhayo
virāgo nibbānan-ti.’ Ayaṃ vuccat’Ānanda virāgasaññā.

“Katamā c’Ānanda, nirodha-saññā? Idh’Ānanda, bhikkhu arañña-gato vā
rukkhamūla-gato vā suññāgāra-gato vā iti paṭisañcikkhati: ‘Etaṃ santaṃ, etaṃ
paṇītaṃ, yad-idaṃ sabba-saṅkhāra-samatho sabbūpadhippaṭinissaggo taṇhākkhayo
nirodho nibbānan-ti.’ Ayaṃ vuccat’Ānanda nirodhasaññā.

“Katamā c’Ānanda, sabba-loke anabhiratasaññā? Idh’Ānanda, bhikkhu ye loke
upādānā cetaso adhiṭṭhānābhinivesānusayā, te pajahanto viharati anupādiyanto.
Ayaṃ vuccat’Ānanda sabba-loke anabhirata-saññā.

“Katamā c’Ānanda sabba-saṅkhāresu aniccāsaññā? Idh’Ānanda bhikkhu
sabba-saṅkhāresu aṭṭīyati, harāyati, jigucchati. Ayaṃ vuccat’ Ānanda,
sabba-saṅkhāresu aniccā-saññā.

“Katamā c’Ānanda ānāpānassati?
Idh’Ānanda, bhikkhu arañña-gato vā rukkhamūla-gato vā suññāgāra-gato vā nisīdati,
pallaṅkaṃ ābhujitvā, ujuṃ kāyaṃ paṇidhāya,
parimukhaṃ satiṃ upaṭṭhapetvā. So sato’va
assasati sato’va passasati.

Dīghaṃ vā assasanto: ‘Dīghaṃ assasāmī-ti’ pajānāti. Dīghaṃ vā passasanto:
‘Dīghaṃ passasāmī-ti’ pajānāti. Rassaṃ vā assasanto: ‘Rassaṃ assasāmī-ti’
pajānāti. Rassaṃ vā passasanto: ‘Rassaṃ passasāmī-ti’ pajānāti.
‘Sabba-kāyapaṭisaṃvedī assasissāmī-ti’ sikkhati. ‘Sabbakāya-paṭisaṃvedī
passasissāmī-ti’ sikkhati. ‘Passambhayaṃ kāya-saṅkhāraṃ assasissāmī-ti’
sikkhati. ‘Passambhayaṃ kāya-saṅkhāraṃ passasissāmī-ti’ sikkhati.

‘Pīti-paṭisaṃvedī assasissāmī-ti’ sikkhati. ‘Pīti-paṭisaṃvedī passasissāmī-ti’
sikkhati. ‘Sukha-paṭisaṃvedī assasissāmī-ti’ sikkhati. ‘Sukha-paṭisaṃvedī
passasissāmī-ti’ sikkhati. ‘Citta-saṅkhāra-paṭisaṃvedī assasissāmī-ti’ sikkhati.
‘Citta-saṅkhāra-paṭisaṃvedī passasissāmī-ti’ sikkhati. ‘Passambhayaṃ
cittasaṅkhāraṃ assasissāmī-ti’ sikkhati. ‘Passambhayaṃ citta-saṅkhāraṃ
passasissāmīti’ sikkhati.

‘Citta-paṭisaṃvedī assasissāmī-ti’ sikkhati. ‘Citta-paṭisaṃvedī passasissāmī-ti’
sikkhati. ‘Abhippamodayaṃ cittaṃ assasissāmī-ti’ sikkhati. ‘Abhippamodayaṃ
cittaṃ passasissāmī-ti’ sikkhati. ‘Samādahaṃ cittaṃ assasissāmī-ti’ sikkhati.
‘Samādahaṃ cittaṃ passasissāmī-ti’ sikkhati. ‘Vimocayaṃ cittaṃ assasissāmī-ti’
sikkhati. ‘Vimocayaṃ cittaṃ passasissāmī-ti’ sikkhati.

‘Aniccānupassī assasissāmī-ti’ sikkhati. ‘Aniccānupassī passasissāmī-ti’
sikkhati. ‘Virāgānupassī assasissāmī-ti’ sikkhati. ‘Virāgānupassī
passasissāmī-ti’ sikkhati. ‘Nirodhānupassī assasissāmī-ti’ sikkhati.
‘Nirodhānupassī passasissāmī-ti’ sikkhati. ‘Paṭinissaggānupassī assasissāmī-ti’
sikkhati. ‘Paṭinissaggānupassī passasissāmī-ti’ sikkhati. Ayaṃ vuccat’ Ānanda,
ānāpānassati.

“Sace kho tvaṃ, Ānanda, Girimānandassa bhikkhuno imā dasa saññā bhāseyyāsi,
ṭhānaṃ kho pan’etaṃ vijjati yaṃ Girimānandassa bhikkhuno imā dasa saññā sutvā so
ābādho ṭhānaso paṭippassambheyyā-ti.”

Atha kho āyasmā Ānando Bhagavato santike imā dasa saññā uggahetvā yen’āyasmā
Girimānando ten’upasaṅkami; upasaṅkamitvā āyasmato Girimānandassa imā dasa saññā
abhāsi.

Atha kho āyasmato Girimānandassa dasa saññā sutvā so ābādho ṭhānaso
paṭippassambhi, vuṭṭhāhi c’āyasmā Girimānando tamhā ābādhā. Tathā pahīno ca
pan’āyasmato Girimānandassa so ābādho ahosī-ti.”

Girimānanda Suttaṃ Niṭṭhitaṃ.

\suttaRef{A.V.108}

