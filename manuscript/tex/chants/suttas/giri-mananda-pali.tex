\section{Girimānanda-sutta}

[Evaṁ me sutaṁ] 
Ekaṁ samayaṁ bhagavā sāvatthiyaṁ viharati jetavane
Anāthapiṇḍikassa ārāme. Tena kho pana samayena āyasmā Girimānando ābādhiko hoti
dukkhito bāḷha-gilāno. Atha kho āyasmā Ānando yena bhagavā ten’upasaṅkami,
upasaṅkamitvā Bhagavantaṁ abhivādetvā ekam-antaṁ nisīdi. Ekam-antaṁ nisinno kho
āyasmā Ānando bhagavantaṁ etad-avoca:

Āyasmā bhante Girimānando ābādhiko hoti dukkhito bāḷha-gilāno. Sādhu bhante
bhagavā yen’āyasmā Girimānando ten’upasaṅkamatu anukampaṁ upādāyā ti.

Sace kho tvaṁ Ānanda Girimānandassa bhikkhuno dasa saññā bhāseyyāsi, ṭhānaṁ kho
pan’etaṁ vijjati yaṁ Girimānandassa bhikkhuno dasa saññā sutvā so ābādho ṭhānaso
paṭipassambheyya.

Katamā dasa? Anicca-saññā, anatta-saññā, asubha-saññā, ādīnava-saññā,
pahāna-saññā, virāga-saññā, nirodha-saññā, sabba-loke anabhirata-saññā,
sabba-saṅkhāresu anicchāsaññā, ānāpānassati.

Katamā c’Ānanda anicca-saññā? Idh’Ānanda, bhikkhu arañña-gato vā
rukkhamūla-gato vā suññāgāra-gato vā iti paṭisañcikkhati: rūpaṁ aniccaṁ, vedanā
aniccā, saññā aniccā, saṅkhārā aniccā, viññāṇaṁ aniccan’ti. Iti imesu pañcasu
upādānakkhandhesu aniccānupassī viharati. Ayaṁ vuccat’Ānanda anicca-saññā.

Katamā c’Ānanda anatta-saññā? Idh’Ānanda, bhikkhu arañña-gato vā
rukkhamūla-gato vā suññāgāra-gato vā iti paṭisañcikkhati: cakkhuṁ anattā, rūpā
anattā, sotaṁ anattā, saddā anattā, ghānaṁ anattā, gandhā anattā, jivhā anattā,
rasā anattā, kāyo anattā, phoṭṭhabbā anattā, mano anattā, dhammā anattā’ti. Iti
imesu chasu ajjhattikabāhiresu āyatanesu anattānupassī viharati. Ayaṁ
vuccat’Ānanda anatta-saññā.

Katamā c’Ānanda asubha-saññā? Idh’Ānanda, bhikkhu imam-eva kāyaṁ uddhaṁ
pāda-talā adho kesa-matthakā taca-pariyantaṁ pūraṁ nānāppakārassa asucino
paccavekkhati: Atthi imasmiṁ kāye kesā, lomā, nakhā, dantā, taco, maṁsaṁ,
nhāru, aṭṭhi, aṭṭhi-miñjaṁ, vakkaṁ, hadayaṁ, yakanaṁ, kilomakaṁ, pihakaṁ,
papphāsaṁ, antaṁ, anta-guṇaṁ, udariyaṁ, karīsaṁ, pittaṁ, semhaṁ, pubbo, lohitaṁ,
sedo, medo, assu, vasā, kheḷo, siṅghāṇikā, lasikā, muttan’ti. Iti imasmiṁ kāye
asubhānupassī viharati. Ayaṁ vuccat’Ānanda asubha-saññā.

Katamā c’Ānanda ādīnava-saññā? Idh’Ānanda, bhikkhu arañña-gato vā
rukkhamūla-gato vā suññāgāra-gato vā iti paṭisañcikkhati: Bahu-dukkho kho ayaṁ
kāyo bahu-ādīnavo. Iti imasmiṁ kāye vividhā ābādhā uppajjanti, seyyathīdaṁ
cakkhu-rogo, sota-rogo, ghāna-rogo, jivhā-rogo, kāya-rogo, sīsa-rogo,
kaṇṇa-rogo, mukha-rogo, dantarogo, oṭṭha-rogo, kāso, sāso, pināso, ḍāho, jaro,
kucchi-rogo, mucchā, pakkhandikā, sūlā, visūcikā, kuṭṭhaṁ, gaṇḍo, kilāso, soso,
apamāro, daddu, kaṇḍu, kacchu, nakhasā, vitacchikā, lohitaṁ, pittaṁ, madhu-meho,
aṁsā, piḷakā, bhagandalā, pitta-samuṭṭhānā ābādhā, semha-samuṭṭhānā ābādhā,
vāta-samuṭṭhānā ābādhā, sannipātikā ābādhā, utupariṇāma-jā ābādhā,
visama-parihāra-jā ābādhā, opakkamikā ābādhā, kamma-vipāka-jā ābādhā, sītaṁ,
uṇhaṁ, jighacchā, pipāsā, uccāro, passāvo’ti. Iti imasmiṁ kāye ādīnavānupassī
viharati. Ayaṁ vuccat’Ānanda ādīnava-saññā.

\ifhandbookedition
\enlargethispage{\baselineskip}
\fi

Katamā c’Ānanda pahāna-saññā? Idh’Ānanda, bhikkhu uppannaṁ kāma-vitakkaṁ
nādhivāseti, pajahati, vinodeti, byantīkaroti, anabhāvaṁ gameti. Uppannaṁ
byāpāda-vitakkaṁ nādhivāseti, pajahati, vinodeti, byantīkaroti, anabhāvaṁ
gameti. Uppannaṁ vihiṁsā-vitakkaṁ nādhivāseti, pajahati, vinodeti, byantīkaroti,
anabhāvaṁ gameti. Uppann’uppanne pāpake akusale dhamme nādhivāseti, pajahati,
vinodeti, byantīkaroti, anabhāvaṁ gameti. Ayaṁ vuccat’Ānanda pahāna-saññā.

Katamā c’Ānanda, virāga-saññā? Idh’Ānanda, bhikkhu arañña-gato vā
rukkhamūla-gato vā suññāgāra-gato vā iti paṭisañcikkhati: Etaṁ santaṁ, etaṁ
paṇītaṁ, yad-idaṁ sabba-saṅkhāra-samatho sabbūpadhippaṭinissaggo taṇhākkhayo
virāgo nibbānan’ti. Ayaṁ vuccat’Ānanda virāgasaññā.

Katamā c’Ānanda, nirodha-saññā? Idh’Ānanda, bhikkhu arañña-gato vā
rukkhamūla-gato vā suññāgāra-gato vā iti paṭisañcikkhati: Etaṁ santaṁ, etaṁ
paṇītaṁ, yad-idaṁ sabba-saṅkhāra-samatho sabbūpadhippaṭinissaggo taṇhākkhayo
nirodho nibbānan’ti. Ayaṁ vuccat’Ānanda nirodhasaññā.

Katamā c’Ānanda, sabba-loke anabhiratasaññā? Idh’Ānanda, bhikkhu ye loke
upādānā cetaso adhiṭṭhānābhinivesānusayā, te pajahanto viharati anupādiyanto.
Ayaṁ vuccat’Ānanda sabba-loke anabhirata-saññā.

Katamā c’Ānanda sabba-saṅkhāresu anicchāsaññā? Idh’Ānanda bhikkhu
sabba-saṅkhāresu aṭṭīyati, harāyati, jigucchati. Ayaṁ vuccat’ Ānanda,
sabba-saṅkhāresu anicchā-saññā.

Katamā c’Ānanda ānāpānassati?
Idh’Ānanda, bhikkhu arañña-gato vā rukkhamūla-gato vā suññāgāra-gato vā nisīdati,
pallaṅkaṁ ābhujitvā ujuṁ kāyaṁ paṇidhāya parimukhaṁ satiṁ upaṭṭhapetvā. So sato’va
assasati sato’va passasati.

Dīghaṁ vā assasanto: Dīghaṁ assasāmī’ti pajānāti. Dīghaṁ vā passasanto:
Dīghaṁ passasāmī’ti pajānāti. Rassaṁ vā assasanto: Rassaṁ assasāmī’ti
pajānāti. Rassaṁ vā passasanto: Rassaṁ passasāmī’ti pajānāti.
Sabba-kāyapaṭisaṁvedī assasissāmī’ti sikkhati. Sabbakāya-paṭisaṁvedī
passasissāmī’ti sikkhati. Passambhayaṁ kāya-saṅkhāraṁ assasissāmī’ti
sikkhati. Passambhayaṁ kāya-saṅkhāraṁ passasissāmī’ti sikkhati.

Pīti-paṭisaṁvedī assasissāmī’ti sikkhati. Pīti-paṭisaṁvedī passasissāmī’ti
sikkhati. Sukha-paṭisaṁvedī assasissāmī’ti sikkhati. Sukha-paṭisaṁvedī
passasissāmī’ti sikkhati. Citta-saṅkhāra-paṭisaṁvedī assasissāmī’ti sikkhati.
Citta-saṅkhāra-paṭisaṁvedī passasissāmī’ti sikkhati. Passambhayaṁ
cittasaṅkhāraṁ assasissāmī’ti sikkhati. Passambhayaṁ citta-saṅkhāraṁ
passasissāmī’ti sikkhati.

Citta-paṭisaṁvedī assasissāmī’ti sikkhati. Citta-paṭisaṁvedī passasissāmī’ti
sikkhati. Abhippamodayaṁ cittaṁ assasissāmī’ti sikkhati. Abhippamodayaṁ
cittaṁ passasissāmī’ti sikkhati. Samādahaṁ cittaṁ assasissāmī’ti sikkhati.
Samādahaṁ cittaṁ passasissāmī’ti sikkhati. Vimocayaṁ cittaṁ assasissāmī’ti
sikkhati. Vimocayaṁ cittaṁ passasissāmī’ti sikkhati.

Aniccānupassī assasissāmī’ti sikkhati. Aniccānupassī passasissāmī’ti
sikkhati. Virāgānupassī assasissāmī’ti sikkhati. Virāgānupassī
passasissāmī’ti sikkhati. Nirodhānupassī assasissāmī’ti sikkhati.
Nirodhānupassī passasissāmī’ti sikkhati. Paṭinissaggānupassī assasissāmī’ti
sikkhati. Paṭinissaggānupassī passasissāmī’ti sikkhati. Ayaṁ vuccat’ Ānanda,
ānāpānassati.

Sace kho tvaṁ Ānanda Girimānandassa bhikkhuno imā dasa saññā bhāseyyāsi,
ṭhānaṁ kho pan’etaṁ vijjati yaṁ Girimānandassa bhikkhuno imā dasa saññā sutvā so
ābādho ṭhānaso paṭippassambheyyā ti.

Atha kho āyasmā Ānando bhagavato santike imā dasa saññā uggahetvā yen’āyasmā
Girimānando ten’upasaṅkami, upasaṅkamitvā āyasmato Girimānandassa imā dasa saññā
abhāsi.

Atha kho āyasmato Girimānandassa dasa saññā sutvā so ābādho ṭhānaso
paṭippassambhi. Vuṭṭhahi c’āyasmā Girimānando tamhā ābādhā. Tathā pahīno ca
pan’āyasmato Girimānandassa so ābādho ahosī ti.

Girimānanda-suttaṁ niṭṭhitaṁ. \suttaRef{A.V.108}

