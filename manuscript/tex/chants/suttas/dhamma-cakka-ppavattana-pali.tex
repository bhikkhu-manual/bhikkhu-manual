\section{Dhammacakkappavattana-sutta}

\firstline{Anuttaraṁ abhisambodhiṁ sambujjhitvā tathāgato}

\soloinstr{Solo introduction}

\ifhandbookedition
\begingroup
\fontsize{9}{13.5}\selectfont
\setlength{\tabcolsep}{0.9em}
\fi
  
\begin{solotwochants}
Anuttaraṁ abhisambodhiṁ & sambujjhitvā tathāgato\\
Pathamaṁ yaṁ adesesi & dhammacakkaṁ anuttaraṁ\\
Sammadeva pavattento & loke appativattiyaṁ\\
Yatthākkhātā ubho antā & paṭipatti ca majjhimā\\
Catūsvāriyasaccesu & visuddhaṁ ñāṇadassanaṁ\\
Desitaṁ dhammarājena & sammāsambodhikittanaṁ\\
Nāmena vissutaṁ suttaṁ & dhammacakkappavattanaṁ\\
Veyyākaraṇapāthena & saṅgītantam bhaṇāma se\\
\end{solotwochants}

\ifhandbookedition
\endgroup
\fi

[Evaṁ me sutaṁ]

Ekaṁ samayaṁ bhagavā bārāṇasiyaṁ viharati isipatane migadāye. Tatra kho
bhagavā pañcavaggiye bhikkhū āmantesi:

Dve'me, bhikkhave, antā pabbajitena na sevitabbā: yo cāyaṁ kāmesu
kāma-sukh'allikānuyogo, hīno, gammo, pothujjaniko, anariyo,
anattha-sañhito; yo cāyaṁ atta-kilamathānuyogo, dukkho, anariyo,
anattha-sañhito.

Ete te, bhikkhave, ubho ante anupagamma majjhimā paṭipadā tathāgatena
abhisambuddhā cakkhukaraṇī, ñāṇakaraṇī, upasamāya, abhiññāya,
sambodhāya, nibbānāya saṁvattati.

Katamā ca sā, bhikkhave, majjhimā paṭipadā tathāgatena abhisambuddhā
cakkhukaraṇī, ñāṇakaraṇī, upasamāya, abhiññāya, sambodhāya, nibbānāya
saṁvattati.

Ayam-eva ariyo aṭṭhaṅgiko maggo seyyathīdaṁ:

Sammā-diṭṭhi, sammā-saṅkappo, sammā-vācā, sammā-kammanto, sammā-ājīvo,
sammā-vāyāmo, sammā-sati, sammā-samādhi.

Ayaṁ kho sā, bhikkhave, majjhimā paṭipadā tathāgatena abhisambuddhā
cakkhukaraṇī, ñāṇakaraṇī, upasamāya, abhiññāya, sambodhāya, nibbānāya
saṁvattati.

Idaṁ kho pana, bhikkhave, dukkhaṁ ariya-saccaṁ:

Jātipi dukkhā, jarāpi dukkhā, maranampi dukkhaṁ,
soka-parideva-dukkha-domanass'upāyāsāpi dukkhā, appiyehi sampayogo
dukkho, piyehi vippayogo dukkho, yamp'icchaṁ na labhati tampi dukkhaṁ,
saṅkhittena pañcupādānakkhandhā dukkhā.

Idaṁ kho pana, bhikkhave, dukkha-samudayo ariya-saccaṁ:

Yā'yaṁ taṇhā ponobbhavikā nandi-rāga-sahagatā tatra-tatrābhinandinī
seyyathīdaṁ: kāma-taṇhā, bhava-taṇhā, vibhava-taṇhā.

Idaṁ kho pana, bhikkhave, dukkha-nirodho ariya-saccaṁ:

Yo tassā yeva taṇhāya asesa-virāga-nirodho, cāgo, paṭinissaggo, mutti,
anālayo.

Idaṁ kho pana, bhikkhave, dukkha-nirodha-gāminī paṭipadā ariya-saccaṁ:

Ayam-eva ariyo aṭṭhaṅgiko maggo seyyathīdam: sammā-diṭṭhi,
sammā-saṅkappo, sammā-vācā, sammā-kammanto, sammā-ājīvo, sammā-vāyāmo,
sammā-sati, sammā-samādhi.

[Idaṁ dukkhaṁ] ariya-saccan'ti me bhikkhave, pubbe ananussutesu dhammesu
cakkhuṁ udapādi, ñāṇaṁ udapādi, paññā udapādi, vijjā udapādi, āloko
udapādi.

Taṁ kho pan'idaṁ dukkhaṁ ariya-saccaṁ pariññeyyan'ti me bhikkhave, pubbe
ananussutesu dhammesu cakkhuṁ udapādi, ñāṇaṁ udapādi, paññā udapādi,
vijjā udapādi, āloko udapādi.

Taṁ kho pan'idaṁ dukkhaṁ ariya-saccaṁ pariññātan'ti me bhikkhave, pubbe
ananussutesu dhammesu cakkhuṁ udapādi, ñāṇaṁ udapādi, paññā udapādi,
vijjā udapādi, āloko udapādi.

Idaṁ dukkha-samudayo ariya-saccan'ti me bhikkhave, pubbe ananussutesu
dhammesu cakkhuṁ udapādi, ñāṇaṁ udapādi, paññā udapādi, vijjā udapādi,
āloko udapādi.

Taṁ kho pan'idaṁ dukkha-samudayo ariyasaccaṁ pahātabban'ti me bhikkhave,
pubbe ananussutesu dhammesu cakkhuṁ udapādi, ñāṇaṁ udapādi, paññā
udapādi, vijjā udapādi, āloko udapādi.

Taṁ kho pan'idaṁ dukkha-samudayo ariya-saccaṁ pahīnan'ti me bhikkhave, pubbe
ananussutesu dhammesu cakkhuṁ udapādi, ñāṇaṁ udapādi, paññā udapādi,
vijjā udapādi, āloko udapādi.

Idaṁ dukkha-nirodho ariya-saccan'ti me bhikkhave, pubbe ananussutesu
dhammesu cakkhuṁ udapādi, ñāṇaṁ udapādi, paññā udapādi, vijjā udapādi,
āloko udapādi.

Taṁ kho pan'idaṁ dukkha-nirodho ariya-saccaṁ sacchikātabban'ti me bhikkhave,
pubbe ananussutesu dhammesu cakkhuṁ udapādi, ñāṇaṁ udapādi, paññā
udapādi, vijjā udapādi, āloko udapādi.

Taṁ kho pan'idaṁ dukkha-nirodho ariya-saccaṁ sacchikatan'ti me bhikkhave,
pubbe ananussutesu dhammesu cakkhuṁ udapādi, ñāṇaṁ udapādi, paññā
udapādi, vijjā udapādi, āloko udapādi.

Idaṁ dukkha-nirodha-gāminī paṭipadā ariya-saccan'ti me bhikkhave, pubbe
ananussutesu dhammesu cakkhuṁ udapādi, ñāṇaṁ udapādi, paññā udapādi,
vijjā udapādi, āloko udapādi.

Taṁ kho pan'idaṁ dukkha-nirodha-gāminī paṭipadā ariya-saccaṁ bhāvetabban'ti
me bhikkhave, pubbe ananussutesu dhammesu cakkhuṁ udapādi, ñāṇaṁ
udapādi, paññā udapādi, vijjā udapādi, āloko udapādi.

Taṁ kho pan'idaṁ dukkha-nirodha-gāminī paṭipadā ariya-saccaṁ bhāvitan'ti me
bhikkhave, pubbe ananussutesu dhammesu cakkhuṁ udapādi, ñāṇaṁ udapādi,
paññā udapādi, vijjā udapādi, āloko udapādi.

[Yāva kīvañca me bhikkhave] imesu catūsu ariya-saccesu evan-ti-parivaṭṭaṁ
dvādas'ākāraṁ yathā-bhūtaṁ ñāṇa-dassanaṁ na suvisuddhaṁ ahosi, n'eva tāv'āhaṁ
bhikkhave, sadevake loke samārake sabrahmake sassamaṇa-brāhmaṇiyā pajāya
sadeva-manussāya anuttaraṁ sammā-sambodhiṁ abhisambuddho paccaññāsiṁ.

Yato ca kho me bhikkhave, imesu catūsu ariya-saccesu evan-ti-parivaṭṭaṁ
dvādas'ākāraṁ yathā-bhūtaṁ ñāṇa-dassanaṁ suvisuddham ahosi, ath'āham
bhikkhave, sadevake loke samārake sabrahmake sassamaṇa-brāhmaṇiyā pajāya
sadeva-manussāya anuttaraṁ sammā-sambodhiṁ abhisambuddho paccaññāsiṁ.

Ñāṇañca pana me dassanaṁ udapādi, akuppā me vimutti ayam-antimā jāti,
natthi dāni punabbhavo'ti.

Idam-avoca bhagavā. Attamanā pañcavaggiyā bhikkhū bhagavato bhāsitaṁ
abhinanduṁ.

Imasmiñca pana veyyākaraṇasmiṁ bhaññamāne āyasmato koṇḍaññassa virajaṁ
vītamalaṁ dhammacakkhuṁ udapādi: yaṁ kiñci samudaya-dhammaṁ sabban-taṁ
nirodha-dhamman'ti.

[Pavattite ca bhagavatā] dhammacakke bhummā devā saddamanussāvesuṁ:

Etaṁ bhagavatā bārāṇasiyaṁ isipatane migadāye anuttaraṁ dhammacakkaṁ
pavattitaṁ appaṭivattiyaṁ samaṇena vā brāhmaṇena vā devena vā mārena vā
brahmunā vā kenaci vā lokasmin'ti.

\ifhandbookedition
\clearpage
\fi

\subsubsection{Bhummānaṁ devānaṁ}

\firstline{Bhummānaṁ devānaṁ saddaṁ sutvā}

Bhummānaṁ devānaṁ saddaṁ sutvā, cātummahārājikā devā
saddamanussāvesuṁ\ldots

Cātummahārājikānaṁ devānaṁ saddaṁ sutvā, tāvatiṁsā devā
saddamanussāvesuṁ\ldots

Tāvatiṁsānaṁ devānaṁ saddaṁ sutvā, yāmā devā saddamanussāvesuṁ\ldots

Yāmānaṁ devānaṁ saddaṁ sutvā, tusitā devā saddamanussāvesuṁ\ldots

Tusitānaṁ devānaṁ saddaṁ sutvā, nimmānaratī devā saddamanussāvesuṁ\ldots

Nimmānaratīnaṁ devānaṁ saddaṁ sutvā, paranimmitavasavattī devā
saddamanussāvesuṁ\ldots

Paranimmitavasavattīnaṁ devānaṁ saddaṁ sutvā, brahmakāyikā devā
saddamanussāvesuṁ:

Etaṁ bhagavatā bārāṇasiyaṁ isipatane migadāye anuttaraṁ dhammacakkaṁ
pavattitaṁ appaṭivattiyaṁ samaṇena vā brāhmaṇena vā devena vā mārena vā
brahmunā vā kenaci vā lokasmin'ti.

Iti'ha tena khaṇena, tena muhuttena, yāva brahmalokā saddo abbhuggacchi.
Ayañca dasa-sahassī lokadhātu saṅkampi sampakampi sampavedhi, appamāṇo ca
oḷāro obhāso loke pāturahosi atikkammeva devānaṁ devānubhāvaṁ.

Atha kho bhagavā udānaṁ udānesi:

Aññāsi vata bho koṇḍañño, aññāsi vata bho koṇḍañño'ti. Iti hidaṁ āyasmato
koṇḍaññassa aññā-koṇḍañño tveva nāmaṁ ahosī'ti.

Dhammacakkappavattana-suttaṁ niṭṭhitaṁ.

\suttaRef{S.V.420; Vin.I.10f}

