\section{Dhammacakkappavattana Sutta}

\firstline{Anuttaraṃ abhisambodhiṃ sambujjhitvā tathāgato}

\begin{leader}
\soloinstr{Solo introduction}

\begin{solotwochants}
Anuttaraṃ abhisambodhiṃ & sambujjhitvā tathāgato\\
Pathamaṃ yaṃ adesesi & dhammacakkaṃ anuttaraṃ\\
Sammadeva pavattento & loke appativattiyaṃ\\
Yatthākkhātā ubho antā & paṭipatti ca majjhimā\\
Catūsvāriyasaccesu & visuddhaṃ ñāṇadassanaṃ\\
Desitaṃ dhammarājena & sammāsambodhikittanaṃ\\
Nāmena vissutaṃ suttaṃ & dhammacakkappavattanaṃ\\
Veyyākaraṇapāthena & saṅgītantam bhaṇāma se\\
\end{solotwochants}
\end{leader}

[Evaṃ me sutaṃ]

Ekaṃ samayaṃ bhagavā bārāṇasiyaṃ viharati isipatane migadāye. Tatra kho
bhagavā pañcavaggiye bhikkhū āmantesi:

Dve'me, bhikkhave, antā pabbajitena na sevitabbā: yo cāyaṃ kāmesu
kāma-sukh'allikānuyogo; hīno, gammo, pothujjaniko, anariyo,
anattha-sañhito; yo cāyaṃ atta-kilamathānuyogo; dukkho, anariyo,
anattha-sañhito.

Ete te, bhikkhave, ubho ante anupagamma majjhimā paṭipadā tathāgatena
abhisambuddhā cakkhukaraṇī, ñāṇakaraṇī, upasamāya, abhiññāya,
sambodhāya, nibbānāya saṃvattati.

Katamā ca sā, bhikkhave, majjhimā paṭipadā tathāgatena abhisambuddhā
cakkhukaraṇī ñāṇakaraṇī, upasamāya, abhiññāya, sambodhāya, nibbānāya
saṃvattati.

Ayam-eva ariyo aṭṭhaṅgiko maggo seyyathīdaṃ:

Sammā-diṭṭhi, sammā-saṅkappo, sammā-vācā, sammā-kammanto, sammā-ājīvo,
sammā-vāyāmo, sammā-sati, sammā-samādhi.

Ayaṃ kho sā, bhikkhave, majjhimā paṭipadā tathāgatena abhisambuddhā
cakkhukaraṇī ñāṇakaraṇī, upasamāya, abhiññāya, sambodhāya, nibbānāya
saṃvattati.

Idaṃ kho pana, bhikkhave, dukkhaṃ ariya-saccaṃ:

Jātipi dukkhā, jarāpi dukkhā, maranampi dukkhaṃ,
soka-parideva-dukkha-domanass'upāyāsāpi dukkhā, appiyehi sampayogo
dukkho, piyehi vippayogo dukkho, yamp'icchaṃ na labhati tampi dukkhaṃ,
saṅkhittena pañcupādānakkhandā dukkhā.

Idaṃ kho pana, bhikkhave, dukkha-samudayo ariya-saccaṃ:

Yā'yaṃ taṇhā ponobbhavikā nandi-rāga-sahagatā tatra-tatrābhinandinī
seyyathīdaṃ: kāma-taṇhā, bhava-taṇhā, vibhava-taṇhā.

Idaṃ kho pana, bhikkhave, dukkha-nirodho ariya-saccaṃ:

Yo tassā yeva taṇhāya asesa-virāga-nirodho, cāgo, paṭinissaggo, mutti,
anālayo.

Idaṃ kho pana, bhikkhave, dukkha-nirodha-gāminī paṭipadā ariya-saccaṃ:

Ayam-eva ariyo aṭṭhaṅgiko maggo seyyathīdam: sammā-diṭṭhi,
sammā-saṅkappo, sammā-vācā, sammā-kammanto, sammā-ājīvo, sammā-vāyāmo,
sammā-sati, sammā-samādhi.

[Idaṃ dukkhaṃ] ariya-saccan'ti me bhikkhave, pubbe ananussutesu dhammesu
cakkhuṃ udapādi, ñāṇaṃ udapādi, paññā udapādi, vijjā udapādi, āloko
udapādi.

Taṃ kho pan'idaṃ dukkhaṃ ariya-saccaṃ pariññeyyan'ti me bhikkhave, pubbe
ananussutesu dhammesu cakkhuṃ udapādi, ñāṇaṃ udapādi, paññā udapādi,
vijjā udapādi, āloko udapādi.

Taṃ kho pan'idaṃ dukkhaṃ ariya-saccaṃ pariññātan'ti me bhikkhave, pubbe
ananussutesu dhammesu cakkhuṃ udapādi, ñāṇaṃ udapādi, paññā udapādi,
vijjā udapādi, āloko udapādi.

Idaṃ dukkha-samudayo ariya-saccan'ti me bhikkhave, pubbe ananussutesu
dhammesu cakkhuṃ udapādi, ñāṇaṃ udapādi, paññā udapādi, vijjā udapādi,
āloko udapādi.

Taṃ kho pan'idaṃ dukkhasamudayo ariyasaccaṃ pahātabban'ti me bhikkhave,
pubbe ananussutesu dhammesu cakkhuṃ udapādi, ñāṇaṃ udapādi, paññā
udapādi, vijjā udapādi, āloko udapādi.

Taṃ kho pan'idaṃ dukkha-samudayo ariya-saccaṃ pahīnan'ti me bhikkhave, pubbe
ananussutesu dhammesu cakkhuṃ udapādi, ñāṇaṃ udapādi, paññā udapādi,
vijjā udapādi, āloko udapādi.

Idaṃ dukkha-nirodho ariya-saccan'ti me bhikkhave, pubbe ananussutesu
dhammesu cakkhuṃ udapādi, ñāṇaṃ udapādi, paññā udapādi, vijjā udapādi,
āloko udapādi.

Taṃ kho pan'idaṃ dukkha-nirodho ariya-saccaṃ sacchikātabban'ti me bhikkhave,
pubbe ananussutesu dhammesu cakkhuṃ udapādi, ñāṇaṃ udapādi, paññā
udapādi, vijjā, udapādi āloko udapādi.

Taṃ kho pan'idaṃ dukkha-nirodho ariya-saccaṃ sacchikatan'ti me bhikkhave,
pubbe ananussutesu dhammesu cakkhuṃ udapādi, ñāṇaṃ udapādi, paññā
udapādi, vijjā udapādi, āloko udapādi.

Idaṃ dukkha-nirodha-gāminī paṭipadā ariya-saccan'ti me bhikkhave, pubbe
ananussutesu dhammesu cakkhuṃ udapādi, ñāṇaṃ udapādi, paññā udapādi,
vijjā udapādi, āloko udapādi.

Taṃ kho pan'idaṃ dukkha-nirodha-gāminī paṭipadā ariya-saccaṃ bhāvetabban'ti
me bhikkhave, pubbe ananussutesu dhammesu cakkhuṃ udapādi, ñāṇaṃ
udapādi, paññā udapādi, vijjā udapādi, āloko udapādi.

Taṃ kho pan'idaṃ dukkha-nirodha-gāminī paṭipadā ariya-saccaṃ bhāvitan'ti me
bhikkhave, pubbe ananussutesu dhammesu cakkhuṃ udapādi, ñāṇaṃ udapādi,
paññā udapādi, vijjā udapādi, āloko udapādi.

[Yāva kīvañca me bhikkhave,] imesu catūsu ariya-saccesu evan-ti-parivaṭṭaṃ
dvādas'ākāraṃ yathā-bhūtaṃ ñāṇa-dassanaṃ na suvisuddhaṃ ahosi, n'eva tāv'āhaṃ
bhikkhave, sadevake loke samārake sabrahmake sassamaṇa-brāhmaṇiyā pajāya
sadeva-manussāya anuttaraṃ sammā-sambodhiṃ abhisambuddho paccaññāsiṃ.

Yato ca kho me bhikkhave, imesu catūsu ariya-saccesu evan-ti-parivaṭṭaṃ
dvādas'ākāraṃ yathā-bhūtaṃ ñāṇa-dassanaṃ suvisuddham ahosi, ath'āham
bhikkhave, sadevake loke samārake sabrahmake sassamaṇa-brāhmaṇiyā pajāya
sadeva-manussāya anuttaraṃ sammā-sambodhiṃ abhisambuddho paccaññāsiṃ.

Ñāṇañca pana me dassanaṃ udapādi, akuppā me vimutti ayam-antimā jāti,
natthi dāni punabbhavo'ti.

Idam-avoca bhagavā. Attamanā pañcavaggiyā bhikkhū bhagavato bhāsitaṃ
abhinanduṃ.

Imasmiñca pana veyyākaraṇasmiṃ bhaññamāne āyasmato koṇḍaññassa virajaṃ
vītamalaṃ dhammacakkhuṃ udapādi: yaṃ kiñci samudaya-dhammaṃ sabban-taṃ
nirodha-dhamman'ti.

[Pavattite ca bhagavatā] dhammacakke bhummā devā saddamanussāvesuṃ:

Etaṃ bhagavatā bārāṇasiyaṃ isipatane migadāye anuttaraṃ dhammacakkaṃ
pavattitaṃ appaṭivattiyaṃ samaṇena vā brāhmaṇena vā devena vā mārena vā
brahmunā vā kenaci vā lokasmin'ti.

\sidepar{\pointerMark}% Bhummānaṃ devānaṃ
Bhummānaṃ devānaṃ saddaṃ sutvā, cātummahārājikā devā
saddamanussāvesuṃ\ldots

Cātummahārājikānaṃ devānaṃ saddaṃ sutvā, tāvatiṃsā devā
saddamanussāvesuṃ\ldots

Tāvatiṃsānaṃ devānaṃ saddaṃ sutvā, yāmā devā saddamanussāvesuṃ\ldots

Yāmānaṃ devānaṃ saddaṃ sutvā, tusitā devā saddamanussāvesuṃ\ldots

Tusitānaṃ devānaṃ saddaṃ sutvā, nimmānaratī devā saddamanussāvesum\ldots

Nimmānaratīnaṃ devānaṃ saddaṃ sutvā, paranimmitavasavattī devā
saddamanussāvesuṃ\ldots

Paranimmitavasavattīnaṃ devānaṃ saddaṃ sutvā, brahmakāyikā devā
saddamanussāvesuṃ:

Etaṃ bhagavatā bārāṇasiyaṃ isipatane migadāye anuttaraṃ dhammacakkaṃ
pavattitaṃ appaṭivattiyaṃ samaṇena vā brāhmaṇena vā devena vā mārena vā
brahmunā vā kenaci vā lokasmin'ti.

Iti'ha tena khaṇena, tena muhuttena, yāva brahmalokā saddo abbhuggacchi.
Ayañca dasa-sahassī lokadhātu saṅkampi sampakampi sampavedhi, appamāṇo ca
oḷāro obhāso loke pāturahosi atikkammeva devānaṃ devānubhāvaṃ.

Atha kho bhagavā udānaṃ udānesi:

Aññāsi vata bho koṇḍañño, aññāsi vata bho koṇḍañño ti. Iti hidaṃ āyasmato
koṇḍaññassa aññā-koṇḍañño tveva nāmaṃ ahosī ti.

Dhammacakkappavattana-suttaṃ niṭṭhitaṃ.

\suttaRef{SN 56.11; Vin.I.10f}

