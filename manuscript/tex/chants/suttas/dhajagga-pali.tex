\section{Dhaj'agga-sutta}

\ifhandbookedition
\enlargethispage{\baselineskip}
\fi

[Evam-me sutaṁ.] Ekaṁ samayaṁ Bhagavā, Sāvatthiyaṁ viharati, Jeta-vane
Anāthapiṇḍikassa ārāme. Tatra kho Bhagavā bhikkhū āmantesi: “bhikkhavo-ti”.
“Bhadante-ti,” te bhikkhū Bhagavato paccassosuṁ. Bhagavā etad avoca:

“Bhūta-pubbaṁ bhikkhave devāsura-saṅgāmo samupabbūḷho ahosi. Atha kho bhikkhave
Sakko devānamindo deve tāva-tiṁse āmantesi: ‘Sace mārisā devānaṁ saṅgāma-gatānaṁ
uppajjeyya bhayaṁ vā chambhitattaṁ vā lomahaṁso vā, mameva tasmiṁ samaye
dhaj’aggaṁ ullokeyyātha. Mamaṁ hi vo dhaj’aggaṁ ullokayataṁ yaṁ bhavissati
bhayaṁ vā chambhitattaṁ vā loma-haṁso vā, so pahīyissati.’

‘No ce me dhaj’aggaṁ ullokeyyātha, atha Pajāpatissa deva-rājassa dhaj’aggaṁ
ullokeyyātha. Pajāpatissa hi vo deva-rājassa dhaj’aggaṁ ullokayataṁ yaṁ
bhavissati bhayaṁ vā chambhitattaṁ vā loma-haṁso vā, so pahīyissati’.

‘No ce Pajāpatissa deva-rājassa dhaj’aggaṁ ullokeyyātha, atha Varuṇassa
deva-rājassa dhaj’aggaṁ ullokeyyātha. Varuṇassa hi vo deva-rājassa dha’jaggaṁ
ullokayataṁ yaṁ bhavissati bhayaṁ vā chambhitattaṁ vā lomahaṁso vā, so
pahīyissati’.

‘No ce Varuṇassa deva-rājassa dhaj’aggaṁ ullokeyyātha, atha Īsānassa
deva-rājassa dhaj’aggaṁ ullokeyyātha. Īsānassa hi vo devarājassa dhaj’aggaṁ
ullokayataṁ yaṁ bhavissati bhayaṁ vā chambhitattaṁ vā loma-haṁso vā, so
pahīyissatī-ti.’

“Taṁ kho pana bhikkhave Sakkassa vā devānam indassa dhaj’aggaṁ ullokayataṁ,
Pajāpatissa vā deva-rājassa dhaj’aggaṁ ullokayataṁ, Varuṇassa vā deva-rājassa
dhaj’aggaṁ ullokayataṁ, Īsānassa vā devarājassa dhaj’aggaṁ ullokayataṁ yaṁ
bhavissati bhayaṁ vā chambhitattaṁ vā loma-haṁso vā, so pahīyethāpi no’pi
pahīyetha.

“Taṁ kissa hetu? Sakko hi, bhikkhave, devānam indo avītarāgo avītadoso avītamoho
bhīru chambhī utrāsī palāyī-ti.

“Ahañ-ca kho, bhikkhave, evaṁ vadāmi: Sace tumhākaṁ, bhikkhave, arañña-gatānaṁ
vā rukkha-mūla-gatānaṁ vā suññāgāra-gatānaṁ vā uppajjeyya bhayaṁ vā
chambhitattaṁ vā loma-haṁso vā, mam eva tasmiṁ samaye anussareyyātha:

‘Iti pi so bhagavā arahaṁ sammā-sambuddho, vijjā-caraṇa-sampanno sugato
loka-vidū, anuttaro purisa-damma-sārathi satthā devamanussānaṁ Buddho
Bhagavā-ti. Mamaṁ hi vo bhikkhave anussarataṁ, yaṁ bhavissati bhayaṁ vā
chambhitattaṁ vā loma-haṁso vā, so pahīyissati.

“No ce maṁ anussareyyātha, atha dhammaṁ anussareyyātha:

‘Svākkhāto Bhagavatā dhammo, sandiṭṭhiko akāliko ehi-passiko, opanayiko
paccattaṁ veditabbo viññūhī-ti. Dhammaṁ hi vo bhikkhave anussarataṁ, yaṁ
bhavissati bhayaṁ vā chambhitattaṁ vā loma-haṁso vā, so pahīyissati.

“No ce dhammaṁ anussareyyātha, atha saṅghaṁ anussareyyātha:

‘Supaṭipanno Bhagavato sāvaka-saṅgho, uju-paṭipanno Bhagavato sāvaka-saṅgho,
ñāya-paṭipanno Bhagavato sāvaka-saṅgho, sāmīci-paṭipanno Bhagavato
sāvaka-saṅgho, yad-idaṁ cattāri purisa-yugāni aṭṭha purisapuggalā, esa Bhagavato
sāvaka-saṅgho, āhuneyyo pāhuneyyo dakkhiṇeyyo añjalikaraṇīyo, anuttaraṁ
puññakkhettaṁ lokassā-ti. Saṅghaṁ hi vo bhikkhave anussarataṁ yaṁ bhavissati
bhayaṁ vā chambhitattaṁ vā lomahaṁso vā, so pahīyissati.

“Taṁ kissa hetu? Tathāgato hi bhikkhave arahaṁ sammā-sambuddho, vītarāgo
vītadoso vītamoho, abhīru acchambhī anutrāsī apalāyīti.”

Idam avoca Bhagavā. Idaṁ vatvā sugato athāparaṁ etad avoca satthā:

“Araññe rukkha-mūle vā,\\
Suññ’āgāre va bhikkhavo;\\
Anussaretha Sambuddhaṁ,\\
Bhayaṁ tumhāka no siyā.\\
No ce Buddhaṁ sareyyātha,\\
Loka-jeṭṭhaṁ narāsabhaṁ;\\
Atha dhammaṁ sareyyātha,\\
Niyyānikaṁ sudesitaṁ.\\
No ce dhammaṁ sareyyātha,\\
Niyyānikaṁ sudesitaṁ;\\
Atha saṅghaṁ sareyyātha,\\
Puññakkhettaṁ anuttaraṁ.\\
Evaṁ-Buddhaṁ sarantānaṁ,\\
Dhammaṁ saṅghañ-ca bhikkhavo;\\
Bhayaṁ vā chambhitattaṁ vā,\\
Loma-haṁso na hessatī-ti.”

Dhaj’agga-suttaṁ niṭṭhitaṁ.

\suttaRef{S.I.218}

