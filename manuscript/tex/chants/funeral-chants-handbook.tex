\chapter{Funeral Chants}

\section{Dhamma-saṅgaṇī-mātikā}

\firstline{Kusalā dhammā akusalā dhammā}

Kusalā dhammā.\\
Akusalā dhammā.\\
Abyākatā dhammā.

Sukhāya vedanāya sampayuttā dhammā.\\
Dukkhāya vedanāya sampayuttā dhammā.\\
Adukkhamasukhāya vedanāya sampayuttā dhammā.

Vipākā dhammā.\\
Vipāka-dhamma-dhammā.\\
N'eva vipāka na vipāka-dhamma-dhammā.

Upādinn'upādāniyā dhammā.\\
Anupādinn'upādāniyā dhammā.\\
Anupādinnānupādāniyā dhammā.

Saṅkiliṭṭha-saṅkilesikā dhammā.\\
Asaṅkiliṭṭha-saṅkilesikā dhammā.\\
Asaṅkiliṭṭhāsaṅkilesikā dhammā.

\clearpage

Savitakka-savicārā dhammā.\\
Avitakka-vicāra-mattā dhammā.\\
Avitakkāvicārā dhammā.

Pīti-saha-gatā dhammā.\\
Sukha-saha-gatā dhammā.\\
Upekkhā-saha-gatā dhammā.

Dassanena pahātabbā dhammā.\\
Bhāvanāya pahātabbā dhammā.\\
N'eva dassanena na bhāvanāya pahātabbā dhammā.

Dassanena pahātabba-hetukā dhammā.\\
Bhāvanāya pahātabba-hetukā dhammā.\\
N'eva dassanena na bhāvanāya pahātabba-hetukā dhammā.

Ācaya-gāmino dhammā.\\
Apacaya-gāmino dhammā.\\
N'ev'ācaya-gāmino nāpacaya-gāmino dhammā.

Sekkhā dhammā.\\
Asekkhā dhammā.\\
N'eva sekkhā nāsekkhā dhammā.

Parittā dhammā.\\
Mahaggatā dhammā.\\
Appamāṇā dhammā.

\clearpage

Paritt'ārammaṇā dhammā.\\
Mahaggat'ārammaṇā dhammā.\\
Appamāṇ'ārammaṇā dhammā.

Hīnā dhammā.\\
Majjhimā dhammā.\\
Paṇītā dhammā.

Micchatta-niyatā dhammā.\\
Sammatta-niyatā dhammā.\\
Aniyatā dhammā.

Magg'ārammaṇā dhammā.\\
Magga-hetukā dhammā.\\
Maggādhipatino dhammā.

Uppannā dhammā.\\
Anuppannā dhammā.\\
Uppādino dhammā.

Atītā dhammā.\\
Anāgatā dhammā.\\
Paccuppannā dhammā.

Atīt'ārammaṇā dhammā.\\
Anāgat'ārammaṇā dhammā.\\
Paccuppann'ārammaṇā dhammā.

\clearpage

Ajjhattā dhammā.\\
Bahiddhā dhammā.\\
Ajjhatta-bahiddhā dhammā.

Ajjhatt'ārammaṇā dhammā.\\
Bahiddh'ārammaṇā dhammā.\\
Ajjhatta-bahiddh'ārammaṇā dhammā.

Sanidassana-sappaṭighā dhammā.\\
Anidassana-sappaṭighā dhammā.\\
Anidassanāppaṭighā dhammā.

\suttaRef{Dhammasaṅganī 1f}

\section{Dhammasaṅgaṇī}

Kusalā dhammā, akusalā dhammā, abyākatā dhammā.

Katame dhammā kusalā.

Yasmiṃ samaye kāmāvacaraṃ kusalaṃ cittaṃ uppannaṃ hoti, somanassa-sahagataṃ
ñāṇa-sampayuttaṃ, rūpārammaṇaṃ vā saddārammaṇaṃ vā gandhārammaṇaṃ vā
rasārammaṇaṃ vā phoṭṭhabbārammaṇaṃ vā dhammārammaṇaṃ vā, yaṃ yaṃ vā panārabbha,
tasmiṃ samaye phasso hoti, avikkhepo hoti, ye vā pana tasmiṃ samaye aññe pi atthi paṭicca-samuppannā arūpino dhammā, ime dhammā kusalā.

\suttaRef{Dhammasaṅganī 56}

% Source: Chomtong chanting book
% Gavesako: These 6.2-6.8 are extracts from the seven books (jet khampi) of the Abhidhamma. They are chanted usually for three days at the home of the deceased before a funeral takes place, depending on local custom.

\section{Vibhaṅga}

Pañcakkhandhā rūpakkhandho, vedanākkhandho, saññākkhandho, saṅkhārakkhandho,
viññāṇakkhandho.

Tattha katamo rūpakkhandho.

Yaṃ kiñci rūpaṃ atītānāgata-paccuppannaṃ ajjhattaṃ vā bahiddhā vā oḷārikaṃ vā
sukhumaṃ vā hīnaṃ vā paṇītaṃ vā yaṃ dūre santike vā, tad ekajjhaṃ
abhisaññūhitvā abhisaṅkhipitvā, ayaṃ vuccati rūpakkhandho.

\suttaRef{Vibhaṅga 1}

% Source: Chomtong chanting book

\section{Dhātukathā}

Saṅgaho asaṅgaho,\\
saṅgahitena asaṅgahitaṃ,\\
asaṅgahitena saṅgahitaṃ,\\
saṅgahitena saṅgahitaṃ,\\
asaṅgahitena asaṅgahitaṃ,\\
sampayogo vippayogo,\\
sampayuttena vippayuttaṃ,\\
vippayuttena sampayuttaṃ,\\
asaṅgahitaṃ.

\suttaRef{Dhātukathā 1}

% Source: Chomtong chanting book

\section{Puggalapaññatti}

Cha paññattiyo khandhapaññatti, āyatanapaññatti, dhātupaññatti, saccapaññatti,
indriyapaññatti, puggalapaññattī'ti.

Kittāvatā puggalānaṃ puggalapaññatti.

Samayavimutto, asamayavimutto,\\
kuppadhammo, akuppadhammo,\\
parihānadhammo, aparihānadhammo,\\
cetanābhabbo, anurakkhaṇābhabbo,\\
puthujjano, gotrabhū,\\
bhayūparato, abhayūparato,\\
bhabbāgamano, abhabbāgamano,\\
niyato, aniyato,\\
paṭipannako, phaleṭhito,\\
arahā, arahattāya paṭipanno.

\suttaRef{Puggalapaññatti 1}

% Source: Chomtong chanting book

\section{Kathāvatthu}

Puggalo upalabbhati saccikaṭṭha-paramatthenā'ti.

Āmantā.

Yo saccikaṭṭho paramattho, tato so puggalo upalabbhati
saccikaṭṭha-paramatthenā'ti.

Na h’evaṃ vattabbe.

Ājānāhi niggahaṃ. Hañci puggalo upalabbhati
saccikaṭṭha-paramatthena, tena vata re vattabbe.

Yo saccikaṭṭho paramattho, tato so puggalo upalabbhati
saccikaṭṭha-paramatthenā'ti micchā.

\suttaRef{Kathāvatthu 1}

% Source: Chomtong chanting book

\section{Yamaka}

Ye keci kusalā dhammā, sabbe te kusalamūlā.\\ 
Ye vā pana kusalamūlā, sabbe te dhammā kusalā.\\
Ye keci kusalā dhammā, sabbe te kusalamūlena ekamūlā.\\ 
Ye vā pana kusalamūlena ekamūlā, sabbe te dhammā kusalā.

\suttaRef{Yamaka 1}

% Source: Chomtong chanting book

\section{Paṭṭhāna-mātikā-pāṭha}

\firstline{Hetu-paccayo ārammaṇa-paccayo}

Hetu-paccayo, ārammaṇa-paccayo,\\
adhipati-paccayo, anantara-paccayo,\\
samanantara-paccayo, saha-jāta-paccayo,\\
aññam-añña-paccayo, nissaya-paccayo,\\
upanissaya-paccayo, pure-jāta-paccayo,\\
pacchā-jāta-paccayo, āsevana-paccayo,\\
kamma-paccayo, vipāka-paccayo,\\
āhāra-paccayo, indriya-paccayo,\\
jhāna-paccayo, magga-paccayo,\\
sampayutta-paccayo, vippayutta-paccayo,\\
atthi-paccayo, n'atthi-paccayo,\\
vigata-paccayo, avigata-paccayo.

\suttaRef{Tika Paṭṭhāna 1}

\section{Vipassanā-bhūmi-pāṭha}

\firstline{Pañcakkhandhā rūpakkhandho vedanākkhandho}

Pañcakkhandhā:\\
Rūpakkhandho, vedanākkhandho, saññākkhandho, saṅkhārakkhandho, viññāṇakkhandho.

Dvā-das'āyatanāni:\\
Cakkhv-āyatanaṃ rūp'āyatanaṃ,\\
Sot'āyatanaṃ sadd'āyatanaṃ,\\
Ghān'āyatanaṃ gandh'āyatanaṃ,\\
Jivh'āyatanaṃ ras'āyatanaṃ\\
Kāy'āyatanaṃ phoṭṭhabb'āyatanaṃ\\
Man'āyatanaṃ dhamm'āyatanaṃ.

Aṭṭhārasa dhātuyo:\\
Cakkhu-dhātu rūpa-dhātu cakkhu-viññāṇa-dhātu,\\
Sota-dhātu sadda-dhātu sota-viññāṇa-dhātu,\\
Ghāna-dhātu gandha-dhātu ghāna-viññāṇa-dhātu,\\
Jivhā-dhātu rasa-dhātu jivhā-viññāṇa-dhātu,\\
Kāya-dhātu phoṭṭhabba-dhātu kāya-viññāṇa-dhātu,\\
Mano-dhātu dhamma-dhātu mano-viññāṇa-dhātu.

Bā-vīsat'indriyāni:\\
Cakkhu'ndriyaṃ sot'indriyaṃ ghān'indriyaṃ,\\
jivh'indriyaṃ kāy'indriyaṃ man'indriyaṃ,\\
Itth'indriyaṃ puris'indriyaṃ jīvit'indriyaṃ,\\
Sukh'indriyaṃ dukkh'indriyaṃ,\\
somanass'indriyaṃ domanass'indriyaṃ upekkh'indriyaṃ,\\
saddh'indriyaṃ viriy'indriyaṃ sat'indriyaṃ\\
samādh'indriyaṃ paññ'indriyaṃ,\\
Anaññātañ-ñassāmī-t'indriyaṃ aññ'indriyaṃ\\
aññātāv'indriyaṃ.

Cattāri ariya-saccāni:\\
Dukkhaṃ ariya-saccaṃ,\\
Dukkha-samudayo ariya-saccaṃ,\\
Dukkha-nirodho ariya-saccaṃ,\\
Dukkha-nirodha-gāminī paṭipadā ariya-saccaṃ.

Avijjā-paccayā saṅkhārā,\\
Saṅkhāra-paccayā viññāṇaṃ,\\
Viññāṇa-paccayā nāma-rūpaṃ,\\
Nāma-rūpa-paccayā saḷ-āyatanaṃ,\\
Saḷ-āyatana-paccayā phasso,\\
Phassa-paccayā vedanā,\\
Vedanā-paccayā taṇhā,\\
Taṇhā-paccayā upādānaṃ,\\
Upādāna-paccayā bhavo,\\
Bhava-paccayā jāti,\\
Jāti-paccayā jarā-maraṇaṃ soka-parideva-dukkha-domanass'upāyāsā sambhavanti.\\
Evam-etassa kevalassa dukkhakkhandhassa samudayo hoti.

Avijjāya tv-eva asesa-virāga-nirodhā saṅkhāra-nirodho,\\
Saṅkhāra-nirodhā viññāṇa-nirodho,\\
Viññāṇa-nirodhā nāma-rūpa-nirodho,\\
Nāma-rūpa-nirodhā saḷ-āyatana-nirodho,\\
Saḷ-āyatana-nirodhā phassa-nirodho,\\
Phassa-nirodhā vedanā-nirodho,\\
Vedanā-nirodhā taṇhā-nirodho,\\
Taṇhā-nirodhā upādāna-nirodho,\\
Upādāna-nirodhā bhava-nirodho,\\
Bhava-nirodhā jāti-nirodho,\\
Jāti-nirodhā jarā-maraṇaṃ soka-parideva-dukkha-domanass'upāyāsā nirujjhanti.\\
Evam-etassa kevalassa dukkhakkhandhassa nirodho hoti.

\suttaRef{cf. M.III.15f; M.III.280f; M.III.62; M.III.249f; S.II.1f}

\clearpage

\section{Paṃsukūla}

\instr{(For the dead)}

\firstline{Aniccā vata saṅkhārā}

Aniccā vata saṅkhārā\\
Uppāda-vaya-dhammino\\
Uppajjitvā nirujjhanti\\
Tesaṃ vūpasamo sukho. \suttaRef{D.II.157; S.I.6}

Sabbe sattā maranti ca\\
Mariṃsu ca marissare\\
Tath'evāhaṃ marissāmi\\
N'atthi me ettha saṃsayo.

\firstline{Addhuvaṃ jīvitaṃ}

Addhuvaṃ jīvitaṃ\\
Dhuvaṃ maraṇaṃ\\
Avassaṃ mayā maritabbaṃ\\
Maraṇapariyosānaṃ me jīvitaṃ\\
Jīvitaṃ me aniyataṃ\\
Maraṇaṃ me niyataṃ \suttaRef{DhpA.III.170}

\instr{(For the living)}

\firstline{Aciraṃ vat'ayaṃ kāyo}

Aciraṃ vat'ayaṃ kāyo\\
Paṭhaviṃ adhisessati\\
Chuḍḍho apeta-viññāṇo\\
Niratthaṃ va kaliṅgaraṃ. \suttaRef{Dhp 41}

