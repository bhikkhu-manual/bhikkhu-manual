\chapter{Funeral Chants}

\section{Dhamma-saṅgaṇī-mātikā}

\firstline{Kusalā dhammā akusalā dhammā}

Kusalā dhammā.\\
Akusalā dhammā.\\
Abyākatā dhammā.

Sukhāya vedanāya sampayuttā dhammā.\\
Dukkhāya vedanāya sampayuttā dhammā.\\
Adukkhamasukhāya vedanāya sampayuttā dhammā.

Vipākā dhammā.\\
Vipāka-dhamma-dhammā.\\
N'eva vipāka na vipāka-dhamma-dhammā.

Upādinn'upādāniyā dhammā.\\
Anupādinn'upādāniyā dhammā.\\
Anupādinnānupādāniyā dhammā.

Saṅkiliṭṭha-saṅkilesikā dhammā.\\
Asaṅkiliṭṭha-saṅkilesikā dhammā.\\
Asaṅkiliṭṭhāsaṅkilesikā dhammā.

\clearpage

Savitakka-savicārā dhammā.\\
Avitakka-vicāra-mattā dhammā.\\
Avitakkāvicārā dhammā.

Pīti-saha-gatā dhammā.\\
Sukha-saha-gatā dhammā.\\
Upekkhā-saha-gatā dhammā.

Dassanena pahātabbā dhammā.\\
Bhāvanāya pahātabbā dhammā.\\
N'eva dassanena na bhāvanāya pahātabbā dhammā.

Dassanena pahātabba-hetukā dhammā.\\
Bhāvanāya pahātabba-hetukā dhammā.\\
N'eva dassanena na bhāvanāya pahātabba-hetukā dhammā.

Ācaya-gāmino dhammā.\\
Apacaya-gāmino dhammā.\\
N'ev'ācaya-gāmino nāpacaya-gāmino dhammā.

Sekkhā dhammā.\\
Asekkhā dhammā.\\
N'eva sekkhā nāsekkhā dhammā.

Parittā dhammā.\\
Mahaggatā dhammā.\\
Appamāṇā dhammā.

\clearpage

Paritt'ārammaṇā dhammā.\\
Mahaggat'ārammaṇā dhammā.\\
Appamāṇ'ārammaṇā dhammā.

Hīnā dhammā.\\
Majjhimā dhammā.\\
Paṇītā dhammā.

Micchatta-niyatā dhammā.\\
Sammatta-niyatā dhammā.\\
Aniyatā dhammā.

Magg'ārammaṇā dhammā.\\
Magga-hetukā dhammā.\\
Maggādhipatino dhammā.

Uppannā dhammā.\\
Anuppannā dhammā.\\
Uppādino dhammā.

Atītā dhammā.\\
Anāgatā dhammā.\\
Paccuppannā dhammā.

Atīt'ārammaṇā dhammā.\\
Anāgat'ārammaṇā dhammā.\\
Paccuppann'ārammaṇā dhammā.

\clearpage

Ajjhattā dhammā.\\
Bahiddhā dhammā.\\
Ajjhatta-bahiddhā dhammā.

Ajjhatt'ārammaṇā dhammā.\\
Bahiddh'ārammaṇā dhammā.\\
Ajjhatta-bahiddh'ārammaṇā dhammā.

Sanidassana-sappaṭighā dhammā.\\
Anidassana-sappaṭighā dhammā.\\
Anidassanāppaṭighā dhammā. \suttaRef{Dhammasaṅganī 1f}

\section{Dhammasaṅgaṇī}

\enlargethispage{\baselineskip}

Kusalā dhammā, akusalā dhammā, abyākatā dhammā.

Katame dhammā kusalā.

Yasmiṁ samaye kāmāvacaraṁ kusalaṁ cittaṁ uppannaṁ hoti, somanassa-sahagataṁ
ñāṇa-sampayuttaṁ, rūpārammaṇaṁ vā saddārammaṇaṁ vā gandhārammaṇaṁ vā
rasārammaṇaṁ vā phoṭṭhabbārammaṇaṁ vā dhammārammaṇaṁ vā, yaṁ yaṁ vā panārabbha,
tasmiṁ samaye phasso hoti, avikkhepo hoti, ye vā pana tasmiṁ samaye aññe pi atthi paṭicca-samuppannā arūpino dhammā, ime dhammā kusalā.

\suttaRef{Dhammasaṅganī 56}

% Source: Chomtong chanting book

% NOTE: These 6.2-6.8 are extracts from the seven books (jet khampi) of the
% Abhidhamma. They are chanted usually for three days at the home of the
% deceased before a funeral takes place, depending on local custom.

\section{Vibhaṅga}

Pañcakkhandhā rūpakkhandho, vedanākkhandho, saññākkhandho, saṅkhārakkhandho,
viññāṇakkhandho.

Tattha katamo rūpakkhandho.

Yaṁ kiñci rūpaṁ atītānāgata-paccuppannaṁ ajjhattaṁ vā bahiddhā vā oḷārikaṁ vā
sukhumaṁ vā hīnaṁ vā paṇītaṁ vā yaṁ dūre santike vā, tad ekajjhaṁ
abhisaññūhitvā abhisaṅkhipitvā, ayaṁ vuccati rūpakkhandho.

\suttaRef{Vibhaṅga 1}

% Source: Chomtong chanting book

\section{Dhātukathā}

Saṅgaho asaṅgaho,\\
saṅgahitena asaṅgahitaṁ,\\
asaṅgahitena saṅgahitaṁ,\\
saṅgahitena saṅgahitaṁ,\\
asaṅgahitena asaṅgahitaṁ,\\
sampayogo vippayogo,\\
sampayuttena vippayuttaṁ,\\
vippayuttena sampayuttaṁ,\\
asaṅgahitaṁ.

\suttaRef{Dhātukathā 1}

% Source: Chomtong chanting book

\section{Puggalapaññatti}

Cha paññattiyo khandhapaññatti, āyatanapaññatti, dhātupaññatti, saccapaññatti,
indriyapaññatti, puggalapaññattī'ti.

Kittāvatā puggalānaṁ puggalapaññatti.

Samayavimutto, asamayavimutto,\\
kuppadhammo, akuppadhammo,\\
parihānadhammo, aparihānadhammo,\\
cetanābhabbo, anurakkhaṇābhabbo,\\
puthujjano, gotrabhū,\\
bhayūparato, abhayūparato,\\
bhabbāgamano, abhabbāgamano,\\
niyato, aniyato,\\
paṭipannako, phaleṭhito,\\
arahā, arahattāya paṭipanno. \suttaRef{Puggalapaññatti 1}

% Source: Chomtong chanting book

\section{Kathāvatthu}

Puggalo upalabbhati saccikaṭṭha-paramatthenā'ti.

Āmantā.

Yo saccikaṭṭho paramattho, tato so puggalo upalabbhati
saccikaṭṭha-paramatthenā'ti.

Na h’evaṁ vattabbe.

Ājānāhi niggahaṁ. Hañci puggalo upalabbhati
saccikaṭṭha-paramatthena, tena vata re vattabbe.

Yo saccikaṭṭho paramattho, tato so puggalo upalabbhati
saccikaṭṭha-paramatthenā'ti micchā.

\suttaRef{Kathāvatthu 1}

% Source: Chomtong chanting book

\section{Yamaka}

Ye keci kusalā dhammā, sabbe te kusalamūlā.\\ 
Ye vā pana kusalamūlā, sabbe te dhammā kusalā.\\
Ye keci kusalā dhammā, sabbe te kusalamūlena ekamūlā.\\ 
Ye vā pana kusalamūlena ekamūlā, sabbe te dhammā kusalā.

\suttaRef{Yamaka 1}

% Source: Chomtong chanting book

\section{Paṭṭhāna-mātikā-pāṭha}

\firstline{Hetu-paccayo ārammaṇa-paccayo}

Hetu-paccayo, ārammaṇa-paccayo,\\
adhipati-paccayo, anantara-paccayo,\\
samanantara-paccayo, saha-jāta-paccayo,\\
aññam-añña-paccayo, nissaya-paccayo,\\
upanissaya-paccayo, pure-jāta-paccayo,\\
pacchā-jāta-paccayo, āsevana-paccayo,\\
kamma-paccayo, vipāka-paccayo,\\
āhāra-paccayo, indriya-paccayo,\\
jhāna-paccayo, magga-paccayo,\\
sampayutta-paccayo, vippayutta-paccayo,\\
atthi-paccayo, n'atthi-paccayo,\\
vigata-paccayo, avigata-paccayo.

\suttaRef{Tika Paṭṭhāna 1}

\section{Vipassanā-bhūmi-pāṭha}

\enlargethispage{\baselineskip}

\firstline{Pañcakkhandhā rūpakkhandho vedanākkhandho}

Pañcakkhandhā:\\
Rūpakkhandho, vedanākkhandho, saññākkhandho, saṅkhārakkhandho, viññāṇakkhandho.

Dvā-das'āyatanāni:\\
Cakkhv-āyatanaṁ rūp'āyatanaṁ,\\
Sot'āyatanaṁ sadd'āyatanaṁ,\\
Ghān'āyatanaṁ gandh'āyatanaṁ,\\
Jivh'āyatanaṁ ras'āyatanaṁ\\
Kāy'āyatanaṁ phoṭṭhabb'āyatanaṁ\\
Man'āyatanaṁ dhamm'āyatanaṁ.

Aṭṭhārasa dhātuyo:\\
Cakkhu-dhātu rūpa-dhātu cakkhu-viññāṇa-dhātu,\\
Sota-dhātu sadda-dhātu sota-viññāṇa-dhātu,\\
Ghāna-dhātu gandha-dhātu ghāna-viññāṇa-dhātu,\\
Jivhā-dhātu rasa-dhātu jivhā-viññāṇa-dhātu,\\
Kāya-dhātu phoṭṭhabba-dhātu kāya-viññāṇa-dhātu,\\
Mano-dhātu dhamma-dhātu mano-viññāṇa-dhātu.

Bā-vīsat'indriyāni:\\
Cakkhu'ndriyaṁ sot'indriyaṁ ghān'indriyaṁ,\\
jivh'indriyaṁ kāy'indriyaṁ man'indriyaṁ,\\
Itth'indriyaṁ puris'indriyaṁ jīvit'indriyaṁ,\\
Sukh'indriyaṁ dukkh'indriyaṁ,\\
somanass'indriyaṁ domanass'indriyaṁ upekkh'indriyaṁ,\\
saddh'indriyaṁ viriy'indriyaṁ sat'indriyaṁ\\
samādh'indriyaṁ paññ'indriyaṁ,\\
Anaññātañ-ñassāmī-t'indriyaṁ aññ'indriyaṁ\\
aññātāv'indriyaṁ.

Cattāri ariya-saccāni:\\
Dukkhaṁ ariya-saccaṁ,\\
Dukkha-samudayo ariya-saccaṁ,\\
Dukkha-nirodho ariya-saccaṁ,\\
Dukkha-nirodha-gāminī paṭipadā ariya-saccaṁ.

Avijjā-paccayā saṅkhārā,\\
Saṅkhāra-paccayā viññāṇaṁ,\\
Viññāṇa-paccayā nāma-rūpaṁ,\\
Nāma-rūpa-paccayā saḷ-āyatanaṁ,\\
Saḷ-āyatana-paccayā phasso,\\
Phassa-paccayā vedanā,\\
Vedanā-paccayā taṇhā,\\
Taṇhā-paccayā upādānaṁ,\\
Upādāna-paccayā bhavo,\\
Bhava-paccayā jāti,\\
Jāti-paccayā jarā-maraṇaṁ soka-parideva-dukkha-domanass'upāyāsā sambhavanti.\\
Evam-etassa kevalassa dukkhakkhandhassa samudayo hoti.

Avijjāya tv-eva asesa-virāga-nirodhā saṅkhāra-nirodho,\\
Saṅkhāra-nirodhā viññāṇa-nirodho,\\
Viññāṇa-nirodhā nāma-rūpa-nirodho,\\
Nāma-rūpa-nirodhā saḷ-āyatana-nirodho,\\
Saḷ-āyatana-nirodhā phassa-nirodho,\\
Phassa-nirodhā vedanā-nirodho,\\
Vedanā-nirodhā taṇhā-nirodho,\\
Taṇhā-nirodhā upādāna-nirodho,\\
Upādāna-nirodhā bhava-nirodho,\\
Bhava-nirodhā jāti-nirodho,\\
Jāti-nirodhā jarā-maraṇaṁ soka-parideva-dukkha-domanass'upāyāsā nirujjhanti.\\
Evam-etassa kevalassa dukkhakkhandhassa nirodho hoti.

\suttaRef{M.III.15f; M.III.280f; M.III.62; M.III.249f; S.II.1f}

\clearpage

\section{Paṁsukūla}

\enlargethispage{\baselineskip}

\instr{The following verses are often repeated three times.}

\instr{(For the dead)}

\firstline{Aniccā vata saṅkhārā}

Aniccā vata saṅkhārā\\
Uppāda-vaya-dhammino\\
Uppajjitvā nirujjhanti\\
Tesaṁ vūpasamo sukho.

Sabbe sattā maranti ca\\
Mariṁsu ca marissare\\
Tath'evāhaṁ marissāmi\\
N'atthi me ettha saṁsayo. \suttaRef{D.II.157; S.I.6}

\firstline{Addhuvaṁ jīvitaṁ}

Addhuvaṁ jīvitaṁ\\
Dhuvaṁ maraṇaṁ\\
Avassaṁ mayā maritabbaṁ\\
Maraṇapariyosānaṁ me jīvitaṁ\\
Jīvitaṁ me aniyataṁ\\
Maraṇaṁ me niyataṁ. \suttaRef{DhpA.III.170}

\instr{(For the living)}

\firstline{Aciraṁ vat'ayaṁ kāyo}

Aciraṁ vat'ayaṁ kāyo\\
Paṭhaviṁ adhisessati\\
Chuḍḍho apeta-viññāṇo\\
Niratthaṁ va kaliṅgaraṁ. \suttaRef{Dhp 41}

