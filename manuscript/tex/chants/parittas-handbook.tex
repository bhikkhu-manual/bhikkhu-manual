\chapter{Paritta Chants}

\section{Thai Tradition}

Paritta chanting ceremonies in Thailand vary regionally but may be outlined as:

\begin{packeditemize}
  \item a layperson chants the invitation for paritta chanting
  \item the third bhikkhu or nun in seniority chants the invitation to the devas
  \item the introductory chants are chanted
  \item the core sequence of paritta chants follow
  \item the closing chants end the ceremony.
\end{packeditemize}

The third introductory chant in the Mahānikāya sect is commonly \emph{Sambuddhe}.
In Dhammayut circles and frequently in the forest tradition, the third chant is
\emph{Yo cakkhumā} instead.

There is a shorter and longer traditional core sequence. The \emph{jet tamnaan}
(\thai{เจ็ดตํานาน}) contains D1-D7 as below, the \emph{sipsong tamnaan}
(\thai{สิบสองตํานาน}) contains S1-S12. Chants that are not numbered `D' or `S' can
be included or not, as wished, but should be recited in the order listed here.

\clearpage

\enlargethispage{\baselineskip}

{\centering
\fontsize{9}{11}\selectfont
\setArrayStretch{1.1}

\begin{tabular}{@{}l l r r@{}}
  & first line & & page \\
  \hline
  i1  & Namo tassa & & \pageref{namo-tassa} \\
  i2  & Buddhaṃ saraṇaṃ gacchāmi & & \pageref{buddham-saranam} \\
  i3/a  & Sambuddhe aṭṭhavīsañca & & \pageref{sambuddhe} \\
  i3/b  & Yo cakkhumā & & \pageref{yo-cakkhuma} \\
  i4  & Namo arahato & & \pageref{namo-arahato} \\
      & & & \\
  D1 & Asevanā ca bālānaṃ & S1 & \pageref{asevana} \\
  D2 & Yaṅkiñci vittaṃ & S2 & \pageref{yankinci-vittam} \\
  D3 & Karaṇīyam-attha-kusalena & S3 & \pageref{karaniyam-attha} \\
  D4 & Virūpakkhehi me mettaṃ & S4 & \pageref{virupakkhehi} \\
  & Vadhissamenanti parāmasanto & & \pageref{vadhissamenanti} \\
  D5 & Udet'ayañ-cakkhumā eka-rājā & S5 & \pageref{udetayan-cakkhuma} \\
  & Atthi loke sīla-guṇo & S6 & \pageref{atthi-loke} \\
  D6 & Iti pi so bhagavā & S7 & \pageref{iti-pi-so} \\
  D7 & Vipassissa nam'atthu & S8 & \pageref{vipassissa} \\
  & Natthi me saraṇaṃ aññaṃ & & \pageref{natthi-me} \\
  & Yaṅkiñci ratanaṃ loke & & \pageref{yankinci-ratanam} \\
  & Sakkatvā buddharatanaṃ & & \pageref{sakkatva} \\
  & Yato'haṃ bhagini & S9 & \pageref{yato-ham-bhagini} \\
  & Bojjh'aṅgo sati-saṅkhāto & S10 & \pageref{bojjhango} \\
  & Yan-dunnimittaṃ & S11 & \pageref{yan-dunnimittam} \\
      & & & \\
  & Dukkhappattā ca niddukkhā & & \pageref{dukkhappatta} \\
  & Bāhuṃ sahassam-abhinimmita & & \pageref{bahum} \\
  & Mahā-kāruṇiko nātho & S12 & \pageref{maha-karuniko} \\
  & Te attha-laddhā sukhitā & & \pageref{te-attha-laddha} \\
  & Bhavatu sabba-maṅgalaṃ & & \pageref{bhavatu} \\
\end{tabular}

\restoreArrayStretch
}

\clearpage

\subsection*{Notes for Particular Chants}

\textbf{Asevanā ca bālānaṃ:} The candles on the shrine during a house invitation
are lit by the senior bhikkhu or nun at \emph{Asevanā}.

\textbf{Yaṅkiñci vittaṃ:} The candles are put out at \emph{Nibbanti
  dhīrā yathā'yam padīpo}.

\textbf{Atthi loke sīla-guṇo:} On the occasion of blessing a new house, this
chant should be included, as it is traditionally considered protection against
fire.

\textbf{Yato'haṃ bhagini:} This chant is to be used for expectant mothers since
the time of the Buddha for the blessing and protection of the mother and child.
It is also a good occasion to chant it when receiving alms from a newly married
couple. Sangha members are encouraged to practise it.

\textbf{Dukkhappattā ca niddukkhā:} This is usually chanted as second to last
before \emph{Bhavatu sabba-maṅgalaṃ}. It is considered necessary to include it
whenever the devas have been invited at the beginning of the paritta chanting
as this chant contains a line inviting them to leave again.

\textbf{Bāhuṃ sahassam-abhinimmita:} This is is a popular later addition to the
present day standard chants. It is not listed in the \emph{jet tamnaan} or
\emph{sipsong tamnaan} sets. Yet these days it is frequently added just before
\emph{Mahā-kāruṇiko nātho}. On some occasions (e.g. public birthdays, jubilees,
inauguration ceremonies, etc.), it is an alternative, instead of chanting
\emph{jet tamnaan} or \emph{sipsong tamnaan}, to do a minimum sequence called
\emph{suat phorn phra} which contains only:

(1)~\emph{Namo Tassa},
(2)~\emph{Iti pi so bhagavā},
(3)~\emph{Bāhuṃ}, 
(4)~\emph{Mahā-kāruṇiko nātho}, and
(5)~\emph{Bhavatu sabba-maṅgalaṃ}.

In this minimal chanting sequence usually one does not invite the devas.

\textbf{Te attha-laddhā sukhitā:} This is sometimes inserted before closing with
\emph{Bhavatu sabba-maṅgalaṃ}, as a special well-wishing when the occasion has
to do with Buddhism in general (e.g. inauguration of a new abbot, or at the end
of an \emph{upasampadā}).

\clearpage

\section{Invitations}

\subsection{Invitation for Paritta Chanting}
\label{paritta-invitation-for-chanting}

\firstline{Vipatti-paṭibāhāya sabba-sampatti-siddhiyā}

\vspace*{5pt}

\begin{paritta}

\instr{(After bowing three times, with hands joined in añjali,\\
  recite the following)\par}

Vipatti-paṭibāhāya sabba-sampatti-siddhiyā\\
Sabbadukkha-vināsāya\\
Parittaṃ brūtha maṅgalaṃ

Vipatti-paṭibāhāya sabba-sampatti-siddhiyā\\
Sabbabhaya-vināsāya\\
Parittaṃ brūtha maṅgalaṃ

Vipatti-paṭibāhāya sabba-sampatti-siddhiyā\\
Sabbaroga-vināsāya\\
Parittaṃ brūtha maṅgalaṃ

\instr{(Bow three times)}
\end{paritta}

%\suttaRef{Thai}

\subsection{Invitation to the Devas}
\label{paritta-devas}

\firstline{Pharitvāna mettaṃ samettā bhadantā}
\firstline{Samantā cakka-vāḷesu}
\firstline{Sarajjaṃ sasenaṃ sabandhuṃ nar'indaṃ}

In Thai custom, the third monk in seniority invites the devas, holding his
hands in \emph{añjali}, and lifting up the ceremonial string.

The string is wound up at the beginning of the last chant, \emph{Mahā-kāruṇiko
  nātho} or \emph{Bhavatu sabba-maṅgalaṃ}, which should be kept in mind by the
last bhikkhu or \emph{sāmaṇera}.

Before royal ceremonies, the invitation starts with A.

Before the shorter \emph{jet tamnaan} set of parittas, B is used and C is
omitted. Before the longer \emph{sipsong tamnaan} set of parittas, B is
omitted and C is used.

The verses at D are always chanted.

When chanting outside the monastery, the invitation is concluded with E. When
chanting at the monastery, the invitation is concluded with either E or F.

\clearpage

\begin{paritta}

\instr{(With hands joined in añjali, recite the following)}

\sidepar{A.}%
Sarajjaṃ sasenaṃ sabandhuṃ nar'indaṃ\\
Paritt'ānubhāvo sadā rakkhatū'ti

\sidepar{B.}%
Pharitvāna mettaṃ samettā bhadantā\\
Avikkhitta-cittā parittaṃ bhaṇantu

\sidepar{C.}%
Samantā cakka-vāḷesu\\
Atr'āgacchantu devatā

\sidepar{D.}%
Sagge kāme ca rūpe\\
Giri-sikhara-taṭe c'antalikkhe vimāne\\
Dīpe raṭṭhe ca gāme\\
Taru-vana-gahane geha-vatthumhi khette\\
Bhummā c'āyantu devā\\
Jala-thala-visame yakkha-gandhabba-nāgā\\
Tiṭṭhantā santike yaṃ\\
Muni-vara-vacanaṃ sādhavo me suṇantu

\sidepar{E.}%
Dhammassavana-kālo ayam-bhadantā

\instr{(Three times, or)}

\sidepar{F.}%
Buddha-dassana-kālo ayam-bhadantā\\
Dhammassavana-kālo ayam-bhadantā\\
Saṅgha-payirūpāsana-kālo ayam-bhadantā
\end{paritta}

%\suttaRef{Thai}

\clearpage

\section{Introductory Chants}

\subsection{Pubba-bhāga-nama-kāra-pāṭho}
\label{namo-tassa}

Namo tassa bhagavato arahato sammā-sambuddhassa\\
Namo tassa bhagavato arahato sammā-sambuddhassa\\
Namo tassa bhagavato arahato sammā-sambuddhassa

\subsection{Saraṇa-gamana-pāṭho}
\label{buddham-saranam}

\begin{paritta}
Buddhaṃ saraṇaṃ gacchāmi\\
Dhammaṃ saraṇaṃ gacchāmi\\
Saṅghaṃ saraṇaṃ gacchāmi

Dutiyam pi buddhaṃ saraṇaṃ gacchāmi\\
Dutiyam pi dhammaṃ saraṇaṃ gacchāmi\\
Dutiyam pi saṅghaṃ saraṇaṃ gacchāmi

Tatiyam pi buddhaṃ saraṇaṃ gacchāmi\\
Tatiyam pi dhammaṃ saraṇaṃ gacchāmi\\
Tatiyam pi saṅghaṃ saraṇaṃ gacchāmi
\end{paritta}

\subsection{Sambuddhe}
\label{sambuddhe}

\firstline{Sambuddhe aṭṭhavīsañca}

\begin{twochants}
Sambuddhe aṭṭhavīsañca & dvādasañca sahassake\\
Pañca-sata-sahassāni & namāmi sirasā ahaṃ\\
Tesaṃ dhammañca saṅghañca & ādarena namāmihaṃ\\
Namakārānubhāvena & hantvā sabbe upaddave\\
Anekā antarāyāpi & vinassantu asesato\\
Sambuddhe pañca-paññāsañca & catuvīsati sahassake\\
Dasa-sata-sahassāni & namāmi sirasā ahaṃ\\
Tesaṃ dhammañca saṅghañca & ādarena namāmihaṃ\\
Namakārānubhāvena & hantvā sabbe upaddave\\
Anekā antarāyāpi & vinassantu asesato\\
Sambuddhe navuttarasate & aṭṭhacattāḷīsa sahassake\\
Vīsati-sata-sahassāni & namāmi sirasā ahaṃ\\
Tesaṃ dhammañca saṅghañca & ādarena namāmihaṃ\\
Namakārānubhāvena & hantvā sabbe upaddave\\
Anekā antarāyāpi & vinassantu asesato\\
\end{twochants}

\subsection{Nama-kāra-siddhi-gāthā}
\label{yo-cakkhuma}

\firstline{Yo cakkhumā moha-malāpakaṭṭho}

\begin{paritta}
Yo cakkhumā moha-malāpakaṭṭho\\
Sāmaṃ va buddho sugato vimutto\\
Mārassa pāsā vinimocayanto\\
Pāpesi khemaṃ janataṃ vineyyaṃ\\
\end{paritta}

\clearpage

\begin{paritta}
Buddhaṃ varan-taṃ sirasā namāmi\\
Lokassa nāthañ-ca vināyakañ-ca\\
Tan-tejasā te jaya-siddhi hotu\\
Sabb'antarāyā ca vināsamentu

Dhammo dhajo yo viya tassa satthu\\
Dassesi lokassa visuddhi-maggaṃ\\
Niyyāniko dhamma-dharassa dhārī\\
Sāt'āvaho santi-karo suciṇṇo\\
Dhammaṃ varan-taṃ sirasā namāmi\\
Mohappadālaṃ upasanta-dāhaṃ\\
Tan-tejasā te jaya-siddhi hotu\\
Sabb'antarāyā ca vināsamentu

Saddhamma-senā sugatānugo yo\\
Lokassa pāpūpakilesa-jetā\\
Santo sayaṃ santi-niyojako ca\\
Svākkhāta-dhammaṃ viditaṃ karoti\\
Saṅghaṃ varan-taṃ sirasā namāmi\\
Buddhānubuddhaṃ sama-sīla-diṭṭhiṃ\\
Tan-tejasā te jaya-siddhi hotu\\
Sabb'antarāyā ca vināsamentu
\end{paritta}

\suttaRef{Thai}

\clearpage

\subsection{Namo-kāra-aṭṭhaka}
\label{namo-arahato}

\firstline{Namo arahato sammā}

\begin{paritta}
  Namo arahato sammā\\
  Sambuddhassa mahesino\\
  Namo uttama-dhammassa\\
  Svākkhātass'eva ten'idha\\
  Namo mahā-saṅghassāpi\\
  Visuddha-sīla-diṭṭhino\\
  Namo omāty-āraddhassa\\
  Ratanattayassa sādhukaṃ\\
  Namo omakātītassa\\
  Tassa vatthuttayassa-pi\\
  Namo-kārappabhāvena\\
  Vigacchantu upaddavā\\
  Namo-kārānubhāvena\\
  Suvatthi hotu sabbadā\\
  Namo-kārassa tejena\\
  Vidhimhi homi tejavā
\end{paritta}

\suttaRef{Thai}

\section{Core Sequence}

\subsection{Maṅgala-sutta}
\label{asevana}

\firstline{Asevanā ca bālānaṃ}

\begin{paritta}
Asevanā ca bālānaṃ\\
Paṇḍitānañ-ca sevanā\\
Pūjā ca pūjanīyānaṃ\\
Etam maṅgalam-uttamaṃ

Paṭirūpa-desa-vāso ca\\
Pubbe ca kata-puññatā\\
Atta-sammā-paṇidhi ca\\
Etam maṅgalam-uttamaṃ

Bāhu-saccañ-ca sippañ-ca,\\
Vinayo ca susikkhito\\
Subhāsitā ca yā vācā\\
Etam maṅgalam-uttamaṃ

Mātā-pitu-upaṭṭhānaṃ\\
Putta-dārassa saṅgaho\\
Anākulā ca kammantā\\
Etam maṅgalam-uttamaṃ

\enlargethispage{\baselineskip}

Dānañ-ca dhamma-cariyā ca\\
Ñātakānañ-ca saṅgaho\\
Anavajjāni kammāni\\
Etam maṅgalam-uttamaṃ

Āratī viratī pāpā\\
Majja-pānā ca saññamo\\
Appamādo ca dhammesu\\
Etam maṅgalam-uttamaṃ

Gāravo ca nivāto ca\\
Santuṭṭhī ca kataññutā\\
Kālena dhammassavanaṃ\\
Etam maṅgalam-uttamaṃ

Khantī ca sovacassatā\\
Samaṇānañ-ca dassanaṃ\\
Kālena dhamma-sākacchā\\
Etam maṅgalam-uttamaṃ

Tapo ca brahma-cariyañ-ca\\
Ariya-saccāna-dassanaṃ\\
Nibbāna-sacchikiriyā ca\\
Etam maṅgalam-uttamaṃ

Phuṭṭhassa loka-dhammehi\\
Cittaṃ yassa na kampati\\
Asokaṃ virajaṃ khemaṃ\\
Etam maṅgalam-uttamaṃ

\clearpage

Etādisāni katvāna\\
Sabbattham-aparājitā\\
Sabbattha sotthiṃ gacchanti\\
Tan-tesaṃ maṅgalam-uttaman'ti
\end{paritta}

\suttaRef{Snp 2.4}

\subsection{Ratana-sutta}

\firstline{Yānīdha bhūtāni samāgatāni}

\instr{(In certain monasteries only the numbered verses are chanted.)}

\bigskip

\begin{paritta}

Yānīdha bhūtāni samāgatāni\\
Bhummāni vā yāni va antalikkhe\\
Sabb'eva bhūtā sumanā bhavantu\\
Atho pi sakkacca suṇantu bhāsitaṃ\\
Tasmā hi bhūtā nisāmetha sabbe\\
Mettaṃ karotha mānusiyā pajāya\\
Divā ca ratto ca haranti ye baliṃ\\
Tasmā hi ne rakkhatha appamattā

\firstline{Yaṅkiñci vittaṃ idha vā huraṃ vā}

\label{yankinci-vittam}
\sidepar{1.}%
Yaṅkiñci vittaṃ idha vā huraṃ vā\\
Saggesu vā yaṃ ratanaṃ paṇītaṃ\\
Na no samaṃ atthi tathāgatena\\
Idam-pi buddhe ratanaṃ paṇītaṃ\\
Etena saccena suvatthi hotu

\sidepar{2.}%
Khayaṃ virāgaṃ amataṃ paṇītaṃ\\
Yad-ajjhagā sakya-munī samāhito\\
Na tena dhammena sam'atthi kiñci\\
Idam-pi dhamme ratanaṃ paṇītaṃ\\
Etena saccena suvatthi hotu

\sidepar{3.}%
Yam buddha-seṭṭho parivaṇṇayī suciṃ\\
Samādhim-ānantarikaññam-āhu\\
Samādhinā tena samo na vijjati\\
Idam-pi dhamme ratanaṃ paṇītaṃ\\
Etena saccena suvatthi hotu

\sidepar{4.}%
Ye puggalā aṭṭha sataṃ pasaṭṭhā\\
Cattāri etāni yugāni honti\\
Te dakkhiṇeyyā sugatassa sāvakā\\
Etesu dinnāni mahapphalāni\\
Idam-pi saṅghe ratanaṃ paṇītaṃ\\
Etena saccena suvatthi hotu

\sidepar{5.}%
Ye suppayuttā manasā daḷhena\\
Nikkāmino gotama-sāsanamhi\\
Te patti-pattā amataṃ vigayha\\
Laddhā mudhā nibbutiṃ bhuñjamānā\\
Idam-pi saṅghe ratanaṃ paṇītaṃ\\
Etena saccena suvatthi hotu

Yath'inda-khīlo paṭhaviṃ sito siyā\\
Catubbhi vātebhi asampakampiyo\\
Tathūpamaṃ sappurisaṃ vadāmi\\
Yo ariya-saccāni avecca passati\\
Idam-pi Saṅghe ratanaṃ paṇītaṃ\\
Etena saccena suvatthi hotu

Ye ariya-saccāni vibhāvayanti\\
Gambhīra-paññena sudesitāni\\
Kiñ-cāpi te honti bhusappamattā\\
Na te bhavaṃ aṭṭhamam-ādiyanti\\
Idam-pi Saṅghe ratanaṃ paṇītaṃ\\
Etena saccena suvatthi hotu

Sahā v'assa dassana-sampadāya\\
Tay'assu dhammā jahitā bhavanti\\
Sakkāya-diṭṭhi vicikicchitañ-ca\\
Sīlabbataṃ vā pi yad-atthi kiñci\\
Catūh'apāyehi ca vippamutto\\
Cha cābhiṭhānāni abhabbo kātuṃ\\
Idam-pi Saṅghe ratanaṃ paṇītaṃ\\
Etena saccena suvatthi hotu

Kiñ-cāpi so kammaṃ karoti pāpakaṃ\\
Kāyena vācā uda cetasā vā\\
Abhabbo so tassa paṭicchādāya\\
Abhabbatā diṭṭha-padassa vuttā\\
Idam-pi Saṅghe ratanaṃ paṇītaṃ\\
Etena saccena suvatthi hotu

Vanappagumbe yathā phussitagge\\
Gimhāna-māse paṭhamasmiṃ gimhe\\
Tathūpamaṃ dhamma-varaṃ adesayi\\
Nibbāna-gāmiṃ paramaṃ hitāya\\
Idam-pi Buddhe ratanaṃ paṇītaṃ\\
Etena saccena suvatthi hotu

Varo varaññū varado var'āharo\\
Anuttaro dhamma-varaṃ adesayi\\
Idam-pi Buddhe ratanaṃ paṇītaṃ\\
Etena saccena suvatthi hotu

\sidepar{6.}%
Khīṇaṃ purāṇaṃ navaṃ n'atthi sambhavaṃ\\
Viratta-citt'āyatike bhavasmiṃ\\
Te khīṇa-bījā aviruḷhi-chandā\\
Nibbanti dhīrā yathā'yam padīpo\\
Idam-pi saṅghe ratanaṃ paṇītaṃ\\
Etena saccena suvatthi hotu.

\enlargethispage{\baselineskip}

Yānīdha bhūtāni samāgatāni\\
Bhummāni vā yāni va antalikkhe\\
Tathāgataṃ deva-manussa-pūjitaṃ\\
Buddhaṃ namassāma suvatthi hotu

Yānīdha bhūtāni samāgatāni\\
Bhummāni vā yāni va antalikkhe\\
Tathāgataṃ deva-manussa-pūjitaṃ\\
Dhammaṃ namassāma suvatthi hotu

Yānīdha bhūtāni samāgatāni\\
Bhummāni vā yāni va antalikkhe\\
Tathāgataṃ deva-manussa-pūjitaṃ\\
Saṅghaṃ namassāma suvatthi hotū'ti. \suttaRef{Snp 2.1}

\end{paritta}

\subsection{The Buddha's Words on Loving-Kindness}
\label{karaniyam-attha}

% TODO Pali title
% Karaṇīya-metta-sutta

\firstline{Karaṇīyam-attha-kusalena}

\begin{paritta}

Karaṇīyam-attha-kusalena\\
Yan-taṃ santaṃ padaṃ abhisamecca\\
Sakko ujū ca suhujū ca\\
Suvaco c'assa mudu anatimānī

Santussako ca subharo ca\\
Appakicco ca sallahuka-vutti\\
Sant'indriyo ca nipako ca\\
Appagabbho kulesu ananugiddho

Na ca khuddaṃ samācare kiñci\\
Yena viññū pare upavadeyyuṃ\\
Sukhino vā khemino hontu\\
Sabbe sattā bhavantu sukhit'attā

Ye keci pāṇa-bhūt'atthi\\
Tasā vā thāvarā vā anavasesā\\
Dīghā vā ye mahantā vā\\
Majjhimā rassakā aṇuka-thūlā

Diṭṭhā vā ye ca adiṭṭhā\\
Ye ca dūre vasanti avidūre\\
Bhūtā vā sambhavesī vā\\
Sabbe sattā bhavantu sukhit'attā

Na paro paraṃ nikubbetha\\
Nātimaññetha katthaci naṃ kiñci\\
Byārosanā paṭighasaññā\\
Nāññam-aññassa dukkham-iccheyya

Mātā yathā niyaṃ puttaṃ\\
Āyusā eka-puttam-anurakkhe\\
Evam'pi sabba-bhūtesu\\
Mānasam-bhāvaye aparimāṇaṃ

\firstline{Mettañ-ca sabba-lokasmiṃ}

Mettañ-ca sabba-lokasmiṃ\\
Mānasam-bhāvaye aparimāṇaṃ\\
Uddhaṃ adho ca tiriyañ-ca\\
Asambādhaṃ averaṃ asapattaṃ

Tiṭṭhañ-caraṃ nisinno vā\\
Sayāno vā yāvat'assa vigata-middho\\
Etaṃ satiṃ adhiṭṭheyya\\
Brahmam-etaṃ vihāraṃ idham-āhu

Diṭṭhiñca anupagamma\\
Sīlavā dassanena sampanno\\
Kāmesu vineyya gedhaṃ\\
Na hi jātu gabbha-seyyaṃ punaretī'ti \suttaRef{Snp 1.8}

\end{paritta}

\subsection{The Buddha's Words on Loving-Kindness (English)}

% TODO ? Pali or English title

\begin{leader}
  [Now let us chant the Buddha's words on loving-kindness]
\end{leader}

\firstline{This is what should be done}

[This is what should be done]\\
By one who is skilled in goodness\\
And who knows the path of peace:\\
Let them be able and upright,\\
Straightforward and gentle in speech,

Humble and not conceited,\\
Contented and easily satisfied,\\
Unburdened with duties and frugal in their ways.\\
Peaceful and calm, and wise and skilful,\\
Not proud and demanding in nature.

\enlargethispage{\baselineskip}

Let them not do the slightest thing\\
That the wise would later reprove,\\
Wishing: In gladness and in safety,\\
May all beings be at ease.

Whatever living beings there may be,\\
Whether they are weak or strong, omitting none,\\
The great or the mighty, medium, short, or small,

The seen and the unseen,\\
Those living near and far away,\\
Those born and to be born,\\
May all beings be at ease.

Let none deceive another\\
Or despise any being in any state.\\
Let none through anger or ill-will\\
Wish harm upon another.

Even as a mother protects with her life\\
Her child, her only child,\\
So with a boundless heart\\
Should one cherish all living beings,\\
Radiating kindness over the entire world:

Spreading upwards to the skies\\
And downwards to the depths,\\
Outwards and unbounded,\\
Freed from hatred and ill-will.

Whether standing or walking, seated, \\
Or lying down --- free from drowsiness ---

\clearpage

One should sustain this recollection.\\
This is said to be the sublime abiding.

By not holding to fixed views,\\
The pure-hearted one, having clarity of vision,\\
Being freed from all sense-desires,\\
Is not born again into this world. \suttaRef{Snp 1.8}

\subsection{Khandha-paritta}
\label{virupakkhehi}

\firstline{Virūpakkhehi me mettaṃ mettaṃ erāpathehi me}

\enlargethispage{\baselineskip}

Virūpakkhehi me mettaṃ\\\vin mettaṃ erāpathehi me\\
Chabyā-puttehi me mettaṃ\\\vin mettaṃ kaṇhā-gotamakehi ca\\
Apādakehi me mettaṃ\\\vin mettaṃ dipādakehi me\\
Catuppadehi me mettaṃ\\\vin mettaṃ bahuppadehi me\\
Mā maṃ apādako hiṃsi\\\vin mā maṃ hiṃsi dipādako\\
Mā maṃ catuppado hiṃsi\\\vin mā maṃ hiṃsi bahuppado\\
Sabbe sattā sabbe pāṇā\\\vin sabbe bhūtā ca kevalā\\
Sabbe bhadrāni passantu\\\vin mā kiñci pāpam-āgamā

\clearpage

\firstline{Appamāṇo buddho appamāṇo dhammo}

Appamāṇo buddho\\\vin appamāṇo dhammo\\\vin appamāṇo saṅgho\\
Pamāṇavantāni siriṃsapāni\\\vin ahi-vicchikā sata-padī\\
Uṇṇā-nābhī sarabhū mūsikā

Katā me rakkhā katā me parittā\\\vin paṭikkamantu bhūtāni\\
So'haṃ namo bhagavato\\\vin namo sattannaṃ\\\vin sammā-sambuddhānaṃ

\suttaRef{A.II.72-73}

\subsection{Chaddanta-paritta}
\label{vadhissamenanti}

% The Great Elephant Protection
% The Ivory Protection

\firstline{Vadhissamenanti parāmasanto}

\begin{paritta}

Vadhissamenanti parāmasanto\\
Kāsāvamaddakkhi dhajaṃ isīnaṃ\\
Dukkhena phuṭṭhassudapādi saññā\\
Arahaddhajo sabbhi avajjharūpo

Sallena viddho byathitopi santo\\
Kāsāvavatthamhi manaṃ na dussayi\\
Sace imaṃ nāgavarena saccaṃ\\
Mā maṃ vane bālamigā agañchunti
\end{paritta}

% Gavesako: Is this ever chanted in our tradition? Or Dhammayut only?

\subsection{Mora-paritta}
\label{udetayan-cakkhuma}

\firstline{Udet'ayañ-cakkhumā eka-rājā}
\firstline{Apet'ayañ-cakkhumā eka-rājā}

\instr{a.m.}

Udet'ayañ-cakkhumā eka-rājā\\
Harissa-vaṇṇo paṭhavippabhāso\\
Taṃ taṃ namassāmi harissa-vaṇṇaṃ paṭhavippabhāsaṃ\\
Tay'ajja guttā viharemu divasaṃ

Ye brāhmaṇā vedagu sabba-dhamme\\
Te me namo te ca maṃ pālayantu\\
Nam'atthu Buddhānaṃ nam'atthu bodhiyā\\
Namo vimuttānaṃ namo vimuttiyā\\
Imaṃ so parittaṃ katvā\\
Moro carati esanā'ti

\instr{p.m.}

\enlargethispage{\baselineskip}

Apet'ayañ-cakkhumā eka-rājā\\
Harissa-vaṇṇo paṭhavippabhāso\\
Taṃ taṃ namassāmi harissa-vaṇṇaṃ paṭhavippabhāsaṃ\\
Tay'ajja guttā viharemu rattiṃ

Ye brāhmaṇā vedagu sabba-dhamme\\
Te me namo te ca maṃ pālayantu\\
Nam'atthu Buddhānaṃ nam'atthu bodhiyā\\
Namo vimuttānaṃ namo vimuttiyā\\
Imaṃ so parittaṃ katvā\\
Moro vāsam-akappayī'ti \suttaRef{Ja.159}

\subsection{Vaṭṭaka-paritta}
\label{atthi-loke}

\firstline{Atthi loke sīla-guṇo saccaṃ soceyy'anuddayā}

\begin{twochants}
Atthi loke sīla-guṇo & saccaṃ soceyy'anuddayā\\
Tena saccena kāhāmi & sacca-kiriyam-anuttaraṃ\\
Āvajjitvā dhamma-balaṃ & saritvā pubbake jine\\
Sacca-balam-avassāya & sacca-kiriyam-akās'ahaṃ\\
Santi pakkhā apattanā & santi pādā avañcanā\\
Mātā pitā ca nikkhantā & jāta-veda paṭikkama\\
Saha sacce kate mayhaṃ & mahā-pajjalito sikhī\\
Vajjesi soḷasa karīsāni & udakaṃ patvā yathā sikhī\\
Saccena me samo n'atthi & esā me sacca-pāramī'ti\\
\end{twochants}

\suttaRef{Cariyāpiṭaka vv.319-322}

\subsection{Buddha-dhamma-saṅgha-guṇā}
\label{iti-pi-so}

\firstline{Iti pi so bhagavā arahaṃ sammā-sambuddho}

\begin{paritta}
Iti pi so bhagavā arahaṃ sammā-sambuddho\\
Vijjā-caraṇa-sampanno sugato loka-vidū\\
Anuttaro purisa-damma-sārathi\\
Satthā devamanussānaṃ buddho bhagavā'ti

Svākkhāto bhagavatā dhammo sandiṭṭhiko\\
\vin akāliko ehi-passiko opanayiko\\
paccattaṃ veditabbo viññūhī'ti

\enlargethispage{\baselineskip}

Supaṭipanno bhagavato sāvaka-saṅgho\\
Uju-paṭipanno bhagavato sāvaka-saṅgho\\
Ñāya-paṭipanno bhagavato sāvaka-saṅgho\\
Sāmīci-paṭipanno bhagavato sāvaka-saṅgho\\
Yad-idaṃ cattāri purisa-yugāni aṭṭha purisa-puggalā\\
Esa bhagavato sāvaka-saṅgho\\
Āhuneyyo pāhuneyyo dakkhiṇeyyo añjali-karaṇīyo\\
Anuttaraṃ puññakkhettaṃ lokassā'ti
\end{paritta}

\subsubsection{Araññe rukkha-mūle vā}

\firstline{Araññe rukkha-mūle vā}

\begin{paritta}
Araññe rukkha-mūle vā\\
Suññāgāre va bhikkhavo\\
Anussaretha sambuddhaṃ\\
Bhayaṃ tumhāka no siyā\\
No ce buddhaṃ sareyyātha\\
Loka-jeṭṭhaṃ nar'āsabhaṃ\\
Atha dhammaṃ sareyyātha\\
Niyyānikaṃ sudesitaṃ\\
No ce dhammaṃ sareyyātha\\
Niyyānikaṃ sudesitaṃ\\
Atha saṅghaṃ sareyyātha\\
Puññakkhettaṃ anuttaraṃ\\
Evam-buddhaṃ sarantānaṃ\\
Dhammaṃ saṅghañ-ca bhikkhavo\\
Bhayaṃ vā chambhitattaṃ vā\\
Loma-haṃso na hessatī'ti. \suttaRef{S.I.219-220}
\end{paritta}

\subsection{Āṭānāṭiya-paritta (short)}
\label{vipassissa}

\firstline{Vipassissa nam'atthu cakkhumantassa sirīmato}

\begin{twochants}
Vipassissa nam'atthu & cakkhumantassa sirīmato\\
Sikhissa pi nam'atthu & sabba-bhūtānukampino\\
Vessabhussa nam'atthu & nhātakassa tapassino\\
Nam'atthu kakusandhassa & māra-senappamaddino\\
Konāgamanassa nam'atthu & brāhmaṇassa vusīmato\\
Kassapassa nam'atthu & vippamuttassa sabbadhi\\
Aṅgīrasassa nam'atthu & sakya-puttassa sirīmato\\
Yo imaṃ dhammam-adesesi & sabba-dukkhāpanūdanaṃ\\
Ye cāpi nibbutā loke & yathā-bhūtaṃ vipassisuṃ\\
Te janā apisuṇā & mahantā vīta-sāradā\\
\end{twochants}

\begin{twochants}
Hitaṃ deva-manussānaṃ & yaṃ namassanti gotamaṃ\\
Vijjā-caraṇa-sampannaṃ & mahantaṃ vīta-sāradaṃ\\
Vijjā-caraṇa-sampannaṃ & buddhaṃ vandāma gotaman'ti\\
\end{twochants}

\suttaRef{D.III.195-196}% TODO Confirm Pali source

\subsection{Sacca-kiriyā-gāthā}
\label{natthi-me}

\firstline{Natthi me saraṇaṃ aññaṃ}

Natthi me saraṇaṃ aññaṃ buddho me saraṇaṃ varaṃ\\
Etena sacca-vajjena sotthi te/me hotu sabbadā

Natthi me saraṇaṃ aññaṃ dhammo me saraṇaṃ varaṃ\\
Etena sacca-vajjena sotthi te/me hotu sabbadā

Natthi me saraṇaṃ aññaṃ saṅgho me saraṇaṃ varaṃ\\
Etena sacca-vajjena sotthi te/me hotu sabbadā

\subsection{Yaṅkiñci ratanaṃ loke}
\label{yankinci-ratanam}

\firstline{Yaṅkiñci ratanaṃ loke}

\begin{twochants}
  Yaṅkiñci ratanaṃ loke & vijjati vividhaṃ puthu\\
  Ratanaṃ buddhasamaṃ & natthi tasmā sotthī bhavantu te\\
  Yaṅkiñci ratanaṃ loke & vijjati vividhaṃ puthu\\
  Ratanaṃ dhammasamaṃ & natthi tasmā sotthī bhavantu te\\
  Yaṅkiñci ratanaṃ loke & vijjati vividhaṃ puthu\\
  Ratanaṃ saṅghasamaṃ & natthi tasmā sotthī bhavantu te\\
\end{twochants}

\subsection{Sakkatvā buddharatanaṃ}
\label{sakkatva}

\firstline{Sakkatvā buddharatanaṃ}

\begin{twochants}
  Sakkatvā buddharatanaṃ & osadhaṃ uttamaṃ varaṃ\\
  Hitaṃ devamanussānaṃ & buddhatejena sotthinā\\
  Nassantupaddavā sabbe & dukkhā vūpasamentu te\\
  Sakkatvā dhammaratanaṃ & osadhaṃ uttamaṃ varaṃ\\
  Pariḷāhūpasamanaṃ & dhammatejena sotthinā\\
  Nassantupaddavā sabbe & bhayā vūpasamentu te\\
  Sakkatvā saṅgharatanaṃ & osadhaṃ uttamaṃ varaṃ\\
  Āhuneyyaṃ pāhuneyyaṃ & saṅghatejena sotthinā\\
  Nassantupaddavā sabbe & rogā vūpasamentu te\\
\end{twochants}

\bigskip

{\centering
  \instr{The \emph{jet tamnaan} sequence ends here\\ and continues with the closing sequence.}
\par}

\subsection{Aṅgulimāla-paritta}
\label{yato-ham-bhagini}

\firstline{Yato'haṃ bhagini ariyāya jātiyā jāto}

\begin{paritta}
Yato'haṃ bhagini ariyāya jātiyā jāto\\
Nābhijānāmi sañcicca pāṇaṃ jīvitā voropetā\\
Tena saccena sotthi te hotu sotthi gabbhassa\\
\suttaRef{M.II.103}% TODO Confirm Pali source
\end{paritta}

\subsection{Bojjhaṅga-paritta}
\label{bojjhango}

\firstline{Bojjhaṅgo sati-saṅkhāto}

\begin{twochants}
Bojjhaṅgo sati-saṅkhāto & dhammānaṃ vicayo tathā\\
Viriyam-pīti-passaddhi & bojjhaṅgā ca tathā'pare\\
Samādh'upekkha-bojjhaṅgā & satt'ete sabba-dassinā\\
Muninā sammad-akkhātā & bhāvitā bahulīkatā\\
Saṃvattanti abhiññāya & nibbānāya ca bodhiyā\\
Etena sacca-vajjena & sotthi te hotu sabbadā\\
Ekasmiṃ samaye nātho & moggallānañ-ca kassapaṃ\\
Gilāne dukkhite disvā & bojjhaṅge satta desayi\\
Te ca taṃ abhinanditvā & rogā mucciṃsu taṅ-khaṇe\\
Etena sacca-vajjena & sotthi te hotu sabbadā\\
Ekadā dhamma-rājā pi & gelaññenābhipīḷito\\
Cundattherena tañ-ñeva & bhaṇāpetvāna sādaraṃ\\
Sammoditvā ca ābādhā & tamhā vuṭṭhāsi ṭhānaso\\
Etena sacca-vajjena & sotthi te hotu sabbadā\\
Pahīnā te ca ābādhā & tiṇṇannam-pi mahesinaṃ\\
Magg'āhata-kilesā va & pattānuppatti-dhammataṃ\\
Etena sacca-vajjena & sotthi te hotu sabbadā\\
\end{twochants}

\suttaRef{S.V.80f}

\subsection{Abhaya-paritta}
\label{yan-dunnimittam}

\firstline{Yan-dunnimittaṃ avamaṅgalañ-ca}

\begin{paritta}
Yan-dunnimittaṃ avamaṅgalañ-ca\\
Yo cāmanāpo sakuṇassa saddo\\
Pāpaggaho dussupinaṃ akantaṃ\\
Buddhānubhāvena vināsamentu

Yan-dunnimittaṃ avamaṅgalañ-ca\\
Yo cāmanāpo sakuṇassa saddo\\
Pāpaggaho dussupinaṃ akantaṃ\\
Dhammānubhāvena vināsamentu

Yan-dunnimittaṃ avamaṅgalañ-ca\\
Yo cāmanāpo sakuṇassa saddo\\
Pāpaggaho dussupinaṃ akantaṃ\\
Saṅghānubhāvena vināsamentu \suttaRef{Trad.}
\end{paritta}

\bigskip

{\centering
  \instr{The \emph{sipsong tamnaan} sequence ends here\\ and continues with the closing sequence.}
\par}

\clearpage

\section{Closing Sequence}

\subsection{Devatā-uyyojana-gāthā}
\label{dukkhappatta}

\firstline{Dukkhappattā ca niddukkhā}
\firstline{Sabbe buddhā balappattā}

\begin{twochants}
Dukkhappattā ca niddukkhā & bhayappattā ca nibbhayā\\
Sokappattā ca nissokā & hontu sabbe pi pāṇino\\
Ettāvatā ca amhehi & sambhataṃ puñña-sampadaṃ\\
Sabbe devānumodantu & sabba-sampatti-siddhiyā\\
Dānaṃ dadantu saddhāya & sīlaṃ rakkhantu sabbadā\\
Bhāvanābhiratā hontu & gacchantu devatā-gatā\\\relax
[Sabbe buddhā] balappattā & paccekānañ-ca yaṃ balaṃ\\
Arahantānañ-ca tejena & rakkhaṃ bandhāmi sabbaso\\
\end{twochants}

%\suttaRef{MJG}

\enlargethispage{\baselineskip}

\subsection{Jaya-maṅgala-aṭṭha-gāthā}
\label{bahum}

\firstline{Bāhuṃ sahassam-abhinimmita sāvudhan-taṃ}

\begin{paritta}
Bāhuṃ sahassam-abhinimmita sāvudhan-taṃ\\
Grīmekhalaṃ udita-ghora-sasena-māraṃ\\
Dān'ādi-dhamma-vidhinā jitavā mun'indo\\
Tan-tejasā bhavatu te jaya-maṅgalāni

Mārātirekam-abhiyujjhita-sabba-rattiṃ\\
Ghoram-pan'āḷavakam-akkhama-thaddha-yakkhaṃ\\
Khantī-sudanta-vidhinā jitavā mun'indo\\
Tan-tejasā bhavatu te jaya-maṅgalāni

\clearpage

Nāḷāgiriṃ gaja-varaṃ atimatta-bhūtaṃ\\
Dāv'aggi-cakkam-asanīva sudāruṇan-taṃ\\
Mett'ambu-seka-vidhinā jitavā mun'indo\\
Tan-tejasā bhavatu te jaya-maṅgalāni

Ukkhitta-khaggam-atihattha-sudāruṇan-taṃ\\
Dhāvan-ti-yojana-path'aṅguli- mālavantaṃ\\
Iddhī'bhisaṅkhata-mano jitavā mun'indo\\
Tan-tejasā bhavatu te jaya-maṅgalāni

Katvāna kaṭṭham-udaraṃ iva gabbhinīyā\\
Ciñcāya duṭṭha-vacanaṃ jana-kāya majjhe\\
Santena soma-vidhinā jitavā mun'indo\\
Tan-tejasā bhavatu te jaya-maṅgalāni

Saccaṃ vihāya-mati-saccaka-vāda-ketuṃ\\
Vādābhiropita-manaṃ ati-andha-bhūtaṃ\\
Paññā-padīpa-jalito jitavā mun'indo\\
Tan-tejasā bhavatu te jaya-maṅgalāni

Nandopananda-bhujagaṃ vibudhaṃ mah'iddhiṃ\\
Puttena thera-bhujagena damāpayanto\\
Iddhūpadesa-vidhinā jitavā mun'indo\\
Tan-tejasā bhavatu te jaya-maṅgalāni

Duggāha-diṭṭhi-bhujagena sudaṭṭha-hatthaṃ\\
Brahmaṃ visuddhi-jutim-iddhi-bakābhidhānaṃ\\
Ñāṇāgadena vidhinā jitavā mun'indo\\
Tan-tejasā bhavatu te jaya-maṅgalāni

Etā pi buddha-jaya-maṅgala-aṭṭha-gāthā\\
Yo vācano dina-dine saratem-atandī\\
Hitvān'aneka-vividhāni c'upaddavāni\\
Mokkhaṃ sukhaṃ adhigameyya naro sapañño \suttaRef{Trad.}
\end{paritta}

\subsection{Jaya-paritta}
\label{maha-karuniko}

\firstline{Mahā-kāruṇiko nātho hitāya sabba-pāṇinaṃ}

\begin{twochants}
Mahā-kāruṇiko nātho & hitāya sabba-pāṇinaṃ\\
Pūretvā pāramī sabbā & patto sambodhim-uttamaṃ\\
Etena sacca-vajjena & hotu te jaya-maṅgalaṃ\\
Jayanto bodhiyā mūle & sakyānaṃ nandi-vaḍḍhano\\
Evaṃ tvaṃ vijayo hohi & jayassu jaya-maṅgale\\
Aparājita-pallaṅke & sīse paṭhavi-pokkhare\\
Abhiseke sabba-buddhānaṃ & aggappatto pamodati\\
Sunakkhattaṃ sumaṅgalaṃ & supabhātaṃ suhuṭṭhitaṃ\\
Sukhaṇo sumuhutto ca & suyiṭṭhaṃ brahma-cārisu\\
Padakkhiṇaṃ kāya-kammaṃ & vācā-kammaṃ padakkhiṇaṃ\\
Padakkhiṇaṃ mano-kammaṃ & paṇidhi te padakkhiṇā\\
Padakkhiṇāni katvāna & labhant'atthe padakkhiṇe
\end{twochants}

%\suttaRef{MJG}
\suttaRef{A.I.294}% TODO Confirm Pali source

\subsection{So attha-laddho}

\firstline{So attha-laddho sukhito viruḷho buddha-sāsane}

\begin{twochants}
So attha-laddho sukhito & viruḷho buddha-sāsane\\
Arogo sukhito hohi & saha sabbehi ñātibhi (×3)\\
\end{twochants}

\subsection{Sā attha-laddhā}

\firstline{Sā attha-laddhā sukhitā viruḷhā buddha-sāsane}

\begin{twochants}
Sā attha-laddhā sukhitā & viruḷhā buddha-sāsane\\
Arogā sukhitā hohi & saha sabbehi ñātibhi (×3)\\
\end{twochants}

\subsection{Te attha-laddhā sukhitā}
\label{te-attha-laddha}

\firstline{Te attha-laddhā sukhitā viruḷhā buddha-sāsane}

\begin{twochants}
Te attha-laddhā sukhitā & viruḷhā buddha-sāsane\\
Arogā sukhitā hotha & saha sabbehi ñātibhi (×3)\\
\end{twochants}

\suttaRef{cf. A.I.294}% TODO Confirm Pali source

\subsection{Bhavatu sabba-maṅgalaṃ}
\label{bhavatu}

\firstline{Bhavatu sabba-maṅgalaṃ}

Bhavatu sabba-maṅgalaṃ rakkhantu sabba-devatā\\
Sabba-buddhānubhāvena sadā sotthī bhavantu te

Bhavatu sabba-maṅgalaṃ rakkhantu sabba-devatā\\
Sabba-dhammānubhāvena sadā sotthī bhavantu te

Bhavatu sabba-maṅgalaṃ rakkhantu sabba-devatā\\
Sabba-saṅghānubhāvena sadā sotthī bhavantu te

\section{Mahā-kāruṇiko nātho'ti ādikā gāthā}

\firstline{Mahā-kāruṇiko nātho atthāya sabba-pāṇinaṃ}

\begin{paritta}
Mahā-kāruṇiko nātho\\
Atthāya sabba-pāṇinaṃ\\
Hitāya sabba-pāṇinaṃ\\
Sukhāya sabba-pāṇinaṃ

Pūretvā pāramī sabbā\\
Patto sambodhim-uttamaṃ\\
Etena sacca-vajjena\\
Mā hontu sabb'upaddavā
\end{paritta}

%\suttaRef{MJG}

\clearpage

\section{Āṭānāṭiya-paritta (long)}

% The Twenty-Eight Buddhas' Protection
% Āṭānāṭiya Paritta

\begin{leader}
\soloinstr{(Solo introduction)}

\firstline{Appasannehi nāthassa sāsane sādhusammate}

\begin{solotwochants}
  Appasannehi nāthassa & sāsane sādhusammate\\
  Amanussehi caṇḍehi & sadā kibbisakāribhi\\
  Parisānañca-tassannam & ahiṃsāya ca guttiyā\\
  Yandesesi mahāvīro & parittan-tam bhaṇāma se\\
\end{solotwochants}
\end{leader}

\firstline{Namo me sabbabuddhānaṃ}

{\centering
  \instr{(If starting with \emph{Vipassissa\ldots}, continue below\\
    without the solo introduction)}
\par}

\enlargethispage{\baselineskip}

\begin{twochants}
  [Namo me sabbabuddhānaṃ] & uppannānaṃ mahesinaṃ\\
  Taṇhaṅkaro mahāvīro & medhaṅkaro mahāyaso\\
  Saraṇaṅkaro lokahito & dīpaṅkaro jutindharo\\
  Koṇḍañño janapāmokkho & maṅgalo purisāsabho\\
  Sumano sumano dhīro & revato rativaḍḍhano\\
  Sobhito guṇasampanno & anomadassī januttamo\\
  Padumo lokapajjoto & nārado varasārathī\\
  Padumuttaro sattasāro & sumedho appaṭipuggalo\\
  Sujāto sabbalokaggo & piyadassī narāsabho\\
  Atthadassī kāruṇiko & dhammadassī tamonudo\\
  Siddhattho asamo loke & tisso ca vadataṃ varo\\
  Phusso ca varado buddho & vipassī ca anūpamo\\
  Sikhī sabbahito satthā & vessabhū sukhadāyako\\
  Kakusandho satthavāho & koṇāgamano raṇañjaho\\
  Kassapo sirisampanno & gotamo sakyapuṅgavo\\
\end{twochants}

\clearpage

\enlargethispage{\baselineskip}

\begin{twochants}
  Ete caññe ca sambuddhā & anekasatakoṭayo\\
  Sabbe buddhā asamasamā & sabbe buddhā mahiddhikā\\
  Sabbe dasabalūpetā & vesārajjehupāgatā\\
  Sabbe te paṭijānanti & āsabhaṇṭhānamuttamaṃ\\
  Sīhanādaṃ nadantete & parisāsu visāradā\\
  Brahmacakkaṃ pavattenti & loke appaṭivattiyaṃ\\
  Upetā buddhadhammehi & aṭṭhārasahi nāyakā\\
  Dvattiṃsa-lakkhaṇūpetā & sītyānubyañjanādharā\\
  Byāmappabhāya suppabhā & sabbe te munikuñjarā\\
  Buddhā sabbaññuno ete & sabbe khīṇāsavā jinā\\
  Mahappabhā mahātejā & mahāpaññā mahabbalā\\
  Mahākāruṇikā dhīrā & sabbesānaṃ sukhāvahā\\
  Dīpā nāthā patiṭṭhā & ca tāṇā leṇā ca pāṇinaṃ\\
  Gatī bandhū mahassāsā & saraṇā ca hitesino\\
  Sadevakassa lokassa & sabbe ete parāyanā\\
  Tesāhaṃ sirasā pāde & vandāmi purisuttame\\
  Vacasā manasā ceva & vandāmete tathāgate\\
  Sayane āsane ṭhāne & gamane cāpi sabbadā\\
  Sadā sukhena rakkhantu & buddhā santikarā tuvaṃ\\
  Tehi tvaṃ rakkhito santo & mutto sabbabhayena ca\\
\end{twochants}

\clearpage

\savenotes

\firstline{Tesaṃ saccena sīlena khantimettābalena ca}

\begin{twochants}
  Sabba-rogavinimutto & sabba-santāpavajjito\\
  Sabba-veramatikkanto & nibbuto ca tuvaṃ bhava\\
  Tesaṃ saccena sīlena & khantimettābalena ca\\
  Tepi tumhe%
  \footnote{If chanting for oneself, change \textit{tumhe} to \textit{amhe} here and in the lines below.}
  anurakkhantu & ārogyena sukhena ca\\
  Puratthimasmiṃ disābhāge & santi bhūtā mahiddhikā\\
  Tepi tumhe anurakkhantu & ārogyena sukhena ca\\
  Dakkhiṇasmiṃ disābhāge & santi devā mahiddhikā\\
  Tepi tumhe anurakkhantu & ārogyena sukhena ca\\
  Pacchimasmiṃ disābhāge & santi nāgā mahiddhikā\\
  Tepi tumhe anurakkhantu & ārogyena sukhena ca\\
  Uttarasmiṃ disābhāge & santi yakkhā mahiddhikā\\
  Tepi tumhe anurakkhantu & ārogyena sukhena ca\\
  Purimadisaṃ dhataraṭṭho & dakkhiṇena viruḷhako\\
  Pacchimena virūpakkho & kuvero uttaraṃ disaṃ\\
  Cattāro te mahārājā & lokapālā yasassino\\
  Tepi tumhe anurakkhantu & ārogyena sukhena ca\\
  Ākāsaṭṭhā ca bhummaṭṭhā & devā nāgā mahiddhikā\\
  Tepi tumhe anurakkhantu & ārogyena sukhena ca\\
\end{twochants}

\spewnotes

\subsection{Natthi me saraṇaṃ aññaṃ}

\firstline{Natthi me saraṇaṃ aññaṃ}

\savenotes

\begin{twochants}
  Natthi me saraṇaṃ aññaṃ & buddho me saraṇaṃ varaṃ\\
  Etena saccavajjena & hotu te%
  \footnote{If chanting for oneself, change \textit{te} to \textit{me} here and in the lines below.}
  jayamaṅgalaṃ\\
  Natthi me saraṇaṃ aññaṃ & dhammo me saraṇaṃ varaṃ\\
  Etena saccavajjena & hotu te jayamaṅgalaṃ\\
  Natthi me saraṇaṃ aññaṃ & saṅgho me saraṇaṃ varaṃ\\
  Etena saccavajjena & hotu te jayamaṅgalaṃ\\
\end{twochants}

\spewnotes

\subsection{Yaṅkiñci ratanaṃ loke}

\firstline{Yaṅkiñci ratanaṃ loke vijjati vividhaṃ puthu}

\begin{twochants}
  Yaṅkiñci ratanaṃ loke & vijjati vividhaṃ puthu\\
  Ratanaṃ buddhasamaṃ & natthi tasmā sotthī bhavantu te\\
  Yaṅkiñci ratanaṃ loke & vijjati vividhaṃ puthu\\
  Ratanaṃ dhammasamaṃ & natthi tasmā sotthī bhavantu te\\
  Yaṅkiñci ratanaṃ loke & vijjati vividhaṃ puthu\\
  Ratanaṃ saṅghasamaṃ & natthi tasmā sotthī bhavantu te\\
\end{twochants}

\subsection{Sakkatvā}

\firstline{Sakkatvā buddha-ratanaṃ osadhaṃ uttamaṃ varaṃ}

\begin{twochants}
  Sakkatvā buddharatanaṃ & osadhaṃ uttamaṃ varaṃ\\
  Hitaṃ devamanussānaṃ & buddhatejena sotthinā\\
  Nassantupaddavā sabbe & dukkhā vūpasamentu te\\
  Sakkatvā dhammaratanaṃ & osadhaṃ uttamaṃ varaṃ\\
  Pariḷāhūpasamanaṃ & dhammatejena sotthinā\\
  Nassantupaddavā sabbe & bhayā vūpasamentu te\\
  Sakkatvā saṅgharatanaṃ & osadhaṃ uttamaṃ varaṃ\\
  Āhuneyyaṃ pāhuneyyaṃ & saṅghatejena sotthinā\\
  Nassantupaddavā sabbe & rogā vūpasamentu te\\
\end{twochants}

\subsection{Sabbītiyo vivajjantu}

\firstline{Sabbītiyo vivajjantu sabbarogo vinassatu}

\begin{twochants}
  Sabbītiyo vivajjantu & sabbarogo vinassatu\\
  Mā te bhavatvantarāyo & sukhī dīghāyuko bhava\\
  Abhivādanasīlissa & niccaṃ vuḍḍhāpacāyino\\
  Cattāro dhammā vaḍḍhanti & āyu vaṇṇo sukhaṃ balaṃ\\
\end{twochants}

%\suttaRef{MJG}

\section{Verses on Mountains}

% TODO Pali title
% Pabbatopama-gāthā

\firstline{Yathā pi selā vipulā nabhaṃ āhacca pabbatā}

\begin{twochants}
Yathā pi selā vipulā & nabhaṃ āhacca pabbatā\\
Samantā anupariyeyyuṃ & nippothentā catuddisā\\
Evaṃ jarā ca maccu ca & adhivattanti pāṇino\\
Khattiye brāhmaṇe vesse & sudde caṇḍāla-pukkuse\\
Na kiñci parivajjeti & sabbam-evābhimaddati\\
Na tattha hatthīnaṃ bhūmi & na rathānaṃ na pattiyā\\
Na cāpi manta-yuddhena & sakkā jetuṃ dhanena vā\\
Tasmā hi paṇḍito poso & sampassaṃ attham-attano\\
Buddhe dhamme ca saṅghe ca & dhīro saddhaṃ nivesaye\\
Yo dhamma-cārī kāyena & vācāya uda cetasā\\
Idh'eva naṃ pasaṃsanti & pecca sagge pamodati
\end{twochants}

\suttaRef{S.I.102}

\section{Verses on the Burden}

% TODO Pali title
% Bhāra-sutta-gāthā

\begin{leader}
  [Handa mayaṃ bhāra-sutta-gāthāyo bhaṇāmase]
\end{leader}

\firstline{Bhārā have pañcakkhandhā}

\begin{twochants}
Bhārā have pañcakkhandhā & bhāra-hāro ca puggalo \\
Bhār'ādānaṃ dukkhaṃ loke & bhāra-nikkhepanaṃ sukhaṃ \\
Nikkhipitvā garuṃ bhāraṃ & aññaṃ bhāraṃ anādiya \\
Samūlaṃ taṇhaṃ abbuyha & nicchāto parinibbuto \\
\end{twochants}

\suttaRef{S.III.26}

\section{True and False Refuges}

% TODO Pali title
% Khemākhema-saraṇa-gamana-paridīpikā-gāthā

\firstline{Bahuṃ ve saraṇaṃ yanti pabbatāni vanāni ca}

\begin{twochants}
Bahuṃ ve saraṇaṃ yanti & pabbatāni vanāni ca\\
Ārāma-rukkha-cetyāni & manussā bhaya-tajjitā\\
N'etaṃ kho saraṇaṃ khemaṃ & n'etaṃ saraṇam-uttamaṃ\\
N'etaṃ saraṇam-āgamma & sabba-dukkhā pamuccati\\
Yo ca buddhañ-ca dhammañ-ca & saṅghañ-ca saraṇaṃ gato\\
Cattāri ariya-saccāni & sammappaññāya passati\\
Dukkhaṃ dukkha-samuppādaṃ & dukkhassa ca atikkamaṃ\\
Ariyañ-c'aṭṭh'aṅgikaṃ maggaṃ & dukkhūpasama-gāminaṃ\\
Etaṃ kho saraṇaṃ khemaṃ & etaṃ saraṇam-uttamaṃ\\
Etaṃ saraṇam-āgamma & sabba-dukkhā pamuccatī'ti.
\end{twochants}

\suttaRef{Dhp 188-192}

\section{Verses on a Shining Night of Prosperity}

% TODO Pali title
% Bhadd'eka-ratta-gāthā

\begin{leader}
  [Handa mayaṃ bhadd'eka-ratta-gāthāyo bhaṇāmase]
\end{leader}

\firstline{Atītaṃ nānvāgameyya nappaṭikaṅkhe anāgataṃ}

\begin{twochants}
  Atītaṃ nānvāgameyya & nappaṭikaṅkhe anāgataṃ \\
  Yad'atītaṃ pahīnan-taṃ & appattañca anāgataṃ \\
  Paccuppannañca yo dhammaṃ & tattha tattha vipassati \\
  Asaṃhiraṃ asaṅkuppaṃ & taṃ viddhām-anubrūhaye \\
  Ajj'eva kiccam-ātappaṃ & ko jaññā maraṇaṃ suve \\
  Na hi no saṅgaran-tena & mahā-senena maccunā \\
\end{twochants}

\begin{twochants}
  Evaṃ vihārim-ātāpiṃ & aho-rattam-atanditaṃ \\
  Taṃ ve bhadd'eka-ratto'ti & santo ācikkhate muni \\
\end{twochants}

\suttaRef{M.III.187}

\section{Verses on the Three Characteristics}

% TODO Pali title
% Ti-lakkhaṇ'ādi-gāthā

\begin{leader}
  [Handa mayaṃ ti-lakkhaṇ'ādi-gāthāyo bhaṇāmase]
\end{leader}

\firstline{Sabbe saṅkhārā aniccā'ti yadā paññāya passati}

\begin{twochants}
  Sabbe saṅkhārā aniccā'ti & yadā paññāya passati \\
  Atha nibbindati dukkhe & esa maggo visuddhiyā \\
  Sabbe saṅkhārā dukkhā'ti & yadā paññāya passati \\
  Atha nibbindati dukkhe & esa maggo visuddhiyā \\
  Sabbe dhammā anattā'ti & yadā paññāya passati \\
  Atha nibbindati dukkhe & esa maggo visuddhiyā \\
\end{twochants}

\suttaRef{Dhp 277-279}

\begin{twochants}
  Appakā te manussesu & ye janā pāra-gāmino \\
  Athāyaṃ itarā pajā & tīram-evānudhāvati \\
  Ye ca kho sammad-akkhāte & dhamme dhammānuvattino \\
  Te janā pāram-essanti & maccu-dheyyaṃ suduttaraṃ \\
  Kaṇhaṃ dhammaṃ vippahāya & sukkaṃ bhāvetha paṇḍito \\
  Okā anokam-āgamma & viveke yattha dūramaṃ \\
  Tatrābhiratim-iccheyya & hitvā kāme akiñcano \\
  Pariyodapeyya attānaṃ & citta-klesehi paṇḍito\\
\end{twochants}

\begin{twochants}
  Yesaṃ sambodhiy-aṅgesu & sammā cittaṃ subhāvitaṃ\\
  Ādāna-paṭinissagge & anupādāya ye ratā\\
  Khīṇ'āsavā jutimanto & te loke parinibbutā'ti
\end{twochants}

\suttaRef{Dhp 85-89}

\section{Verses on Respect for the Dhamma}

% TODO Pali title
% Dhamma-gārav'ādi-gāthā

\begin{leader}
  [Handa mayaṃ dhamma-gārav'ādi-gāthāyo bhaṇāmase]
\end{leader}

\firstline{Ye ca atītā sambuddhā ye ca buddhā anāgatā}

\begin{twochants}
  Ye ca atītā sambuddhā & ye ca buddhā anāgatā \\
  Yo c'etarahi sambuddho & bahunnaṃ soka-nāsano \\
  Sabbe saddhamma-garuno & vihariṃsu viharanti ca \\
  Atho pi viharissanti & esā buddhāna dhammatā \\
  Tasmā hi atta-kāmena & mahattam-abhikaṅkhatā \\
  Saddhammo garu-kātabbo & saraṃ buddhāna sāsanaṃ \\
\end{twochants}

\suttaRef{S.I.140}

\begin{paritta}
Na hi dhammo adhammo ca\\
Ubho sama-vipākino \\
Adhammo nirayaṃ neti\\
Dhammo pāpeti suggatiṃ

Dhammo have rakkhati dhamma-cāriṃ\\
Dhammo suciṇṇo sukham-āvahāti\\
Esānisaṃso dhamme suciṇṇe\\{}
[Na duggatiṃ gacchati dhamma-cārī.] \suttaRef{Thag 303-304}
\end{paritta}

\instr{(The last line is sometimes omitted when chanting.)}

\section{Verses on the Buddha's First Exclamation}

% TODO Pali title
% Paṭhama-buddha-bhāsita-gāthā

\begin{leader}
  [Handa mayaṃ paṭhama-buddha-bhāsita-gāthāyo bhaṇāmase]
\end{leader}

\firstline{Aneka-jāti-saṃsāraṃ sandhāvissaṃ anibbisaṃ}

\begin{twochants}
  Aneka-jāti-saṃsāraṃ & sandhāvissaṃ anibbisaṃ \\
  Gaha-kāraṃ gavesanto & dukkhā jāti punappunaṃ \\
  Gaha-kāraka diṭṭho'si & puna gehaṃ na kāhasi \\
  Sabbā te phāsukā bhaggā & gaha-kūṭaṃ visaṅkhataṃ \\
  Visaṅkhāra-gataṃ cittaṃ & taṇhānaṃ khayam-ajjhagā \\
\end{twochants}

\suttaRef{Dhp 153-154}

\section{Arising From a Cause}

% TODO Pali title
% Ye dhammā hetuppabhavā

\firstline{Ye dhammā hetuppabhavā}

\begin{paritta}
  Ye dhammā hetuppabhavā\\
  Tesaṃ hetuṃ tathāgato āha\\
  Tesañca yo nirodho\\
  Evaṃ-vādī mahāsamaṇo'ti
\end{paritta}

\begin{english}
  Whatever phenomena arise from a cause,\\
  The Tathāgata has explained their cause,\\
  And also their cessation.\\
  That is the teaching of the Great Ascetic.
\end{english}

\suttaRef{Mv.1.23.5}

\section{Nakkhattayakkha}

\instr{The paritta chanting may be closed with the following:}

\firstline{Nakkhatta-yakkha-bhūtānaṃ}

\begin{twochants}
  Nakkhatta-yakkha-bhūtānaṃ & pāpa-ggaha-nivāraṇā\\
  Parittassānubhāvena & hantvā tesaṃ upaddave\\
\end{twochants}

\instr{(Three times)}

% Gavesako: Is this actually done anywhere in our tradition, or is it Dhammayut style?

