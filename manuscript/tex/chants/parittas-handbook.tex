\chapter{Paritta Chants}

\section{Thai Tradition}

{\fontsize{9}{13}\selectfont

Paritta chanting ceremonies in Thailand vary regionally but may be outlined as:

\begin{packeditemize}
  \item a layperson chants the invitation for paritta chanting
  \item the third bhikkhu or nun in seniority chants the invitation to the devas
  \item the introductory chants are chanted
  \item the core sequence of paritta chants follow
  \item the closing chants end the ceremony.
\end{packeditemize}

The third introductory chant in the Mahānikāya sect is commonly \emph{Sambuddhe}.
In Dhammayut circles and frequently in the forest tradition, the third chant is
\emph{Yo cakkhumā} instead.

There is a shorter and longer traditional core sequence. The \emph{jet tamnaan}
(\thai{เจ็ดตํานาน}) contains D1-D7 as below, the \emph{sipsong tamnaan}
(\thai{สิบสองตํานาน}) contains S1-S12. Chants that are not numbered `D' or `S' can
be included or not, as wished, but should be recited in the order listed here.

}

\clearpage

\enlargethispage{\baselineskip}

{\centering
\fontsize{9}{11}\selectfont
\setArrayStretch{1.1}

\begin{tabular}{@{}l l r r@{}}
  & first line & & page \\
  \hline
  i1  & Namo tassa & & \pageref{namo-tassa} \\
  i2  & Buddhaṁ saraṇaṁ gacchāmi & & \pageref{buddham-saranam} \\
  i3/a  & Sambuddhe aṭṭhavīsañca & & \pageref{sambuddhe} \\
  i3/b  & Yo cakkhumā & & \pageref{yo-cakkhuma} \\
  i4  & Namo arahato & & \pageref{namo-arahato} \\
      & & & \\
  D1 & Asevanā ca bālānaṁ & S1 & \pageref{asevana} \\
  D2 & Yaṅkiñci vittaṁ & S2 & \pageref{yankinci-vittam} \\
  D3 & Karaṇīyam-attha-kusalena & S3 & \pageref{karaniyam-attha} \\
  D4 & Virūpakkhehi me mettaṁ & S4 & \pageref{virupakkhehi} \\
  & Vadhissamenanti parāmasanto & & \pageref{vadhissamenanti} \\
  D5 & Udet'ayañ-cakkhumā eka-rājā & S5 & \pageref{udetayan-cakkhuma} \\
  & Atthi loke sīla-guṇo & S6 & \pageref{atthi-loke} \\
  D6 & Iti pi so bhagavā & S7 & \pageref{iti-pi-so} \\
  D7 & Vipassissa nam'atthu & S8 & \pageref{vipassissa} \\
  & Natthi me saraṇaṁ aññaṁ & & \pageref{natthi-me} \\
  & Yaṅkiñci ratanaṁ loke & & \pageref{yankinci-ratanam} \\
  & Sakkatvā buddharatanaṁ & & \pageref{sakkatva} \\
  & Yato'haṁ bhagini & S9 & \pageref{yato-ham-bhagini} \\
  & Bojjh'aṅgo sati-saṅkhāto & S10 & \pageref{bojjhango} \\
  & Yan-dunnimittaṁ & S11 & \pageref{yan-dunnimittam} \\
      & & & \\
  & Dukkhappattā ca niddukkhā & & \pageref{dukkhappatta} \\
  & Bāhuṁ sahassam-abhinimmita & & \pageref{bahum} \\
  & Mahā-kāruṇiko nātho & S12 & \pageref{maha-karuniko} \\
  & Te attha-laddhā sukhitā & & \pageref{te-attha-laddha} \\
  & Bhavatu sabba-maṅgalaṁ & & \pageref{bhavatu} \\
\end{tabular}

\restoreArrayStretch
}

\clearpage

\subsection*{Notes for Particular Chants}

\textbf{Asevanā ca bālānaṁ:} The candles on the shrine during a house invitation
are lit by the senior bhikkhu or nun at \emph{Asevanā}.

\textbf{Yaṅkiñci vittaṁ:} The candles are put out at \emph{Nibbanti
  dhīrā yathā'yam padīpo}.

\textbf{Atthi loke sīla-guṇo:} On the occasion of blessing a new house, this
chant should be included, as it is traditionally considered protection against
fire.

\textbf{Yato'haṁ bhagini:} This chant is to be used for expectant mothers since
the time of the Buddha for the blessing and protection of the mother and child.
It is also a good occasion to chant it when receiving alms from a newly married
couple. Sangha members are encouraged to practise it.

\textbf{Dukkhappattā ca niddukkhā:} This is usually chanted as second to last
before \emph{Bhavatu sabba-maṅgalaṁ}. It is considered necessary to include it
whenever the devas have been invited at the beginning of the paritta chanting
as this chant contains a line inviting them to leave again.

\clearpage

\textbf{Bāhuṁ sahassam-abhinimmita:} This is is a popular later addition to the
present day standard chants. It is not listed in the \emph{jet tamnaan} or
\emph{sipsong tamnaan} sets. Yet these days it is frequently added just before
\emph{Mahā-kāruṇiko nātho}. On some occasions (e.g. public birthdays, jubilees,
inauguration ceremonies, etc.), it is an alternative, instead of chanting
\emph{jet tamnaan} or \emph{sipsong tamnaan}, to do a minimum sequence called
\emph{suat phorn phra} which contains only:

(1)~\emph{Namo Tassa},\\
(2)~\emph{Iti pi so bhagavā},\\
(3)~\emph{Bāhuṁ},\\
(4)~\emph{Mahā-kāruṇiko nātho}, and\\
(5)~\emph{Bhavatu sabba-maṅgalaṁ}.

In this minimal chanting sequence usually one does not invite the devas.

\textbf{Te attha-laddhā sukhitā:} This is sometimes inserted before closing with
\emph{Bhavatu sabba-maṅgalaṁ}, as a special well-wishing when the occasion has
to do with Buddhism in general (e.g. inauguration of a new abbot, or at the end
of an \emph{upasampadā}).

\clearpage

\section{Invitations}

\subsection{Invitation for Paritta Chanting}
\label{paritta-invitation-for-chanting}

\firstline{Vipatti-paṭibāhāya sabba-sampatti-siddhiyā}

\vspace*{5pt}

\begin{paritta}

\instr{(After bowing three times, with hands joined in añjali,\\
  recite the following)\par}

Vipatti-paṭibāhāya sabba-sampatti-siddhiyā\\
Sabbadukkha-vināsāya\\
Parittaṁ brūtha maṅgalaṁ

Vipatti-paṭibāhāya sabba-sampatti-siddhiyā\\
Sabbabhaya-vināsāya\\
Parittaṁ brūtha maṅgalaṁ

Vipatti-paṭibāhāya sabba-sampatti-siddhiyā\\
Sabbaroga-vināsāya\\
Parittaṁ brūtha maṅgalaṁ

\instr{(Bow three times)}
\end{paritta}

\begin{english}
  For warding off misfortune, for the arising of good fortune,\\
  For the dispelling of all dukkha,\\
  May you chant a blessing and protection.

  For warding off misfortune, for the arising of good fortune,\\
  For the dispelling of all fear,\\
  May you chant a blessing and protection.

  For warding off misfortune, for the arising of good fortune,\\
  For the dispelling of all sickness,\\
  May you chant a blessing and protection.
\end{english}

\subsection{Invitation to the Devas}
\label{paritta-devas}

\firstline{Pharitvāna mettaṁ samettā bhadantā}
\firstline{Samantā cakka-vāḷesu}
\firstline{Sarajjaṁ sasenaṁ sabandhuṁ nar'indaṁ}

In Thai custom, the third monk in seniority invites the devas, holding his
hands in \emph{añjali}, and lifting up the ceremonial string.

The string is wound up at the beginning of the last chant, \emph{Mahā-kāruṇiko
  nātho} or \emph{Bhavatu sabba-maṅgalaṁ}, which should be kept in mind by the
last bhikkhu or \emph{sāmaṇera}.

Before royal ceremonies, the invitation starts with A.

Before the shorter \emph{jet tamnaan} set of parittas, B is used and C is
omitted. Before the longer \emph{sipsong tamnaan} set of parittas, B is
omitted and C is used.

The verses at D are always chanted.

When chanting outside the monastery, the invitation is concluded with E. When
chanting at the monastery, the invitation is concluded with either E or F.

\clearpage

\enlargethispage{\baselineskip}

\begin{paritta}

\instr{(With hands joined in añjali, recite the following)}

\sidepar{A.}%
Sarajjaṁ sasenaṁ sabandhuṁ nar'indaṁ\\
Paritt'ānubhāvo sadā rakkhatū'ti

\sidepar{B.}%
Pharitvāna mettaṁ samettā bhadantā\\
Avikkhitta-cittā parittaṁ bhaṇantu

\sidepar{C.}%
Samantā cakka-vāḷesu\\
Atr'āgacchantu devatā\\
Saddhammaṁ muni-rājassa\\
Suṇantu sagga-mokkha-daṁ

\sidepar{D.}%
Sagge kāme ca rūpe\\
Giri-sikhara-taṭe c'antalikkhe vimāne\\
Dīpe raṭṭhe ca gāme\\
Taru-vana-gahane geha-vatthumhi khette\\
Bhummā c'āyantu devā\\
Jala-thala-visame yakkha-gandhabba-nāgā\\
Tiṭṭhantā santike yaṁ\\
Muni-vara-vacanaṁ sādhavo me suṇantu

\sidepar{E.}%
Dhammassavana-kālo ayam-bhadantā (×3)

\instr{Or, end with:}

\sidepar{F.}%
Buddha-dassana-kālo ayam-bhadantā\\
Dhammassavana-kālo ayam-bhadantā\\
Saṅgha-payirūpāsana-kālo ayam-bhadantā
\end{paritta}

\clearpage

\section{Introductory Chants}

\subsection{Pubba-bhāga-nama-kāra-pāṭha}
\label{namo-tassa}

Namo tassa bhagavato arahato sammā-sambuddhassa\\
Namo tassa bhagavato arahato sammā-sambuddhassa\\
Namo tassa bhagavato arahato sammā-sambuddhassa

\subsection{Saraṇa-gamana-pāṭha}
\label{buddham-saranam}

\begin{paritta}
Buddhaṁ saraṇaṁ gacchāmi\\
Dhammaṁ saraṇaṁ gacchāmi\\
Saṅghaṁ saraṇaṁ gacchāmi

Dutiyam pi buddhaṁ saraṇaṁ gacchāmi\\
Dutiyam pi dhammaṁ saraṇaṁ gacchāmi\\
Dutiyam pi saṅghaṁ saraṇaṁ gacchāmi

Tatiyam pi buddhaṁ saraṇaṁ gacchāmi\\
Tatiyam pi dhammaṁ saraṇaṁ gacchāmi\\
Tatiyam pi saṅghaṁ saraṇaṁ gacchāmi
\end{paritta}

\subsection{Sambuddhe}
\label{sambuddhe}

\enlargethispage{\baselineskip}

\firstline{Sambuddhe aṭṭhavīsañca}

\begin{paritta}
Sambuddhe aṭṭhavīsañca\\
Dvādasañca sahassake\\
Pañca-sata-sahassāni\\
Namāmi sirasā ahaṁ

Tesaṁ dhammañca saṅghañca\\
Ādarena namāmihaṁ\\
Namakārānubhāvena\\
Hantvā sabbe upaddave\\
Anekā antarāyāpi\\
Vinassantu asesato

Sambuddhe pañca-paññāsañca\\
Catuvīsati sahassake\\
Dasa-sata-sahassāni\\
Namāmi sirasā ahaṁ

Tesaṁ dhammañca saṅghañca\\
Ādarena namāmihaṁ\\
Namakārānubhāvena\\
Hantvā sabbe upaddave\\
Anekā antarāyāpi\\
Vinassantu asesato

Sambuddhe navuttarasate\\
Aṭṭhacattāḷīsa sahassake\\
Vīsati-sata-sahassāni\\
Namāmi sirasā ahaṁ

Tesaṁ dhammañca saṅghañca\\
Ādarena namāmihaṁ\\
Namakārānubhāvena\\
Hantvā sabbe upaddave\\
Anekā antarāyāpi\\
Vinassantu asesato
\end{paritta}

\enlargethispage{\baselineskip}

\subsection{Nama-kāra-siddhi-gāthā}
\label{yo-cakkhuma}

\firstline{Yo cakkhumā moha-malāpakaṭṭho}

\begin{paritta}
Yo cakkhumā moha-malāpakaṭṭho\\
Sāmaṁ va buddho sugato vimutto\\
Mārassa pāsā vinimocayanto\\
Pāpesi khemaṁ janataṁ vineyyaṁ\\
Buddhaṁ varan-taṁ sirasā namāmi\\
Lokassa nāthañ-ca vināyakañ-ca\\
Tan-tejasā te jaya-siddhi hotu\\
Sabb'antarāyā ca vināsamentu

Dhammo dhajo yo viya tassa satthu\\
Dassesi lokassa visuddhi-maggaṁ\\
Niyyāniko dhamma-dharassa dhārī\\
Sāt'āvaho santi-karo suciṇṇo\\
Dhammaṁ varan-taṁ sirasā namāmi\\
Mohappadālaṁ upasanta-dāhaṁ\\
Tan-tejasā te jaya-siddhi hotu\\
Sabb'antarāyā ca vināsamentu

Saddhamma-senā sugatānugo yo\\
Lokassa pāpūpakilesa-jetā\\
Santo sayaṁ santi-niyojako ca\\
Svākkhāta-dhammaṁ viditaṁ karoti\\
Saṅghaṁ varan-taṁ sirasā namāmi\\
Buddhānubuddhaṁ sama-sīla-diṭṭhiṁ\\
Tan-tejasā te jaya-siddhi hotu\\
Sabb'antarāyā ca vināsamentu
\end{paritta}

\subsection{Namo-kāra-aṭṭhaka}
\label{namo-arahato}

\firstline{Namo arahato sammā}

\begin{paritta}
  Namo arahato sammā\\
  Sambuddhassa mahesino\\
  Namo uttama-dhammassa\\
  Svākkhātass'eva ten'idha\\
  Namo mahā-saṅghassāpi\\
  Visuddha-sīla-diṭṭhino\\
  Namo omāty-āraddhassa\\
  Ratanattayassa sādhukaṁ\\
  Namo omakātītassa\\
  Tassa vatthuttayassa-pi\\
  Namo-kārappabhāvena\\
  Vigacchantu upaddavā\\
  Namo-kārānubhāvena\\
  Suvatthi hotu sabbadā\\
  Namo-kārassa tejena\\
  Vidhimhi homi tejavā
\end{paritta}

\section{Core Sequence}

\subsection{Maṅgala-sutta}
\label{asevana}

\firstline{Asevanā ca bālānaṁ}

\enlargethispage{\baselineskip}

\begin{paritta}
Asevanā ca bālānaṁ\\
Paṇḍitānañ-ca sevanā\\
Pūjā ca pūjanīyānaṁ\\
Etam maṅgalam-uttamaṁ

Paṭirūpa-desa-vāso ca\\
Pubbe ca kata-puññatā\\
Atta-sammā-paṇidhi ca\\
Etam maṅgalam-uttamaṁ

Bāhu-saccañ-ca sippañ-ca,\\
Vinayo ca susikkhito\\
Subhāsitā ca yā vācā\\
Etam maṅgalam-uttamaṁ

Mātā-pitu-upaṭṭhānaṁ\\
Putta-dārassa saṅgaho\\
Anākulā ca kammantā\\
Etam maṅgalam-uttamaṁ

Dānañ-ca dhamma-cariyā ca\\
Ñātakānañ-ca saṅgaho\\
Anavajjāni kammāni\\
Etam maṅgalam-uttamaṁ

Āratī viratī pāpā\\
Majja-pānā ca saññamo\\
Appamādo ca dhammesu\\
Etam maṅgalam-uttamaṁ

Gāravo ca nivāto ca\\
Santuṭṭhī ca kataññutā\\
Kālena dhammassavanaṁ\\
Etam maṅgalam-uttamaṁ

Khantī ca sovacassatā\\
Samaṇānañ-ca dassanaṁ\\
Kālena dhamma-sākacchā\\
Etam maṅgalam-uttamaṁ

Tapo ca brahma-cariyañ-ca\\
Ariya-saccāna-dassanaṁ\\
Nibbāna-sacchikiriyā ca\\
Etam maṅgalam-uttamaṁ

Phuṭṭhassa loka-dhammehi\\
Cittaṁ yassa na kampati\\
Asokaṁ virajaṁ khemaṁ\\
Etam maṅgalam-uttamaṁ

Etādisāni katvāna\\
Sabbattham-aparājitā\\
Sabbattha sotthiṁ gacchanti\\
Tan-tesaṁ maṅgalam-uttaman'ti \suttaRef{Snp 2.4}
\end{paritta}

\subsubsection{The Thirty-Eight Highest Blessings}

\firstline{Avoiding those of foolish ways}

Avoiding those of foolish ways,\\
Associating with the wise,\\
And honouring those worthy of honour.\\
These are the highest blessings.

Living in places of suitable kinds,\\
With the fruits of past good deeds\\
And guided by the rightful way.\\
These are the highest blessings.

Accomplished in learning and craftsman's skills,\\
With discipline, highly trained,\\
And speech that is true and pleasant to hear.\\
These are the highest blessings.

Providing for mother and father's support\\
And cherishing family,\\
And ways of work that harm no being,\\
These are the highest blessings.

Generosity and a righteous life,\\
Offering help to relatives and kin,\\
And acting in ways that leave no blame.\\
These are the highest blessings.

Steadfast in restraint, and shunning evil ways,\\
Avoiding intoxicants that dull the mind,\\
And heedfulness in all things that arise.\\
These are the highest blessings.

Respectfulness and being of humble ways,\\
Contentment and gratitude,\\
And hearing the Dhamma frequently taught.\\
These are the highest blessings.

Patience and willingness to accept one's faults,\\
Seeing venerated seekers of the truth,\\
And sharing often the words of Dhamma.\\
These are the highest blessings.

Ardent, committed to the Holy Life,\\
Seeing for oneself the Noble Truths\\
And the realization of Nibbāna.\\
These are the highest blessings.

Although in contact with the world,\\
Unshaken the mind remains\\
Beyond all sorrow, spotless, secure.\\
These are the highest blessings.

They who live by following this path\\
Know victory wherever they go,\\
And every place for them is safe.\\
These are the highest blessings. \suttaRef{Snp 2.4}

\subsection{Ratana-sutta}

\firstline{Yānīdha bhūtāni samāgatāni}

\instr{(In certain monasteries only the numbered verses are chanted.)}

\bigskip

\begin{paritta}

Yānīdha bhūtāni samāgatāni\\
Bhummāni vā yāni va antalikkhe\\
Sabb'eva bhūtā sumanā bhavantu\\
Atho pi sakkacca suṇantu bhāsitaṁ\\
Tasmā hi bhūtā nisāmetha sabbe\\
Mettaṁ karotha mānusiyā pajāya\\
Divā ca ratto ca haranti ye baliṁ\\
Tasmā hi ne rakkhatha appamattā

\firstline{Yaṅkiñci vittaṁ idha vā huraṁ vā}

\label{yankinci-vittam}
\sidepar{1.}%
Yaṅkiñci vittaṁ idha vā huraṁ vā\\
Saggesu vā yaṁ ratanaṁ paṇītaṁ\\
Na no samaṁ atthi tathāgatena\\
Idam-pi buddhe ratanaṁ paṇītaṁ\\
Etena saccena suvatthi hotu

\sidepar{2.}%
Khayaṁ virāgaṁ amataṁ paṇītaṁ\\
Yad-ajjhagā sakya-munī samāhito\\
Na tena dhammena sam'atthi kiñci\\
Idam-pi dhamme ratanaṁ paṇītaṁ\\
Etena saccena suvatthi hotu

\sidepar{3.}%
Yam buddha-seṭṭho parivaṇṇayī suciṁ\\
Samādhim-ānantarikaññam-āhu\\
Samādhinā tena samo na vijjati\\
Idam-pi dhamme ratanaṁ paṇītaṁ\\
Etena saccena suvatthi hotu

\sidepar{4.}%
Ye puggalā aṭṭha sataṁ pasaṭṭhā\\
Cattāri etāni yugāni honti\\
Te dakkhiṇeyyā sugatassa sāvakā\\
Etesu dinnāni mahapphalāni\\
Idam-pi saṅghe ratanaṁ paṇītaṁ\\
Etena saccena suvatthi hotu

\sidepar{5.}%
Ye suppayuttā manasā daḷhena\\
Nikkāmino gotama-sāsanamhi\\
Te patti-pattā amataṁ vigayha\\
Laddhā mudhā nibbutiṁ bhuñjamānā\\
Idam-pi saṅghe ratanaṁ paṇītaṁ\\
Etena saccena suvatthi hotu

\enlargethispage{\baselineskip}

Yath'inda-khīlo paṭhaviṁ sito siyā\\
Catubbhi vātebhi asampakampiyo\\
Tathūpamaṁ sappurisaṁ vadāmi\\
Yo ariya-saccāni avecca passati\\
Idam-pi Saṅghe ratanaṁ paṇītaṁ\\
Etena saccena suvatthi hotu

Ye ariya-saccāni vibhāvayanti\\
Gambhīra-paññena sudesitāni\\
Kiñ-cāpi te honti bhusappamattā\\
Na te bhavaṁ aṭṭhamam-ādiyanti\\
Idam-pi Saṅghe ratanaṁ paṇītaṁ\\
Etena saccena suvatthi hotu

Sahā v'assa dassana-sampadāya\\
Tay'assu dhammā jahitā bhavanti\\
Sakkāya-diṭṭhi vicikicchitañ-ca\\
Sīlabbataṁ vā pi yad-atthi kiñci\\
Catūh'apāyehi ca vippamutto\\
Cha cābhiṭhānāni abhabbo kātuṁ\\
Idam-pi Saṅghe ratanaṁ paṇītaṁ\\
Etena saccena suvatthi hotu

Kiñ-cāpi so kammaṁ karoti pāpakaṁ\\
Kāyena vācā uda cetasā vā\\
Abhabbo so tassa paṭicchādāya\\
Abhabbatā diṭṭha-padassa vuttā\\
Idam-pi Saṅghe ratanaṁ paṇītaṁ\\
Etena saccena suvatthi hotu

Vanappagumbe yathā phussitagge\\
Gimhāna-māse paṭhamasmiṁ gimhe\\
Tathūpamaṁ dhamma-varaṁ adesayi\\
Nibbāna-gāmiṁ paramaṁ hitāya\\
Idam-pi Buddhe ratanaṁ paṇītaṁ\\
Etena saccena suvatthi hotu

Varo varaññū varado var'āharo\\
Anuttaro dhamma-varaṁ adesayi\\
Idam-pi Buddhe ratanaṁ paṇītaṁ\\
Etena saccena suvatthi hotu

\sidepar{6.}%
Khīṇaṁ purāṇaṁ navaṁ n'atthi sambhavaṁ\\
Viratta-citt'āyatike bhavasmiṁ\\
Te khīṇa-bījā aviruḷhi-chandā\\
Nibbanti dhīrā yathā'yam padīpo\\
Idam-pi saṅghe ratanaṁ paṇītaṁ\\
Etena saccena suvatthi hotu.

Yānīdha bhūtāni samāgatāni\\
Bhummāni vā yāni va antalikkhe\\
Tathāgataṁ deva-manussa-pūjitaṁ\\
Buddhaṁ namassāma suvatthi hotu

Yānīdha bhūtāni samāgatāni\\
Bhummāni vā yāni va antalikkhe\\
Tathāgataṁ deva-manussa-pūjitaṁ\\
Dhammaṁ namassāma suvatthi hotu

Yānīdha bhūtāni samāgatāni\\
Bhummāni vā yāni va antalikkhe\\
Tathāgataṁ deva-manussa-pūjitaṁ\\
Saṅghaṁ namassāma suvatthi hotū'ti. \suttaRef{Snp 2.1}

\end{paritta}

\subsubsection{Verses from the Discourse on Treasures}

\instr{(The translations correspond to the numbered verses above.)}

\sidepar{1.}%
Whatever wealth in this world or the next,\\
whatever exquisite treasure in the heavens,\\
is not, for us, equal to the Tathāgata.\\
This, too, is an exquisite treasure in the Buddha.\\
By this truth may there be well-being.

\sidepar{2.}%
The exquisite Deathless -- dispassion, ending --\\
discovered by the Sakyan Sage while in concentration:\\
There is nothing equal to that Dhamma.\\
This, too, is an exquisite treasure in the Dhamma.\\
By this truth may there be well-being.

\sidepar{3.}%
What the excellent Awakened One extolled as pure\\
and called the concentration of unmediated knowing:\\
No equal to that concentration can be found.\\
This, too, is an exquisite treasure in the Dhamma.\\
By this truth may there be well-being.

\sidepar{4.}%
The eight persons -- the four pairs --\\
praised by those at peace:\\
They, disciples of the One Well-Gone, deserve offerings.\\
What is given to them bears great fruit.\\
This, too, is an exquisite treasure in the Saṅgha.\\
By this truth may there be well-being.

\sidepar{5.}%
Those who, devoted, firm-minded,\\
apply themselves to Gotama's message,\\
on attaining their goal, plunge into the Deathless,\\
freely enjoying the Unbinding they've gained.\\
This, too, is an exquisite treasure in the Saṅgha.\\
By this truth may there be well-being.

\sidepar{6.}%
Ended the old, there is no new taking birth.\\
Dispassioned their minds toward further becoming,\\
they -- with no seed, no desire for growth,\\
enlightened -- go out like this flame.\\
This, too, is an exquisite treasure in the Saṅgha.\\
By this truth may there be well-being.

\clearpage

\subsection{Karaṇīya-metta-sutta}
\label{karaniyam-attha}

\firstline{Karaṇīyam-attha-kusalena}

\enlargethispage{\baselineskip}

\begin{paritta}

Karaṇīyam-attha-kusalena\\
Yan-taṁ santaṁ padaṁ abhisamecca\\
Sakko ujū ca suhujū ca\\
Suvaco c'assa mudu anatimānī

Santussako ca subharo ca\\
Appakicco ca sallahuka-vutti\\
Sant'indriyo ca nipako ca\\
Appagabbho kulesu ananugiddho

Na ca khuddaṁ samācare kiñci\\
Yena viññū pare upavadeyyuṁ\\
Sukhino vā khemino hontu\\
Sabbe sattā bhavantu sukhit'attā

Ye keci pāṇa-bhūt'atthi\\
Tasā vā thāvarā vā anavasesā\\
Dīghā vā ye mahantā vā\\
Majjhimā rassakā aṇuka-thūlā

Diṭṭhā vā ye ca adiṭṭhā\\
Ye ca dūre vasanti avidūre\\
Bhūtā vā sambhavesī vā\\
Sabbe sattā bhavantu sukhit'attā

Na paro paraṁ nikubbetha\\
Nātimaññetha katthaci naṁ kiñci\\
Byārosanā paṭighasaññā\\
Nāññam-aññassa dukkham-iccheyya

Mātā yathā niyaṁ puttaṁ\\
Āyusā eka-puttam-anurakkhe\\
Evam'pi sabba-bhūtesu\\
Mānasam-bhāvaye aparimāṇaṁ

\subsubsection{Mettañ-ca sabba-lokasmiṁ}

\instr{(A shorter form is sometimes started here)}

\firstline{Mettañ-ca sabba-lokasmiṁ}

Mettañ-ca sabba-lokasmiṁ\\
Mānasam-bhāvaye aparimāṇaṁ\\
Uddhaṁ adho ca tiriyañ-ca\\
Asambādhaṁ averaṁ asapattaṁ

Tiṭṭhañ-caraṁ nisinno vā\\
Sayāno vā yāvat'assa vigata-middho\\
Etaṁ satiṁ adhiṭṭheyya\\
Brahmam-etaṁ vihāraṁ idham-āhu

Diṭṭhiñca anupagamma\\
Sīlavā dassanena sampanno\\
Kāmesu vineyya gedhaṁ\\
Na hi jātu gabbha-seyyaṁ punaretī'ti \suttaRef{Snp 1.8}

\end{paritta}

\subsubsection{The Buddha's Words on Loving-Kindness}

\begin{leader}
  [Now let us chant the Buddha's words on loving-kindness]
\end{leader}

\firstline{This is what should be done}

[This is what should be done]\\
By one who is skilled in goodness\\
And who knows the path of peace:\\
Let them be able and upright,\\
Straightforward and gentle in speech,

Humble and not conceited,\\
Contented and easily satisfied,\\
Unburdened with duties and frugal in their ways.\\
Peaceful and calm, and wise and skilful,\\
Not proud and demanding in nature.

Let them not do the slightest thing\\
That the wise would later reprove,\\
Wishing: In gladness and in safety,\\
May all beings be at ease.

Whatever living beings there may be,\\
Whether they are weak or strong, omitting none,\\
The great or the mighty,\\
\vin medium, short, or small,\\
The seen and the unseen,\\
Those living near and far away,\\
Those born and to be born,\\
May all beings be at ease.

Let none deceive another\\
Or despise any being in any state.\\
Let none through anger or ill-will\\
Wish harm upon another.

Even as a mother protects with her life\\
Her child, her only child,\\
So with a boundless heart\\
Should one cherish all living beings,\\
Radiating kindness over the entire world:

Spreading upwards to the skies\\
And downwards to the depths,\\
Outwards and unbounded,\\
Freed from hatred and ill-will.

Whether standing or walking, seated, \\
Or lying down -- free from drowsiness --\\
One should sustain this recollection.\\
This is said to be the sublime abiding.

By not holding to fixed views,\\
The pure-hearted one, having clarity of vision,\\
Being freed from all sense-desires,\\
Is not born again into this world. \suttaRef{Snp 1.8}

\subsection{Khandha-paritta}
\label{virupakkhehi}

\firstline{Virūpakkhehi me mettaṁ mettaṁ erāpathehi me}

Virūpakkhehi me mettaṁ\\\vin mettaṁ erāpathehi me\\
Chabyā-puttehi me mettaṁ\\\vin mettaṁ kaṇhā-gotamakehi ca\\
Apādakehi me mettaṁ\\\vin mettaṁ dipādakehi me\\
Catuppadehi me mettaṁ\\\vin mettaṁ bahuppadehi me\\
Mā maṁ apādako hiṁsi\\\vin mā maṁ hiṁsi dipādako\\
Mā maṁ catuppado hiṁsi\\\vin mā maṁ hiṁsi bahuppado\\
Sabbe sattā sabbe pāṇā\\\vin sabbe bhūtā ca kevalā\\
Sabbe bhadrāni passantu\\\vin mā kiñci pāpam-āgamā

\subsubsection{Appamāṇo buddho appamāṇo dhammo}

\instr{(This part is sometimes chanted on its own)}

\firstline{Appamāṇo buddho appamāṇo dhammo}

Appamāṇo buddho\\\vin appamāṇo dhammo\\\vin appamāṇo saṅgho\\
Pamāṇavantāni siriṁsapāni\\\vin ahi-vicchikā sata-padī\\
Uṇṇā-nābhī sarabhū mūsikā

Katā me rakkhā katā me parittā\\\vin paṭikkamantu bhūtāni\\
So'haṁ namo bhagavato\\\vin namo sattannaṁ\\\vin sammā-sambuddhānaṁ \suttaRef{A.II.72-73}

\subsection{Chaddanta-paritta}
\label{vadhissamenanti}

\englishTitle{The Great Elephant Protection}

\firstline{Vadhissamenanti parāmasanto}

\begin{paritta}
Vadhissamenanti parāmasanto\\
Kāsāvamaddakkhi dhajaṁ isīnaṁ\\
Dukkhena phuṭṭhassudapādi saññā\\
Arahaddhajo sabbhi avajjharūpo

Sallena viddho byathitopi santo\\
Kāsāvavatthamhi manaṁ na dussayi\\
Sace imaṁ nāgavarena saccaṁ\\
Mā maṁ vane bālamigā agañchunti
\end{paritta}

\clearpage

\subsection{Mora-paritta}
\label{udetayan-cakkhuma}

\englishTitle{The Peacock's Protection}

\firstline{Udet'ayañ-cakkhumā eka-rājā}
\firstline{Apet'ayañ-cakkhumā eka-rājā}

\vspace*{-.7\baselineskip}

\instr{(a.m.)}

\vspace*{-.2\baselineskip}

Udet'ayañ-cakkhumā eka-rājā\\
Harissa-vaṇṇo paṭhavippabhāso\\
Taṁ taṁ namassāmi harissa-vaṇṇaṁ paṭhavippabhāsaṁ\\
Tay'ajja guttā viharemu divasaṁ

Ye brāhmaṇā vedagu sabba-dhamme\\
Te me namo te ca maṁ pālayantu\\
Nam'atthu Buddhānaṁ nam'atthu bodhiyā\\
Namo vimuttānaṁ namo vimuttiyā\\
Imaṁ so parittaṁ katvā\\
Moro carati esanā'ti

\vspace*{-.1\baselineskip}

\instr{(p.m.)}

\vspace*{-.2\baselineskip}

\enlargethispage{2\baselineskip}

Apet'ayañ-cakkhumā eka-rājā\\
Harissa-vaṇṇo paṭhavippabhāso\\
Taṁ taṁ namassāmi harissa-vaṇṇaṁ paṭhavippabhāsaṁ\\
Tay'ajja guttā viharemu rattiṁ

Ye brāhmaṇā vedagu sabba-dhamme\\
Te me namo te ca maṁ pālayantu\\
Nam'atthu Buddhānaṁ nam'atthu bodhiyā\\
Namo vimuttānaṁ namo vimuttiyā\\
Imaṁ so parittaṁ katvā\\
Moro vāsam-akappayī'ti \suttaRef{Ja.159}

\clearpage

\subsection{Vaṭṭaka-paritta}
\label{atthi-loke}

\englishTitle{The Quail's Protection}

\firstline{Atthi loke sīla-guṇo saccaṁ soceyy'anuddayā}

\begin{twochants}
Atthi loke sīla-guṇo & saccaṁ soceyy'anuddayā\\
Tena saccena kāhāmi & sacca-kiriyam-anuttaraṁ\\
Āvajjitvā dhamma-balaṁ & saritvā pubbake jine\\
Sacca-balam-avassāya & sacca-kiriyam-akās'ahaṁ\\
Santi pakkhā apattanā & santi pādā avañcanā\\
Mātā pitā ca nikkhantā & jāta-veda paṭikkama\\
Saha sacce kate mayhaṁ & mahā-pajjalito sikhī\\
Vajjesi soḷasa karīsāni & udakaṁ patvā yathā sikhī\\
Saccena me samo n'atthi & esā me sacca-pāramī'ti\\
\end{twochants}

\suttaRef{Cariyāpiṭaka vv.319-322}

\subsection{Buddha-dhamma-saṅgha-guṇā}
\label{iti-pi-so}

\firstline{Iti pi so bhagavā arahaṁ sammā-sambuddho}

\begin{paritta}
Iti pi so bhagavā arahaṁ sammā-sambuddho\\
Vijjā-caraṇa-sampanno sugato loka-vidū\\
Anuttaro purisa-damma-sārathi\\
Satthā devamanussānaṁ buddho bhagavā'ti

Svākkhāto bhagavatā dhammo sandiṭṭhiko\\
\vin akāliko ehi-passiko opanayiko\\
paccattaṁ veditabbo viññūhī'ti

Supaṭipanno bhagavato sāvaka-saṅgho\\
Uju-paṭipanno bhagavato sāvaka-saṅgho\\
Ñāya-paṭipanno bhagavato sāvaka-saṅgho\\
Sāmīci-paṭipanno bhagavato sāvaka-saṅgho\\
Yad-idaṁ cattāri purisa-yugāni aṭṭha purisa-puggalā\\
Esa bhagavato sāvaka-saṅgho\\
Āhuneyyo pāhuneyyo dakkhiṇeyyo añjali-karaṇīyo\\
Anuttaraṁ puññakkhettaṁ lokassā'ti
\end{paritta}

\subsection{Araññe rukkha-mūle vā}

\firstline{Araññe rukkha-mūle vā}

\begin{paritta}
Araññe rukkha-mūle vā\\
Suññāgāre va bhikkhavo\\
Anussaretha sambuddhaṁ\\
Bhayaṁ tumhāka no siyā\\
No ce buddhaṁ sareyyātha\\
Loka-jeṭṭhaṁ nar'āsabhaṁ\\
Atha dhammaṁ sareyyātha\\
Niyyānikaṁ sudesitaṁ\\
No ce dhammaṁ sareyyātha\\
Niyyānikaṁ sudesitaṁ\\
Atha saṅghaṁ sareyyātha\\
Puññakkhettaṁ anuttaraṁ\\
Evam-buddhaṁ sarantānaṁ\\
Dhammaṁ saṅghañ-ca bhikkhavo\\
Bhayaṁ vā chambhitattaṁ vā\\
Loma-haṁso na hessatī'ti. \suttaRef{S.I.219-220}
\end{paritta}

\subsection{Āṭānāṭiya-paritta (short)}
\label{vipassissa}

\englishTitle{Homage to the Seven Past Buddhas}

\firstline{Vipassissa nam'atthu cakkhumantassa sirīmato}

\begin{paritta}
Vipassissa nam'atthu\\\vin cakkhumantassa sirīmato\\
Sikhissa pi nam'atthu\\\vin sabba-bhūtānukampino\\
Vessabhussa nam'atthu\\\vin nhātakassa tapassino\\
Nam'atthu kakusandhassa\\\vin māra-senappamaddino\\
Konāgamanassa nam'atthu\\\vin brāhmaṇassa vusīmato\\
Kassapassa nam'atthu\\\vin vippamuttassa sabbadhi\\
Aṅgīrasassa nam'atthu\\\vin sakya-puttassa sirīmato\\
Yo imaṁ dhammam-adesesi\\\vin sabba-dukkhāpanūdanaṁ\\
Ye cāpi nibbutā loke\\\vin yathā-bhūtaṁ vipassisuṁ\\
Te janā apisuṇā\\\vin mahantā vīta-sāradā\\
Hitaṁ deva-manussānaṁ\\\vin yaṁ namassanti gotamaṁ\\
Vijjā-caraṇa-sampannaṁ\\\vin mahantaṁ vīta-sāradaṁ\\
Vijjā-caraṇa-sampannaṁ\\\vin buddhaṁ vandāma gotaman'ti \suttaRef{D.III.195-196}
\end{paritta}

\subsection{Sacca-kiriyā-gāthā}
\label{natthi-me}

\firstline{Natthi me saraṇaṁ aññaṁ}

Natthi me saraṇaṁ aññaṁ buddho me saraṇaṁ varaṁ\\
Etena sacca-vajjena sotthi te/me hotu sabbadā

Natthi me saraṇaṁ aññaṁ dhammo me saraṇaṁ varaṁ\\
Etena sacca-vajjena sotthi te/me hotu sabbadā

Natthi me saraṇaṁ aññaṁ saṅgho me saraṇaṁ varaṁ\\
Etena sacca-vajjena sotthi te/me hotu sabbadā

\subsection{Yaṅkiñci ratanaṁ loke}
\label{yankinci-ratanam}

\firstline{Yaṅkiñci ratanaṁ loke}

\begin{paritta}
  Yaṅkiñci ratanaṁ loke\\\vin vijjati vividhaṁ puthu\\
  Ratanaṁ buddhasamaṁ\\\vin natthi tasmā sotthī bhavantu te\\
  Yaṅkiñci ratanaṁ loke\\\vin vijjati vividhaṁ puthu\\
  Ratanaṁ dhammasamaṁ\\\vin natthi tasmā sotthī bhavantu te\\
  Yaṅkiñci ratanaṁ loke\\\vin vijjati vividhaṁ puthu\\
  Ratanaṁ saṅghasamaṁ\\\vin natthi tasmā sotthī bhavantu te\\
\end{paritta}

\subsection{Sakkatvā buddharatanaṁ}
\label{sakkatva}

\firstline{Sakkatvā buddharatanaṁ}

\begin{twochants}
  Sakkatvā buddharatanaṁ & osadhaṁ uttamaṁ varaṁ\\
  Hitaṁ devamanussānaṁ & buddhatejena sotthinā\\
  Nassantupaddavā sabbe & dukkhā vūpasamentu te\\
  Sakkatvā dhammaratanaṁ & osadhaṁ uttamaṁ varaṁ\\
  Pariḷāhūpasamanaṁ & dhammatejena sotthinā\\
\end{twochants}

\begin{twochants}
  Nassantupaddavā sabbe & bhayā vūpasamentu te\\
  Sakkatvā saṅgharatanaṁ & osadhaṁ uttamaṁ varaṁ\\
  Āhuneyyaṁ pāhuneyyaṁ & saṅghatejena sotthinā\\
  Nassantupaddavā sabbe & rogā vūpasamentu te\\
\end{twochants}

\bigskip

{\centering
  \instr{The \emph{jet tamnaan} sequence ends here\\ and continues with the closing sequence.}
\par}

\subsection{Aṅgulimāla-paritta}
\label{yato-ham-bhagini}

\firstline{Yato'haṁ bhagini ariyāya jātiyā jāto}

\begin{paritta}
Yato'haṁ bhagini ariyāya jātiyā jāto\\
Nābhijānāmi sañcicca pāṇaṁ jīvitā voropetā\\
Tena saccena sotthi te hotu sotthi gabbhassa \suttaRef{M.II.103}
\end{paritta}

\instr{(Three times)}

\enlargethispage{\baselineskip}

\subsection{Bojjhaṅga-paritta}
\label{bojjhango}

\englishTitle{The Factors of Awakening Protection}

\firstline{Bojjhaṅgo sati-saṅkhāto}

\begin{twochants}
Bojjhaṅgo sati-saṅkhāto & dhammānaṁ vicayo tathā\\
Viriyam-pīti-passaddhi & bojjhaṅgā ca tathā'pare\\
Samādh'upekkha-bojjhaṅgā & satt'ete sabba-dassinā\\
Muninā sammad-akkhātā & bhāvitā bahulīkatā\\
Saṁvattanti abhiññāya & nibbānāya ca bodhiyā\\
Etena sacca-vajjena & sotthi te hotu sabbadā\\
Ekasmiṁ samaye nātho & moggallānañ-ca kassapaṁ\\
Gilāne dukkhite disvā & bojjhaṅge satta desayi\\
Te ca taṁ abhinanditvā & rogā mucciṁsu taṅ-khaṇe\\
Etena sacca-vajjena & sotthi te hotu sabbadā\\
Ekadā dhamma-rājā pi & gelaññenābhipīḷito\\
Cundattherena tañ-ñeva & bhaṇāpetvāna sādaraṁ\\
Sammoditvā ca ābādhā & tamhā vuṭṭhāsi ṭhānaso\\
Etena sacca-vajjena & sotthi te hotu sabbadā\\
\end{twochants}

\begin{twochants}
Pahīnā te ca ābādhā & tiṇṇannam-pi mahesinaṁ\\
Magg'āhata-kilesā va & pattānuppatti-dhammataṁ\\
Etena sacca-vajjena & sotthi te hotu sabbadā\\
\end{twochants}

\suttaRef{S.V.80f}

\vspace*{-\baselineskip}

\subsection{Abhaya-paritta}
\label{yan-dunnimittam}

\englishTitle{The Danger-free Protection}

\firstline{Yan-dunnimittaṁ avamaṅgalañ-ca}

\begin{paritta}
Yan-dunnimittaṁ avamaṅgalañ-ca\\
Yo cāmanāpo sakuṇassa saddo\\
Pāpaggaho dussupinaṁ akantaṁ\\
Buddhānubhāvena vināsamentu

Yan-dunnimittaṁ avamaṅgalañ-ca\\
Yo cāmanāpo sakuṇassa saddo\\
Pāpaggaho dussupinaṁ akantaṁ\\
Dhammānubhāvena vināsamentu

Yan-dunnimittaṁ avamaṅgalañ-ca\\
Yo cāmanāpo sakuṇassa saddo\\
Pāpaggaho dussupinaṁ akantaṁ\\
Saṅghānubhāvena vināsamentu
\end{paritta}

\bigskip

{\centering
  \instr{The \emph{sipsong tamnaan} sequence ends here\\ and continues with the closing sequence.}
\par}

\clearpage

\section{Closing Sequence}

\subsection{Devatā-uyyojana-gāthā}
\label{dukkhappatta}

\englishTitle{Verses on Sending Off the Devatā}

\firstline{Dukkhappattā ca niddukkhā}
\firstline{Sabbe buddhā balappattā}

\begin{paritta}
Dukkhappattā ca niddukkhā\\\vin bhayappattā ca nibbhayā\\
Sokappattā ca nissokā\\\vin hontu sabbe pi pāṇino\\
Ettāvatā ca amhehi\\\vin sambhataṁ puñña-sampadaṁ\\
Sabbe devānumodantu\\\vin sabba-sampatti-siddhiyā\\
Dānaṁ dadantu saddhāya\\\vin sīlaṁ rakkhantu sabbadā\\
Bhāvanābhiratā hontu\\\vin gacchantu devatā-gatā\\\relax
[Sabbe buddhā] balappattā\\\vin paccekānañ-ca yaṁ balaṁ\\
Arahantānañ-ca tejena\\\vin rakkhaṁ bandhāmi sabbaso\\
\end{paritta}

\subsection{Jaya-maṅgala-aṭṭha-gāthā}
\label{bahum}

\englishTitle{Verses on the Buddha's Victories}

\firstline{Bāhuṁ sahassam-abhinimmita sāvudhan-taṁ}

\begin{paritta}
Bāhuṁ sahassam-abhinimmita sāvudhan-taṁ\\
Grīmekhalaṁ udita-ghora-sasena-māraṁ\\
Dān'ādi-dhamma-vidhinā jitavā mun'indo\\
Tan-tejasā bhavatu te jaya-maṅgalāni

Mārātirekam-abhiyujjhita-sabba-rattiṁ\\
Ghoram-pan'āḷavakam-akkhama-thaddha-yakkhaṁ\\
Khantī-sudanta-vidhinā jitavā mun'indo\\
Tan-tejasā bhavatu te jaya-maṅgalāni

Nāḷāgiriṁ gaja-varaṁ atimatta-bhūtaṁ\\
Dāv'aggi-cakkam-asanīva sudāruṇan-taṁ\\
Mett'ambu-seka-vidhinā jitavā mun'indo\\
Tan-tejasā bhavatu te jaya-maṅgalāni

\enlargethispage{\baselineskip}

Ukkhitta-khaggam-atihattha-sudāruṇan-taṁ\\
Dhāvan-ti-yojana-path'aṅguli-mālavantaṁ\\
Iddhī'bhisaṅkhata-mano jitavā mun'indo\\
Tan-tejasā bhavatu te jaya-maṅgalāni

Katvāna kaṭṭham-udaraṁ iva gabbhinīyā\\
Ciñcāya duṭṭha-vacanaṁ jana-kāya majjhe\\
Santena soma-vidhinā jitavā mun'indo\\
Tan-tejasā bhavatu te jaya-maṅgalāni

Saccaṁ vihāya-mati-saccaka-vāda-ketuṁ\\
Vādābhiropita-manaṁ ati-andha-bhūtaṁ\\
Paññā-padīpa-jalito jitavā mun'indo\\
Tan-tejasā bhavatu te jaya-maṅgalāni

Nandopananda-bhujagaṁ vibudhaṁ mah'iddhiṁ\\
Puttena thera-bhujagena damāpayanto\\
Iddhūpadesa-vidhinā jitavā mun'indo\\
Tan-tejasā bhavatu te jaya-maṅgalāni

Duggāha-diṭṭhi-bhujagena sudaṭṭha-hatthaṁ\\
Brahmaṁ visuddhi-jutim-iddhi-bakābhidhānaṁ\\
Ñāṇāgadena vidhinā jitavā mun'indo\\
Tan-tejasā bhavatu te jaya-maṅgalāni

Etā pi buddha-jaya-maṅgala-aṭṭha-gāthā\\
Yo vācano dina-dine saratem-atandī\\
Hitvān'aneka-vividhāni c'upaddavāni\\
Mokkhaṁ sukhaṁ adhigameyya naro sapañño
\end{paritta}

\subsection{Jaya-paritta}
\label{maha-karuniko}

\englishTitle{The Victory Protection}

\firstline{Mahā-kāruṇiko nātho hitāya sabba-pāṇinaṁ}

\begin{paritta}
  Mahā-kāruṇiko nātho\\
  Hitāya sabba-pāṇinaṁ\\
  Pūretvā pāramī sabbā\\
  Patto sambodhim-uttamaṁ\\
  Etena sacca-vajjena\\
  Hotu te jaya-maṅgalaṁ
\end{paritta}

\vspace*{-0.5\baselineskip}

\subsubsection{Jayanto bodhiyā mūle}

\instr{(This part is sometimes chanted on its own)}

\firstline{Jayanto bodhiyā mūle}

\bigskip

\begin{paritta}
  Jayanto bodhiyā mūle\\
  Sakyānaṁ nandi-vaḍḍhano\\
  Evaṁ tvaṁ vijayo hohi\\
  Jayassu jaya-maṅgale\\
  Aparājita-pallaṅke\\
  Sīse paṭhavi-pokkhare

\enlargethispage{\baselineskip}

  Abhiseke sabba-buddhānaṁ\\
  Aggappatto pamodati\\
  Sunakkhattaṁ sumaṅgalaṁ\\
  Supabhātaṁ suhuṭṭhitaṁ\\
  Sukhaṇo sumuhutto ca\\
  Suyiṭṭhaṁ brahma-cārisu

  Padakkhiṇaṁ kāya-kammaṁ\\
  Vācā-kammaṁ padakkhiṇaṁ\\
  Padakkhiṇaṁ mano-kammaṁ\\
  Paṇidhi te padakkhiṇā\\
  Padakkhiṇāni katvāna\\
  Labhant'atthe padakkhiṇe \suttaRef{A.I.294}
\end{paritta}

\subsection{So attha-laddho}

\firstline{So attha-laddho sukhito viruḷho buddha-sāsane}

\begin{twochants}
So attha-laddho sukhito & viruḷho buddha-sāsane\\
Arogo sukhito hohi & saha sabbehi ñātibhi (×3)\\
\end{twochants}

\subsection{Sā attha-laddhā}

\firstline{Sā attha-laddhā sukhitā viruḷhā buddha-sāsane}

\begin{twochants}
Sā attha-laddhā sukhitā & viruḷhā buddha-sāsane\\
Arogā sukhitā hohi & saha sabbehi ñātibhi (×3)\\
\end{twochants}

\subsection{Te attha-laddhā sukhitā}
\label{te-attha-laddha}

\firstline{Te attha-laddhā sukhitā viruḷhā buddha-sāsane}

\begin{twochants}
Te attha-laddhā sukhitā & viruḷhā buddha-sāsane\\
Arogā sukhitā hotha & saha sabbehi ñātibhi (×3)\\
\end{twochants}

\suttaRef{A.I.294}

\subsection{Bhavatu sabba-maṅgalaṁ}
\label{bhavatu}

\firstline{Bhavatu sabba-maṅgalaṁ}

Bhavatu sabba-maṅgalaṁ rakkhantu sabba-devatā\\
Sabba-buddhānubhāvena sadā sotthī bhavantu te

Bhavatu sabba-maṅgalaṁ rakkhantu sabba-devatā\\
Sabba-dhammānubhāvena sadā sotthī bhavantu te

Bhavatu sabba-maṅgalaṁ rakkhantu sabba-devatā\\
Sabba-saṅghānubhāvena sadā sotthī bhavantu te

\section{Mahā-kāruṇiko nātho'ti ādikā gāthā}

\firstline{Mahā-kāruṇiko nātho atthāya sabba-pāṇinaṁ}

\begin{paritta}
Mahā-kāruṇiko nātho\\
Atthāya sabba-pāṇinaṁ\\
Hitāya sabba-pāṇinaṁ\\
Sukhāya sabba-pāṇinaṁ

Pūretvā pāramī sabbā\\
Patto sambodhim-uttamaṁ\\
Etena sacca-vajjena\\
Mā hontu sabb'upaddavā
\end{paritta}

\clearpage

\section{Āṭānāṭiya-paritta (long)}

\englishTitle{The Twenty-Eight Buddhas' Protection}

\begin{leader}
\soloinstr{(Solo introduction)}

\firstline{Appasannehi nāthassa sāsane sādhusammate}

\begin{solotwochants}
  Appasannehi nāthassa & sāsane sādhusammate\\
  Amanussehi caṇḍehi & sadā kibbisakāribhi\\
  Parisānañca-tassannam & ahiṁsāya ca guttiyā\\
  Yandesesi mahāvīro & parittan-tam bhaṇāma se\\
\end{solotwochants}
\end{leader}

\firstline{Namo me sabbabuddhānaṁ}

{\centering
  \instr{(If starting with \emph{Vipassissa\ldots}, continue below\\
    without the solo introduction)}
\par}

\bigskip

\begin{twochants}
  [Namo me sabbabuddhānaṁ] & uppannānaṁ mahesinaṁ\\
  Taṇhaṅkaro mahāvīro & medhaṅkaro mahāyaso\\
  Saraṇaṅkaro lokahito & dīpaṅkaro jutindharo\\
  Koṇḍañño janapāmokkho & maṅgalo purisāsabho\\
  Sumano sumano dhīro & revato rativaḍḍhano\\
  Sobhito guṇasampanno & anomadassī januttamo\\
  Padumo lokapajjoto & nārado varasārathī\\
  Padumuttaro sattasāro & sumedho appaṭipuggalo\\
  Sujāto sabbalokaggo & piyadassī narāsabho\\
  Atthadassī kāruṇiko & dhammadassī tamonudo\\
  Siddhattho asamo loke & tisso ca vadataṁ varo\\
  Phusso ca varado buddho & vipassī ca anūpamo\\
  Sikhī sabbahito satthā & vessabhū sukhadāyako\\
\end{twochants}

\begin{twochants}
  Kakusandho satthavāho & koṇāgamano raṇañjaho\\
  Kassapo sirisampanno & gotamo sakyapuṅgavo\\
  Ete caññe ca sambuddhā & anekasatakoṭayo\\
  Sabbe buddhā asamasamā & sabbe buddhā mahiddhikā\\
  Sabbe dasabalūpetā & vesārajjehupāgatā\\
  Sabbe te paṭijānanti & āsabhaṇṭhānamuttamaṁ\\
  Sīhanādaṁ nadantete & parisāsu visāradā\\
  Brahmacakkaṁ pavattenti & loke appaṭivattiyaṁ\\
  Upetā buddhadhammehi & aṭṭhārasahi nāyakā\\
  Dvattiṁsa-lakkhaṇūpetā & sītyānubyañjanādharā\\
  Byāmappabhāya suppabhā & sabbe te munikuñjarā\\
  Buddhā sabbaññuno ete & sabbe khīṇāsavā jinā\\
  Mahappabhā mahātejā & mahāpaññā mahabbalā\\
  Mahākāruṇikā dhīrā & sabbesānaṁ sukhāvahā\\
  Dīpā nāthā patiṭṭhā & ca tāṇā leṇā ca pāṇinaṁ\\
  Gatī bandhū mahassāsā & saraṇā ca hitesino\\
  Sadevakassa lokassa & sabbe ete parāyanā\\
  Tesāhaṁ sirasā pāde & vandāmi purisuttame\\
  Vacasā manasā ceva & vandāmete tathāgate\\
  Sayane āsane ṭhāne & gamane cāpi sabbadā\\
  Sadā sukhena rakkhantu & buddhā santikarā tuvaṁ\\
  Tehi tvaṁ rakkhito santo & mutto sabbabhayena ca\\
\end{twochants}

\clearpage

\savenotes

\firstline{Tesaṁ saccena sīlena khantimettābalena ca}

\begin{twochants}
  Sabba-rogavinimutto & sabba-santāpavajjito\\
  Sabba-veramatikkanto & nibbuto ca tuvaṁ bhava\\
  Tesaṁ saccena sīlena & khantimettābalena ca\\
  Tepi tumhe%
  \footnote{If chanting for oneself, change \textit{tumhe} to \textit{amhe} here and in the lines below.}
  anurakkhantu & ārogyena sukhena ca\\
  Puratthimasmiṁ disābhāge & santi bhūtā mahiddhikā\\
  Tepi tumhe anurakkhantu & ārogyena sukhena ca\\
  Dakkhiṇasmiṁ disābhāge & santi devā mahiddhikā\\
  Tepi tumhe anurakkhantu & ārogyena sukhena ca\\
  Pacchimasmiṁ disābhāge & santi nāgā mahiddhikā\\
  Tepi tumhe anurakkhantu & ārogyena sukhena ca\\
  Uttarasmiṁ disābhāge & santi yakkhā mahiddhikā\\
  Tepi tumhe anurakkhantu & ārogyena sukhena ca\\
  Purimadisaṁ dhataraṭṭho & dakkhiṇena viruḷhako\\
  Pacchimena virūpakkho & kuvero uttaraṁ disaṁ\\
  Cattāro te mahārājā & lokapālā yasassino\\
  Tepi tumhe anurakkhantu & ārogyena sukhena ca\\
  Ākāsaṭṭhā ca bhummaṭṭhā & devā nāgā mahiddhikā\\
  Tepi tumhe anurakkhantu & ārogyena sukhena ca\\
\end{twochants}

\spewnotes

\clearpage

\subsubsection{Natthi me saraṇaṁ aññaṁ}

\vspace*{\parskip}

\firstline{Natthi me saraṇaṁ aññaṁ}

\savenotes

\begin{paritta}
  Natthi me saraṇaṁ aññaṁ\\\vin buddho me saraṇaṁ varaṁ\\
  Etena saccavajjena\\\vin hotu te%
  \footnote{If chanting for oneself, change \textit{te} to \textit{me} here and in the lines below.}
  jayamaṅgalaṁ\\
  Natthi me saraṇaṁ aññaṁ\\\vin dhammo me saraṇaṁ varaṁ\\
  Etena saccavajjena\\\vin hotu te jayamaṅgalaṁ\\
  Natthi me saraṇaṁ aññaṁ\\\vin saṅgho me saraṇaṁ varaṁ\\
  Etena saccavajjena\\\vin hotu te jayamaṅgalaṁ\\
\end{paritta}

\spewnotes

\subsubsection{Yaṅkiñci ratanaṁ loke}

\vspace*{\parskip}

\firstline{Yaṅkiñci ratanaṁ loke vijjati vividhaṁ puthu}

\begin{paritta}
  Yaṅkiñci ratanaṁ loke\\\vin vijjati vividhaṁ puthu\\
  Ratanaṁ buddhasamaṁ\\\vin natthi tasmā sotthī bhavantu te\\
  Yaṅkiñci ratanaṁ loke\\\vin vijjati vividhaṁ puthu\\
  Ratanaṁ dhammasamaṁ\\\vin natthi tasmā sotthī bhavantu te\\
  Yaṅkiñci ratanaṁ loke\\\vin vijjati vividhaṁ puthu\\
  Ratanaṁ saṅghasamaṁ\\\vin natthi tasmā sotthī bhavantu te\\
\end{paritta}

\subsubsection{Sakkatvā}

\vspace*{\parskip}

\firstline{Sakkatvā buddha-ratanaṁ osadhaṁ uttamaṁ varaṁ}

\begin{twochants}
  Sakkatvā buddharatanaṁ & osadhaṁ uttamaṁ varaṁ\\
  Hitaṁ devamanussānaṁ & buddhatejena sotthinā\\
  Nassantupaddavā sabbe & dukkhā vūpasamentu te\\
  Sakkatvā dhammaratanaṁ & osadhaṁ uttamaṁ varaṁ\\
  Pariḷāhūpasamanaṁ & dhammatejena sotthinā\\
  Nassantupaddavā sabbe & bhayā vūpasamentu te\\
  Sakkatvā saṅgharatanaṁ & osadhaṁ uttamaṁ varaṁ\\
  Āhuneyyaṁ pāhuneyyaṁ & saṅghatejena sotthinā\\
  Nassantupaddavā sabbe & rogā vūpasamentu te\\
\end{twochants}

\subsubsection{Sabbītiyo vivajjantu}

\vspace*{\parskip}

\firstline{Sabbītiyo vivajjantu sabbarogo vinassatu}

\begin{twochants}
  Sabbītiyo vivajjantu & sabbarogo vinassatu\\
  Mā te bhavatvantarāyo & sukhī dīghāyuko bhava\\
  Abhivādanasīlissa & niccaṁ vuḍḍhāpacāyino\\
  Cattāro dhammā vaḍḍhanti & āyu vaṇṇo sukhaṁ balaṁ\\
\end{twochants}

\clearpage

\section{Pabbatopama-gāthā}

\englishTitle{Verses on Mountains}

\firstline{Yathā pi selā vipulā nabhaṁ āhacca pabbatā}

\enlargethispage{\baselineskip}

\begin{paritta}
Yathā pi selā vipulā\\\vin nabhaṁ āhacca pabbatā\\
Samantā anupariyeyyuṁ\\\vin nippothentā catuddisā\\
Evaṁ jarā ca maccu ca\\\vin adhivattanti pāṇino\\
Khattiye brāhmaṇe vesse\\\vin sudde caṇḍāla-pukkuse\\
Na kiñci parivajjeti\\\vin sabbam-evābhimaddati\\
Na tattha hatthīnaṁ bhūmi\\\vin na rathānaṁ na pattiyā\\
Na cāpi manta-yuddhena\\\vin sakkā jetuṁ dhanena vā\\
Tasmā hi paṇḍito poso\\\vin sampassaṁ attham-attano\\
Buddhe dhamme ca saṅghe ca\\\vin dhīro saddhaṁ nivesaye\\
Yo dhamma-cārī kāyena\\\vin vācāya uda cetasā\\
Idh'eva naṁ pasaṁsanti\\\vin pecca sagge pamodati \suttaRef{S.I.102}
\end{paritta}

\section{Bhāra-sutta-gāthā}

\englishTitle{Verses on the Burden}

\firstline{Bhārā have pañcakkhandhā}

\begin{paritta}
Bhārā have pañcakkhandhā\\\vin bhāra-hāro ca puggalo \\
Bhār'ādānaṁ dukkhaṁ loke\\\vin bhāra-nikkhepanaṁ sukhaṁ \\
Nikkhipitvā garuṁ bhāraṁ\\\vin aññaṁ bhāraṁ anādiya \\
Samūlaṁ taṇhaṁ abbuyha\\\vin nicchāto parinibbuto \suttaRef{S.III.26}
\end{paritta}

\section{Khemākhema-saraṇa-gamana-paridīpikā-gāthā}

\englishTitle{True and False Refuges}

\firstline{Bahuṁ ve saraṇaṁ yanti pabbatāni vanāni ca}

\enlargethispage{\baselineskip}

\begin{paritta}
Bahuṁ ve saraṇaṁ yanti\\\vin pabbatāni vanāni ca\\
Ārāma-rukkha-cetyāni\\\vin manussā bhaya-tajjitā\\
N'etaṁ kho saraṇaṁ khemaṁ\\\vin n'etaṁ saraṇam-uttamaṁ\\
N'etaṁ saraṇam-āgamma\\\vin sabba-dukkhā pamuccati\\
Yo ca buddhañ-ca dhammañ-ca\\\vin saṅghañ-ca saraṇaṁ gato\\
Cattāri ariya-saccāni\\\vin sammappaññāya passati\\
Dukkhaṁ dukkha-samuppādaṁ\\\vin dukkhassa ca atikkamaṁ\\
Ariyañ-c'aṭṭh'aṅgikaṁ maggaṁ\\\vin dukkhūpasama-gāminaṁ\\
Etaṁ kho saraṇaṁ khemaṁ\\\vin etaṁ saraṇam-uttamaṁ\\
Etaṁ saraṇam-āgamma\\\vin sabba-dukkhā pamuccatī'ti. \suttaRef{Dhp 188-192}
\end{paritta}

\section{Bhadd'eka-ratta-gāthā}

\englishTitle{Verses on a Shining Night of Prosperity}

\firstline{Atītaṁ nānvāgameyya nappaṭikaṅkhe anāgataṁ}

\begin{paritta}
  Atītaṁ nānvāgameyya\\\vin nappaṭikaṅkhe anāgataṁ \\
  Yad'atītaṁ pahīnan-taṁ\\\vin appattañca anāgataṁ \\
  Paccuppannañca yo dhammaṁ\\\vin tattha tattha vipassati \\
  Asaṁhiraṁ asaṅkuppaṁ\\\vin taṁ viddhām-anubrūhaye \\
  Ajj'eva kiccam-ātappaṁ\\\vin ko jaññā maraṇaṁ suve \\
  Na hi no saṅgaran-tena\\\vin mahā-senena maccunā \\
  Evaṁ vihārim-ātāpiṁ\\\vin aho-rattam-atanditaṁ \\
  Taṁ ve bhadd'eka-ratto'ti\\\vin santo ācikkhate muni \suttaRef{M.III.187}
\end{paritta}

\section{Ti-lakkhaṇ'ādi-gāthā}

\englishTitle{Verses on the Three Characteristics}

\firstline{Sabbe saṅkhārā aniccā'ti yadā paññāya passati}

\begin{paritta}
  Sabbe saṅkhārā aniccā'ti\\\vin yadā paññāya passati \\
  Atha nibbindati dukkhe\\\vin esa maggo visuddhiyā \\
  Sabbe saṅkhārā dukkhā'ti\\\vin yadā paññāya passati \\
  Atha nibbindati dukkhe\\\vin esa maggo visuddhiyā \\
  Sabbe dhammā anattā'ti\\\vin yadā paññāya passati \\
  Atha nibbindati dukkhe\\\vin esa maggo visuddhiyā \suttaRef{Dhp 277-279}
\end{paritta}

\clearpage

\begin{paritta}
  Appakā te manussesu\\\vin ye janā pāra-gāmino \\
  Athāyaṁ itarā pajā\\\vin tīram-evānudhāvati \\
  Ye ca kho sammad-akkhāte\\\vin dhamme dhammānuvattino \\
  Te janā pāram-essanti\\\vin maccu-dheyyaṁ suduttaraṁ \\
  Kaṇhaṁ dhammaṁ vippahāya\\\vin sukkaṁ bhāvetha paṇḍito \\
  Okā anokam-āgamma\\\vin viveke yattha dūramaṁ \\
  Tatrābhiratim-iccheyya\\\vin hitvā kāme akiñcano \\
  Pariyodapeyya attānaṁ\\\vin citta-klesehi paṇḍito\\
  Yesaṁ sambodhiy-aṅgesu\\\vin sammā cittaṁ subhāvitaṁ\\
  Ādāna-paṭinissagge\\\vin anupādāya ye ratā\\
  Khīṇ'āsavā jutimanto\\\vin te loke parinibbutā'ti \suttaRef{Dhp 85-89}
\end{paritta}

\clearpage

\section{Dhamma-gārav'ādi-gāthā}

\englishTitle{Verses on Respect for the Dhamma}

\firstline{Ye ca atītā sambuddhā ye ca buddhā anāgatā}

\begin{paritta}
  Ye ca atītā sambuddhā\\
  Ye ca buddhā anāgatā\\
  Yo c'etarahi sambuddho\\
  Bahunnaṁ soka-nāsano

  Sabbe saddhamma-garuno\\
  Vihariṁsu viharanti ca\\
  Atho pi viharissanti\\
  Esā buddhāna dhammatā

  Tasmā hi atta-kāmena\\
  Mahattam-abhikaṅkhatā\\
  Saddhammo garu-kātabbo\\
  Saraṁ buddhāna sāsanaṁ \suttaRef{S.I.140}
\end{paritta}

\vspace*{\parskip}

\begin{paritta}
  Na hi dhammo adhammo ca\\
  Ubho sama-vipākino\\
  Adhammo nirayaṁ neti\\
  Dhammo pāpeti suggatiṁ

  Dhammo have rakkhati dhamma-cāriṁ\\
  Dhammo suciṇṇo sukham-āvahāti\\
  Esānisaṁso dhamme suciṇṇe \suttaRef{Thag 303-304}
\end{paritta}

% NOTE: The last line of the verse is usually omitted --
% [Na duggatiṁ gacchati dhamma-cārī.]

\section{Paṭhama-buddha-bhāsita-gāthā}

\englishTitle{Verses on the Buddha's First Exclamation}

\firstline{Aneka-jāti-saṁsāraṁ sandhāvissaṁ anibbisaṁ}

\begin{twochants}
  Aneka-jāti-saṁsāraṁ & sandhāvissaṁ anibbisaṁ \\
  Gaha-kāraṁ gavesanto & dukkhā jāti punappunaṁ \\
  Gaha-kāraka diṭṭho'si & puna gehaṁ na kāhasi \\
  Sabbā te phāsukā bhaggā & gaha-kūṭaṁ visaṅkhataṁ \\
  Visaṅkhāra-gataṁ cittaṁ & taṇhānaṁ khayam-ajjhagā \\
\end{twochants}

\suttaRef{Dhp 153-154}

\section{Pacchima-ovāda-gāthā}

\englishTitle{Verses on the Last Instructions}

\firstline{Handa dāni bhikkhave āmantayāmi vo}

\begin{paritta}
Handa dāni bhikkhave āmantayāmi vo\\
Vaya-dhammā saṅkhārā\\
Appamādena sampādethā'ti\\
Ayaṁ tathāgatassa pacchimā vācā
\end{paritta}

\begin{english}
  ‘Now, take heed, bhikkhus, I caution you thus: Dissolution is the nature of
  all conditions. Therefore strive on with diligence!’ These are the final words
  of the Tathāgata.
\end{english}

\suttaRef{D.II.156}

\clearpage

\section{Ye dhammā hetuppabhavā}

\englishTitle{Arising From a Cause}

\firstline{Ye dhammā hetuppabhavā}

\begin{paritta}
  Ye dhammā hetuppabhavā\\
  Tesaṁ hetuṁ tathāgato āha\\
  Tesañca yo nirodho\\
  Evaṁ-vādī mahāsamaṇo'ti
\end{paritta}

\begin{english}
  Whatever phenomena arise from a cause,\\
  The Tathāgata has explained their cause,\\
  And also their cessation.\\
  That is the teaching of the Great Ascetic.
\end{english}

\suttaRef{Mv.1.23.5}

\section{Nakkhattayakkha}

\firstline{Nakkhatta-yakkha-bhūtānaṁ}

\instr{The paritta chanting may be closed with the following:}

\bigskip

\begin{paritta}
  Nakkhatta-yakkha-bhūtānaṁ\\
  Pāpa-ggaha-nivāraṇā\\
  Parittassānubhāvena\\
  Hantvā tesaṁ upaddave
\end{paritta}

\instr{(Three times)}

