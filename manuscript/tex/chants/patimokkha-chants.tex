\chapter{Pāṭimokkha Chants}

\section{Verses on the Training Code}

\begin{leader}
  [Handa mayaṃ ovāda-pāṭimokkha-gāthāyo bhaṇāmase]
\end{leader}

\firstline{Sabba-pāpassa akaraṇaṃ}
\firstline{Khantī paramaṃ tapo tītikkhā}

Sabba-pāpassa akaraṇaṃ\\
Kusalassūpasampadā\\
Sacitta-pariyodapanaṃ\\
Etaṃ buddhāna sāsanaṃ\\
Khantī paramaṃ tapo tītikkhā\\
Nibbānaṃ paramaṃ vadanti buddhā\\
Na hi pabbajito parūpaghātī\\
Samaṇo hoti paraṃ viheṭhayanto\\
Anūpavādo anūpaghāto\\
Pāṭimokkhe ca saṃvaro\\
Mattaññutā ca bhattasmiṃ\\
Pantañca sayan'āsanaṃ\\
Adhicitte ca āyogo\\
Etaṃ buddhāna sāsanaṃ

\suttaRef{Dhp 183-185}

\clearpage

\section{Sacca-kiriyā-gāthā}

\begin{leader}
  [Handa mayaṃ sacca-kiriyā-gāthāyo bhaṇāmase]
\end{leader}

\firstline{Natthi me saraṇaṃ aññaṃ}

Natthi me saraṇaṃ aññaṃ buddho me saraṇaṃ varaṃ\\
Etena sacca-vajjena sotthi me hotu sabbadā

Natthi me saraṇaṃ aññaṃ dhammo me saraṇaṃ varaṃ\\
Etena sacca-vajjena sotthi me hotu sabbadā

Natthi me saraṇaṃ aññaṃ saṅgho me saraṇaṃ varaṃ\\
Etena sacca-vajjena sotthi me hotu sabbadā

% English source: Mahamakut Patimokkha, p.138

\begin{english}
  For me there is no other Refuge, the Buddha \ldots\ Dhamma \ldots\ Sangha is
  my excellent refuge. By the utterance of this Truth, may there be blessings
  for me.
\end{english}

\section[Sīl'uddesa-pāṭho]{Sīl'uddesa-pāṭho Uposath'āvasāne Sajjhāyitabbo}

\begin{leader}
  [Handa mayaṃ sīl'uddesa-pāṭho bhaṇāmase]
\end{leader}

\firstline{Bhāsitam idaṃ tena bhagavatā jānatā passatā}

Bhāsitam idaṃ tena bhagavatā jānatā passatā\\
arahatā sammā-sambuddhena,\\
Sampanna-sīlā bhikkhave viharatha\\
sampanna-pāṭimokkhā,\\
Pāṭimokkha-saṃvara-saṃvutā viharatha\\
ācāra-gocara-sampannā,\\
Aṇu-mattesu vajjesu bhaya-dassāvī\\
samādāya sikkhatha sikkhāpadesū-ti.

% English source: Mahamakut Patimokkha, p.138

\begin{english}
  This has been said by the Lord, One-who-knows, One-who-sees, the Arahant, the
  Perfect Buddha enlightened by himself: `Bhikkhus, be perfect in moral
  conduct. Be perfect in the Pāṭimokkha. Dwell restrained in accordance with the
  the Pāṭimokkha. Be perfect in conduct and resort, seeing danger even in the
  slightest faults. Train yourselves by undertaking rightly the rules of training.'
\end{english}

Tasmā-tih'amhehi sikkhitabbaṃ,\\
Sampanna-sīlā viharissāma sampannapāṭimokkhā,\\
Pāṭimokkha-saṃvara-saṃvutā viharissāma\\
ācāra-gocara-sampannā,\\
Aṇu-mattesu vajjesu bhaya-dassāvī\\
samādāya sikkhissāma sikkhāpadesū-ti,\\
Evañ hi no sikkhitabbaṃ.

\begin{english}
  Therefore we should train ourselves thus: `We will be perfect in the
  Pāṭimokkha. We will dwell restrained in accordance with the Pāṭimokkha. We
  will be perfect in conduct and resort, seeing danger even in the slightest
  faults.' Thus indeed we should train ourselves.
\end{english}

\suttaRef{cf. D.I.63; D.III.266f}


\section{The Verses of Tāyana}

\begin{leader}
  [Handa mayaṃ tāyana-gāthāyo bhaṇāmase]
\end{leader}

\firstline{Chinda sotaṃ parakkamma}

\begin{twochants}
  Chinda sotaṃ parakkamma & kāme panūda brāhmaṇa \\
  Nappahāya muni kāme & n'ekattam-upapajjati \\
  Kayirā ce kayirāthenaṃ & daḷham-enaṃ parakkame \\
  Sithilo hi paribbājo & bhiyyo ākirate rajaṃ \\
\end{twochants}

\begin{english}
  Exert yourself and cut the stream.\\
  Discard sense-pleasures, Holy Man;\\
  Not letting sensual pleasures go,\\
  A sage will not reach unity.
  Vigorously, with all one's strength,\\
  It should be done, what should be done;\\
  A lax monastic life stirs up\\
  The dust of passions all the more.
\end{english}

\begin{twochants}
  Akataṃ dukkaṭaṃ seyyo & pacchā tappati dukkaṭaṃ \\
  Katañca sukataṃ seyyo & yaṃ katvā nānutappati \\
  Kuso yathā duggahito & hattham-evānukantati \\
  Sāmaññaṃ dupparāmaṭṭhaṃ & nirayāyūpakaḍḍhati \\
  Yaṃ kiñci sithilaṃ kammaṃ & saṅkiliṭṭhañca yaṃ vataṃ \\
  Saṅkassaraṃ brahma-cariyaṃ & na taṃ hoti mahapphalan'ti \\
\end{twochants}

\begin{english}
  Better is not to do bad deeds\\
  That afterwards would bring remorse;\\
  It's rather good deeds one should do\\
  Which having done one won't regret.

  As Kusa-grass, when wrongly grasped,\\
  Will only cut into one's hand\\
  So does the monk's life wrongly led\\
  Indeed drag one to hellish states.

  Whatever deed that's slackly done,\\
  Whatever vow corruptly kept,\\
  The Holy Life led in doubtful ways ---\\
  All these will never bear great fruit.
\end{english}

\suttaRef{S.I.49f}

\section{Sāmaṇera Sikkhā}

\firstline{Anuññāsi kho bhagavā, sāmaṇerānaṃ dasa}

Anuññāsi kho bhagavā,\\
Sāmaṇerānaṃ dasa sikkhā-padāni,\\
Tesu ca sāmaṇerehi sikkhituṃ:\\

\begin{english}
  Ten novice training rules\\
  were established by the Blessed One.\\
  They are the things in which a novice should train
\end{english}

Pāṇātipātā veramaṇī,\\
Adinn'ādānā veramaṇī,\\
Abrahma-cariyā veramaṇī,\\
Musā-vādā veramaṇī,\\
Surā-meraya-majja-pamādaṭṭhānā veramaṇī,\\
Vikāla-bhojanā veramaṇī,\\
Nacca-gīta-vādita-visūka-dassanā veramaṇī,\\
Mālā-gandha-vilepana-dhāraṇa-maṇḍanavibhūsanaṭṭhānā veramaṇī,\\
Uccā-sayana-mahā-sayanā veramaṇī,\\
Jāta-rūpa-rajata-paṭiggahaṇā veramaṇī-ti.

\begin{english}
  Abstaining from killing living beings\\
  Abstaining from taking what is not given\\
  Abstaining from unchastity\\
  Abstaining from false speech\\
  Abstaining from intoxicants that dull the mind\\
  Abstaining from eating at the wrong time\\
  Abstaining from dancing, singing, music and watching shows\\
  Abstaining from perfumes, beautification and adornment\\
  Abstaining from lying on high or luxurious beds\\
  Abstaining from using gold, silver or money.
\end{english}

\suttaRef{Vin.I.83f}

Anuññāsi kho Bhagavā,\\
Dasahi aṅgehi samannāgataṃ sāmaṇeraṃ nāsetuṃ.\\
Katamehi dasahi?

\begin{english}
  Ten grounds for a novice to be dismissed\\
  were established by the Blessed One.\\
  What are these ten?
\end{english}

Pāṇātipātī hoti,\\
Adinn'ādāyī hoti,\\
Abrahma-carī hoti,\\
Musā-vādī hoti,\\
Majja-pāyī hoti,\\
Buddhassa avaṇṇaṃ bhāsati,\\
Dhammassa avaṇṇaṃ bhāsati,\\
Saṅghassa avaṇṇaṃ bhāsati,\\
Micchā-diṭṭhiko hoti,\\
Bhikkhunī-dūsako hoti,

\begin{english}
  He is a killer of living beings\\
  He is a taker of what is not given\\
  He is a practicioner of unchastity\\
  He is a speaker of falsity\\
  He is a consumer of intoxicants\\
  He speaks in dispraise of the Buddha\\
  He speaks in dispraise of the Dhamma\\
  He speaks in dispraise of the Saṅgha\\
  He is a holder of wrong views\\
  He has corrupted a nun
\end{english}

Anuññāsi kho Bhagavā,\\
Imehi dasahi aṅgehi samannāgataṃ sāmaṇeraṃ nāsetun-ti.

\begin{english}
  These are the ten grounds for a novice to be dismissed\\
  which were established by the Blessed One.
  \suttaRef{Vin.I.85}
\end{english}

Anuññāsi kho Bhagavā,\\
Pañcahi aṅgehi samannāgatassa sāmaṇerassa daṇḍa-kammaṃ kātuṃ.\\
Katamehi pañcahi?

\begin{english}
  Five grounds for a novice to be punished\\
  were established by the Blessed One.\\
  What are these five?
\end{english}

Bhikkhūnaṃ alābhāya parisakkati,\\
Bhikkhūnaṃ anatthāya parisakkati,\\
Bhikkhūnaṃ anāvāsāya parisakkati,\\
Bhikkhū akkosati paribhāsati,\\
Bhikkhū bhikkhūhi bhedeti,

\begin{english}
  He strives for the loss of the Bhikkhus\\
  He strives for the non-benefit of the Bhikkhus\\
  He strives for the non-residence of the Bhikkhus\\
  He insults or abuses the Bhikkhus\\
  He causes a split between the Bhikkhus
\end{english}

Anuññāsi kho Bhagavā,\\
Imehi pañcahi aṅgehi samannāgatassa\\
sāmaṇerassa daṇḍa-kammaṃ kātun-ti.

\begin{english}
  These are the ten grounds for a novice to be punished\\
  that were established by the Blessed One.
  \suttaRef{Vin.I.84}
\end{english}

