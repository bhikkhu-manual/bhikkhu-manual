\chapter{Pāṭimokkha Chants}

\section{Verses on the Training Code}

\firstline{Sabba-pāpassa akaraṇaṃ}
\firstline{Khantī paramaṃ tapo tītikkhā}

\begin{leader}
  [Handa mayaṃ ovāda-pāṭimokkha-gāthāyo bhaṇāmase]
\end{leader}

Sabba-pāpassa akaraṇaṃ\\
Kusalassūpasampadā\\
Sacitta-pariyodapanaṃ\\
Etaṃ buddhāna sāsanaṃ\\
Khantī paramaṃ tapo tītikkhā\\
Nibbānaṃ paramaṃ vadanti buddhā\\
Na hi pabbajito parūpaghātī\\
Samaṇo hoti paraṃ viheṭhayanto\\
Anūpavādo anūpaghāto\\
Pāṭimokkhe ca saṃvaro\\
Mattaññutā ca bhattasmiṃ\\
Pantañca sayan'āsanaṃ\\
Adhicitte ca āyogo\\
Etaṃ buddhāna sāsanaṃ

\suttaRef{Dhp 183-185}

\clearpage

\section[Sīl'uddesa-pāṭho]{Sīl'uddesa-pāṭho Uposath'āvasāne Sajjhāyitabbo}

\firstline{Bhāsitam idaṃ tena bhagavatā jānatā passatā}

Bhāsitam idaṃ tena bhagavatā jānatā passatā\\
arahatā sammā-sambuddhena,\\
Sampanna-sīlā bhikkhave viharatha\\
sampanna-pāṭimokkhā,\\
Pāṭimokkha-saṃvara-saṃvutā viharatha\\
ācāra-gocara-sampannā,\\
Aṇu-mattesu vajjesu bhaya-dassāvī\\
samādāya sikkhatha sikkhāpadesū-ti.\\
Tasmā-tih'amhehi sikkhitabbaṃ,\\
Sampanna-sīlā viharissāma sampannapāṭimokkhā,\\
Pāṭimokkha-saṃvara-saṃvutā viharissāma\\
ācāra-gocara-sampannā,\\
Aṇu-mattesu vajjesu bhaya-dassāvī\\
samādāya sikkhissāma sikkhāpadesū-ti,\\
Evañ hi no sikkhitabbaṃ. \suttaRef{cf. D.I.63; D.III.266f}

\section{The Verses of Tāyana}

\firstline{Chinda sotaṃ parakkamma}

\begin{leader}
  [Handa mayaṃ tāyana-gāthāyo bhaṇāmase]
\end{leader}

\begin{twochants}
  Chinda sotaṃ parakkamma & kāme panūda brāhmaṇa \\
  Nappahāya muni kāme & n'ekattam-upapajjati \\
  Kayirā ce kayirāthenaṃ & daḷham-enaṃ parakkame \\
  Sithilo hi paribbājo & bhiyyo ākirate rajaṃ \\
  Akataṃ dukkaṭaṃ seyyo & pacchā tappati dukkaṭaṃ \\
\end{twochants}

\begin{twochants}
  Katañca sukataṃ seyyo & yaṃ katvā nānutappati \\
  Kuso yathā duggahito & hattham-evānukantati \\
  Sāmaññaṃ dupparāmaṭṭhaṃ & nirayāyūpakaḍḍhati \\
  Yaṃ kiñci sithilaṃ kammaṃ & saṅkiliṭṭhañca yaṃ vataṃ \\
  Saṅkassaraṃ brahma-cariyaṃ & na taṃ hoti mahapphalan'ti \\
\end{twochants}

\suttaRef{S.I.49f}

\section{Sāmaṇera Sikkhā}

\firstline{Anuññāsi kho bhagavā, sāmaṇerānaṃ dasa}

Anuññāsi kho bhagavā,\\
Sāmaṇerānaṃ dasa sikkhā-padāni,

\begin{english}
  Ten novice training rules\\
  were established by the Blessed One.
\end{english}

Tesu ca sāmaṇerehi sikkhituṃ:

\begin{english}
  They are the things in which a novice should train
\end{english}

Pāṇātipātā veramaṇī,

\begin{english}
  Abstaining from killing living beings
\end{english}

Adinn'ādānā veramaṇī,

\begin{english}
  Abstaining from taking what is not given
\end{english}

Abrahma-cariyā veramaṇī,

\begin{english}
  Abstaining from unchastity
\end{english}

Musā-vādā veramaṇī,

\begin{english}
  Abstaining from false speech
\end{english}

Surā-meraya-majja-pamādaṭṭhānā veramaṇī,

\begin{english}
  Abstaining from intoxicants that dull the mind
\end{english}

Vikāla-bhojanā veramaṇī,

\begin{english}
  Abstaining from eating at the wrong time
\end{english}

Nacca-gīta-vādita-visūka-dassanā veramaṇī,

\begin{english}
  Abstaining from dancing, singing, music and watching shows
\end{english}

Mālā-gandha-vilepana-dhāraṇa-maṇḍanavibhūsanaṭṭhānā veramaṇī,

\begin{english}
  Abstaining from perfumes, beautification and adornment
\end{english}

Uccā-sayana-mahā-sayanā veramaṇī,

\begin{english}
  Abstaining from lying on high or luxurious beds
\end{english}

Jāta-rūpa-rajata-paṭiggahaṇā veramaṇī-ti.

\begin{english}
  Abstaining from using gold, silver or money
\end{english}

\suttaRef{Vin.I.83f}

Anuññāsi kho Bhagavā,\\
Dasahi aṅgehi samannāgataṃ sāmaṇeraṃ nāsetuṃ.\\

\begin{english}
  Ten grounds for a novice to be dismissed\\
  were established by the Blessed One.
\end{english}

Katamehi dasahi?

\begin{english}
  What are these ten?
\end{english}

Pāṇātipātī hoti,

\begin{english}
  He is a killer of living beings
\end{english}

Adinn'ādāyī hoti,

\begin{english}
  He is a taker of what is not given
\end{english}

Abrahma-carī hoti,

\begin{english}
  He is a practicioner of unchastity
\end{english}

Musā-vādī hoti,

\begin{english}
  He is a speaker of falsity
\end{english}

Majja-pāyī hoti,

\begin{english}
  He is a consumer of intoxicants
\end{english}

Buddhassa avaṇṇaṃ bhāsati,

\begin{english}
  He speaks in dispraise of the Buddha
\end{english}

Dhammassa avaṇṇaṃ bhāsati,

\begin{english}
  He speaks in dispraise of the Dhamma
\end{english}

Saṅghassa avaṇṇaṃ bhāsati,

\begin{english}
  He speaks in dispraise of the Saṅgha
\end{english}

Micchā-diṭṭhiko hoti,

\begin{english}
  He is a holder of wrong views
\end{english}

Bhikkhunī-dūsako hoti,

\begin{english}
  He has corrupted a nun
\end{english}

Anuññāsi kho Bhagavā,\\
Imehi dasahi aṅgehi samannāgataṃ sāmaṇeraṃ nāsetun-ti.

\begin{english}
  These are the ten grounds for a novice to be dismissed\\
  which were established by the Blessed One.
\end{english}

\suttaRef{Vin.I.85}

Anuññāsi kho Bhagavā,\\
Pañcahi aṅgehi samannāgatassa sāmaṇerassa daṇḍa-kammaṃ kātuṃ.

\begin{english}
  Five grounds for a novice to be punished\\
  were established by the Blessed One.
\end{english}

Katamehi pañcahi?

\begin{english}
  What are these five?
\end{english}

Bhikkhūnaṃ alābhāya parisakkati,

\begin{english}
  He strives for the loss of the Bhikkhus
\end{english}

Bhikkhūnaṃ anatthāya parisakkati,

\begin{english}
  He strives for the non-benefit of the Bhikkhus
\end{english}

Bhikkhūnaṃ anāvāsāya parisakkati,

\begin{english}
  He strives for the non-residence of the Bhikkhus
\end{english}

Bhikkhū akkosati paribhāsati,

\begin{english}
  He insults or abuses the Bhikkhus
\end{english}

Bhikkhū bhikkhūhi bhedeti,

\begin{english}
  He causes a split between the Bhikkhus
\end{english}

Anuññāsi kho Bhagavā,\\
Imehi pañcahi aṅgehi samannāgatassa\\
sāmaṇerassa daṇḍa-kammaṃ kātun-ti.

\begin{english}
  These are the ten grounds for a novice to be punished\\
  that were established by the Blessed One.
\end{english}

\suttaRef{Vin.I.84}

