\chapter{Chants Used in Sri Lanka}

%
%\section{Salutation to the Three Main Objects of Venerations}
%
%Vandāmi cetiyaṃ sabbaṃ\\
%Sabba-ṭhānesu patiṭṭhitaṃ\\
%Sārīrīka-dhātu-Mahā-bodhiṃ\\
%Buddha-rūpaṃ sakalaṃ sadā.
%
%\section{Salutation to the Bodhi-Tree}
%
%Yassa mūle nissino va\\
%Sabbāri vijayaṃ akā,\\
%Patto sabbaññutaṃ Satthā\\
%Vande taṃ Bodhi-pādapaṃ.\\
%Ime ete Mahā-Bodhi\\
%Loka-nāthena pūjitā,\\
%Aham-pi te namassāmi\\
%Bodhi-Rājā nam'atthu te!
%
%\section{Offering of Lights}
%
%Ghana-sārappadittena\\
%Dīpena tama-dhaṃsinā\\
%Tīloka-dīpam sambuddhaṃ\\
%Pūjayāmi tamo-nudaṃ.
%
%\section{Offering of Incense}
%
%Gandha-sambhāra-yuttena\\
%Dhūpenāhaṃ sugandhinā\\
%Pūjaye pūjaneyyan-taṃ\\
%Pūjā-bhājanam-uttamaṃ.
%
%\section{Offering of Flowers}
%
%Vaṇṇa-gandha-guṇopetaṃ\\
%Etaṃ kusuma-santatiṃ.\\
%Pūjayāmi munindassa\\
%Sirīpāda-saroruhe.\\
%Pūjemi Buddhaṃ kusumena'nena\\
%Puññenam-etena ca hotu mokkhaṃ\\
%Pupphaṃ milāyāti yathā idaṃ me\\
%Kāyo tathā yāti vināsa-bhāvaṃ.
%
%\section{Transference of Merit to Devas}
%
%Ākāsatthā ca bhummatthā\\
%Devā nāgā mah'iddhikā\\
%Puññaṃ taṃ anumoditvā
%
%Ciraṃ rakkhantu ...
%
%% FIXME formatting
%
%%  /loka/ sāsanaṃ.
%% Ciraṃ rakkhantu   desanaṃ
%%  maṃ paraṃ
%
%Ettāvatā ca amhehi\\
%Sambhataṃ puñña-sampadaṃ\\
%Sabbe devā/ bhūtā/ sattā anumodantu\\
%Sabba-sampatti siddhiyā.
%
%\section{Blessing to the World}
%
%Devo vassatu kālena\\
%Sassa-sampatti-hetu ca\\
%Phīto bhavatu loko ca\\
%Rajā bhavatu dhammiko.
%
%\section{Transference of Merits to Departed Ones}
%
%Idam te...
%
%% Idam te/ vo/ no/ me * ñātīnam hotu
%% sukhitā hontu ñātayo.
%% (×3)
%
%\section{The Aspirations}
%
%Iminā puñña-kammena\\
%Mā me bāla-samāgamo,\\
%Sataṃ samāgamo hotu,\\
%Yāva nibbāna-pattiyā.\\
%Kāyena vācā-cittena\\
%Pamādena mayā kataṃ\\
%Accayaṃ khama me bhante\\
%Bhūri-pañña Tathāgata.
%
%\section{Blessing and Protection}
%
%Sabb'ītiyo vivajjantu,\\
%Sabba-rogo vinassatu;\\
%Mā me/no bhavatvantarāyo,\\
%Sukhī dīghāyuko bhava.\\
%/Sukhī dīghāyukā homa.\\
%Bhavatu sabba-maṅgalaṃ.\\
%Rakkhantu sabba-devatā.
%
%When chanting for one person use ‘te’; when for laypeople use ‘vo’; when chanting together in a group use
%‘no’; when alone use ‘me’.
%
%Sabba-buddhānubhāvena,\\
%Sadā sotthi bhavantu me.\\
%Bhavatu sabba-maṅgalaṃ.\\
%Rakkhantu sabba-devatā.\\
%Sabba-dhammānunbhāvena,\\
%Sadā sotthi bhavantu me.\\
%Bhavatu sabba-maṅgalaṃ.\\
%Rakkhantu sabba-devatā.\\
%Sabba-saṅghānubhāvena,\\
%Sadā sotthi bhavantu me.\\
%Nakkhatta-yakkha-bhūtānaṃ\\
%Pāpaggaha-nivāraṇā\\
%Parittassānubhāvena\\
%Hantvā mayhaṃ/amhe upaddave.\\
%Devo vassatu kālena.\\
%Sassa-sampatti-hetu ca.\\
%Phīto bhavatu loko ca.\\
%Rājā bhavatu dhammiko.\\
%Sabbe buddhā balappattā,\\
%Paccekānañca yaṃ balaṃ\\
%Arahantānañca tejena,\\
%Rakkhaṃ bandhāmi sabbaso.
%
%\section{Mettā Bhāvanā}
%
%1. Attūpamāya sabbesaṃ\\
%Sattānaṃ sukhakāmataṃ,\\
%Passitvā kamato mettaṃ\\
%Sabbasattesu bhāvaye.\\
%2. Sukhi bhaveyyaṃ niddukkho\\
%Ahaṃ niccaṃ ahaṃ viya,\\
%Hitā ca me sukhī hontu\\
%Majjhatthā c'atha verino.\\
%3. Imamhi gāmakkhettamhi\\
%Sattā hontu sukhī sadā,\\
%Tato parañ ca-rajjesu\\
%Cakkavāḷesu jantuno.\\
%4. Samantā cakkavāḷesu\\
%Sattānan-tesu pāṇino,\\
%Sukhino puggala bhūtā\\
%Attabhāvagatā siyuṃ.\\
%5. Tathā itthī pumā ce'va\\
%Ariya anariya’ pi ca,\\
%Devā narā apāyaṭṭhā\\
%Tathā dasa disāsu cā-ti.
%
%\section{Pattanumodana}
%
%(Sharing Merits)
%
%Idaṃ te...
%
%% Idaṃ te/ vo/ no/ me* ñātīnaṃ hotu
%% Sukhitā hontu ñātayo (×3)
%
%Yathā vāri-vahā pūrā\\
%Paripūrenti sāgaraṃ,\\
%Evaṃ eva ito dinnaṃ\\
%Petānaṃ upakappatu.\\
%Unname udakaṃ vattaṃ\\
%Yathā ninnaṃ pavattati,\\
%Evaṃ eva ito dinnaṃ\\
%Petānaṃ upakappatu.\\
%Āyūr-arogya-sampatti\\
%Sagga-sampattiṃ eva ca,\\
%Atho nibbāna-sampatti,\\
%Iminā te/* samijjhatu.\\
%Icchitaṃ patthitaṃ tuyhaṃ\\
%Sabbam-eva samijjhatu,\\
%Pūrentu citta-saṅkappā\\
%Maṇi-joti-raso yathā.\\
%Icchitaṃ patthitaṃ tuyhaṃ,\\
%Sabbam-eva samijjhatu,\\
%Pūrentu citta-saṅkappā\\
%Cando paṇṇa-rasī yathā.\\
%Icchitaṃ patthitaṃ tuyhaṃ\\
%Khippam-eva samijjhatu,\\
%Sabbe pūrentu saṅkappā\\
%Cando paṇṇa-rasī yathā.
