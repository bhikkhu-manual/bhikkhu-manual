\chapter{Chants Used in Sri Lanka}

\section{Devotional Chants}

\subsection[Salutation to the Three Main Objects]{Salutation to the Three Main Objects of Venerations}

\firstline{Vandāmi cetiyaṁ sabbaṁ}

\begin{paritta}
Vandāmi cetiyaṁ sabbaṁ\\
Sabba-ṭhānesu patiṭṭhitaṁ\\
Sārīrīka-dhātu-Mahā-bodhiṁ\\
Buddha-rūpaṁ sakalaṁ sadā.
\end{paritta}

\subsection{Salutation to the Bodhi-Tree}

\firstline{Yassa mūle nissino va sabbāri vijayaṁ akā}

\begin{twochants}
Yassa mūle nissino va & sabbāri vijayaṁ akā,\\
Patto sabbaññutaṁ Satthā & vande taṁ Bodhi-pādapaṁ.\\
Ime ete Mahā-Bodhi & loka-nāthena pūjitā,\\
Aham-pi te namassāmi & bodhi-Rājā nam'atthu te!
\end{twochants}

\subsection{Offering of Lights}

\firstline{Ghana-sārappadittena}

\begin{paritta}
Ghana-sārappadittena\\
Dīpena tama-dhaṁsinā\\
Tīloka-dīpam sambuddhaṁ\\
Pūjayāmi tamo-nudaṁ.
\end{paritta}

\clearpage

\subsection{Offering of Incense}

\firstline{Gandha-sambhāra-yuttena}

\begin{paritta}
Gandha-sambhāra-yuttena\\
Dhūpenāhaṁ sugandhinā\\
Pūjaye pūjaneyyan-taṁ\\
Pūjā-bhājanam-uttamaṁ.
\end{paritta}

\subsection{Offering of Flowers}

\firstline{Vaṇṇa-gandha-guṇopetaṁ}

\begin{paritta}
Vaṇṇa-gandha-guṇopetaṁ\\
Etaṁ kusuma-santatiṁ.\\
Pūjayāmi munindassa\\
Sirīpāda-saroruhe.\\
Pūjemi Buddhaṁ kusumena'nena\\
Puññenam-etena ca hotu mokkhaṁ\\
Pupphaṁ milāyāti yathā idaṁ me\\
Kāyo tathā yāti vināsa-bhāvaṁ.
\end{paritta}

\subsection{Transference of Merit to Devas}

\firstline{Ākāsatthā ca bhummatthā}

\begin{paritta}
Ākāsatthā ca bhummatthā\\
Devā nāgā mah'iddhikā\\
Puññaṁ taṁ anumoditvā\\
Ciraṁ rakkhantu [loka] sāsanaṁ\\
Ciraṁ rakkhantu desanaṁ\\
Ciraṁ rakkhantu maṁ paraṁ
\end{paritta}

\begin{paritta}
Ettāvatā ca amhehi\\
Sambhataṁ puñña-sampadaṁ\\
Sabbe devā/ bhūtā/ sattā anumodantu\\
Sabba-sampatti siddhiyā.
\end{paritta}

\subsection{Blessing to the World}

\firstline{Devo vassatu kālena}

\begin{paritta}
Devo vassatu kālena\\
Sassa-sampatti-hetu ca\\
Phīto bhavatu loko ca\\
Rajā bhavatu dhammiko.
\end{paritta}

\subsection{Transference of Merits to Departed Ones}

\firstline{Idaṁ te/vo/no/me ñātīnaṁ hotu}

\begin{paritta}
  Idaṁ te/vo/no/me ñātīnaṁ hotu\\
  sukhitā hontu ñātayo. (×3)

  \instr{(When chanting for one person use `te'; when for laypeople use `vo';
    when chanting together in a group use `no'; when alone use `me'.)}
\end{paritta}

\subsection{The Aspirations}

\firstline{Iminā puññakammena mā me bāla-samāgamo}

\begin{twochants}
Iminā puññakammena & mā me bāla-samāgamo,\\
Sataṁ samāgamo hotu, & yāva nibbāna-pattiyā.\\
Kāyena vācā-cittena & pamādena mayā kataṁ\\
Accayaṁ khama me bhante & bhūri-pañña Tathāgata.
\end{twochants}

\subsection{Blessing and Protection}

\firstline{Sabb'ītiyo vivajjantu sabba-rogo vinassatu}

\begin{twochants}
Sabb'ītiyo vivajjantu & sabba-rogo vinassatu;\\
Mā me/no bhavatvantarāyo & sukhī dīghāyuko/ā bhava/homa.\\
Bhavatu sabba-maṅgalaṁ & rakkhantu sabba-devatā.\\
Sabba-buddhānubhāvena & sadā sotthi bhavantu me.\\
Bhavatu sabba-maṅgalaṁ & rakkhantu sabba-devatā.\\
Sabba-dhammānunbhāvena & sadā sotthi bhavantu me.\\
Bhavatu sabba-maṅgalaṁ. & rakkhantu sabba-devatā.\\
Sabba-saṅghānubhāvena, & sadā sotthi bhavantu me.\\
Nakkhatta-yakkha-bhūtānaṁ & pāpaggaha-nivāraṇā\\
Parittassānubhāvena & hantvā mayhaṁ/amhe upaddave.\\
Devo vassatu kālena. & sassa-sampatti-hetu ca.\\
Phīto bhavatu loko ca. & rājā bhavatu dhammiko.\\
Sabbe buddhā balappattā, & paccekānañca yaṁ balaṁ\\
Arahantānañca tejena, & rakkhaṁ bandhāmi sabbaso.
\end{twochants}

\subsection{Mettā Bhāvanā}

\firstline{Attūpamāya sabbesaṁ sattānaṁ sukhakāmataṁ}

\begin{twochants}
Attūpamāya sabbesaṁ & sattānaṁ sukhakāmataṁ,\\
Passitvā kamato mettaṁ & sabbasattesu bhāvaye.\\
Sukhi bhaveyyaṁ niddukkho & ahaṁ niccaṁ ahaṁ viya,\\
Hitā ca me sukhī hontu & majjhatthā c'atha verino.\\
Imamhi gāmakkhettamhi & sattā hontu sukhī sadā,\\
Tato parañ ca-rajjesu & cakkavāḷesu jantuno.\\
\end{twochants}

\begin{twochants}
Samantā cakkavāḷesu & sattānan-tesu pāṇino,\\
Sukhino puggala bhūtā & attabhāvagatā siyuṁ.\\
Tathā itthī pumā ce'va & ariya anariya’ pi ca,\\
Devā narā apāyaṭṭhā & tathā dasa disāsu cā-ti.\\
\end{twochants}

\subsection{Pattanumodana (Sharing Merits)}

\firstline{Idaṁ te/vo/no/me ñātīnaṁ hotu}

Idaṁ te/vo/no/me ñātīnaṁ hotu\\
Sukhitā hontu ñātayo (×3)

\begin{twochants}
Yathā vāri-vahā pūrā & paripūrenti sāgaraṁ,\\
Evaṁ eva ito dinnaṁ & petānaṁ upakappatu.\\
Unname udakaṁ vattaṁ & yathā ninnaṁ pavattati,\\
Evaṁ eva ito dinnaṁ & petānaṁ upakappatu.\\
Āyūr-arogya-sampatti & sagga-sampattiṁ eva ca,\\
Atho nibbāna-sampatti & iminā te/vo/no/me samijjhatu.\\
Icchitaṁ patthitaṁ tuyhaṁ & sabbam-eva samijjhatu,\\
Pūrentu citta-saṅkappā & maṇi-joti-raso yathā.\\
Icchitaṁ patthitaṁ tuyhaṁ & sabbam-eva samijjhatu,\\
Pūrentu citta-saṅkappā & cando paṇṇa-rasī yathā.\\
Icchitaṁ patthitaṁ tuyhaṁ & khippam-eva samijjhatu,\\
Sabbe pūrentu saṅkappā & cando paṇṇa-rasī yathā.
\end{twochants}

\suttaRef{Petavatthu p.19-31 \& KhpA. 206-215}

\section{Offences}

\subsection{Āpatti-paṭidesanā (Confession of Offences)}

\subsubsection{Method of confessing light offences}

\prul{JCB:} Junior Confessing Bhikkhu\\
\prul{SAB:} Senior Acknowledging Bhikkhu

\hangindent=25pt%
\parbox{22pt}{\prul{JCB:}} Okāsa, ahaṁ bhante, sabbā āpattiyo ārocemi.\\
Dutiyam-pi ahaṁ bhante, sabbā āpattiyo ārocemi.\\
Tatiyam-pi ahaṁ bhante, sabbā āpattiyo ārocemi.\\
\emph{I ven. sir, declare all offences. For the second time… For the third time…}

\hangindent=25pt%
\parbox{22pt}{\prul{SAB:}} Sādhu, sādhu.\\ \emph{It is good, it is good.}

\hangindent=25pt%
\parbox{22pt}{\prul{JCB:}} Okāsa ahaṁ bhante, sambahulā nānā-vatthukā āpattiyo āpajjiṁ, tā tumha-mūle paṭidesemi.\\ \emph{I, ven. sir, having many times fallen into many different offences with different bases, these I confess.}

\hangindent=25pt%
\parbox{22pt}{\prul{SAB:}} Passasi āvuso tā āpattiyo?\\ \emph{Do you see, friend, those offences?}

\hangindent=25pt%
\parbox{22pt}{\prul{JCB:}} Āma bhante passāmi.\\ \emph{Yes, ven. sir, I see.}

\hangindent=25pt%
\parbox{22pt}{\prul{SAB:}} Āyatiṁ āvuso saṁvareyyāsi.\\ \emph{In the future, friend, you should be restrained.}

\hangindent=25pt%
\parbox{22pt}{\prul{JCB:}} Sādhu suṭṭhu bhante āyatiṁ saṁvarissāmi.\\
Dutiyam-pi sādhu suṭṭhu bhante āyatiṁ saṁvarissāmi.\\
Tatiyam-pi sādhu suṭṭhu bhante āyatiṁ saṁvarissāmi.\\
\emph{It is well indeed, ven. sir, in future I shall be restrained. For the second time…For the third time…}

\hangindent=25pt%
\parbox{22pt}{\prul{SAB:}} Sādhu, sādhu.\\ \emph{It is good, it is good.}

\hangindent=25pt%
\parbox{22pt}{\prul{JCB:}} Okāsa ahaṁ bhante,\\
sabbā tā garukāpattiyo āvikaromi.\\
Dutiyam-pi okāsa ahaṁ bhante,\\
sabbā tā garukāpattiyo āvikaromi.\\
Tatiyam-pi okāsa ahaṁ bhante,\\
sabbā tā garukāpattiyo āvikaromi.\\
\emph{Ven. sir, I reveal all heavy offences. For the second time… For the third time…}

This final declaration is only used in some communities. Also, some communities
will acknowledge with a ‘\emph{Sādhu}’ after each declaration rather than as
shown above. That is, after each ‘\emph{ārocemi}’ and each
‘\emph{saṁvarissāmi}’.

\subsubsection{Formula for same base offences}

\hangindent=25pt%
\parbox{22pt}{\prul{JCB:}} Okāsa ahaṁ bhante, desanādukkaṭāpattiṁ āpajjiṁ, taṁ tumha-mūle paṭidesemi.\\ \emph{I, ven. sir, confess an offence of wrong-doing through having confessed the same-based offences.}

\hangindent=25pt%
\parbox{22pt}{\prul{SAB:}} Passasi āvuso taṁ āpaṭṭiṁ?\\ \emph{Do you see, friend, that offence?}

\hangindent=25pt%
\parbox{22pt}{\prul{JCB:}} Āma bhante passāmi.\\ \emph{Yes, ven. sir, I see.}

\hangindent=25pt%
\parbox{22pt}{\prul{SAB:}} Āyatiṁ āvuso saṁvareyyāsi.\\ \emph{In the future, friend, you should be restrained.}

\hangindent=25pt%
\parbox{22pt}{\prul{JCB:}} Sādhu suṭṭhu bhante āyatiṁ saṁvarissāmi. Dutiyam-pi sādhu suṭṭhu … . Tatiyam-pi … saṁvarissāmi.\\ \emph{It is well indeed, ven. sir, in future I shall be restrained. For the second time… For the third time…}

\hangindent=25pt%
\parbox{22pt}{\prul{SAB:}} Sādhu, sādhu.\\ \emph{It is good, it is good.} \suttaRef{Vin.II.102}

\section{Rains and Kathina}

\subsection{Entering the Rains}

‘Imasmiṁ vihāre imaṁ te-māsaṁ vassaṁ upemi. Idha vassaṁ upemi.’\\
‘\emph{I enter the Rains in this kuṭi for three months. I enter the Rains
  here.}’

\section{Uposatha-day for Lay-followers}

\subsection{Eight Precepts}

With hands in \emph{añjali}, the laypeople recite the following request:

‘Sādhu! Sādhu! Sādhu! Okāsa ahaṁ bhante ti-saraṇena saddhiṁ aṭṭh'aṅga sīlaṁ
dhammaṁ yācāmi, anuggahaṁ katvā sīlaṁ detha me bhante. Dutiyam-pi okāsa… detha
me bhante. Tatiyam-pi okāsa… detha me bhante.’

\emph{Bhk}: ‘Yaṁ ahaṁ vadāmi taṁ vadetha.’

\emph{Laypeople}: ‘Āma, bhante.’

\emph{Bhk}: ‘Namo…’ (×3)

\emph{Laypeople}: repeat.

\emph{Bhk}:

‘Buddhaṁ saraṇaṁ gacchāmi.\\
Dhammaṁ saraṇaṁ gacchāmi.\\
Saṅghaṁ saraṇaṁ gacchāmi.\\
Dutiyam-pi Buddhaṁ saraṇaṁ gacchāmi.\\
Dutiyam-pi Dhammaṁ saraṇaṁ gacchāmi.\\
Dutiyam-pi Saṅghaṁ saraṇaṁ gacchāmi.\\
Tatiyam-pi Buddhaṁ saraṇaṁ gacchāmi.\\
Tatiyam-pi Dhammaṁ saraṇaṁ gacchāmi.\\
Tatiyam-pi Saṅghaṁ saraṇaṁ gacchāmi.’

\emph{Laypeople}: repeat line by line.

\emph{Bhk}: ‘Saraṇagamanaṁ sampuṇṇaṁ.’

\emph{Laypeople}: ‘Āma, bhante.’

Then the bhikkhu recites, with the laypeople repeating line by line:

\begin{packeditemize}

\item Pāṇātipātā veramaṇī sikkhā-padaṁ samādiyāmi.
\item Adinnādānā veramaṇī sikkhā-padaṁ samādiyāmi.
\item Abrahma-cariyā veramaṇī sikkhā-padaṁ samādiyāmi.
\item Musāvādā veramaṇī sikkhā-padaṁ samādiyāmi.
\item Surā-meraya-majja-pamādaṭṭhānā veramaṇī sikkhā-padaṁ samādiyāmi.
\item Vikāla-bhojanā veramaṇī sikkhā-padaṁ samādiyāmi.
\item Nacca-gīta vādita visūka-dassana mālāgandha vilepana dhāraṇa maṇḍana vibhūsanaṭṭhānā veramaṇī sikkhā-padaṁ samādiyāmi.
\item Uccā-sayana mahā-sayanā veramaṇī sikkhā-padaṁ samādiyāmi.

\end{packeditemize}

\suttaRef{A.IV.248–250}

I undertake the precept to refrain from:

\begin{packeditemize}

\item destroying living beings.
\item taking that which is not given.
\item any kind of intentional sexual behaviour.
\item false speech.
\item intoxicating drinks and drugs that lead to carelessness.
\item eating at wrong times.
\item dancing, singing, music and going to entertainments, perfumes, beautification and adornment.
\item lying on a high or luxurious sleeping place.

\end{packeditemize}

\emph{Bhk}: ‘Imaṁ aṭṭh'aṅga-sīlaṁ samādiyāmi.’

\emph{Laypeople}: ‘Imaṁ aṭṭh'aṅga-sīlaṁ samādiyāmi.’ (×3)

\emph{Bhk}: ‘Ti-saraṇena saddhiṁ aṭṭh'aṅga-sīlaṁ dhammaṁ sādhukaṁ surakkhitaṁ
katvā appamādena sampādetha.’

\emph{Laypeople}: ‘Āma, bhante.’

\emph{Bhk}:

‘Sīlena sugatiṁ yanti,\\
Sīlena bhoga-sampadā,\\
Sīlena nibbutiṁ yanti,\\
Tasmā sīlaṁ visodhaye.’

‘\emph{These Eight Precepts\\
  Have morality as a vehicle for happiness,\\
  Have morality as a vehicle for good fortune,\\
  Have morality as a vehicle for liberation,\\
  Let morality therefore be purified.}’

The Laypeople may respond with:

‘Sādhu, sādhu, sādhu!’

\subsection{Five Precepts}

With hands in \emph{añjali}, the laypeople recite the following request:

‘Sādhu! Sādhu! Sādhu! Okāsa ahaṁ bhante tisaraṇena saddhiṁ pañca-sīlaṁ dhammaṁ
yācāmi, anuggahaṁ katvā sīlaṁ detha me bhante. Dutiyam-pi okāsa… Tatiyam-pi
okāsa…’

\emph{Bhikkhu}: ‘Yaṁ ahaṁ vadāmi taṁ vadetha.’

\emph{Laypeople}: ‘Āma, bhante.’

\emph{Bhk}: ‘Namo…’ (×3)

\emph{Laypeople}: repeat.

\emph{Bhk}: ‘Saraṇagamanaṁ sampuṇṇaṁ.’

\emph{Laypeople}: ‘Āma, bhante.’

Then the bhikkhu recites, with the laypeople repeating line by line:

\begin{packeditemize}

\item Pāṇātipātā veramaṇī sikkhā-padaṁ samādiyāmi.
\item Adinnādānā veramaṇī sikkhā-padaṁ samādiyāmi.
\item Kāmesu micchā-cārā veramaṇī sikkhā-padaṁ samādiyāmi.
\item Musā-vādā veramaṇī sikkhā-padaṁ samādiyāmi.
\item Surā-meraya-majja-pamādaṭṭhānā veramaṇī sikkhā-padaṁ samādiyāmi.

\end{packeditemize}

\suttaRef{A.IV.248–250}

\emph{Bhk}:

‘Tisaraṇena saddhiṁ pañcasīlaṁ dhammaṁ sādhukaṁ surakkhitaṁ katvā appamādena
sampādetha.’

\emph{Laypeople}: ‘Āma, bhante.’

\emph{Bhk}:

‘Sīlena sugatiṁ yanti\\
Sīlena bhoga-sampadā,\\
Sīlena nibbutiṁ yanti,\\
Tasmā sīlaṁ visodhaye.’

