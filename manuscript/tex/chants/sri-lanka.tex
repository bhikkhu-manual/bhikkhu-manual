\chapter{Chants Used in Sri Lanka}

\section[Salutation to the Three Main Objects]{Salutation to the Three Main Objects of Venerations}

\firstline{Vandāmi cetiyaṃ sabbaṃ}

\begin{paritta}
Vandāmi cetiyaṃ sabbaṃ\\
Sabba-ṭhānesu patiṭṭhitaṃ\\
Sārīrīka-dhātu-Mahā-bodhiṃ\\
Buddha-rūpaṃ sakalaṃ sadā.
\end{paritta}

\section{Salutation to the Bodhi-Tree}

\firstline{Yassa mūle nissino va sabbāri vijayaṃ akā}

\begin{twochants}
Yassa mūle nissino va & sabbāri vijayaṃ akā,\\
Patto sabbaññutaṃ Satthā & vande taṃ Bodhi-pādapaṃ.\\
Ime ete Mahā-Bodhi & loka-nāthena pūjitā,\\
Aham-pi te namassāmi & bodhi-Rājā nam'atthu te!
\end{twochants}

\section{Offering of Lights}

\firstline{Ghana-sārappadittena}

\begin{paritta}
Ghana-sārappadittena\\
Dīpena tama-dhaṃsinā\\
Tīloka-dīpam sambuddhaṃ\\
Pūjayāmi tamo-nudaṃ.
\end{paritta}

\section{Offering of Incense}

\firstline{Gandha-sambhāra-yuttena}

\begin{paritta}
Gandha-sambhāra-yuttena\\
Dhūpenāhaṃ sugandhinā\\
Pūjaye pūjaneyyan-taṃ\\
Pūjā-bhājanam-uttamaṃ.
\end{paritta}

\section{Offering of Flowers}

\firstline{Vaṇṇa-gandha-guṇopetaṃ}

\begin{paritta}
Vaṇṇa-gandha-guṇopetaṃ\\
Etaṃ kusuma-santatiṃ.\\
Pūjayāmi munindassa\\
Sirīpāda-saroruhe.\\
Pūjemi Buddhaṃ kusumena'nena\\
Puññenam-etena ca hotu mokkhaṃ\\
Pupphaṃ milāyāti yathā idaṃ me\\
Kāyo tathā yāti vināsa-bhāvaṃ.
\end{paritta}

\section{Transference of Merit to Devas}

\firstline{Ākāsatthā ca bhummatthā}

\begin{paritta}
Ākāsatthā ca bhummatthā\\
Devā nāgā mah'iddhikā\\
Puññaṃ taṃ anumoditvā\\
Ciraṃ rakkhantu /loka/ sāsanaṃ\\
Ciraṃ rakkhantu desanaṃ\\
Ciraṃ rakkhantu maṃ paraṃ
\end{paritta}

\clearpage

\begin{paritta}
Ettāvatā ca amhehi\\
Sambhataṃ puñña-sampadaṃ\\
Sabbe devā/ bhūtā/ sattā anumodantu\\
Sabba-sampatti siddhiyā.
\end{paritta}

\section{Blessing to the World}

\firstline{Devo vassatu kālena}

\begin{paritta}
Devo vassatu kālena\\
Sassa-sampatti-hetu ca\\
Phīto bhavatu loko ca\\
Rajā bhavatu dhammiko.
\end{paritta}

\section{Transference of Merits to Departed Ones}

\firstline{Idaṃ te/vo/no/me ñātīnaṃ hotu}

\begin{paritta}
  Idaṃ te/vo/no/me ñātīnaṃ hotu\\
  sukhitā hontu ñātayo. (×3)

  \instr{(When chanting for one person use `te'; when for laypeople use `vo';
    when chanting together in a group use `no'; when alone use `me'.)}
\end{paritta}

\section{The Aspirations}

\firstline{Iminā puññakammena mā me bāla-samāgamo}

\begin{twochants}
Iminā puññakammena & mā me bāla-samāgamo,\\
Sataṃ samāgamo hotu, & yāva nibbāna-pattiyā.\\
Kāyena vācā-cittena & pamādena mayā kataṃ\\
Accayaṃ khama me bhante & bhūri-pañña Tathāgata.
\end{twochants}

\section{Blessing and Protection}

\firstline{Sabb'ītiyo vivajjantu sabba-rogo vinassatu}

\begin{twochants}
Sabb'ītiyo vivajjantu & sabba-rogo vinassatu;\\
Mā me/no bhavatvantarāyo & sukhī dīghāyuko/ā bhava/homa.\\
Bhavatu sabba-maṅgalaṃ & rakkhantu sabba-devatā.\\
Sabba-buddhānubhāvena & sadā sotthi bhavantu me.\\
Bhavatu sabba-maṅgalaṃ & rakkhantu sabba-devatā.\\
Sabba-dhammānunbhāvena & sadā sotthi bhavantu me.\\
Bhavatu sabba-maṅgalaṃ. & rakkhantu sabba-devatā.\\
Sabba-saṅghānubhāvena, & sadā sotthi bhavantu me.\\
Nakkhatta-yakkha-bhūtānaṃ & pāpaggaha-nivāraṇā\\
Parittassānubhāvena & hantvā mayhaṃ/amhe upaddave.\\
Devo vassatu kālena. & sassa-sampatti-hetu ca.\\
Phīto bhavatu loko ca. & rājā bhavatu dhammiko.\\
Sabbe buddhā balappattā, & paccekānañca yaṃ balaṃ\\
Arahantānañca tejena, & rakkhaṃ bandhāmi sabbaso.
\end{twochants}

%\suttaRef{MJG}

\section{Mettā Bhāvanā}

\firstline{Attūpamāya sabbesaṃ sattānaṃ sukhakāmataṃ}

\begin{twochants}
Attūpamāya sabbesaṃ & sattānaṃ sukhakāmataṃ,\\
Passitvā kamato mettaṃ & sabbasattesu bhāvaye.\\
Sukhi bhaveyyaṃ niddukkho & ahaṃ niccaṃ ahaṃ viya,\\
Hitā ca me sukhī hontu & majjhatthā c'atha verino.\\
Imamhi gāmakkhettamhi & sattā hontu sukhī sadā,\\
Tato parañ ca-rajjesu & cakkavāḷesu jantuno.\\
\end{twochants}

\begin{twochants}
Samantā cakkavāḷesu & sattānan-tesu pāṇino,\\
Sukhino puggala bhūtā & attabhāvagatā siyuṃ.\\
Tathā itthī pumā ce'va & ariya anariya’ pi ca,\\
Devā narā apāyaṭṭhā & tathā dasa disāsu cā-ti.\\
\end{twochants}

\section{Pattanumodana (Sharing Merits)}

\firstline{Idaṃ te/vo/no/me ñātīnaṃ hotu}

Idaṃ te/vo/no/me ñātīnaṃ hotu\\
Sukhitā hontu ñātayo (×3)

\begin{twochants}
Yathā vāri-vahā pūrā & paripūrenti sāgaraṃ,\\
Evaṃ eva ito dinnaṃ & petānaṃ upakappatu.\\
Unname udakaṃ vattaṃ & yathā ninnaṃ pavattati,\\
Evaṃ eva ito dinnaṃ & petānaṃ upakappatu.\\
Āyūr-arogya-sampatti & sagga-sampattiṃ eva ca,\\
Atho nibbāna-sampatti & iminā te/vo/no/me samijjhatu.\\
Icchitaṃ patthitaṃ tuyhaṃ & sabbam-eva samijjhatu,\\
Pūrentu citta-saṅkappā & maṇi-joti-raso yathā.\\
Icchitaṃ patthitaṃ tuyhaṃ & sabbam-eva samijjhatu,\\
Pūrentu citta-saṅkappā & cando paṇṇa-rasī yathā.\\
Icchitaṃ patthitaṃ tuyhaṃ & khippam-eva samijjhatu,\\
Sabbe pūrentu saṅkappā & cando paṇṇa-rasī yathā.
\end{twochants}

\suttaRef{cf. Petavatthu p.19-31 \& KhpA. 206-215}

\section{Greeting Used in Sri-Lanka}

(FIXME placeholder)
