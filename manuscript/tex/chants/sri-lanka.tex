\chapter{Chants Used in Sri Lanka}

\section{Devotional Chants}

\subsection[Salutation to the Three Main Objects]{Salutation to the Three Main Objects of Venerations}

\firstline{Vandāmi cetiyaṃ sabbaṃ}

\begin{paritta}
Vandāmi cetiyaṃ sabbaṃ\\
Sabba-ṭhānesu patiṭṭhitaṃ\\
Sārīrīka-dhātu-Mahā-bodhiṃ\\
Buddha-rūpaṃ sakalaṃ sadā.
\end{paritta}

\subsection{Salutation to the Bodhi-Tree}

\firstline{Yassa mūle nissino va sabbāri vijayaṃ akā}

\begin{twochants}
Yassa mūle nissino va & sabbāri vijayaṃ akā,\\
Patto sabbaññutaṃ Satthā & vande taṃ Bodhi-pādapaṃ.\\
Ime ete Mahā-Bodhi & loka-nāthena pūjitā,\\
Aham-pi te namassāmi & bodhi-Rājā nam'atthu te!
\end{twochants}

\subsection{Offering of Lights}

\firstline{Ghana-sārappadittena}

\begin{paritta}
Ghana-sārappadittena\\
Dīpena tama-dhaṃsinā\\
Tīloka-dīpam sambuddhaṃ\\
Pūjayāmi tamo-nudaṃ.
\end{paritta}

\subsection{Offering of Incense}

\firstline{Gandha-sambhāra-yuttena}

\begin{paritta}
Gandha-sambhāra-yuttena\\
Dhūpenāhaṃ sugandhinā\\
Pūjaye pūjaneyyan-taṃ\\
Pūjā-bhājanam-uttamaṃ.
\end{paritta}

\subsection{Offering of Flowers}

\firstline{Vaṇṇa-gandha-guṇopetaṃ}

\begin{paritta}
Vaṇṇa-gandha-guṇopetaṃ\\
Etaṃ kusuma-santatiṃ.\\
Pūjayāmi munindassa\\
Sirīpāda-saroruhe.\\
Pūjemi Buddhaṃ kusumena'nena\\
Puññenam-etena ca hotu mokkhaṃ\\
Pupphaṃ milāyāti yathā idaṃ me\\
Kāyo tathā yāti vināsa-bhāvaṃ.
\end{paritta}

\subsection{Transference of Merit to Devas}

\firstline{Ākāsatthā ca bhummatthā}

\begin{paritta}
Ākāsatthā ca bhummatthā\\
Devā nāgā mah'iddhikā\\
Puññaṃ taṃ anumoditvā\\
Ciraṃ rakkhantu /loka/ sāsanaṃ\\
Ciraṃ rakkhantu desanaṃ\\
Ciraṃ rakkhantu maṃ paraṃ
\end{paritta}

\begin{paritta}
Ettāvatā ca amhehi\\
Sambhataṃ puñña-sampadaṃ\\
Sabbe devā/ bhūtā/ sattā anumodantu\\
Sabba-sampatti siddhiyā.
\end{paritta}

\subsection{Blessing to the World}

\firstline{Devo vassatu kālena}

\begin{paritta}
Devo vassatu kālena\\
Sassa-sampatti-hetu ca\\
Phīto bhavatu loko ca\\
Rajā bhavatu dhammiko.
\end{paritta}

\subsection{Transference of Merits to Departed Ones}

\firstline{Idaṃ te/vo/no/me ñātīnaṃ hotu}

\begin{paritta}
  Idaṃ te/vo/no/me ñātīnaṃ hotu\\
  sukhitā hontu ñātayo. (×3)

  \instr{(When chanting for one person use `te'; when for laypeople use `vo';
    when chanting together in a group use `no'; when alone use `me'.)}
\end{paritta}

\subsection{The Aspirations}

\firstline{Iminā puññakammena mā me bāla-samāgamo}

\begin{twochants}
Iminā puññakammena & mā me bāla-samāgamo,\\
Sataṃ samāgamo hotu, & yāva nibbāna-pattiyā.\\
Kāyena vācā-cittena & pamādena mayā kataṃ\\
Accayaṃ khama me bhante & bhūri-pañña Tathāgata.
\end{twochants}

\subsection{Blessing and Protection}

\firstline{Sabb'ītiyo vivajjantu sabba-rogo vinassatu}

\begin{twochants}
Sabb'ītiyo vivajjantu & sabba-rogo vinassatu;\\
Mā me/no bhavatvantarāyo & sukhī dīghāyuko/ā bhava/homa.\\
Bhavatu sabba-maṅgalaṃ & rakkhantu sabba-devatā.\\
Sabba-buddhānubhāvena & sadā sotthi bhavantu me.\\
Bhavatu sabba-maṅgalaṃ & rakkhantu sabba-devatā.\\
Sabba-dhammānunbhāvena & sadā sotthi bhavantu me.\\
Bhavatu sabba-maṅgalaṃ. & rakkhantu sabba-devatā.\\
Sabba-saṅghānubhāvena, & sadā sotthi bhavantu me.\\
Nakkhatta-yakkha-bhūtānaṃ & pāpaggaha-nivāraṇā\\
Parittassānubhāvena & hantvā mayhaṃ/amhe upaddave.\\
Devo vassatu kālena. & sassa-sampatti-hetu ca.\\
Phīto bhavatu loko ca. & rājā bhavatu dhammiko.\\
Sabbe buddhā balappattā, & paccekānañca yaṃ balaṃ\\
Arahantānañca tejena, & rakkhaṃ bandhāmi sabbaso.
\end{twochants}

\subsection{Mettā Bhāvanā}

\firstline{Attūpamāya sabbesaṃ sattānaṃ sukhakāmataṃ}

\begin{twochants}
Attūpamāya sabbesaṃ & sattānaṃ sukhakāmataṃ,\\
Passitvā kamato mettaṃ & sabbasattesu bhāvaye.\\
Sukhi bhaveyyaṃ niddukkho & ahaṃ niccaṃ ahaṃ viya,\\
Hitā ca me sukhī hontu & majjhatthā c'atha verino.\\
Imamhi gāmakkhettamhi & sattā hontu sukhī sadā,\\
Tato parañ ca-rajjesu & cakkavāḷesu jantuno.\\
\end{twochants}

\begin{twochants}
Samantā cakkavāḷesu & sattānan-tesu pāṇino,\\
Sukhino puggala bhūtā & attabhāvagatā siyuṃ.\\
Tathā itthī pumā ce'va & ariya anariya’ pi ca,\\
Devā narā apāyaṭṭhā & tathā dasa disāsu cā-ti.\\
\end{twochants}

\subsection{Pattanumodana (Sharing Merits)}

\firstline{Idaṃ te/vo/no/me ñātīnaṃ hotu}

Idaṃ te/vo/no/me ñātīnaṃ hotu\\
Sukhitā hontu ñātayo (×3)

\begin{twochants}
Yathā vāri-vahā pūrā & paripūrenti sāgaraṃ,\\
Evaṃ eva ito dinnaṃ & petānaṃ upakappatu.\\
Unname udakaṃ vattaṃ & yathā ninnaṃ pavattati,\\
Evaṃ eva ito dinnaṃ & petānaṃ upakappatu.\\
Āyūr-arogya-sampatti & sagga-sampattiṃ eva ca,\\
Atho nibbāna-sampatti & iminā te/vo/no/me samijjhatu.\\
Icchitaṃ patthitaṃ tuyhaṃ & sabbam-eva samijjhatu,\\
Pūrentu citta-saṅkappā & maṇi-joti-raso yathā.\\
Icchitaṃ patthitaṃ tuyhaṃ & sabbam-eva samijjhatu,\\
Pūrentu citta-saṅkappā & cando paṇṇa-rasī yathā.\\
Icchitaṃ patthitaṃ tuyhaṃ & khippam-eva samijjhatu,\\
Sabbe pūrentu saṅkappā & cando paṇṇa-rasī yathā.
\end{twochants}

\suttaRef{Petavatthu p.19-31 \& KhpA. 206-215}

\subsection{Greeting Used in Sri Lanka}

(FIXME placeholder)

\section{Offences}

\subsection{Āpatti-paṭidesanā (Confession of Offences)}

\subsubsection{Method of confessing light offences}

\hangindent=25pt%
\parbox{22pt}{\prul{JCB:}} Okāsa, ahaṃ bhante, sabbā āpattiyo ārocemi.\\
Dutiyam-pi ahaṃ bhante, sabbā āpattiyo ārocemi.\\
Tatiyam-pi ahaṃ bhante, sabbā āpattiyo ārocemi.\\
\emph{I ven. sir, declare all offences. For the second time… For the third time…}

\hangindent=25pt%
\parbox{22pt}{\prul{SAB:}} Sādhu, sādhu.\\ \emph{It is good, it is good.}

\hangindent=25pt%
\parbox{22pt}{\prul{JCB:}} Okāsa ahaṃ bhante, sambahulā nānā-vatthukā āpattiyo āpajjiṃ, tā tumha-mūle paṭidesemi.\\ \emph{I, ven. sir, having many times fallen into many different offences with different bases, these I confess.}

\hangindent=25pt%
\parbox{22pt}{\prul{SAB:}} Passasi āvuso tā āpattiyo?\\ \emph{Do you see, friend, those offences?}

\hangindent=25pt%
\parbox{22pt}{\prul{JCB:}} Āma bhante passāmi.\\ \emph{Yes, ven. sir, I see.}

\hangindent=25pt%
\parbox{22pt}{\prul{SAB:}} Āyatiṃ āvuso saṃvareyyāsi.\\ \emph{In the future, friend, you should be restrained.}

\hangindent=25pt%
\parbox{22pt}{\prul{JCB:}} Sādhu suṭṭhu bhante āyatiṃ saṃvarissāmi.\\
Dutiyam-pi sādhu suṭṭhu bhante āyatiṃ saṃvarissāmi.\\
Tatiyam-pi sādhu suṭṭhu bhante āyatiṃ saṃvarissāmi.\\
\emph{It is well indeed, ven. sir, in future I shall be restrained. For the second time…For the third time…}

\hangindent=25pt%
\parbox{22pt}{\prul{SAB:}} Sādhu, sādhu.\\ \emph{It is good, it is good.}

\hangindent=25pt%
\parbox{22pt}{\prul{JCB:}} Okāsa ahaṃ bhante,\\
sabbā tā garukāpattiyo āvikaromi.\\
Dutiyam-pi okāsa ahaṃ bhante,\\
sabbā tā garukāpattiyo āvikaromi.\\
Tatiyam-pi okāsa ahaṃ bhante,\\
sabbā tā garukāpattiyo āvikaromi.\\
\emph{Ven. sir, I reveal all heavy offences. For the second time… For the third time…}

This final declaration is only used in some communities. Also, some communities
will acknowledge with a ‘\emph{Sādhu}’ after each declaration rather than as
shown above. That is, after each ‘\emph{ārocemi}’ and each
‘\emph{saṃvarissāmi}’.

\subsubsection{Formula for same base offences}

\hangindent=25pt%
\parbox{22pt}{\prul{JCB:}} Okāsa ahaṃ bhante, desanādukkaṭāpattiṃ āpajjiṃ, taṃ tumha-mūle paṭidesemi.\\ \emph{I, ven. sir, confess an offence of wrong-doing through having confessed the same-based offences.}

\hangindent=25pt%
\parbox{22pt}{\prul{SAB:}} Passasi āvuso taṃ āpaṭṭiṃ?\\ \emph{Do you see, friend, that offence?}

\hangindent=25pt%
\parbox{22pt}{\prul{JCB:}} Āma bhante passāmi.\\ \emph{Yes, ven. sir, I see.}

\hangindent=25pt%
\parbox{22pt}{\prul{SAB:}} Āyatiṃ āvuso saṃvareyyāsi.\\ \emph{In the future, friend, you should be restrained.}

\hangindent=25pt%
\parbox{22pt}{\prul{JCB:}} Sādhu suṭṭhu bhante āyatiṃ saṃvarissāmi. Dutiyam-pi sādhu suṭṭhu … . Tatiyam-pi … saṃvarissāmi.\\ \emph{It is well indeed, ven. sir, in future I shall be restrained. For the second time… For the third time…}

\hangindent=25pt%
\parbox{22pt}{\prul{SAB:}} Sādhu, sādhu.\\ \emph{It is good, it is good.} \suttaRef{Vin.II.102}

\section{Rains and Kathina}

\subsection{Entering the Rains}

‘Imasmiṃ vihāre imaṃ te-māsaṃ vassaṃ upemi. Idha vassaṃ upemi.’\\
‘\emph{I enter the Rains in this kuṭi for three months. I enter the Rains
  here.}’

\section{Uposatha-day for Sāmaṇeras and Lay-followers}

\subsection{Eight Precepts}

With hands in \emph{añjali}, the laypeople recite the following request:

‘Sādhu! Sādhu! Sādhu! Okāsa ahaṃ bhante ti-saraṇena saddhiṃ aṭṭh'aṅga sīlaṃ
dhammaṃ yācāmi, anuggahaṃ katvā sīlaṃ detha me bhante. Dutiyam-pi okāsa… detha
me bhante. Tatiyam-pi okāsa… detha me bhante.’

\emph{Bhk}: ‘Yaṃ ahaṃ vadāmi taṃ vadetha.’

\emph{Laypeople}: ‘Āma, bhante.’

\emph{Bhk}: ‘Namo…’ (×3)

\emph{Laypeople}: repeat.

\emph{Bhk}:

‘Buddhaṃ saraṇaṃ gacchāmi.\\
Dhammaṃ saraṇaṃ gacchāmi.\\
Saṅghaṃ saraṇaṃ gacchāmi.\\
Dutiyam-pi Buddhaṃ saraṇaṃ gacchāmi.\\
Dutiyam-pi Dhammaṃ saraṇaṃ gacchāmi.\\
Dutiyam-pi Saṅghaṃ saraṇaṃ gacchāmi.\\
Tatiyam-pi Buddhaṃ saraṇaṃ gacchāmi.\\
Tatiyam-pi Dhammaṃ saraṇaṃ gacchāmi.\\
Tatiyam-pi Saṅghaṃ saraṇaṃ gacchāmi.’

\emph{Laypeople}: repeat line by line.

\emph{Bhk}: ‘Saraṇagamanaṃ sampuṇṇaṃ.’

\emph{Laypeople}: ‘Āma, bhante.’

Then the bhikkhu recites, with the laypeople repeating line by line:

‘Pāṇātipātā veramaṇī sikkhā-padaṃ samādiyāmi.\\
Adinnādānā veramaṇī sikkhā-padaṃ samādiyāmi.\\
Abrahma-cariyā veramaṇī sikkhā-padaṃ samādiyāmi.\\
Musāvādā veramaṇī sikkhā-padaṃ samādiyāmi.\\
Surā-meraya-majja-pamādaṭṭhānā veramaṇī sikkhā-padaṃ samādiyāmi.\\
Vikāla-bhojanā veramaṇī sikkhā-padaṃ samādiyāmi.\\
Nacca-gīta vādita visūka-dassana mālāgandha vilepana dhāraṇa maṇḍana
vibhūsanaṭṭhānā veramaṇī sikkhā-padaṃ samādiyāmi.\\
Uccā-sayana mahā-sayanā veramaṇī sikkhā-padaṃ samādiyāmi.’

\suttaRef{A.IV.248–250}

{\itshape

  ‘I undertake the precept to refrain from:

  \begin{packeditemize}

  \item destroying living beings.
  \item taking that which is not given.
  \item any kind of intentional sexual behaviour.
  \item false speech.
  \item intoxicating drinks and drugs that lead to carelessness.
  \item eating at wrong times.
  \item dancing, singing, music and going to entertainments.
  \item perfumes, beautification and adornment.
  \item lying on a high or luxurious sleeping place.
  \item accepting gold or silver.’

  \end{packeditemize}

}

\emph{Bhk}: ‘Imaṃ aṭṭh'aṅga-sīlaṃ samādiyāmi.’

\emph{Laypeople}: ‘Imaṃ aṭṭh'aṅga-sīlaṃ samādiyāmi.’ (×3)

\emph{Bhk}: ‘Ti-saraṇena saddhiṃ aṭṭh'aṅga-sīlaṃ dhammaṃ sādhukaṃ surakkhitaṃ
katvā appamādena sampādetha.’

\emph{Laypeople}: ‘Āma, bhante.’

\emph{Bhk}:

‘Sīlena sugatiṃ yanti,\\
Sīlena bhoga-sampadā,\\
Sīlena nibbutiṃ yanti,\\
Tasmā sīlaṃ visodhaye.’

‘\emph{These Eight Precepts\\
  Have morality as a vehicle for happiness,\\
  Have morality as a vehicle for good fortune,\\
  Have morality as a vehicle for liberation,\\
  Let morality therefore be purified.}’

The Laypeople may respond with:

‘Sādhu, sādhu, sādhu!’

\subsection{Five Precepts}

With hands in \emph{añjali}, the laypeople recite the following request:

‘Sādhu! Sādhu! Sādhu! Okāsa ahaṃ bhante tisaraṇena saddhiṃ pañca-sīlaṃ dhammaṃ
yācāmi, anuggahaṃ katvā sīlaṃ detha me bhante. Dutiyam-pi okāsa… Tatiyam-pi
okāsa…’

\emph{Bhikkhu}: ‘Yaṃ ahaṃ vadāmi taṃ vadetha.’

\emph{Laypeople}: ‘Āma, bhante.’

\emph{Bhk}: ‘Namo…’ (×3)

\emph{Laypeople}: repeat.

\emph{Bhk}: ‘Saraṇagamanaṃ sampuṇṇaṃ.’

\emph{Laypeople}: ‘Āma, bhante.’

Then the bhikkhu recites, with the laypeople repeating line by line:

‘Pāṇātipātā veramaṇī sikkhā-padaṃ samādiyāmi.\\
Adinnādānā veramaṇī sikkhā-padaṃ samādiyāmi.\\
Kāmesu micchā-cārā veramaṇī sikkhā-padaṃ samādiyāmi.\\
Musā-vādā veramaṇī sikkhā-padaṃ samādiyāmi.\\
Surā-meraya-majja-pamādaṭṭhānā veramaṇī sikkhā-padaṃ samādiyāmi.’

\suttaRef{A.IV.248–250}

\emph{Bhk}:

‘Tisaraṇena saddhiṃ pañcasīlaṃ dhammaṃ sādhukaṃ surakkhitaṃ katvā appamādena
sampādetha.’

\emph{Laypeople}: ‘Āma, bhante.’

\emph{Bhk}:

‘Sīlena sugatiṃ yanti\\
Sīlena bhoga-sampadā,\\
Sīlena nibbutiṃ yanti,\\
Tasmā sīlaṃ visodhaye.’

