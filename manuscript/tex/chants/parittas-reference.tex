\chapter{Paritta Chants}

\section{Thai Tradition}

Paritta chanting ceremonies in Thailand vary regionally but may be outlined as:

\begin{packeditemize}
  \item a layperson chants the invitation for paritta chanting
  \item the third bhikkhu or nun in seniority chants the invitation to the devas
  \item the introductory chants are chanted
  \item the core sequence of paritta chants follow
  \item the closing chants end the ceremony.
\end{packeditemize}

The third introductory chant in the Mahānikāya sect is commonly \emph{Sambuddhe}.
In Dhammayut circles and frequently in the forest tradition, the third chant is
\emph{Yo cakkhumā} instead.

There is a shorter and longer traditional core sequence. The \emph{jet tamnaan}
(\thai{เจ็ดตํานาน}) contains D1-D7 as below, the \emph{sipsong tamnaan}
(\thai{สิบสองตํานาน}) contains S1-S12. Chants that are not numbered `D' or `S' can
be included or not, as wished, but should be recited in the order listed here.

\clearpage

{\centering
\fontsize{12.5}{16}\selectfont

\begin{tabular}{@{}l l r r@{}}
  & first line & & page \\
  \hline
  i1  & Namo tassa & & \pageref{namo-tassa} \\
  i2  & Buddhaṁ saraṇaṁ gacchāmi & & \pageref{buddham-saranam} \\
  i3/a  & Sambuddhe aṭṭhavīsañca & & \pageref{sambuddhe} \\
  i3/b  & Yo cakkhumā & & \pageref{yo-cakkhuma} \\
  i4  & Namo arahato & & \pageref{namo-arahato} \\
      & & & \\
  D1 & Asevanā ca bālānaṁ & S1 & \pageref{asevana} \\
  D2 & Yaṅkiñci vittaṁ & S2 & \pageref{yankinci-vittam} \\
  D3 & Karaṇīyam-attha-kusalena & S3 & \pageref{karaniyam-attha} \\
  D4 & Virūpakkhehi me mettaṁ & S4 & \pageref{virupakkhehi} \\
  & Vadhissamenanti parāmasanto & & \pageref{vadhissamenanti} \\
  D5 & Udet'ayañ-cakkhumā eka-rājā & S5 & \pageref{udetayan-cakkhuma} \\
  & Atthi loke sīla-guṇo & S6 & \pageref{atthi-loke} \\
  D6 & Iti pi so bhagavā & S7 & \pageref{iti-pi-so} \\
  D7 & Vipassissa nam'atthu & S8 & \pageref{vipassissa} \\
  & Natthi me saraṇaṁ aññaṁ & & \pageref{natthi-me} \\
  & Yaṅkiñci ratanaṁ loke & & \pageref{yankinci-ratanam} \\
  & Sakkatvā buddharatanaṁ & & \pageref{sakkatva} \\
  & Yato'haṁ bhagini & S9 & \pageref{yato-ham-bhagini} \\
  & Bojjh'aṅgo sati-saṅkhāto & S10 & \pageref{bojjhango} \\
  & Yan-dunnimittaṁ & S11 & \pageref{yan-dunnimittam} \\
      & & & \\
  & Dukkhappattā ca niddukkhā & & \pageref{dukkhappatta} \\
  & Bāhuṁ sahassam-abhinimmita & & \pageref{bahum} \\
  & Mahā-kāruṇiko nātho & S12 & \pageref{maha-karuniko} \\
  & Te attha-laddhā sukhitā & & \pageref{te-attha-laddha} \\
  & Bhavatu sabba-maṅgalaṁ & & \pageref{bhavatu} \\
\end{tabular}

}

\subsection*{Notes for Particular Chants}

\textbf{Asevanā ca bālānaṁ:} The candles on the shrine during a house invitation
are lit by the senior bhikkhu or nun at \emph{Asevanā}.

\textbf{Yaṅkiñci vittaṁ:} The candles are put out at \emph{Nibbanti
  dhīrā yathā'yam padīpo}.

\textbf{Atthi loke sīla-guṇo:} On the occasion of blessing a new house, this
chant should be included, as it is traditionally considered protection against
fire.

\textbf{Yato'haṁ bhagini:} This chant is to be used for expectant mothers since
the time of the Buddha for the blessing and protection of the mother and child.
It is also a good occasion to chant it when receiving alms from a newly married
couple. Sangha members are encouraged to practise it.

\textbf{Dukkhappattā ca niddukkhā:} This is usually chanted as second to last
before \emph{Bhavatu sabba-maṅgalaṁ}. It is considered necessary to include it
whenever the devas have been invited at the beginning of the paritta chanting
as this chant contains a line inviting them to leave again.

\textbf{Bāhuṁ sahassam-abhinimmita:} This is is a popular later addition to the
present day standard chants. It is not listed in the \emph{jet tamnaan} or
\emph{sipsong tamnaan} sets. Yet these days it is frequently added just before
\emph{Mahā-kāruṇiko nātho}. On some occasions (e.g. public birthdays, jubilees,
inauguration ceremonies, etc.), it is an alternative, instead of chanting
\emph{jet tamnaan} or \emph{sipsong tamnaan}, to do a minimum sequence called
\emph{suat phorn phra} which contains only:

(1)~\emph{Namo Tassa},\\
(2)~\emph{Iti pi so bhagavā},\\
(3)~\emph{Bāhuṁ},\\
(4)~\emph{Mahā-kāruṇiko nātho}, and\\
(5)~\emph{Bhavatu sabba-maṅgalaṁ}.

In this minimal chanting sequence usually one does not invite the devas.

\textbf{Te attha-laddhā sukhitā:} This is sometimes inserted before closing with
\emph{Bhavatu sabba-maṅgalaṁ}, as a special well-wishing when the occasion has
to do with Buddhism in general (e.g. inauguration of a new abbot, or at the end
of an \emph{upasampadā}).

\section{Invitations}

\subsection{Invitation for Paritta Chanting}
\label{paritta-invitation-for-chanting}

\firstline{Vipatti-paṭibāhāya sabba-sampatti-siddhiyā}

\vspace*{5pt}

\enlargethispage{\baselineskip}

\begin{paritta}

\instr{(After bowing three times, with hands joined in añjali,\\
  recite the following)\par}

Vipatti-paṭibāhāya sabba-sampatti-siddhiyā\\
Sabbadukkha-vināsāya\\
Parittaṁ brūtha maṅgalaṁ

Vipatti-paṭibāhāya sabba-sampatti-siddhiyā\\
Sabbabhaya-vināsāya\\
Parittaṁ brūtha maṅgalaṁ

Vipatti-paṭibāhāya sabba-sampatti-siddhiyā\\
Sabbaroga-vināsāya\\
Parittaṁ brūtha maṅgalaṁ

\instr{(Bow three times)}
\end{paritta}

\begin{english}
  For warding off misfortune, for the arising of good fortune,\\
  For the dispelling of all dukkha,\\
  May you chant a blessing and protection.

  For warding off misfortune, for the arising of good fortune,\\
  For the dispelling of all fear,\\
  May you chant a blessing and protection.

  For warding off misfortune, for the arising of good fortune,\\
  For the dispelling of all sickness,\\
  May you chant a blessing and protection.
\end{english}

\subsection{Invitation to the Devas}
\label{paritta-devas}

\firstline{Pharitvāna mettaṁ samettā bhadantā}
\firstline{Samantā cakka-vāḷesu}
\firstline{Sarajjaṁ sasenaṁ sabandhuṁ nar'indaṁ}

\enlargethispage{\baselineskip}

In Thai custom, the third monk in seniority invites the devas, holding his
hands in \emph{añjali}, and lifting up the ceremonial string.

The string is wound up at the beginning of the last chant, \emph{Mahā-kāruṇiko
  nātho} or \emph{Bhavatu sabba-maṅgalaṁ}, which should be kept in mind by the
last bhikkhu or \emph{sāmaṇera}.

Before royal ceremonies, the invitation starts with A.

Before the shorter \emph{jet tamnaan} set of parittas, B is used and C is
omitted. Before the longer \emph{sipsong tamnaan} set of parittas, B is
omitted and C is used.

The verses at D are always chanted.

When chanting outside the monastery, the invitation is concluded with E. When
chanting at the monastery, the invitation is concluded with either E or F.

\clearpage

\begin{paritta}

\instr{(With hands joined in añjali, recite the following)}

\sidepar{A.}%
Sarajjaṁ sasenaṁ sabandhuṁ nar'indaṁ\\
Paritt'ānubhāvo sadā rakkhatū'ti

\sidepar{B.}%
Pharitvāna mettaṁ samettā bhadantā\\
Avikkhitta-cittā parittaṁ bhaṇantu

\sidepar{C.}%
Samantā cakka-vāḷesu\\
Atr'āgacchantu devatā\\
Saddhammaṁ muni-rājassa\\
Suṇantu sagga-mokkha-daṁ

\sidepar{D.}%
Sagge kāme ca rūpe\\
Giri-sikhara-taṭe c'antalikkhe vimāne\\
Dīpe raṭṭhe ca gāme\\
Taru-vana-gahane geha-vatthumhi khette\\
Bhummā c'āyantu devā\\
Jala-thala-visame yakkha-gandhabba-nāgā\\
Tiṭṭhantā santike yaṁ\\
Muni-vara-vacanaṁ sādhavo me suṇantu

\sidepar{E.}%
Dhammassavana-kālo ayam-bhadantā (×3)

\instr{Or, end with:}

\sidepar{F.}%
Buddha-dassana-kālo ayam-bhadantā\\
Dhammassavana-kālo ayam-bhadantā\\
Saṅgha-payirūpāsana-kālo ayam-bhadantā
\end{paritta}

\begin{english}
  Benevolent, venerable sirs: having spread thoughts of goodwill, listen to the
  chant with undistracted mind.

  From all around the ten-thousand world-systems, may the devas come here.\\
  May they listen to the True Dhamma of the King of Sages,\\
  leading to heaven and liberation.

  Those in the heavens of sensuality and form,\\
  on peaks and mountain precipices, in palaces floating in the sky,\\
  in islands, countries, and towns,\\
  in groves of trees and thickets, around home sites and fields.

  And the earth-devas, spirits, heavenly minstrels, and nagas\\
  in water, on land, in bad lands, and nearby:\\
  May they come and listen with approval\\
  as I recite the word of the excellent sage.

  This is the time to see the Buddha, venerable sirs.\\
  This is the time to listen to the Dhamma, venerable sirs.\\
  This is the time to attend to the Saṅgha, venerable sirs.
\end{english}

\clearpage

\section{Introductory Chants}

\subsection{Pubba-bhāga-nama-kāra-pāṭha}
\label{namo-tassa}

Namo tassa bhagavato arahato sammā-sambuddhassa\\
Namo tassa bhagavato arahato sammā-sambuddhassa\\
Namo tassa bhagavato arahato sammā-sambuddhassa

\subsection{Saraṇa-gamana-pāṭha}
\label{buddham-saranam}

\begin{paritta}
Buddhaṁ saraṇaṁ gacchāmi\\
Dhammaṁ saraṇaṁ gacchāmi\\
Saṅghaṁ saraṇaṁ gacchāmi

Dutiyam pi buddhaṁ saraṇaṁ gacchāmi\\
Dutiyam pi dhammaṁ saraṇaṁ gacchāmi\\
Dutiyam pi saṅghaṁ saraṇaṁ gacchāmi

Tatiyam pi buddhaṁ saraṇaṁ gacchāmi\\
Tatiyam pi dhammaṁ saraṇaṁ gacchāmi\\
Tatiyam pi saṅghaṁ saraṇaṁ gacchāmi
\end{paritta}

\subsection{Sambuddhe}
\label{sambuddhe}

\firstline{Sambuddhe aṭṭhavīsañca}

\begin{paritta}
Sambuddhe aṭṭhavīsañca\\
Dvādasañca sahassake\\
Pañca-sata-sahassāni\\
Namāmi sirasā ahaṁ

Tesaṁ dhammañca saṅghañca\\
Ādarena namāmihaṁ\\
Namakārānubhāvena\\
Hantvā sabbe upaddave\\
Anekā antarāyāpi\\
Vinassantu asesato

Sambuddhe pañca-paññāsañca\\
Catuvīsati sahassake\\
Dasa-sata-sahassāni\\
Namāmi sirasā ahaṁ

Tesaṁ dhammañca saṅghañca\\
Ādarena namāmihaṁ\\
Namakārānubhāvena\\
Hantvā sabbe upaddave\\
Anekā antarāyāpi\\
Vinassantu asesato

Sambuddhe navuttarasate\\
Aṭṭhacattāḷīsa sahassake\\
Vīsati-sata-sahassāni\\
Namāmi sirasā ahaṁ

Tesaṁ dhammañca saṅghañca\\
Ādarena namāmihaṁ\\
Namakārānubhāvena\\
Hantvā sabbe upaddave\\
Anekā antarāyāpi\\
Vinassantu asesato
\end{paritta}

\clearpage

\subsubsection{The Buddhas}

% English source: Bodhivana

I pay homage with my head to\\
the 512,028 Buddhas.

I pay devoted homage to their Dhamma and Saṅgha.\\
Through the power of this homage,\\
having demolished all misfortunes,\\
may countless dangers be destroyed without trace.

I pay homage with my head to\\
the 1,024,055 Buddhas.

I pay devoted homage to their Dhamma and Saṅgha.\\
Through the power of this homage,\\
having demolished all misfortunes,\\
may countless dangers be destroyed without trace.

I pay homage with my head to\\
the 2,048,109 Buddhas.

I pay devoted homage to their Dhamma and Saṅgha.\\
Through the power of this homage,\\
having demolished all misfortunes,\\
may countless dangers be destroyed without trace.

\subsection{Nama-kāra-siddhi-gāthā}
\label{yo-cakkhuma}

\firstline{Yo cakkhumā moha-malāpakaṭṭho}

\begin{paritta}
Yo cakkhumā moha-malāpakaṭṭho\\
Sāmaṁ va buddho sugato vimutto\\
Mārassa pāsā vinimocayanto\\
Pāpesi khemaṁ janataṁ vineyyaṁ\\
Buddhaṁ varan-taṁ sirasā namāmi\\
Lokassa nāthañ-ca vināyakañ-ca\\
Tan-tejasā te jaya-siddhi hotu\\
Sabb'antarāyā ca vināsamentu

Dhammo dhajo yo viya tassa satthu\\
Dassesi lokassa visuddhi-maggaṁ\\
Niyyāniko dhamma-dharassa dhārī\\
Sāt'āvaho santi-karo suciṇṇo\\
Dhammaṁ varan-taṁ sirasā namāmi\\
Mohappadālaṁ upasanta-dāhaṁ\\
Tan-tejasā te jaya-siddhi hotu\\
Sabb'antarāyā ca vināsamentu

Saddhamma-senā sugatānugo yo\\
Lokassa pāpūpakilesa-jetā\\
Santo sayaṁ santi-niyojako ca\\
Svākkhāta-dhammaṁ viditaṁ karoti\\
Saṅghaṁ varan-taṁ sirasā namāmi\\
Buddhānubuddhaṁ sama-sīla-diṭṭhiṁ\\
Tan-tejasā te jaya-siddhi hotu\\
Sabb'antarāyā ca vināsamentu
\end{paritta}

\subsubsection{The Verses of Success through Homage}

% English source: Bodhivana

The One with Vision, with the stain of delusion removed,\\
Self-awakened, Well-Gone, and Released.\\
Releasing them from the Māra's snare,\\
he leads humanity from evils to security.

\clearpage

I pay homage with my head to that excellent Buddha,\\
the Protector and Mentor for the world.\\
By the majesty of this, may you have triumph and success,\\
and may all your dangers be destroyed.

The Teacher's Dhamma, like a banner,\\
shows the path of purity to the world.\\
Leading out, upholding those who uphold it,\\
rightly accomplished, it brings pleasure, makes peace.

I pay homage with my head to that excellent Dhamma,\\
which pierces delusion and makes fever grow calm.\\
By the majesty of this, may you have triumph and success,\\
and may all your dangers be destroyed.

The True Dhamma's army, following the One Well-Gone,\\
is victor over the evils and corruptions of the world.\\
Self-calmed, it is calming and unfettering,\\
and makes the well-taught Dhamma be known.

I pay homage with my head to that excellent Saṅgha,\\
awakened following the Awakened One,\\\vin harmonious in virtue and view.\\
By the majesty of this, may you have triumph and success,\\
and may all your dangers be destroyed.

\subsection{Namo-kāra-aṭṭhaka}
\label{namo-arahato}

\firstline{Namo arahato sammā}

\begin{paritta}
  Namo arahato sammā\\
  Sambuddhassa mahesino\\
  Namo uttama-dhammassa\\
  Svākkhātass'eva ten'idha\\
  Namo mahā-saṅghassāpi\\
  Visuddha-sīla-diṭṭhino\\
  Namo omāty-āraddhassa\\
  Ratanattayassa sādhukaṁ\\
  Namo omakātītassa\\
  Tassa vatthuttayassa-pi\\
  Namo-kārappabhāvena\\
  Vigacchantu upaddavā\\
  Namo-kārānubhāvena\\
  Suvatthi hotu sabbadā\\
  Namo-kārassa tejena\\
  Vidhimhi homi tejavā
\end{paritta}

\subsubsection{The Homage Octet}

Homage to the Great Seer, the Worthy One, Rightly Self-awakened.

Homage to the highest Dhamma, well-taught by him here.

And homage to the Great Saṅgha, pure in virtue and view.

Homage to the Triple Gem beginning auspiciously with AUM.

And homage to those three objects that have left base things behind.

By the potency of this homage, may misfortunes disappear.

By the potency of this homage, may there always be well-being.

By the majesty of this homage, may I be successful in this ceremony.

\section{Core Sequence}

\subsection{Maṅgala-sutta}
\label{asevana}

[Evam-me sutaṁ: ekaṁ samayaṁ bhagavā, sāvatthiyaṁ viharati, jeta-vane
anāthapiṇḍikassa ārāme. Atha kho aññatarā devatā abhikkantāya rattiyā
abhikkanta-vaṇṇā kevala-kappaṁ jetavanaṁ obhāsetvā, yena bhagavā ten'upasaṅkami.
Upasaṅkamitvā bhagavantaṁ abhivādetvā ekam-antaṁ aṭṭhāsi. Ekam-antaṁ ṭhitā kho
sā devatā bhagavantaṁ gāthāya ajjhabhāsi:

Bahū devā manussā ca,\\
Maṅgalāni acintayuṁ;\\
Ākaṅkhamānā sotthānaṁ,\\
Brūhi maṅgalam-uttamaṁ.]

\bigskip

\firstline{Asevanā ca bālānaṁ}

\begin{paritta}
Asevanā ca bālānaṁ\\
Paṇḍitānañ-ca sevanā\\
Pūjā ca pūjanīyānaṁ\\
Etam maṅgalam-uttamaṁ

Paṭirūpa-desa-vāso ca\\
Pubbe ca kata-puññatā\\
Atta-sammā-paṇidhi ca\\
Etam maṅgalam-uttamaṁ

\clearpage

Bāhu-saccañ-ca sippañ-ca,\\
Vinayo ca susikkhito\\
Subhāsitā ca yā vācā\\
Etam maṅgalam-uttamaṁ

Mātā-pitu-upaṭṭhānaṁ\\
Putta-dārassa saṅgaho\\
Anākulā ca kammantā\\
Etam maṅgalam-uttamaṁ

Dānañ-ca dhamma-cariyā ca\\
Ñātakānañ-ca saṅgaho\\
Anavajjāni kammāni\\
Etam maṅgalam-uttamaṁ

Āratī viratī pāpā\\
Majja-pānā ca saññamo\\
Appamādo ca dhammesu\\
Etam maṅgalam-uttamaṁ

Gāravo ca nivāto ca\\
Santuṭṭhī ca kataññutā\\
Kālena dhammassavanaṁ\\
Etam maṅgalam-uttamaṁ

Khantī ca sovacassatā\\
Samaṇānañ-ca dassanaṁ\\
Kālena dhamma-sākacchā\\
Etam maṅgalam-uttamaṁ

\clearpage

Tapo ca brahma-cariyañ-ca\\
Ariya-saccāna-dassanaṁ\\
Nibbāna-sacchikiriyā ca\\
Etam maṅgalam-uttamaṁ

Phuṭṭhassa loka-dhammehi\\
Cittaṁ yassa na kampati\\
Asokaṁ virajaṁ khemaṁ\\
Etam maṅgalam-uttamaṁ

Etādisāni katvāna\\
Sabbattham-aparājitā\\
Sabbattha sotthiṁ gacchanti\\
Tan-tesaṁ maṅgalam-uttaman'ti
\end{paritta}

\suttaRef{Snp 2.4}

\subsubsection{The Thirty-Eight Highest Blessings}

\firstline{Thus have I heard that the Blessed One}

\bigskip

\begin{leader}
  [Now let us chant the verses on the Highest Blessings]
\end{leader}

[Thus have I heard that the Blessed One]\\
Was staying at Sāvatthī,\\
Residing at the Jeta's Grove\\
In Anāthapiṇḍika's Park.

Then in the dark of the night, a radiant deva\\
Illuminated all Jeta's Grove.\\
She bowed down low before the Blessed One\\
Then standing to one side she said:

\clearpage

`Devas are concerned for happiness\\
And ever long for peace.\\
The same is true for humankind.\\
What then are the highest blessings?'

\firstline{Avoiding those of foolish ways}

Avoiding those of foolish ways,\\
Associating with the wise,\\
And honouring those worthy of honour.\\
These are the highest blessings.

Living in places of suitable kinds,\\
With the fruits of past good deeds\\
And guided by the rightful way.\\
These are the highest blessings.

Accomplished in learning and craftsman's skills,\\
With discipline, highly trained,\\
And speech that is true and pleasant to hear.\\
These are the highest blessings.

Providing for mother and father's support\\
And cherishing family,\\
And ways of work that harm no being,\\
These are the highest blessings.

Generosity and a righteous life,\\
Offering help to relatives and kin,\\
And acting in ways that leave no blame.\\
These are the highest blessings.

Steadfast in restraint, and shunning evil ways,\\
Avoiding intoxicants that dull the mind,\\
And heedfulness in all things that arise.\\
These are the highest blessings.

Respectfulness and being of humble ways,\\
Contentment and gratitude,\\
And hearing the Dhamma frequently taught.\\
These are the highest blessings.

Patience and willingness to accept one's faults,\\
Seeing venerated seekers of the truth,\\
And sharing often the words of Dhamma.\\
These are the highest blessings.

Ardent, committed to the Holy Life,\\
Seeing for oneself the Noble Truths\\
And the realization of Nibbāna.\\
These are the highest blessings.

Although in contact with the world,\\
Unshaken the mind remains\\
Beyond all sorrow, spotless, secure.\\
These are the highest blessings.

They who live by following this path\\
Know victory wherever they go,\\
And every place for them is safe.\\
These are the highest blessings.

\suttaRef{Snp 2.4}

\clearpage

\subsection{Ratana-sutta}

\firstline{Yānīdha bhūtāni samāgatāni}

\instr{(In certain monasteries only the numbered verses are chanted.)}

\bigskip

\begin{paritta}

Yānīdha bhūtāni samāgatāni\\
Bhummāni vā yāni va antalikkhe\\
Sabb'eva bhūtā sumanā bhavantu\\
Atho pi sakkacca suṇantu bhāsitaṁ\\
Tasmā hi bhūtā nisāmetha sabbe\\
Mettaṁ karotha mānusiyā pajāya\\
Divā ca ratto ca haranti ye baliṁ\\
Tasmā hi ne rakkhatha appamattā

\firstline{Yaṅkiñci vittaṁ idha vā huraṁ vā}

\label{yankinci-vittam}
\sidepar{1.}%
Yaṅkiñci vittaṁ idha vā huraṁ vā\\
Saggesu vā yaṁ ratanaṁ paṇītaṁ\\
Na no samaṁ atthi tathāgatena\\
Idam-pi buddhe ratanaṁ paṇītaṁ\\
Etena saccena suvatthi hotu

\sidepar{2.}%
Khayaṁ virāgaṁ amataṁ paṇītaṁ\\
Yad-ajjhagā sakya-munī samāhito\\
Na tena dhammena sam'atthi kiñci\\
Idam-pi dhamme ratanaṁ paṇītaṁ\\
Etena saccena suvatthi hotu

\sidepar{3.}%
Yam buddha-seṭṭho parivaṇṇayī suciṁ\\
Samādhim-ānantarikaññam-āhu\\
Samādhinā tena samo na vijjati\\
Idam-pi dhamme ratanaṁ paṇītaṁ\\
Etena saccena suvatthi hotu

\sidepar{4.}%
Ye puggalā aṭṭha sataṁ pasaṭṭhā\\
Cattāri etāni yugāni honti\\
Te dakkhiṇeyyā sugatassa sāvakā\\
Etesu dinnāni mahapphalāni\\
Idam-pi saṅghe ratanaṁ paṇītaṁ\\
Etena saccena suvatthi hotu

\sidepar{5.}%
Ye suppayuttā manasā daḷhena\\
Nikkāmino gotama-sāsanamhi\\
Te patti-pattā amataṁ vigayha\\
Laddhā mudhā nibbutiṁ bhuñjamānā\\
Idam-pi saṅghe ratanaṁ paṇītaṁ\\
Etena saccena suvatthi hotu

Yath'inda-khīlo paṭhaviṁ sito siyā\\
Catubbhi vātebhi asampakampiyo\\
Tathūpamaṁ sappurisaṁ vadāmi\\
Yo ariya-saccāni avecca passati\\
Idam-pi Saṅghe ratanaṁ paṇītaṁ\\
Etena saccena suvatthi hotu

Ye ariya-saccāni vibhāvayanti\\
Gambhīra-paññena sudesitāni\\
Kiñ-cāpi te honti bhusappamattā\\
Na te bhavaṁ aṭṭhamam-ādiyanti\\
Idam-pi Saṅghe ratanaṁ paṇītaṁ\\
Etena saccena suvatthi hotu

Sahā v'assa dassana-sampadāya\\
Tay'assu dhammā jahitā bhavanti\\
Sakkāya-diṭṭhi vicikicchitañ-ca\\
Sīlabbataṁ vā pi yad-atthi kiñci\\
Catūh'apāyehi ca vippamutto\\
Cha cābhiṭhānāni abhabbo kātuṁ\\
Idam-pi Saṅghe ratanaṁ paṇītaṁ\\
Etena saccena suvatthi hotu

Kiñ-cāpi so kammaṁ karoti pāpakaṁ\\
Kāyena vācā uda cetasā vā\\
Abhabbo so tassa paṭicchādāya\\
Abhabbatā diṭṭha-padassa vuttā\\
Idam-pi Saṅghe ratanaṁ paṇītaṁ\\
Etena saccena suvatthi hotu

Vanappagumbe yathā phussitagge\\
Gimhāna-māse paṭhamasmiṁ gimhe\\
Tathūpamaṁ dhamma-varaṁ adesayi\\
Nibbāna-gāmiṁ paramaṁ hitāya\\
Idam-pi Buddhe ratanaṁ paṇītaṁ\\
Etena saccena suvatthi hotu

Varo varaññū varado var'āharo\\
Anuttaro dhamma-varaṁ adesayi\\
Idam-pi Buddhe ratanaṁ paṇītaṁ\\
Etena saccena suvatthi hotu

\sidepar{6.}%
Khīṇaṁ purāṇaṁ navaṁ n'atthi sambhavaṁ\\
Viratta-citt'āyatike bhavasmiṁ\\
Te khīṇa-bījā aviruḷhi-chandā\\
Nibbanti dhīrā yathā'yam padīpo\\
Idam-pi saṅghe ratanaṁ paṇītaṁ\\
Etena saccena suvatthi hotu.

Yānīdha bhūtāni samāgatāni\\
Bhummāni vā yāni va antalikkhe\\
Tathāgataṁ deva-manussa-pūjitaṁ\\
Buddhaṁ namassāma suvatthi hotu

Yānīdha bhūtāni samāgatāni\\
Bhummāni vā yāni va antalikkhe\\
Tathāgataṁ deva-manussa-pūjitaṁ\\
Dhammaṁ namassāma suvatthi hotu

Yānīdha bhūtāni samāgatāni\\
Bhummāni vā yāni va antalikkhe\\
Tathāgataṁ deva-manussa-pūjitaṁ\\
Saṅghaṁ namassāma suvatthi hotū'ti. \suttaRef{Snp 2.1}

\end{paritta}

\subsubsection{Verses from the Discourse on Treasures}

\instr{(The translations correspond to the numbered verses above.)}

\sidepar{1.}%
Whatever wealth in this world or the next,\\
whatever exquisite treasure in the heavens,\\
is not, for us, equal to the Tathāgata.\\
This, too, is an exquisite treasure in the Buddha.\\
By this truth may there be well-being.

\enlargethispage{\baselineskip}

\sidepar{2.}%
The exquisite Deathless -- dispassion, ending --\\
discovered by the Sakyan Sage while in concentration:\\
There is nothing equal to that Dhamma.\\
This, too, is an exquisite treasure in the Dhamma.\\
By this truth may there be well-being.

\clearpage

\sidepar{3.}%
What the excellent Awakened One extolled as pure\\
and called the concentration of unmediated knowing:\\
No equal to that concentration can be found.\\
This, too, is an exquisite treasure in the Dhamma.\\
By this truth may there be well-being.

\sidepar{4.}%
The eight persons -- the four pairs --\\
praised by those at peace:\\
They, disciples of the One Well-Gone, deserve offerings.\\
What is given to them bears great fruit.\\
This, too, is an exquisite treasure in the Saṅgha.\\
By this truth may there be well-being.

\sidepar{5.}%
Those who, devoted, firm-minded,\\
apply themselves to Gotama's message,\\
on attaining their goal, plunge into the Deathless,\\
freely enjoying the Unbinding they've gained.\\
This, too, is an exquisite treasure in the Saṅgha.\\
By this truth may there be well-being.

\sidepar{6.}%
Ended the old, there is no new taking birth.\\
Dispassioned their minds toward further becoming,\\
they -- with no seed, no desire for growth,\\
enlightened -- go out like this flame.\\
This, too, is an exquisite treasure in the Saṅgha.\\
By this truth may there be well-being.

\clearpage

\subsection{Karaṇīya-metta-sutta}
\label{karaniyam-attha}

\firstline{Karaṇīyam-attha-kusalena}

\begin{paritta}

Karaṇīyam-attha-kusalena\\
Yan-taṁ santaṁ padaṁ abhisamecca\\
Sakko ujū ca suhujū ca\\
Suvaco c'assa mudu anatimānī

Santussako ca subharo ca\\
Appakicco ca sallahuka-vutti\\
Sant'indriyo ca nipako ca\\
Appagabbho kulesu ananugiddho

Na ca khuddaṁ samācare kiñci\\
Yena viññū pare upavadeyyuṁ\\
Sukhino vā khemino hontu\\
Sabbe sattā bhavantu sukhit'attā

Ye keci pāṇa-bhūt'atthi\\
Tasā vā thāvarā vā anavasesā\\
Dīghā vā ye mahantā vā\\
Majjhimā rassakā aṇuka-thūlā

Diṭṭhā vā ye ca adiṭṭhā\\
Ye ca dūre vasanti avidūre\\
Bhūtā vā sambhavesī vā\\
Sabbe sattā bhavantu sukhit'attā

Na paro paraṁ nikubbetha\\
Nātimaññetha katthaci naṁ kiñci\\
Byārosanā paṭighasaññā\\
Nāññam-aññassa dukkham-iccheyya

Mātā yathā niyaṁ puttaṁ\\
Āyusā eka-puttam-anurakkhe\\
Evam'pi sabba-bhūtesu\\
Mānasam-bhāvaye aparimāṇaṁ

\subsubsection{Mettañ-ca sabba-lokasmiṁ}

\instr{(A shorter form is sometimes started here)}

\firstline{Mettañ-ca sabba-lokasmiṁ}

Mettañ-ca sabba-lokasmiṁ\\
Mānasam-bhāvaye aparimāṇaṁ\\
Uddhaṁ adho ca tiriyañ-ca\\
Asambādhaṁ averaṁ asapattaṁ

Tiṭṭhañ-caraṁ nisinno vā\\
Sayāno vā yāvat'assa vigata-middho\\
Etaṁ satiṁ adhiṭṭheyya\\
Brahmam-etaṁ vihāraṁ idham-āhu

Diṭṭhiñca anupagamma\\
Sīlavā dassanena sampanno\\
Kāmesu vineyya gedhaṁ\\
Na hi jātu gabbha-seyyaṁ punaretī'ti

\suttaRef{Snp 1.8}

\end{paritta}

\clearpage

\subsubsection{The Buddha's Words on Loving-Kindness}

\begin{leader}
  [Now let us chant the Buddha's words on loving-kindness]
\end{leader}

\firstline{This is what should be done}

[This is what should be done]\\
By one who is skilled in goodness\\
And who knows the path of peace:\\
Let them be able and upright,\\
Straightforward and gentle in speech,

Humble and not conceited,\\
Contented and easily satisfied,\\
Unburdened with duties and frugal in their ways.\\
Peaceful and calm, and wise and skilful,\\
Not proud and demanding in nature.

Let them not do the slightest thing\\
That the wise would later reprove,\\
Wishing: In gladness and in safety,\\
May all beings be at ease.

Whatever living beings there may be,\\
Whether they are weak or strong, omitting none,\\
The great or the mighty,\\
\vin medium, short, or small,\\
The seen and the unseen,\\
Those living near and far away,\\
Those born and to be born,\\
May all beings be at ease.

Let none deceive another\\
Or despise any being in any state.\\
Let none through anger or ill-will\\
Wish harm upon another.

Even as a mother protects with her life\\
Her child, her only child,\\
So with a boundless heart\\
Should one cherish all living beings,\\
Radiating kindness over the entire world:

Spreading upwards to the skies\\
And downwards to the depths,\\
Outwards and unbounded,\\
Freed from hatred and ill-will.

Whether standing or walking, seated, \\
Or lying down -- free from drowsiness --\\
One should sustain this recollection.\\
This is said to be the sublime abiding.

By not holding to fixed views,\\
The pure-hearted one, having clarity of vision,\\
Being freed from all sense-desires,\\
Is not born again into this world.

\suttaRef{Snp 1.8}

\subsection{Khandha-paritta}
\label{virupakkhehi}

\firstline{Virūpakkhehi me mettaṁ mettaṁ erāpathehi me}

Virūpakkhehi me mettaṁ\\\vin mettaṁ erāpathehi me\\
Chabyā-puttehi me mettaṁ\\\vin mettaṁ kaṇhā-gotamakehi ca\\
Apādakehi me mettaṁ\\\vin mettaṁ dipādakehi me\\
Catuppadehi me mettaṁ\\\vin mettaṁ bahuppadehi me\\
Mā maṁ apādako hiṁsi\\\vin mā maṁ hiṁsi dipādako\\
Mā maṁ catuppado hiṁsi\\\vin mā maṁ hiṁsi bahuppado\\
Sabbe sattā sabbe pāṇā\\\vin sabbe bhūtā ca kevalā\\
Sabbe bhadrāni passantu\\\vin mā kiñci pāpam-āgamā

\subsubsection{Appamāṇo buddho appamāṇo dhammo}

\instr{(This part is sometimes chanted on its own)}

\firstline{Appamāṇo buddho appamāṇo dhammo}

Appamāṇo buddho\\\vin appamāṇo dhammo\\\vin appamāṇo saṅgho\\
Pamāṇavantāni siriṁsapāni\\\vin ahi-vicchikā sata-padī\\
Uṇṇā-nābhī sarabhū mūsikā

Katā me rakkhā katā me parittā\\\vin paṭikkamantu bhūtāni\\
So'haṁ namo bhagavato\\\vin namo sattannaṁ\\\vin sammā-sambuddhānaṁ \suttaRef{A.II.72-73}

\subsubsection{The Group Protection}

I have goodwill for the Virupakkhas, the Erapathas,\\
goodwill for the Chabya descendants, and the Black Gotamakas.

I have goodwill for footless beings, two-footed beings,\\
goodwill for four-footed, and many-footed beings.

May footless beings, two-footed beings do me no harm.\\
May four-footed beings and many-footed beings do me no harm.

May all creatures, all breathing things, all beings\\\vin -- each and every one --\\
meet with good fortune. May none of them come to any evil.

Limitless is the Buddha, limitless the Dhamma,\\\vin limitless the Saṅgha.

There is a limit to creeping things -- snakes, scorpions, centipedes, spiders,
lizards and rats.

I have made this protection, I have made this spell.\\
May the beings depart.\\
I pay homage to the Blessed One,\\
homage to the seven Rightly Self-awakened Ones.

\subsection{Chaddanta-paritta}
\label{vadhissamenanti}

\englishTitle{The Great Elephant Protection}

\firstline{Vadhissamenanti parāmasanto}

\begin{paritta}
Vadhissamenanti parāmasanto\\
Kāsāvamaddakkhi dhajaṁ isīnaṁ\\
Dukkhena phuṭṭhassudapādi saññā\\
Arahaddhajo sabbhi avajjharūpo

Sallena viddho byathitopi santo\\
Kāsāvavatthamhi manaṁ na dussayi\\
Sace imaṁ nāgavarena saccaṁ\\
Mā maṁ vane bālamigā agañchunti
\end{paritta}

\subsection{Mora-paritta}
\label{udetayan-cakkhuma}

\firstline{Udet'ayañ-cakkhumā eka-rājā}
\firstline{Apet'ayañ-cakkhumā eka-rājā}

\enlargethispage{\baselineskip}

\vspace*{-.5\baselineskip}

\instr{(a.m.)}

\vspace*{-.2\baselineskip}

Udet'ayañ-cakkhumā eka-rājā\\
Harissa-vaṇṇo paṭhavippabhāso\\
Taṁ taṁ namassāmi harissa-vaṇṇaṁ paṭhavippabhāsaṁ\\
Tay'ajja guttā viharemu divasaṁ

Ye brāhmaṇā vedagu sabba-dhamme\\
Te me namo te ca maṁ pālayantu\\
Nam'atthu Buddhānaṁ nam'atthu bodhiyā\\
Namo vimuttānaṁ namo vimuttiyā\\
Imaṁ so parittaṁ katvā\\
Moro carati esanā'ti

\vspace*{-.1\baselineskip}

\instr{(p.m.)}

\vspace*{-.2\baselineskip}

Apet'ayañ-cakkhumā eka-rājā\\
Harissa-vaṇṇo paṭhavippabhāso\\
Taṁ taṁ namassāmi harissa-vaṇṇaṁ paṭhavippabhāsaṁ\\
Tay'ajja guttā viharemu rattiṁ

Ye brāhmaṇā vedagu sabba-dhamme\\
Te me namo te ca maṁ pālayantu\\
Nam'atthu Buddhānaṁ nam'atthu bodhiyā\\
Namo vimuttānaṁ namo vimuttiyā\\
Imaṁ so parittaṁ katvā\\
Moro vāsam-akappayī'ti \suttaRef{Ja.159}

\subsubsection{The Peacock's Protection}

The One King, rising, with Vision,\\
golden-hued, illuminating the Earth: I pay homage to you,\\
golden-hued, illuminating the Earth.\\
Guarded today by you, may I live through the day.

Those Brahmans who are knowers of all truths,\\
I pay homage to them; may they keep watch over me.\\
Homage to the Awakened Ones. Homage to Awakening.\\
Homage to the Released Ones. Homage to Release.

Having made this protection, the peacock sets out in search for food.

The One King, setting, with Vision,\\
golden-hued, illuminating the Earth: I pay homage to you,\\
golden-hued, illuminating the Earth.\\
Guarded today by you, may I live through the night.

Those Brahmans who are knowers of all truths,\\
I pay homage to them; may they keep watch over me.\\
Homage to the Awakened Ones. Homage to Awakening.\\
Homage to the Released Ones. Homage to Release.

Having made this protection, the peacock arranges his nest.

\subsection{Vaṭṭaka-paritta}
\label{atthi-loke}

\firstline{Atthi loke sīla-guṇo saccaṁ soceyy'anuddayā}

\begin{twochants}
Atthi loke sīla-guṇo & saccaṁ soceyy'anuddayā\\
Tena saccena kāhāmi & sacca-kiriyam-anuttaraṁ\\
Āvajjitvā dhamma-balaṁ & saritvā pubbake jine\\
Sacca-balam-avassāya & sacca-kiriyam-akās'ahaṁ\\
Santi pakkhā apattanā & santi pādā avañcanā\\
Mātā pitā ca nikkhantā & jāta-veda paṭikkama\\
Saha sacce kate mayhaṁ & mahā-pajjalito sikhī\\
Vajjesi soḷasa karīsāni & udakaṁ patvā yathā sikhī\\
Saccena me samo n'atthi & esā me sacca-pāramī'ti\\
\end{twochants}

\suttaRef{Cariyāpiṭaka vv.319-322}

\subsubsection{The Quail's Protection}

There is in this world the quality of virtue,\\
truth, purity, tenderness.\\
In accordance with this truth I will make\\
an unsurpassed vow of truth.

Sensing the strength of the Dhamma,\\
calling to mind the victors of the past,\\
in dependence on the strength of truth,\\
I made an unsurpassed vow of truth:

Here are wings with no feathers;\\
here are feet that can't walk.\\
My mother and father have left me.\\
Fire, go back!

When I made my vow with truth,\\
the great crested flames\\
avoided the sixteen acres around me\\
as if they had come to a body of water.\\
My truth has no equal:\\
Such is my perfection of truth.

\subsection{Buddha-dhamma-saṅgha-guṇā}
\label{iti-pi-so}

\firstline{Iti pi so bhagavā arahaṁ sammā-sambuddho}

\begin{paritta}
Iti pi so bhagavā arahaṁ sammā-sambuddho\\
Vijjā-caraṇa-sampanno sugato loka-vidū\\
Anuttaro purisa-damma-sārathi\\
Satthā devamanussānaṁ buddho bhagavā'ti

Svākkhāto bhagavatā dhammo sandiṭṭhiko\\
\vin akāliko ehi-passiko opanayiko\\
paccattaṁ veditabbo viññūhī'ti

Supaṭipanno bhagavato sāvaka-saṅgho\\
Uju-paṭipanno bhagavato sāvaka-saṅgho\\
Ñāya-paṭipanno bhagavato sāvaka-saṅgho\\
Sāmīci-paṭipanno bhagavato sāvaka-saṅgho\\
Yad-idaṁ cattāri purisa-yugāni aṭṭha purisa-puggalā\\
Esa bhagavato sāvaka-saṅgho\\
Āhuneyyo pāhuneyyo dakkhiṇeyyo añjali-karaṇīyo\\
Anuttaraṁ puññakkhettaṁ lokassā'ti
\end{paritta}

\subsection{Araññe rukkha-mūle vā}

\firstline{Araññe rukkha-mūle vā}

\begin{paritta}
Araññe rukkha-mūle vā\\
Suññāgāre va bhikkhavo\\
Anussaretha sambuddhaṁ\\
Bhayaṁ tumhāka no siyā\\
No ce buddhaṁ sareyyātha\\
Loka-jeṭṭhaṁ nar'āsabhaṁ\\
Atha dhammaṁ sareyyātha\\
Niyyānikaṁ sudesitaṁ\\
No ce dhammaṁ sareyyātha\\
Niyyānikaṁ sudesitaṁ\\
Atha saṅghaṁ sareyyātha\\
Puññakkhettaṁ anuttaraṁ\\
Evam-buddhaṁ sarantānaṁ\\
Dhammaṁ saṅghañ-ca bhikkhavo\\
Bhayaṁ vā chambhitattaṁ vā\\
Loma-haṁso na hessatī'ti. \suttaRef{S.I.219-220}
\end{paritta}

\subsection{Āṭānāṭiya-paritta (short)}
\label{vipassissa}

\firstline{Vipassissa nam'atthu cakkhumantassa sirīmato}

\begin{twochants}
Vipassissa nam'atthu & cakkhumantassa sirīmato\\
Sikhissa pi nam'atthu & sabba-bhūtānukampino\\
Vessabhussa nam'atthu & nhātakassa tapassino\\
Nam'atthu kakusandhassa & māra-senappamaddino\\
Konāgamanassa nam'atthu & brāhmaṇassa vusīmato\\
Kassapassa nam'atthu & vippamuttassa sabbadhi\\
Aṅgīrasassa nam'atthu & sakya-puttassa sirīmato\\
\end{twochants}

\begin{twochants}
Yo imaṁ dhammam-adesesi & sabba-dukkhāpanūdanaṁ\\
Ye cāpi nibbutā loke & yathā-bhūtaṁ vipassisuṁ\\
Te janā apisuṇā & mahantā vīta-sāradā\\
Hitaṁ deva-manussānaṁ & yaṁ namassanti gotamaṁ\\
Vijjā-caraṇa-sampannaṁ & mahantaṁ vīta-sāradaṁ\\
Vijjā-caraṇa-sampannaṁ & buddhaṁ vandāma gotaman'ti\\
\end{twochants}

\suttaRef{D.III.195-196}

\subsubsection{Homage to the Seven Past Buddhas}

% English source: Bodhivana

Homage to Vipassī, possessed of vision and splendor.\\
Homage to Sikhī, sympathetic to all beings.\\
Homage to Vesabhū, cleansed, austere.\\
Homage to Kakusandha, crusher of Māra's host.\\
Homage to Konāgamana, the Brahman who lived the life perfected.\\
Homage to Kassapa, everywhere released.\\
Homage to Aṅgīrasa, splendid son of the Sakyans,\\
Who taught this Dhamma -- the dispelling of all stress.\\
Those unbound in the world,\\\vin who have seen things as they have come to be,\\
Great Ones of gentle speech, thoroughly mature:\\
Even they pay homage to Gotama,\\\vin the benefit of human and heavenly beings,\\
consummate in knowledge and conduct,\\\vin the Great One, thoroughly mature.\\
We revere the Buddha Gotama,\\\vin consummate in knowledge and conduct.

\subsection{Sacca-kiriyā-gāthā}
\label{natthi-me}

\firstline{Natthi me saraṇaṁ aññaṁ}

Natthi me saraṇaṁ aññaṁ buddho me saraṇaṁ varaṁ\\
Etena sacca-vajjena sotthi te/me hotu sabbadā

Natthi me saraṇaṁ aññaṁ dhammo me saraṇaṁ varaṁ\\
Etena sacca-vajjena sotthi te/me hotu sabbadā

Natthi me saraṇaṁ aññaṁ saṅgho me saraṇaṁ varaṁ\\
Etena sacca-vajjena sotthi te/me hotu sabbadā

\subsection{Yaṅkiñci ratanaṁ loke}
\label{yankinci-ratanam}

\firstline{Yaṅkiñci ratanaṁ loke}

\begin{twochants}
  Yaṅkiñci ratanaṁ loke & vijjati vividhaṁ puthu\\
  Ratanaṁ buddhasamaṁ & natthi tasmā sotthī bhavantu te\\
  Yaṅkiñci ratanaṁ loke & vijjati vividhaṁ puthu\\
  Ratanaṁ dhammasamaṁ & natthi tasmā sotthī bhavantu te\\
  Yaṅkiñci ratanaṁ loke & vijjati vividhaṁ puthu\\
  Ratanaṁ saṅghasamaṁ & natthi tasmā sotthī bhavantu te\\
\end{twochants}

\subsection{Sakkatvā buddharatanaṁ}
\label{sakkatva}

\firstline{Sakkatvā buddharatanaṁ}

\begin{twochants}
  Sakkatvā buddharatanaṁ & osadhaṁ uttamaṁ varaṁ\\
  Hitaṁ devamanussānaṁ & buddhatejena sotthinā\\
  Nassantupaddavā sabbe & dukkhā vūpasamentu te\\
  Sakkatvā dhammaratanaṁ & osadhaṁ uttamaṁ varaṁ\\
  Pariḷāhūpasamanaṁ & dhammatejena sotthinā\\
  Nassantupaddavā sabbe & bhayā vūpasamentu te\\
\end{twochants}

\begin{twochants}
  Sakkatvā saṅgharatanaṁ & osadhaṁ uttamaṁ varaṁ\\
  Āhuneyyaṁ pāhuneyyaṁ & saṅghatejena sotthinā\\
  Nassantupaddavā sabbe & rogā vūpasamentu te\\
\end{twochants}

\bigskip

{\centering
  \instr{The \emph{jet tamnaan} sequence ends here\\ and continues with the closing sequence.}
\par}

\subsubsection{Having Revered}

% Sakkatvā buddharatanaṁ

% English source: Bodhivana

Having revered the jewel of the Buddha, the highest, most excellent medicine,
the welfare of human and heavenly beings: Through the Buddha's majesty and
safety, may all obstacles vanish. May your sufferings grow totally calm.

Having revered the jewel of the Dhamma, the highest, most excellent medicine,
the stiller of feverish passion: Through the Dhamma's majesty and safety, may
all obstacles vanish. May your fears grow totally calm.

Having revered the jewel of the Saṅgha, the highest, most excellent medicine,
worthy of gifts, worthy of hospitality: Through the Saṅgha's majesty and safety,
may all obstacles vanish. May your diseases grow totally calm.

\subsection{Aṅgulimāla-paritta}
\label{yato-ham-bhagini}

\firstline{Yato'haṁ bhagini ariyāya jātiyā jāto}

\begin{paritta}
Yato'haṁ bhagini ariyāya jātiyā jāto\\
Nābhijānāmi sañcicca pāṇaṁ jīvitā voropetā\\
Tena saccena sotthi te hotu sotthi gabbhassa \suttaRef{M.II.103}
\end{paritta}

\instr{(Three times)}

% English source: Bodhivana

\begin{english}
  Sister, since being born in the Noble Birth,\\
  I am not aware that I have intentionally deprived a being of life.\\
  By this truth may you be well,\\
  and so may the child in your womb.
\end{english}

\subsection{Bojjhaṅga-paritta}
\label{bojjhango}

\firstline{Bojjhaṅgo sati-saṅkhāto}

\begin{twochants}
Bojjhaṅgo sati-saṅkhāto & dhammānaṁ vicayo tathā\\
Viriyam-pīti-passaddhi & bojjhaṅgā ca tathā'pare\\
Samādh'upekkha-bojjhaṅgā & satt'ete sabba-dassinā\\
Muninā sammad-akkhātā & bhāvitā bahulīkatā\\
Saṁvattanti abhiññāya & nibbānāya ca bodhiyā\\
Etena sacca-vajjena & sotthi te hotu sabbadā\\
Ekasmiṁ samaye nātho & moggallānañ-ca kassapaṁ\\
Gilāne dukkhite disvā & bojjhaṅge satta desayi\\
Te ca taṁ abhinanditvā & rogā mucciṁsu taṅ-khaṇe\\
Etena sacca-vajjena & sotthi te hotu sabbadā\\
Ekadā dhamma-rājā pi & gelaññenābhipīḷito\\
Cundattherena tañ-ñeva & bhaṇāpetvāna sādaraṁ\\
Sammoditvā ca ābādhā & tamhā vuṭṭhāsi ṭhānaso\\
Etena sacca-vajjena & sotthi te hotu sabbadā\\
Pahīnā te ca ābādhā & tiṇṇannam-pi mahesinaṁ\\
Magg'āhata-kilesā va & pattānuppatti-dhammataṁ\\
Etena sacca-vajjena & sotthi te hotu sabbadā\\
\end{twochants}

\suttaRef{S.V.80f}

\subsubsection{The Factors of Awakening Protection}

% English source: Bodhivana

The factors for Awakening include: mindfulness, analysis of qualities,
persistence, rapture, and calm as factors for Awakening, plus concentration and
equanimity.

These seven, which the All-seeing Sage has rightly taught, when developed and
matured, bring about heightened knowledge, Unbinding and Awakening.

By the utterance of this truth, may you always be well.

At one time, our Protector -- seeing that Moggallāna and Kassapa were sick and
in pain -- taught them the seven factors for Awakening.

They, delighting in that, were instantly freed from their illness.

By the utterance of this truth, may you always be well.

Once, when the Dhamma King was afflicted with fever, he had the Elder Cunda
recite that very teaching with devotion. And as he approved, he rose up from
that disease.

By the utterance of this truth, may you always be well.

Those diseases were abandoned by the three great seers, just as defilements are
demolished by the Path in accordance with step-by-step attainment.

By the utterance of this truth, may you always be well.

\subsection{Abhaya-paritta}
\label{yan-dunnimittam}

\firstline{Yan-dunnimittaṁ avamaṅgalañ-ca}

\begin{paritta}
Yan-dunnimittaṁ avamaṅgalañ-ca\\
Yo cāmanāpo sakuṇassa saddo\\
Pāpaggaho dussupinaṁ akantaṁ\\
Buddhānubhāvena vināsamentu

Yan-dunnimittaṁ avamaṅgalañ-ca\\
Yo cāmanāpo sakuṇassa saddo\\
Pāpaggaho dussupinaṁ akantaṁ\\
Dhammānubhāvena vināsamentu

Yan-dunnimittaṁ avamaṅgalañ-ca\\
Yo cāmanāpo sakuṇassa saddo\\
Pāpaggaho dussupinaṁ akantaṁ\\
Saṅghānubhāvena vināsamentu
\end{paritta}

\bigskip

{\centering
  \instr{The \emph{sipsong tamnaan} sequence ends here\\ and continues with the closing sequence.}
\par}

\subsubsection{The Danger-free Protection}

% English source: Bodhivana

Whatever unlucky portents and ill omens,\\
and whatever distressing bird calls,\\
evil planets, upsetting nightmares:

By the Buddha's power may they be destroyed.

Whatever unlucky portents and ill omens,\\
and whatever distressing bird calls,\\
evil planets, upsetting nightmares:

By the Dhamma's power may they be destroyed.

Whatever unlucky portents and ill omens,\\
and whatever distressing bird calls,\\
evil planets, upsetting nightmares:

By the Saṅgha's power may they be destroyed.

\section{Closing Sequence}

\subsection{Devatā-uyyojana-gāthā}
\label{dukkhappatta}

\firstline{Dukkhappattā ca niddukkhā}
\firstline{Sabbe buddhā balappattā}

\begin{twochants}
Dukkhappattā ca niddukkhā & bhayappattā ca nibbhayā\\
Sokappattā ca nissokā & hontu sabbe pi pāṇino\\
Ettāvatā ca amhehi & sambhataṁ puñña-sampadaṁ\\
Sabbe devānumodantu & sabba-sampatti-siddhiyā\\
Dānaṁ dadantu saddhāya & sīlaṁ rakkhantu sabbadā\\
Bhāvanābhiratā hontu & gacchantu devatā-gatā\\\relax
[Sabbe buddhā] balappattā & paccekānañ-ca yaṁ balaṁ\\
Arahantānañ-ca tejena & rakkhaṁ bandhāmi sabbaso\\
\end{twochants}

\subsubsection{Verses on Sending Off the Devatā}

% English source: Bodhivana

May all beings: who have fallen into suffering be without suffering,\\
who have fallen into danger be without danger,\\
who have fallen into sorrow be without sorrow.

For the sake of all attainment and success, may all heavenly beings\\
rejoice in the extent to which we have gathered a consummation of merit.

May they give gifts with conviction, may they always maintain virtue.\\
May they delight in meditation. May they go to a heavenly destination.

From the strength attained by all the Buddhas,\\
the strength of the Private Buddhas,\\
by the majesty of the arahants,\\
I bind this protection all around.

\subsection{Jaya-maṅgala-aṭṭha-gāthā}
\label{bahum}

\firstline{Bāhuṁ sahassam-abhinimmita sāvudhan-taṁ}

\begin{paritta}
Bāhuṁ sahassam-abhinimmita sāvudhan-taṁ\\
Grīmekhalaṁ udita-ghora-sasena-māraṁ\\
Dān'ādi-dhamma-vidhinā jitavā mun'indo\\
Tan-tejasā bhavatu te jaya-maṅgalāni

Mārātirekam-abhiyujjhita-sabba-rattiṁ\\
Ghoram-pan'āḷavakam-akkhama-thaddha-yakkhaṁ\\
Khantī-sudanta-vidhinā jitavā mun'indo\\
Tan-tejasā bhavatu te jaya-maṅgalāni

Nāḷāgiriṁ gaja-varaṁ atimatta-bhūtaṁ\\
Dāv'aggi-cakkam-asanīva sudāruṇan-taṁ\\
Mett'ambu-seka-vidhinā jitavā mun'indo\\
Tan-tejasā bhavatu te jaya-maṅgalāni

Ukkhitta-khaggam-atihattha-sudāruṇan-taṁ\\
Dhāvan-ti-yojana-path'aṅguli- mālavantaṁ\\
Iddhī'bhisaṅkhata-mano jitavā mun'indo\\
Tan-tejasā bhavatu te jaya-maṅgalāni

\clearpage

Katvāna kaṭṭham-udaraṁ iva gabbhinīyā\\
Ciñcāya duṭṭha-vacanaṁ jana-kāya majjhe\\
Santena soma-vidhinā jitavā mun'indo\\
Tan-tejasā bhavatu te jaya-maṅgalāni

Saccaṁ vihāya-mati-saccaka-vāda-ketuṁ\\
Vādābhiropita-manaṁ ati-andha-bhūtaṁ\\
Paññā-padīpa-jalito jitavā mun'indo\\
Tan-tejasā bhavatu te jaya-maṅgalāni

Nandopananda-bhujagaṁ vibudhaṁ mah'iddhiṁ\\
Puttena thera-bhujagena damāpayanto\\
Iddhūpadesa-vidhinā jitavā mun'indo\\
Tan-tejasā bhavatu te jaya-maṅgalāni

Duggāha-diṭṭhi-bhujagena sudaṭṭha-hatthaṁ\\
Brahmaṁ visuddhi-jutim-iddhi-bakābhidhānaṁ\\
Ñāṇāgadena vidhinā jitavā mun'indo\\
Tan-tejasā bhavatu te jaya-maṅgalāni

Etā pi buddha-jaya-maṅgala-aṭṭha-gāthā\\
Yo vācano dina-dine saratem-atandī\\
Hitvān'aneka-vividhāni c'upaddavāni\\
Mokkhaṁ sukhaṁ adhigameyya naro sapañño
\end{paritta}

\subsubsection{Verses on the Buddha's Victories}

% English source: Bodhivana

Creating a form with a thousand arms, each equipped with a weapon,\\
Māra, on the elephant Girimekhala,\\
uttered a frightening roar together with his troops.\\
The Lord of Sages defeated him by means of such qualities as generosity:\\
By the majesty of this, may you have blessings of victory.

Even more frightful than Māra making war all night,\\
was Āḷavaka, the arrogant unstable ogre.\\
The Lord of Sages defeated him by means of well-trained endurance:\\
By the majesty of this, may you have blessings of victory.

Nāḷāgiri, the excellent elephant, when maddened,\\
was very horrific, like a forest fire, a flaming discus, a lightning bolt.\\
The Lord of Sages defeated him by sprinkling the water of goodwill:\\
By the majesty of this, may you have blessings of victory.

Very horrific, with a sword upraised in his expert hand,\\
Garlanded-with-Fingers ran three leages along the path.\\
The Lord of Sages defeated him with mind-fashioned marvels:\\
By the majesty of this, may you have blessings of victory.

Having made a wooden belly to appear pregnant,\\
Ciñcā made a lewd accusation in the midst of the gathering.\\
The Lord of Sages defeated her with peaceful, gracious means:\\
By the majesty of this, may you have blessings of victory.

Saccaka, whose provocative views had abandoned the truth,\\
his mind delighting in argument, had become thoroughly blind.\\
The Lord of Sages defeated him with the light of discernment:\\
By the majesty of this, may you have blessings of victory.

Nandopananda was a serpent with great power but wrong views.\\
The Lord of Sages defeated him by means of a display of marvels,\\
sending his son (Moggallāna), the serpent-elder, to tame him:\\
By the majesty of this, may you have blessings of victory.

His hands bound tight by the serpent of wrongly held views,\\
Baka, the Brahmā, thought himself pure in his radiance and power.\\
The Lord of Sages defeated him by means of his words of knowledge:
By the majesty of this, may you have blessings of victory.

These eight verses of the Buddha's blessings of victory:\\
Whatever person of discernment\\
recites or recalls them day after day without lapsing,\\
destroying all kinds of obstacles,\\
will attain liberation and happiness.

\subsection{Jaya-paritta}
\label{maha-karuniko}

\firstline{Mahā-kāruṇiko nātho hitāya sabba-pāṇinaṁ}

\begin{paritta}
  Mahā-kāruṇiko nātho\\
  Hitāya sabba-pāṇinaṁ\\
  Pūretvā pāramī sabbā\\
  Patto sambodhim-uttamaṁ\\
  Etena sacca-vajjena\\
  Hotu te jaya-maṅgalaṁ
\end{paritta}

\clearpage

\subsubsection{Jayanto bodhiyā mūle}

\instr{(This part is sometimes chanted on its own)}

\firstline{Jayanto bodhiyā mūle}

\bigskip

\begin{paritta}
  Jayanto bodhiyā mūle\\
  Sakyānaṁ nandi-vaḍḍhano\\
  Evaṁ tvaṁ vijayo hohi\\
  Jayassu jaya-maṅgale\\
  Aparājita-pallaṅke\\
  Sīse paṭhavi-pokkhare

  Abhiseke sabba-buddhānaṁ\\
  Aggappatto pamodati\\
  Sunakkhattaṁ sumaṅgalaṁ\\
  Supabhātaṁ suhuṭṭhitaṁ\\
  Sukhaṇo sumuhutto ca\\
  Suyiṭṭhaṁ brahma-cārisu

  Padakkhiṇaṁ kāya-kammaṁ\\
  Vācā-kammaṁ padakkhiṇaṁ\\
  Padakkhiṇaṁ mano-kammaṁ\\
  Paṇidhi te padakkhiṇā\\
  Padakkhiṇāni katvāna\\
  Labhant'atthe padakkhiṇe \suttaRef{A.I.294}
\end{paritta}

\subsubsection{Victory Protection}

% English source: Bodhivana

(The Buddha), our protector, with great compassion,\\
for the welfare of all beings,\\
having fulfilled all the perfections,\\
attained the highest self-awakening.\\
By the utterance of this truth,\\
may you have a blessing of victory.

Victorious at the foot of the Bodhi tree,\\
was he who increased the Sakyans' delight.\\
May you have the same sort of victory.\\
May you win blessings of victory.

At the head of the lotus leaf of the world\\
on the undefeated seat\\
consecrated by all the Buddhas,\\
he rejoiced in the utmost attainment.

A lucky star it is, a lucky blessing,\\
a lucky dawn, a lucky sacrifice,\\
a lucky instant, a lucky moment,\\
a lucky offering: i.e., a rightful bodily act\\
a rightful verbal act, a rightful mental act,\\
your rightful intentions\\
with regard to those who lead the holy life.\\
Doing these rightful things,
your rightful aims are achieved.

\subsection{So attha-laddho}

\firstline{So attha-laddho sukhito viruḷho buddha-sāsane}

\begin{twochants}
So attha-laddho sukhito & viruḷho buddha-sāsane\\
Arogo sukhito hohi & saha sabbehi ñātibhi (×3)\\
\end{twochants}

\begin{english}
  May he gain in his aims, be happy, and flourish in the Buddha's teachings. May
  you, together with all your relatives, be happy and free from disease.
\end{english}

\subsection{Sā attha-laddhā}

\firstline{Sā attha-laddhā sukhitā viruḷhā buddha-sāsane}

\begin{twochants}
Sā attha-laddhā sukhitā & viruḷhā buddha-sāsane\\
Arogā sukhitā hohi & saha sabbehi ñātibhi (×3)\\
\end{twochants}

\subsection{Te attha-laddhā sukhitā}
\label{te-attha-laddha}

\firstline{Te attha-laddhā sukhitā viruḷhā buddha-sāsane}

\begin{twochants}
Te attha-laddhā sukhitā & viruḷhā buddha-sāsane\\
Arogā sukhitā hotha & saha sabbehi ñātibhi (×3) \suttaRef{A.I.294}
\end{twochants}

\subsection{Bhavatu sabba-maṅgalaṁ}
\label{bhavatu}

\firstline{Bhavatu sabba-maṅgalaṁ}

Bhavatu sabba-maṅgalaṁ rakkhantu sabba-devatā\\
Sabba-buddhānubhāvena sadā sotthī bhavantu te

Bhavatu sabba-maṅgalaṁ rakkhantu sabba-devatā\\
Sabba-dhammānubhāvena sadā sotthī bhavantu te

Bhavatu sabba-maṅgalaṁ rakkhantu sabba-devatā\\
Sabba-saṅghānubhāvena sadā sotthī bhavantu te

\section{Mahā-kāruṇiko nātho'ti ādikā gāthā}

\firstline{Mahā-kāruṇiko nātho atthāya sabba-pāṇinaṁ}

\begin{paritta}
Mahā-kāruṇiko nātho\\
Atthāya sabba-pāṇinaṁ\\
Hitāya sabba-pāṇinaṁ\\
Sukhāya sabba-pāṇinaṁ

Pūretvā pāramī sabbā\\
Patto sambodhim-uttamaṁ\\
Etena sacca-vajjena\\
Mā hontu sabb'upaddavā
\end{paritta}

\clearpage

\section{Āṭānāṭiya-paritta (long)}

\begin{leader}
\soloinstr{(Solo introduction)}

\firstline{Appasannehi nāthassa sāsane sādhusammate}

\begin{solotwochants}
  Appasannehi nāthassa & sāsane sādhusammate\\
  Amanussehi caṇḍehi & sadā kibbisakāribhi\\
  Parisānañca-tassannam & ahiṁsāya ca guttiyā\\
  Yandesesi mahāvīro & parittan-tam bhaṇāma se\\
\end{solotwochants}
\end{leader}

\firstline{Namo me sabbabuddhānaṁ}

{\centering
  \instr{(If starting with \emph{Vipassissa\ldots}, continue below\\
    without the solo introduction)}
\par}

%\enlargethispage{\baselineskip}

\begin{twochants}
  [Namo me sabbabuddhānaṁ] & uppannānaṁ mahesinaṁ\\
  Taṇhaṅkaro mahāvīro & medhaṅkaro mahāyaso\\
  Saraṇaṅkaro lokahito & dīpaṅkaro jutindharo\\
  Koṇḍañño janapāmokkho & maṅgalo purisāsabho\\
  Sumano sumano dhīro & revato rativaḍḍhano\\
  Sobhito guṇasampanno & anomadassī januttamo\\
  Padumo lokapajjoto & nārado varasārathī\\
  Padumuttaro sattasāro & sumedho appaṭipuggalo\\
  Sujāto sabbalokaggo & piyadassī narāsabho\\
  Atthadassī kāruṇiko & dhammadassī tamonudo\\
  Siddhattho asamo loke & tisso ca vadataṁ varo\\
  Phusso ca varado buddho & vipassī ca anūpamo\\
  Sikhī sabbahito satthā & vessabhū sukhadāyako\\
  Kakusandho satthavāho & koṇāgamano raṇañjaho\\
  Kassapo sirisampanno & gotamo sakyapuṅgavo\\
  Ete caññe ca sambuddhā & anekasatakoṭayo\\
  Sabbe buddhā asamasamā & sabbe buddhā mahiddhikā\\
  Sabbe dasabalūpetā & vesārajjehupāgatā\\
\end{twochants}

\clearpage

\savenotes

\begin{twochants}
  Sabbe te paṭijānanti & āsabhaṇṭhānamuttamaṁ\\
  Sīhanādaṁ nadantete & parisāsu visāradā\\
  Brahmacakkaṁ pavattenti & loke appaṭivattiyaṁ\\
  Upetā buddhadhammehi & aṭṭhārasahi nāyakā\\
  Dvattiṁsa-lakkhaṇūpetā & sītyānubyañjanādharā\\
  Byāmappabhāya suppabhā & sabbe te munikuñjarā\\
  Buddhā sabbaññuno ete & sabbe khīṇāsavā jinā\\
  Mahappabhā mahātejā & mahāpaññā mahabbalā\\
  Mahākāruṇikā dhīrā & sabbesānaṁ sukhāvahā\\
  Dīpā nāthā patiṭṭhā & ca tāṇā leṇā ca pāṇinaṁ\\
  Gatī bandhū mahassāsā & saraṇā ca hitesino\\
  Sadevakassa lokassa & sabbe ete parāyanā\\
  Tesāhaṁ sirasā pāde & vandāmi purisuttame\\
  Vacasā manasā ceva & vandāmete tathāgate\\
  Sayane āsane ṭhāne & gamane cāpi sabbadā\\
  Sadā sukhena rakkhantu & buddhā santikarā tuvaṁ\\
  Tehi tvaṁ rakkhito santo & mutto sabbabhayena ca\\
  Sabba-rogavinimutto & sabba-santāpavajjito\\
  Sabba-veramatikkanto & nibbuto ca tuvaṁ bhava\\
  Tesaṁ saccena sīlena & khantimettābalena ca\\
  Tepi tumhe%
  \footnote{If chanting for oneself, change \textit{tumhe} to \textit{amhe} here and in the lines below.}
  anurakkhantu & ārogyena sukhena ca\\
  Puratthimasmiṁ disābhāge & santi bhūtā mahiddhikā\\
  Tepi tumhe anurakkhantu & ārogyena sukhena ca\\
\end{twochants}

\spewnotes

\clearpage

\firstline{Tesaṁ saccena sīlena khantimettābalena ca}

\begin{twochants}
  Dakkhiṇasmiṁ disābhāge & santi devā mahiddhikā\\
  Tepi tumhe anurakkhantu & ārogyena sukhena ca\\
  Pacchimasmiṁ disābhāge & santi nāgā mahiddhikā\\
  Tepi tumhe anurakkhantu & ārogyena sukhena ca\\
  Uttarasmiṁ disābhāge & santi yakkhā mahiddhikā\\
  Tepi tumhe anurakkhantu & ārogyena sukhena ca\\
  Purimadisaṁ dhataraṭṭho & dakkhiṇena viruḷhako\\
  Pacchimena virūpakkho & kuvero uttaraṁ disaṁ\\
  Cattāro te mahārājā & lokapālā yasassino\\
  Tepi tumhe anurakkhantu & ārogyena sukhena ca\\
  Ākāsaṭṭhā ca bhummaṭṭhā & devā nāgā mahiddhikā\\
  Tepi tumhe anurakkhantu & ārogyena sukhena ca\\
\end{twochants}

\subsubsection{Natthi me saraṇaṁ aññaṁ}

\vspace*{\parskip}

\firstline{Natthi me saraṇaṁ aññaṁ}

\savenotes

\begin{twochants}
  Natthi me saraṇaṁ aññaṁ & buddho me saraṇaṁ varaṁ\\
  Etena saccavajjena & hotu te%
  \footnote{If chanting for oneself, change \textit{te} to \textit{me} here and in the lines below.}
  jayamaṅgalaṁ\\
  Natthi me saraṇaṁ aññaṁ & dhammo me saraṇaṁ varaṁ\\
  Etena saccavajjena & hotu te jayamaṅgalaṁ\\
  Natthi me saraṇaṁ aññaṁ & saṅgho me saraṇaṁ varaṁ\\
  Etena saccavajjena & hotu te jayamaṅgalaṁ\\
\end{twochants}

\spewnotes

\subsubsection{Yaṅkiñci ratanaṁ loke}

\vspace*{\parskip}

\firstline{Yaṅkiñci ratanaṁ loke vijjati vividhaṁ puthu}

\begin{twochants}
  Yaṅkiñci ratanaṁ loke & vijjati vividhaṁ puthu\\
  Ratanaṁ buddhasamaṁ & natthi tasmā sotthī bhavantu te\\
  Yaṅkiñci ratanaṁ loke & vijjati vividhaṁ puthu\\
  Ratanaṁ dhammasamaṁ & natthi tasmā sotthī bhavantu te\\
  Yaṅkiñci ratanaṁ loke & vijjati vividhaṁ puthu\\
  Ratanaṁ saṅghasamaṁ & natthi tasmā sotthī bhavantu te\\
\end{twochants}

\subsubsection{Sakkatvā}

\vspace*{\parskip}

\firstline{Sakkatvā buddha-ratanaṁ osadhaṁ uttamaṁ varaṁ}

\begin{twochants}
  Sakkatvā buddharatanaṁ & osadhaṁ uttamaṁ varaṁ\\
  Hitaṁ devamanussānaṁ & buddhatejena sotthinā\\
  Nassantupaddavā sabbe & dukkhā vūpasamentu te\\
  Sakkatvā dhammaratanaṁ & osadhaṁ uttamaṁ varaṁ\\
  Pariḷāhūpasamanaṁ & dhammatejena sotthinā\\
  Nassantupaddavā sabbe & bhayā vūpasamentu te\\
  Sakkatvā saṅgharatanaṁ & osadhaṁ uttamaṁ varaṁ\\
  Āhuneyyaṁ pāhuneyyaṁ & saṅghatejena sotthinā\\
  Nassantupaddavā sabbe & rogā vūpasamentu te\\
\end{twochants}

\subsubsection{Sabbītiyo vivajjantu}

\vspace*{\parskip}

\firstline{Sabbītiyo vivajjantu sabbarogo vinassatu}

\begin{twochants}
  Sabbītiyo vivajjantu & sabbarogo vinassatu\\
  Mā te bhavatvantarāyo & sukhī dīghāyuko bhava\\
  Abhivādanasīlissa & niccaṁ vuḍḍhāpacāyino\\
  Cattāro dhammā vaḍḍhanti & āyu vaṇṇo sukhaṁ balaṁ\\
\end{twochants}

\section{The Twenty-Eight Buddhas' Protection}

{\setlength{\parskip}{0pt}%
  \soloinstr{Solo introduction}

  \begin{soloonechants}
    We will now recite the discourse given by the Great Hero\\
    (the Buddha), as a protection for virtue-loving human beings,\\
    Against harm from all evil-doing, malevolent non-humans who are\\
    displeased with the Buddha's Teachings.\\
  \end{soloonechants}%
}

Homage to all Buddhas, the mighty who have arisen:\\
Taṇhaṅkara, the great hero, Medhaṅkara, the renowned,\\
Saraṇaṅkara, who guarded the world, Dīpaṅkara, the light-bearer,\\
Koṇḍañña, liberator of people, Maṅgala, great leader of people,\\
Sumana, kindly and wise, Revata, increaser of joy,\\
Sobhita, perfected in virtues, Anomadassī, greatest of beings,\\
Paduma, illuminer of the world, Nārada, true charioteer,\\
Padumuttara, most excellent of beings, Sumedha,\\\vin the unequalled one,\\
Sujāta, summit of the world,  Piyadassī, great leader of men,\\
Atthadassī, the compassionate, Dhammadassī,\\\vin destroyer of darkness,\\
Siddhattha, unequalled in the world,  and Tissa, speaker of Truth,\\
Phussa, bestower of blessings, Vipassī, the incomparable,\\
Sikhī, the bliss-bestowing teacher, Vessabhū, giver of happiness,\\
Kakusandha, the caravan leader, Koṇāgamana, abandoner of ills,\\
Kassapa, perfect in glory, Gotama, chief of the Sakyans.

These and all self-enlightened Buddhas are also peerless ones,\\
All the Buddhas together, all of mighty power,\\
All endowed with the Ten Powers, attained to highest knowledge,\\
All of these are accorded the supreme place of leadership.\\
They roar the lion's roar with confidence among their followers,\\
They observe with the divine eye, unhindered, all the world.\\
The leaders endowed with the eighteen kinds of Buddha-Dhamma,\\
The thirty-two major and eighty minor marks of a great being,\\
Shining with fathom-wide haloes, all these elephant-like sages,\\
All these omniscient Buddhas, conquerors free of corruption,\\
Of mighty brilliance, mighty power, of mighty wisdom,\\\vin mighty strength,\\
Of mighty compassion and wisdom, bearing bliss to all,\\
Islands, guardians and supports, shelters and caves for all beings,\\
Resorts, kinsmen and comforters, benevolent givers of refuge,\\
These are all the final resting place for the world with its deities.\\
With my head at their feet I salute these greatest of humans.\\
With both speech and thought I venerate those Tathāgatas,\\
Whether lying down, seated or standing, or walking anywhere.\\
May they ever guard your happiness, the Buddhas,\\\vin bringers of peace,\\
And may you, guarded by them, at peace, freed from all fear,\\
Released from all illness, safe from all torments,\\
Having transcended hatred, may you gain cessation.

By the power of their truth, their virtue and love,\\
May they protect and guard you in health and happiness.\\
In the Eastern quarter are beings of great power,\\
May they protect and guard you in health and happiness.\\
In the Southern quarter are deities of great power,\\
May they protect and guard you in health and happiness.\\
In the Western quarter are dragons of great power,\\
May they protect and guard you in health and happiness.\\
In the Northern quarter are spirits of great power,\\
May they protect and guard you in health and happiness.\\
In the East is Dhataraṭṭha, in the South is Viruḷhaka,\\
In the West is Virūpakkha, Kuvera rules the North.\\
These Four Mighty Kings, far-famed guardians of the world,\\
May they all be your protectors in health and happiness.\\
Sky-dwelling and earth-dwelling gods and dragons of great power,\\
May they all be your protectors in health and happiness.\\
For me there is no other refuge, the Buddha is my excellent refuge:\\
By this declaration of truth may the blessings of victory be yours.\\
For me there is no other refuge,\\\vin the Dhamma is my excellent refuge:\\
By this declaration of truth may the blessings of victory be yours.\\
For me there is no other refuge, the Saṅgha is my excellent refuge:\\
By this declaration of truth may the blessings of victory be yours.

Whatever jewel may be found in the world, however splendid,\\
There is no jewel equal to the Buddha,\\\vin therefore may you be blessed.\\
Whatever jewel may be found in the world, however splendid,\\
There is no jewel equal to the Dhamma,\\\vin therefore may you be blessed.\\
Whatever jewel may be found in the world, however splendid,\\
There is no jewel equal to the Saṅgha,\\\vin therefore may you be blessed.\\
If you venerate the Buddha jewel, the supreme,\\\vin excellent protection,\\
Which benefits gods and humans, then in safety,\\\vin by the Buddha's power,\\
All dangers will be prevented, your sorrows will pass away.\\
If you venerate the Dhamma jewel, the supreme,\\\vin excellent protection,\\
Which calms all fevered states, then in safety,\\\vin by the Dhamma's power,\\
All dangers will be prevented, your fears will pass away.\\
If you venerate the Saṅgha jewel, the supreme,\\\vin excellent protection,\\
Worthy of gifts and hospitality, then in safety,\\\vin by the Saṅgha's power,\\
All dangers will be prevented, your sicknesses will pass away.\\
May all calamities be avoided, may all illness pass away,\\
May no dangers threaten you, may you be happy and long-lived,\\
Greeted kindly and welcome everywhere.\\
May four things accrue to you: long life, beauty, bliss, and strength.

\section{Pabbatopama-gāthā}

\englishTitle{Verses on Mountains}

\firstline{Yathā pi selā vipulā nabhaṁ āhacca pabbatā}

\begin{twochants}
Yathā pi selā vipulā & nabhaṁ āhacca pabbatā\\
Samantā anupariyeyyuṁ & nippothentā catuddisā\\
Evaṁ jarā ca maccu ca & adhivattanti pāṇino\\
Khattiye brāhmaṇe vesse & sudde caṇḍāla-pukkuse\\
Na kiñci parivajjeti & sabbam-evābhimaddati\\
\end{twochants}

\begin{twochants}
Na tattha hatthīnaṁ bhūmi & na rathānaṁ na pattiyā\\
Na cāpi manta-yuddhena & sakkā jetuṁ dhanena vā\\
Tasmā hi paṇḍito poso & sampassaṁ attham-attano\\
Buddhe dhamme ca saṅghe ca & dhīro saddhaṁ nivesaye\\
Yo dhamma-cārī kāyena & vācāya uda cetasā\\
Idh'eva naṁ pasaṁsanti & pecca sagge pamodati
\end{twochants}

\suttaRef{S.I.102}

\section{Bhāra-sutta-gāthā}

\englishTitle{Verses on the Burden}

\firstline{Bhārā have pañcakkhandhā}

\begin{twochants}
Bhārā have pañcakkhandhā & bhāra-hāro ca puggalo \\
Bhār'ādānaṁ dukkhaṁ loke & bhāra-nikkhepanaṁ sukhaṁ \\
\end{twochants}

\begin{english}
  The five aggregates indeed are burdens,\\
  The beast of burden though is man.\\
  In this world to take up burdens is dukkha.\\
  Putting them down brings happiness.
\end{english}

\begin{twochants}
Nikkhipitvā garuṁ bhāraṁ & aññaṁ bhāraṁ anādiya \\
Samūlaṁ taṇhaṁ abbuyha & nicchāto parinibbuto \\
\end{twochants}

\begin{english}
  A heavy burden cast away,\\
  Not taking on another load,\\
  With craving pulled out from the root,\\
  Desires stilled, one is released.
\end{english}

\suttaRef{S.III.26}

\clearpage

\section{Khemākhema-saraṇa-gamana-paridīpikā-gāthā}

\englishTitle{True and False Refuges}

\firstline{Bahuṁ ve saraṇaṁ yanti pabbatāni vanāni ca}

\begin{twochants}
Bahuṁ ve saraṇaṁ yanti & pabbatāni vanāni ca\\
Ārāma-rukkha-cetyāni & manussā bhaya-tajjitā\\
\end{twochants}

\begin{english}
  To many refuges they go ---\\
  To mountain slopes and forest glades,\\
  To parkland shrines and sacred sites ---\\
  People overcome by fear.
\end{english}

\begin{twochants}
N'etaṁ kho saraṇaṁ khemaṁ & n'etaṁ saraṇam-uttamaṁ\\
N'etaṁ saraṇam-āgamma & sabba-dukkhā pamuccati\\
\end{twochants}

\begin{english}
  Such a refuge is not secure,\\
  Such a refuge is not supreme,\\
  Such a refuge does not bring\\
  Complete release from suffering.
\end{english}

\begin{twochants}
Yo ca buddhañ-ca dhammañ-ca & saṅghañ-ca saraṇaṁ gato\\
Cattāri ariya-saccāni & sammappaññāya passati\\
\end{twochants}

\begin{english}
  Whoever goes to refuge\\
  In the Triple Gem\\
  Sees with right discernment\\
  The Four Noble Truths:
\end{english}

\begin{twochants}
Dukkhaṁ dukkha-samuppādaṁ & dukkhassa ca atikkamaṁ\\
Ariyañ-c'aṭṭh'aṅgikaṁ maggaṁ & dukkhūpasama-gāminaṁ\\
\end{twochants}

\begin{english}
  Suffering and its origin\\
  And that which lies beyond ---\\
  The Noble Eightfold Path\\
  That leads the way to suff'ring's end.
\end{english}

\begin{twochants}
Etaṁ kho saraṇaṁ khemaṁ & etaṁ saraṇam-uttamaṁ\\
Etaṁ saraṇam-āgamma & sabba-dukkhā pamuccatī'ti.
\end{twochants}

\begin{english}
  Such a refuge is secure,\\
  Such a refuge is supreme,\\
  Such a refuge truly brings\\
  Complete release from all suffering.
\end{english}

\suttaRef{Dhp 188-192}

\section{Bhadd'eka-ratta-gāthā}

\englishTitle{Verses on a Shining Night of Prosperity}

\firstline{Atītaṁ nānvāgameyya nappaṭikaṅkhe anāgataṁ}

\begin{twochants}
  Atītaṁ nānvāgameyya & nappaṭikaṅkhe anāgataṁ \\
  Yad'atītaṁ pahīnan-taṁ & appattañca anāgataṁ \\
\end{twochants}

\begin{english}
  One should not revive the past\\
  Nor speculate on what's to come;\\
  The past is left behind,\\
  The future is un-realized.
\end{english}

\begin{twochants}
  Paccuppannañca yo dhammaṁ & tattha tattha vipassati \\
  Asaṁhiraṁ asaṅkuppaṁ & taṁ viddhām-anubrūhaye \\
\end{twochants}

\begin{english}
  In every presently arisen state\\
  There, just there, one clearly sees;\\
  Unmoved, unagitated,\\
  Such insight is one's strength.
\end{english}

\begin{twochants}
  Ajj'eva kiccam-ātappaṁ & ko jaññā maraṇaṁ suve \\
  Na hi no saṅgaran-tena & mahā-senena maccunā \\
\end{twochants}

\begin{english}
  Ardently doing one's task today,\\
  Tomorrow, who knows, death may come;\\
  Facing the mighty hordes of death,\\
  Indeed one cannot strike a deal.
\end{english}

\begin{twochants}
  Evaṁ vihārim-ātāpiṁ & aho-rattam-atanditaṁ \\
  Taṁ ve bhadd'eka-ratto'ti & santo ācikkhate muni \\
\end{twochants}

\begin{english}
  To dwell with energy aroused\\
  Thus for a night of non-decline,\\
  That is a `night of shining prosperity.'\\
  So it was taught by the Peaceful Sage.
\end{english}

\suttaRef{M.III.187}

\section{Ti-lakkhaṇ'ādi-gāthā}

\firstline{Sabbe saṅkhārā aniccā'ti yadā paññāya passati}

\begin{twochants}
  Sabbe saṅkhārā aniccā'ti & yadā paññāya passati \\
  Atha nibbindati dukkhe & esa maggo visuddhiyā \\
  Sabbe saṅkhārā dukkhā'ti & yadā paññāya passati \\
  Atha nibbindati dukkhe & esa maggo visuddhiyā \\
  Sabbe dhammā anattā'ti & yadā paññāya passati \\
  Atha nibbindati dukkhe & esa maggo visuddhiyā \\
\end{twochants}

\suttaRef{Dhp 277-279}

\begin{twochants}
  Appakā te manussesu & ye janā pāra-gāmino \\
  Athāyaṁ itarā pajā & tīram-evānudhāvati \\
  Ye ca kho sammad-akkhāte & dhamme dhammānuvattino \\
  Te janā pāram-essanti & maccu-dheyyaṁ suduttaraṁ \\
  Kaṇhaṁ dhammaṁ vippahāya & sukkaṁ bhāvetha paṇḍito \\
  Okā anokam-āgamma & viveke yattha dūramaṁ \\
  Tatrābhiratim-iccheyya & hitvā kāme akiñcano \\
  Pariyodapeyya attānaṁ & citta-klesehi paṇḍito\\
  Yesaṁ sambodhiy-aṅgesu & sammā cittaṁ subhāvitaṁ\\
  Ādāna-paṭinissagge & anupādāya ye ratā\\
  Khīṇ'āsavā jutimanto & te loke parinibbutā'ti
\end{twochants}

\suttaRef{Dhp 85-89}

\subsubsection{Verses on the Three Characteristics}

`Impermanent are all conditioned things' ---\\
When with wisdom this is seen\\
One feels weary of all dukkha;\\
This is the path to purity.

`Dukkha are all conditioned things' ---\\
When with wisdom this is seen\\
One feels weary of all dukkha;\\
This is the path to purity.

`There is no self in anything' ---\\
When with wisdom this is seen\\
One feels weary of all dukkha;\\
This is the path to purity.

Few amongst humankind\\
Are those who go beyond,\\
Yet there are the many folks\\
Ever wand'ring on this shore.

Wherever Dhamma is well-taught,\\
Those who train in line with it\\
Are the ones who will cross over\\
The realm of death so hard to flee.

Abandoning the darker states,\\
The wise pursue the bright;\\
From the floods dry land they reach\\
Living withdrawn so hard to do.\\
Such rare delight one should desire,\\
Sense pleasures cast away,\\
Not having anything.

\section{Dhamma-gārav'ādi-gāthā}

\englishTitle{Verses on Respect for the Dhamma}

\firstline{Ye ca atītā sambuddhā ye ca buddhā anāgatā}

\begin{twochants}
  Ye ca atītā sambuddhā & ye ca buddhā anāgatā \\
  Yo c'etarahi sambuddho & bahunnaṁ soka-nāsano \\
\end{twochants}

\begin{english}
  All the Buddhas of the past,\\
  All the Buddhas yet to come,\\
  The Buddha of this current age ---\\
  Dispellers of much sorrow.
\end{english}

\begin{twochants}
  Sabbe saddhamma-garuno & vihariṁsu viharanti ca \\
  Atho pi viharissanti & esā buddhāna dhammatā \\
\end{twochants}

\begin{english}
  Those having lived or living now,\\
  Those living in the future,\\
  All do revere the True Dhamma ---\\
  That is the nature of all Buddhas.
\end{english}

\begin{twochants}
  Tasmā hi atta-kāmena & mahattam-abhikaṅkhatā \\
  Saddhammo garu-kātabbo & saraṁ buddhāna sāsanaṁ \\
\end{twochants}

\begin{english}
  Therefore desiring one's own welfare,\\
  Pursuing greatest aspirations,\\
  One should revere the True Dhamma ---\\
  Recollecting the Buddha's teaching.
\end{english}

\suttaRef{S.I.140}

\begin{paritta}
Na hi dhammo adhammo ca\\
Ubho sama-vipākino \\
Adhammo nirayaṁ neti\\
Dhammo pāpeti suggatiṁ
\end{paritta}

\begin{english}
  What is true Dhamma and what not\\
  Will never have the same results,\\
  While lack of Dhamma leads to hell-realms ---\\
  True Dhamma takes one on a good course.
\end{english}

\begin{paritta}
Dhammo have rakkhati dhamma-cāriṁ\\
Dhammo suciṇṇo sukham-āvahāti\\
Esānisaṁso dhamme suciṇṇe
\end{paritta}

% NOTE: The last line of the verse is usually omitted --
% [Na duggatiṁ gacchati dhamma-cārī.]

\clearpage

\begin{english}
  The Dhamma guards who lives in line with it\\
  And leads to happiness when practised well ---\\
  This is the blessing of well-practised Dhamma.
\end{english}

\suttaRef{Thag 303-304}

\section{Paṭhama-buddha-bhāsita-gāthā}

\englishTitle{Verses on the Buddha's First Exclamation}

\firstline{Aneka-jāti-saṁsāraṁ sandhāvissaṁ anibbisaṁ}

\begin{twochants}
  Aneka-jāti-saṁsāraṁ & sandhāvissaṁ anibbisaṁ \\
  Gaha-kāraṁ gavesanto & dukkhā jāti punappunaṁ \\
\end{twochants}

\begin{english}
  For many lifetimes in the round of birth,\\
  Wandering on endlessly,\\
  For the builder of this house I searched ---\\
  How painful is repeated birth.
\end{english}

\begin{twochants}
  Gaha-kāraka diṭṭho'si & puna gehaṁ na kāhasi \\
  Sabbā te phāsukā bhaggā & gaha-kūṭaṁ visaṅkhataṁ \\
  Visaṅkhāra-gataṁ cittaṁ & taṇhānaṁ khayam-ajjhagā \\
\end{twochants}

\begin{english}
  House-builder you've been seen,\\
  Another home you will not build,\\
  All your rafters have been snapped,\\
  Dismantled is your ridge-pole;\\
  The non-constructing mind\\
  Has come to craving's end.
\end{english}

\suttaRef{Dhp 153-154}

\clearpage

\section{Pacchima-ovāda-gāthā}

\englishTitle{Verses on the Last Instructions}

\firstline{Handa dāni bhikkhave āmantayāmi vo}

\begin{paritta}
Handa dāni bhikkhave āmantayāmi vo\\
Vaya-dhammā saṅkhārā\\
Appamādena sampādethā'ti\\
Ayaṁ tathāgatassa pacchimā vācā
\end{paritta}

\begin{english}
  ‘Now, take heed, bhikkhus, I caution you thus: Dissolution is the nature of
  all conditions. Therefore strive on with diligence!’ These are the final words
  of the Tathāgata.
\end{english}

\suttaRef{D.II.156}

\section{Ye dhammā hetuppabhavā}

\englishTitle{Arising From a Cause}

\firstline{Ye dhammā hetuppabhavā}

\begin{paritta}
  Ye dhammā hetuppabhavā\\
  Tesaṁ hetuṁ tathāgato āha\\
  Tesañca yo nirodho\\
  Evaṁ-vādī mahāsamaṇo'ti
\end{paritta}

\begin{english}
  Whatever phenomena arise from a cause,\\
  The Tathāgata has explained their cause,\\
  And also their cessation.\\
  That is the teaching of the Great Ascetic.
\end{english}

\suttaRef{Mv.1.23.5}

\clearpage

\section{Nakkhattayakkha}

\firstline{Nakkhatta-yakkha-bhūtānaṁ}

\instr{The paritta chanting may be closed with the following:}

\bigskip

\begin{paritta}
  Nakkhatta-yakkha-bhūtānaṁ\\
  Pāpa-ggaha-nivāraṇā\\
  Parittassānubhāvena\\
  Hantvā tesaṁ upaddave
\end{paritta}

\instr{(Three times)}

\section{Verses on Respect}

% Gārav'ādi gāthā

% Pali and English source: Bodhivana

Satthu-garu dhamma-garu,\\
Saṅghe ca tibba-gāravo,\\
Samādhi-garu ātāpī,\\
Sikkhāya tibba-gāravo,\\
Appamāda-garu bhikkhu,\\
Paṭisanthāra-gāravo:\\
Abhabbo parihānāya,\\
Nibbānasseva santike.

\begin{english}
  One with respect for the Buddha and Dhamma,\\
  and strong respect for the Saṅgha,\\
  one who is ardent, with respect for concentration,\\
  and strong respect for the Training,\\
  one who sees danger and respects being heedful,\\
  and shows respect in welcoming guests.\\
  A person like this cannot decline,\\
  stands right in the presence of Nibbāna. \suttaRef{A.IV.28}
\end{english}
