\chapter{Paritta Chants}

\section{Thai Traditions}

Paritta chanting ceremonies in Thailand vary regionally but may be outlined as:

\begin{packeditemize}
  \item a layperson chants the invitation for paritta chanting
  \item the third bhikkhu or nun in seniority chants the invitation to the devas
  \item the introductory chants are chanted
  \item the core sequence of paritta chants follow
  \item the closing chants end the ceremony.
\end{packeditemize}

The 3rd introductory chant in the Mahānikāya sect is commonly \emph{Sambuddhe}.
In Thammayut circles and frequently in the Forest Tradition, the 3rd chant is
\emph{Yo cakkhumā}.

There is a shorter and longer traditional core sequence. The \emph{djet-damnahn}
(\thai{เจ็ดตํานาน}) contains D1-D7 as below, the \emph{sipsong-damnahn}
(\thai{สิบสองตํานาน}) contains S1-S12. Chants that are not numbered `D' or `S' can
be included or not, as wished, but should be recited in the order listed here.

\clearpage

{\centering
\fontsize{11}{15}\selectfont

\begin{tabular}{@{}l l r r@{}}
  & first line & & page \\
  \hline
  i1  & Namo tassa & & \pageref{namo-tassa} \\
  i2  & Buddhaṃ saraṇaṃ gacchāmi & & \pageref{buddham-saranam} \\
  i3/a  & Sambuddhe aṭṭhavīsañca & & \pageref{sambuddhe} \\
  i3/b  & Yo cakkhumā & & \pageref{yo-cakkhuma} \\
  i4  & Namo arahato & & \pageref{namo-arahato} \\
      & & & \\
  D1 & Asevanā ca bālānaṃ & S1 & \pageref{asevana} \\
  D2 & Yaṅkiñci vittaṃ & S2 & \pageref{yankinci-vittam} \\
  D3 & Karaṇīyam-attha-kusalena & S3 & \pageref{karaniyam-attha} \\
  D4 & Virūpakkhehi me mettaṃ & S4 & \pageref{virupakkhehi} \\
  & Vadhissamenanti parāmasanto & & \pageref{vadhissamenanti} \\
  D5 & Udet'ayañ-cakkhumā eka-rājā & S5 & \pageref{udetayan-cakkhuma} \\
  & Atthi loke sīla-guṇo & S6 & \pageref{atthi-loke} \\
  D6 & Iti pi so bhagavā & S7 & \pageref{iti-pi-so} \\
  D7 & Vipassissa nam'atthu & S8 & \pageref{vipassissa} \\
  & Natthi me saraṇaṃ aññaṃ & & \pageref{natthi-me} \\
  & Yaṅkiñci ratanaṃ loke & & \pageref{yankinci-ratanam} \\
  & Sakkatvā buddharatanaṃ & & \pageref{sakkatva} \\
  & Yato'haṃ bhagini & S9 & \pageref{yato-ham-bhagini} \\
  & Bojjh'aṅgo sati-saṅkhāto & S10 & \pageref{bojjhango} \\
  & Yan-dunnimittaṃ & S11 & \pageref{yan-dunnimittam} \\
      & & & \\
  & Dukkhappattā ca niddukkhā & & \pageref{dukkhappatta} \\
  & Bāhuṃ sahassam-abhinimmita & & \pageref{bahum} \\
  & Mahā-kāruṇiko nātho & S12 & \pageref{maha-karuniko} \\
  & Te attha-laddhā sukhitā & & \pageref{te-attha-laddha} \\
  & Bhavatu sabba-maṅgalaṃ & & \pageref{bhavatu} \\
\end{tabular}

}

\subsection{Notes for Particular Chants}

\textbf{Asevanā ca bālānaṃ:} The candles at the shrine during a house invitation
are lit by the senior bhikkhu or nun at \emph{Asevanā}.

  \textbf{Yaṅkiñci vittaṃ:} The candles are put out at \emph{Nibbanti
  dhīrā yathā'yam padīpo}.

\textbf{Atthi loke sīla-guṇo:} On the occasion of blessing a new house, this
chant should be included, as it is traditionally considered protection against
fire.

\textbf{Yato'haṃ bhagini:} This chant is to be used for expectant mothers since
the time of the Buddha for the blessing and protection of the mother and child.
It is also a good occasion to chant it when receiving alms from a newly married
couple. Sangha members are encouraged to practice it.

\textbf{Dukkhappattā ca niddukkhā:} This is usually chanted as second to last
before \emph{Bhavatu sabba-maṅgalaṃ}. It is considered necessary to include it
whenever the devas have been invited at the beginning of the paritta chanting
as this chant contains a line inviting them to leave again.

\textbf{Bāhuṃ sahassam-abhinimmita:} This is is a popular later addition to the
present day standard chants. It is not listed in the \emph{djet-} or
\emph{sipsong-damnahn} sets. Yet these days it is frequently added just before
\emph{Mahā-kāruṇiko nātho}. On some occasions (e.g. public birthdays, jubilees,
inauguration ceremonies, etc.), it is an alternative, instead of chanting
\emph{djet-} or \emph{sipsong-damnahn}, to do a minimum sequence called
\emph{suat-phorn-phra} which contains only:

(1)~\emph{Namo Tassa},
(2)~\emph{Iti pi so bhagavā},
(3)~\emph{Bāhuṃ}, 
(4)~\emph{Mahā-kāruṇiko nātho}, and
(5)~\emph{Bhavatu sabba-maṅgalaṃ}.

In this minimal chanting sequence usually one does not invite the devas.

\textbf{Te attha-laddhā sukhitā:} This is sometimes inserted before closing with
\emph{Bhavatu sabba-maṅgalaṃ}, as a special well-wishing when the occasion has
to do with Buddhism in general (e.g. inauguration of a new abbot, or at the end
of an \emph{upasampadā}).

\section{Invitations}

\subsection{Invitation for Paritta Chanting}
\label{paritta-invitation-for-chanting}

\firstline{Vipatti-paṭibāhāya sabba-sampatti-siddhiyā}

\vspace*{5pt}

\begin{paritta}

\instr{(After bowing three times, with hands joined in añjali,\\
  recite the following)\par}

Vipatti-paṭibāhāya sabba-sampatti-siddhiyā\\
Sabbadukkha-vināsāya\\
Parittaṃ brūtha maṅgalaṃ

Vipatti-paṭibāhāya sabba-sampatti-siddhiyā\\
Sabbabhaya-vināsāya\\
Parittaṃ brūtha maṅgalaṃ

Vipatti-paṭibāhāya sabba-sampatti-siddhiyā\\
Sabbaroga-vināsāya\\
Parittaṃ brūtha maṅgalaṃ

\instr{(Bow three times)}
\end{paritta}

%\suttaRef{Thai}

\subsection{Invitation to the Devas}
\label{paritta-devas}

\firstline{Pharitvāna mettaṃ samettā bhadantā}
\firstline{Samantā cakka-vāḷesu}
\firstline{Sarajjaṃ sasenaṃ sabandhuṃ nar'indaṃ}

\enlargethispage{\baselineskip}

In Thai custom, the third monk in seniority invites the devas, holding his
hands in \emph{añjali}, and lifting up the ceremonial string.

The string is wound up at the beginning of the last chant, \emph{Mahā-kāruṇiko
  nātho} or \emph{Bhavatu sabba-maṅgalaṃ}, which should be kept in mind by the
last bhikkhu or \emph{sāmaṇera}.

Before royal ceremonies, the invitation starts with A.

Before the shorter, \emph{djet-damnahn} set of parittas, B. is used and C. is
omitted. Before the longer, \emph{sipsong-damnahn} set of parittas, B. is
omitted and C. is used.

The verses at D. are always chanted.

When chanting outside the monastery, the invitation is concluded with E. When
chanting at the monastery, the invitation is concluded either with E. or F.

\begin{paritta}

\instr{(With hands joined in añjali, recite the following)}

\sidepar{A.}%
Sarajjaṃ sasenaṃ sabandhuṃ nar'indaṃ\\
Paritt'ānubhavo sadā rakkhatū-ti

\sidepar{B.}%
Pharitvāna mettaṃ samettā bhadantā\\
Avikkhitta-cittā parittaṃ bhaṇantu

\sidepar{C.}%
Samantā cakka-vāḷesu\\
Atr'āgacchantu devatā

\sidepar{D.}%
Sagge kāme ca rūpe\\
Giri-sikhara-taṭe c'antalikkhe vimāne\\
Dīpe raṭṭhe ca gāme\\
Taru-vana-gahane geha-vatthumhi khette\\
Bhummā c'āyantu devā\\
Jala-thala-visame yakkha-gandhabba-nāgā\\
Tiṭṭhantā santike yaṃ\\
Muni-vara-vacanaṃ sādhavo me suṇantu

\sidepar{E.}%
Dhammassavana-kālo ayam-bhadantā

\instr{(Three times, or)}

\sidepar{F.}%
Buddha-dassana-kālo ayam-bhadantā\\
Dhammassavana-kālo ayam-bhadantā\\
Saṅgha-payirūpāsana-kālo ayam-bhadantā
\end{paritta}

%\suttaRef{Thai}

\begin{english}
  Benevolent, venerable sirs: having spread thoughts of goodwill, listen to the
  chant with undistracted mind.

  From all around the ten-thousand world-systems, may the devas come here.\\
  May they listen to the True Dhamma of the King of Sages,\\
  leading to heaven and liberation.

  Those in the heavens of sensuality and form,\\
  on peaks and mountain precipices, in palaces floating in the sky,\\
  in islands, countries, and towns,\\
  in groves of trees and thickets, around home sites and fields.

  And the earth-devas, spirits, heavenly minstrels, and nagas\\
  in water, on land, in bad lands, and nearby:\\
  May they come and listen with approval\\
  as I recite the word of the excellent sage.

  This is the time to see the Buddha, venerable sirs.\\
  This is the time to listen to the Dhamma, venerable sirs.\\
  This is the time to attend to the Saṅgha, venerable sirs.
\end{english}

\clearpage

\section{Introductory Chants}

\subsection{Pubba-bhāga-nama-kāra-pāṭho}
\label{namo-tassa}

Namo tassa bhagavato arahato sammā-sambuddhassa\\
Namo tassa bhagavato arahato sammā-sambuddhassa\\
Namo tassa bhagavato arahato sammā-sambuddhassa

\subsection{Saraṇa-gamana-pāṭho}
\label{buddham-saranam}

\begin{paritta}
Buddhaṃ saraṇaṃ gacchāmi\\
Dhammaṃ saraṇaṃ gacchāmi\\
Saṅghaṃ saraṇaṃ gacchāmi

Dutiyam pi buddhaṃ saraṇaṃ gacchāmi\\
Dutiyam pi dhammaṃ saraṇaṃ gacchāmi\\
Dutiyam pi saṅghaṃ saraṇaṃ gacchāmi

Tatiyam pi buddhaṃ saraṇaṃ gacchāmi\\
Tatiyam pi dhammaṃ saraṇaṃ gacchāmi\\
Tatiyam pi saṅghaṃ saraṇaṃ gacchāmi
\end{paritta}

\subsection{Sambuddhe}
\label{sambuddhe}

\firstline{Sambuddhe aṭṭhavīsañca}

\begin{twochants}
Sambuddhe aṭṭhavīsañca & dvādasañca sahassake\\
Pañca-sata-sahassāni & namāmi sirasā ahaṃ\\
Tesaṃ dhammañca saṅghañca & ādarena namāmihaṃ\\
Namakārānubhāvena & hantvā sabbe upaddave\\
Anekā antarāyāpi & vinassantu asesato\\
Sambuddhe pañca-paññāsañca & catuvīsati sahassake\\
Dasa-sata-sahassāni & namāmi sirasā ahaṃ\\
Tesaṃ dhammañca saṅghañca & ādarena namāmihaṃ\\
Namakārānubhāvena & hantvā sabbe upaddave\\
Anekā antarāyāpi & vinassantu asesato\\
Sambuddhe navuttarasate & aṭṭhacattāḷīsa sahassake\\
Vīsati-sata-sahassāni & namāmi sirasā ahaṃ\\
Tesaṃ dhammañca saṅghañca & ādarena namāmihaṃ\\
Namakārānubhāvena & hantvā sabbe upaddave\\
Anekā antarāyāpi & vinassantu asesato\\
\end{twochants}

\subsection{The Buddhas}

% English source: Bodhivana

I pay homage with my head to\\
the 512,028 Buddhas.

I pay devoted homage to their Dhamma and Saṅgha.\\
Through the power of this homage,\\
having demolished all misfortunes,\\
may countless dangers be destroyed without trace.

I pay homage with my head to\\
the 1,024,055 Buddhas.

I pay devoted homage to their Dhamma and Saṅgha.\\
Through the power of this homage,\\
having demolished all misfortunes,\\
may countless dangers be destroyed without trace.

I pay homage with my head to\\
the 2,048,109 Buddhas.

I pay devoted homage to their Dhamma and Saṅgha.\\
Through the power of this homage,\\
having demolished all misfortunes,\\
may countless dangers be destroyed without trace.

\subsection{Nama-kāra-siddhi-gāthā}
\label{yo-cakkhuma}

\firstline{Yo cakkhumā moha-malāpakaṭṭho}

\begin{paritta}
Yo cakkhumā moha-malāpakaṭṭho\\
Sāmaṃ va buddho sugato vimutto\\
Mārassa pāsā vinimocayanto\\
Pāpesi khemaṃ janataṃ vineyyaṃ\\
Buddhaṃ varan-taṃ sirasā namāmi\\
Lokassa nāthañ-ca vināyakañ-ca\\
Tan-tejasā te jaya-siddhi hotu\\
Sabb'antarāyā ca vināsamentu

Dhammo dhajo yo viya tassa satthu\\
Dassesi lokassa visuddhi-maggaṃ\\
Niyyāniko dhamma-dharassa dhārī\\
Sāt'āvaho santi-karo suciṇṇo\\
Dhammaṃ varan-taṃ sirasā namāmi\\
Mohappadālaṃ upasanta-dāhaṃ\\
Tan-tejasā te jaya-siddhi hotu\\
Sabb'antarāyā ca vināsamentu

Saddhamma-senā sugatānugo yo\\
Lokassa pāpūpakilesa-jetā\\
Santo sayaṃ santi-niyojako ca\\
Svākkhāta-dhammaṃ viditaṃ karoti\\
Saṅghaṃ varan-taṃ sirasā namāmi\\
Buddhānubuddhaṃ sama-sīla-diṭṭhiṃ\\
Tan-tejasā te jaya-siddhi hotu\\
Sabb'antarāyā ca vināsamentu \suttaRef{Thai}
\end{paritta}

\subsection{The Verses of Success through Homage}

% English source: Bodhivana

The One with Vision, with the stain of delusion removed,\\
Self-awakened, Well-Gone, and Released.\\
Releasing them from the Māra's snare,\\
he leads humanity from evils to security.

I pay homage with my head to that excellent Buddha,\\
the Protector and Mentor for the world.\\
By the majesty of this, may you have triumph and success,\\
and may all your dangers be destroyed.

The Teacher's Dhamma, like a banner,\\
shows the path of purity to the world.\\
Leading out, upholding those who uphold it,\\
rightly accomplished, it brings pleasure, makes peace.

I pay homage with my head to that excellent Dhamma,\\
which pierces delusion and makes fever grow calm.\\
By the majesty of this, may you have triumph and success,\\
and may all your dangers be destroyed.

The True Dhamma's army, following the One Well-Gone,\\
is victor over the evils and corruptions of the world.\\
Self-calmed, it is calming and unfettering,\\
and makes the well-taught Dhamma be known.

I pay homage with my head to that excellent Saṅgha,\\
awakened following the Awakened One, harmonious in virtue and view.\\
By the majesty of this, may you have triumph and success,\\
and may all your dangers be destroyed.

\subsection{Namo-kāra-aṭṭhaka}
\label{namo-arahato}

\firstline{Namo arahato sammā}

\begin{paritta}
  Namo arahato sammā\\
  Sambuddhassa mahesino\\
  Namo uttama-dhammassa\\
  Svākkhātass'eva ten'idha\\
  Namo mahā-saṅghassāpi\\
  Visuddha-sīla-diṭṭhino\\
  Namo omāty-āraddhassa\\
  Ratanattayassa sādhukaṃ\\
  Namo omakātītassa\\
  Tassa vatthuttayassa-pi\\
  Namo-kārappabhāvena\\
  Vigacchantu upaddavā\\
  Namo-kārānubhāvena\\
  Suvatthi hotu sabbadā\\
  Namo-kārassa tejena\\
  Vidhimhi homi tejavā \suttaRef{Thai}
\end{paritta}

\subsection{The Homage Octet}

Homage to the Great Seer, the Worthy One, Rightly Self-awakened.

Homage to the highest Dhamma, well-taught by him here.

And homage to the Great Saṅgha, pure in virtue and view.

Homage to the Triple Gem beginning auspiciously with AUM.

And homage to those three objects that have left base things behind.

By the potency of this homage, may misfortunes disappear.

By the potency of this homage, may there always be well-being.

By the majesty of this homage, may I be successful in this ceremony.

\section{Core Sequence}

\subsection{Maṅgala-sutta}
\label{asevana}

[Evam-me sutaṃ: ekaṃ samayaṃ bhagavā, sāvatthiyaṃ viharati, jeta-vane
anāthapiṇḍikassa ārāme. Atha kho aññatarā devatā abhikkantāya rattiyā
abhikkanta-vaṇṇā kevala-kappaṃ jetavanaṃ obhāsetvā, yena bhagavā ten'upasaṅkami.
Upasaṅkamitvā bhagavantaṃ abhivādetvā ekam-antaṃ aṭṭhāsi. Ekam-antaṃ ṭhitā kho
sā devatā bhagavantaṃ gāthāya ajjhabhāsi:

Bahū devā manussā ca,\\
Maṅgalāni acintayuṃ;\\
Ākaṅkhamānā sotthānaṃ,\\
Brūhi maṅgalam-uttamaṃ.]

\bigskip

\firstline{Asevanā ca bālānaṃ}

\begin{paritta}
Asevanā ca bālānaṃ\\
Paṇḍitānañ-ca sevanā\\
Pūjā ca pūjanīyānaṃ\\
Etam maṅgalam-uttamaṃ

Paṭirūpa-desa-vāso ca\\
Pubbe ca kata-puññatā\\
Atta-sammā-paṇidhi ca\\
Etam maṅgalam-uttamaṃ

Bāhu-saccañ-ca sippañ-ca,\\
Vinayo ca susikkhito\\
Subhāsitā ca yā vācā\\
Etam maṅgalam-uttamaṃ

Mātā-pitu-upaṭṭhānaṃ\\
Putta-dārassa saṅgaho\\
Anākulā ca kammantā\\
Etam maṅgalam-uttamaṃ

Dānañ-ca dhamma-cariyā ca\\
Ñātakānañ-ca saṅgaho\\
Anavajjāni kammāni\\
Etam maṅgalam-uttamaṃ

Āratī viratī pāpā\\
Majja-pānā ca saññamo\\
Appamādo ca dhammesu\\
Etam maṅgalam-uttamaṃ

Gāravo ca nivāto ca\\
Santuṭṭhī ca kataññutā\\
Kālena dhammassavanaṃ\\
Etam maṅgalam-uttamaṃ

Khantī ca sovacassatā\\
Samaṇānañ-ca dassanaṃ\\
Kālena dhamma-sākacchā\\
Etam maṅgalam-uttamaṃ

Tapo ca brahma-cariyañ-ca\\
Ariya-saccāna-dassanaṃ\\
Nibbāna-sacchikiriyā ca\\
Etam maṅgalam-uttamaṃ

Phuṭṭhassa loka-dhammehi\\
Cittaṃ yassa na kampati\\
Asokaṃ virajaṃ khemaṃ\\
Etam maṅgalam-uttamaṃ

Etādisāni katvāna\\
Sabbattham-aparājitā\\
Sabbattha sotthiṃ gacchanti\\
Tan-tesaṃ maṅgalam-uttaman'ti \suttaRef{Snp 2.4}

\end{paritta}

\subsection{The Highest Blessings (English)}

\firstline{Thus have I heard that the Blessed One}

\begin{leader}
  [Now let us chant the verses on the Highest Blessings]
\end{leader}

%\suttaref{Sn 2.4}%
[\prul{Thus} have I \prul{heard} that the Blessed One]\\
Was staying at Sāvatthī,\\
\prul{Residing} at the Jeta's Grove\\
In Anāthapiṇḍika's Park.

\prul{Then} in the dark of the night, a radiant deva\\
Illuminated \prul{all} Jeta's Grove.\\
She bowed down low before the Blessed One\\
Then standing to one side she said:

`Devas are concerned for happiness\\
And ever long for peace.\\
The same is true for humankind.\\
What \prul{then} are the highest blessings?'

`Avoiding those of foolish ways,\\
Associating with the wise,\\
And honouring those worthy of honour.\\
\prul{These} are the highest blessings.

`Living in places of suitable kinds,\\
With the fruits of past good deeds\\
And guided by the rightful way.\\
\prul{These} are the highest blessings.

`Accomplished in learning and craftsman's skills,\\% {{{1
With discipline, highly trained,\\
And \prul{speech} that is true and pleasant to hear.\\
\prul{These} are the highest blessings.

`Providing for mother and father's support\\
And cherishing family,\\
And ways of work that harm no being,\\
\prul{These} are the highest blessings.

`Generosity and a righteous life,\\
Offering help to relatives and kin,\\
And acting in ways that leave no blame.\\
\prul{These} are the highest blessings.

`Steadfast in restraint, and shunning evil ways,\\
Avoiding intoxicants that dull the mind,\\
And heedfulness in all things that arise.\\
\prul{These} are the highest blessings.

`Respectfulness and being of humble ways,\\
Contentment and gratitude,\\
And hearing the Dhamma frequently taught.\\
\prul{These} are the highest blessings.

`Patience and willingness to accept one's faults,\\
Seeing venerated seekers of the truth,\\
And sharing often the words of Dhamma.\\
\prul{These} are the highest blessings.

`Ardent, committed to the Holy Life,\\% {{{1
Seeing for oneself the Noble Truths\\
And the realization of Nibbāna.\\
\prul{These} are the highest blessings.

`Although in contact with the world,\\
Unshaken the mind remains\\
Beyond all sorrow, spotless, secure.\\
\prul{These} are the highest blessings.

`They who \prul{live} by following this path\\
Know victory wherever they go,\\
And every \prul{place} for them is safe.\\
\prul{These} are the highest blessings.'

\subsection{Ratana Sutta}

\firstline{Yānīdha bhūtāni samāgatāni}

\emph{(In certain monasteries the custom is to chant only the numbered verses.)}

\begin{paritta}

Yānīdha bhūtāni samāgatāni,\\
Bhummāni vā yāni va antalikkhe.\\
Sabb'eva bhūtā sumanā bhavantu,\\
Atho pi sakkacca suṇantu bhāsitaṃ.\\
Tasmā hi bhūtā nisāmetha sabbe,\\
Mettaṃ karotha mānusiyā pajāya.\\
Divā ca ratto ca haranti ye baliṃ,\\
Tasmā hi ne rakkhatha appamattā.

\firstline{Yaṅkiñci vittaṃ idha vā huraṃ vā}

\label{yankinci-vittam}
\sidepar{1.}%
Yaṅkiñci vittaṃ idha vā huraṃ vā\\
Saggesu vā yaṃ ratanaṃ paṇītaṃ\\
Na no samaṃ atthi tathāgatena\\
Idam-pi buddhe ratanaṃ paṇītaṃ\\
Etena saccena suvatthi hotu

\clearpage

\sidepar{2.}%
Khayaṃ virāgaṃ amataṃ paṇītaṃ\\
Yad-ajjhagā sakya-munī samāhito\\
Na tena dhammena sam'atthi kiñci\\
Idam-pi dhamme ratanaṃ paṇītaṃ\\
Etena saccena suvatthi hotu

\sidepar{3.}%
Yam buddha-seṭṭho parivaṇṇayī suciṃ\\
Samādhim-ānantarikaññam-āhu\\
Samādhinā tena samo na vijjati\\
Idam-pi dhamme ratanaṃ paṇītaṃ\\
Etena saccena suvatthi hotu

\sidepar{4.}%
Ye puggalā aṭṭha sataṃ pasaṭṭhā\\
Cattāri etāni yugāni honti\\
Te dakkhiṇeyyā sugatassa sāvakā\\
Etesu dinnāni mahapphalāni\\
Idam-pi saṅghe ratanaṃ paṇītaṃ\\
Etena saccena suvatthi hotu

\sidepar{5.}%
Ye suppayuttā manasā daḷhena\\
Nikkāmino gotama-sāsanamhi\\
Te patti-pattā amataṃ vigayha\\
Laddhā mudhā nibbutiṃ bhuñjamānā\\
Idam-pi saṅghe ratanaṃ paṇītaṃ\\
Etena saccena suvatthi hotu

Yath'inda-khīlo paṭhaviṃ sito siyā,\\
Catubbhi vātebhi asampakampiyo.\\
Tathūpamaṃ sappurisaṃ vadāmi,\\
Yo ariya-saccāni avecca passati.\\
Idam-pi Saṅghe ratanaṃ paṇītaṃ,\\
Etena saccena suvatthi hotu.

Ye ariya-saccāni vibhāvayanti,\\
Gambhīra-paññena sudesitāni.\\
Kiñ-cāpi te honti bhusappamattā,\\
Na te bhavaṃ aṭṭhamam-ādiyanti.\\
Idam-pi Saṅghe ratanaṃ paṇītaṃ,\\
Etena saccena suvatthi hotu.

Sahā v'assa dassana-sampadāya,\\
Tay'assu dhammā jahitā bhavanti.\\
Sakkāya-diṭṭhi vicikicchitañ-ca,\\
Sīlabbataṃ vā pi yad-atthi kiñci.\\
Catūh'apāyehi ca vippamutto,\\
Cha cābhiṭhānāni abhabbo kātuṃ.\\
Idam-pi Saṅghe ratanaṃ paṇītaṃ,\\
Etena saccena suvatthi hotu.

Kiñ-cāpi so kammaṃ karoti pāpakaṃ,\\
Kāyena vācā uda cetasā vā.\\
Abhabbo so tassa paṭicchadāya,\\
Abhabbatā diṭṭha-padassa vuttā.\\
Idam-pi Saṅghe ratanaṃ paṇītaṃ,\\
Etena saccena suvatthi hotu.

Vanappagumbe yathā phussi-t-agge,\\
Gimhāna-māse paṭhamasmiṃ gimhe.\\
Tathūpamaṃ dhamma-varaṃ adesayi,\\
Nibbāna-gāmiṃ paramaṃ hitāya.\\
Idam-pi Buddhe ratanaṃ paṇītaṃ,\\
Etena saccena suvatthi hotu.

Varo varañ-ñū vara-do var'āharo,\\
Anuttaro dhamma-varaṃ adesayi.\\
Idam-pi Buddhe ratanaṃ paṇītaṃ,\\
Etena saccena suvatthi hotu.

\sidepar{6.}%
Khīṇaṃ purāṇaṃ navaṃ n'atthi sambhavaṃ\\
Viratta-citt'āyatike bhavasmiṃ\\
Te khīṇa-bījā aviruḷhi-chandā\\
Nibbanti dhīrā yathā'yam padīpo\\
Idam-pi saṅghe ratanaṃ paṇītaṃ\\
Etena saccena suvatthi hotu

Yānīdha bhūtāni samāgatāni,\\
Bhummāni vā yāni va antalikkhe.\\
Tathāgataṃ deva-manussa-pūjitaṃ,\\
Buddhaṃ namassāma suvatthi hotu.

Yānīdha bhūtāni samāgatāni,\\
Bhummāni vā yāni va antalikkhe.\\
Tathāgataṃ deva-manussa-pūjitaṃ,\\
Dhammaṃ namassāma suvatthi hotu.

Yānīdha bhūtāni samāgatāni,\\
Bhummāni vā yāni va antalikkhe.\\
Tathāgataṃ deva-manussa-pūjitaṃ,\\
Saṅghaṃ namassāma suvatthi hotū-ti. \suttaRef{Snp 2.1}

\end{paritta}

\subsection{The Six Protective Verses from the Discourse on Treasures}

Whatever wealth in this world or the next,\\
whatever exquisite treasure in the heavens,\\
is not, for us, equal to the Tathāgata.\\
This, too, is an exquisite treasure in the Buddha.\\
By this truth may there be well-being.

The exquisite Deathless -- dispassion, ending --\\
discovered by the Sakyan Sage while in concentration:\\
There is nothing equal to that Dhamma.\\
This, too, is an exquisite treasure in the Dhamma.\\
By this truth may there be well-being.

What the excellent Awakened One extolled as pure\\
and called the concentration of unmediated knowing:\\
No equal to that concentration can be found.\\
This, too, is an exquisite treasure in the Dhamma.\\
By this truth may there be well-being.

The eight persons -- the four pairs --\\
praised by those at peace:\\
They, disciples of the One Well-Gone, deserve offerings.\\
What is given to them bears great fruit.\\
This, too, is an exquisite treasure in the Saṅgha.\\
By this truth may there be well-being.

Those who, devoted, firm-minded,\\
apply themselves to Gotama's message,\\
on attaining their goal, plunge into the Deathless,\\
freely enjoying the Unbinding they've gained.\\
This, too, is an exquisite treasure in the Saṅgha.\\
By this truth may there be well-being.

Ended the old, there is no new taking birth.\\
Dispassioned their minds toward further becoming,\\
they -- with no seed, no desire for growth,\\
enlightened -- go out like this flame.\\
This, too, is an exquisite treasure in the Saṅgha.\\
By this truth may there be well-being.

\subsection{The Buddha's Words on Loving-Kindness}
\label{karaniyam-attha}

% Karaṇīya-metta-sutta

\firstline{Karaṇīyam-attha-kusalena}

\begin{paritta}

Karaṇīyam-attha-kusalena\\
Yan-taṃ santaṃ padaṃ abhisamecca\\
Sakko ujū ca suhujū ca\\
Suvaco c'assa mudu anatimānī

Santussako ca subharo ca\\
Appakicco ca sallahuka-vutti\\
Sant'indriyo ca nipako ca\\
Appagabbho kulesu ananugiddho

Na ca khuddaṃ samācare kiñci\\
Yena viññū pare upavadeyyuṃ\\
Sukhino vā khemino hontu\\
Sabbe sattā bhavantu sukhit'attā

Ye keci pāṇa-bhūt'atthi\\
Tasā vā thāvarā vā anavasesā\\
Dīghā vā ye mahantā vā\\
Majjhimā rassakā aṇuka-thūlā

Diṭṭhā vā ye ca adiṭṭhā\\
Ye ca dūre vasanti avidūre\\
Bhūtā vā sambhavesī vā\\
Sabbe sattā bhavantu sukhit'attā

Na paro paraṃ nikubbetha\\
Nātimaññetha katthaci naṃ kiñci\\
Byārosanā paṭighasaññā\\
Nāññam-aññassa dukkham-iccheyya

Mātā yathā niyaṃ puttaṃ\\
Āyusā eka-puttam-anurakkhe\\
Evam'pi sabba-bhūtesu\\
Mānasam-bhāvaye aparimāṇaṃ

\firstline{Mettañ-ca sabba-lokasmiṃ}

Mettañ-ca sabba-lokasmiṃ\\
Mānasam-bhāvaye aparimāṇaṃ\\
Uddhaṃ adho ca tiriyañ-ca\\
Asambādhaṃ averaṃ asapattaṃ

Tiṭṭhañ-caraṃ nisinno vā\\
Sayāno vā yāvat'assa vigata-middho\\
Etaṃ satiṃ adhiṭṭheyya\\
Brahmam-etaṃ vihāraṃ idham-āhu

Diṭṭhiñca anupagamma\\
Sīlavā dassanena sampanno\\
Kāmesu vineyya gedhaṃ\\
Na hi jātu gabbha-seyyaṃ punaretī'ti

\suttaRef{Snp 1.8}

\end{paritta}

\subsection{The Buddha's Words on Loving-Kindness (English)}

\begin{leader}
  [Now let us chant the Buddha's words on loving-kindness.]
\end{leader}

\firstline{This is what should be done}

[This is what should be done]\\
By one who is skilled in goodness\\
And who knows the path of peace:\\
Let them be able and upright,\\
Straightforward and gentle in speech,

Humble and not conceited,\\
Contented and easily satisfied,\\
Unburdened with duties and frugal in their ways.\\
Peaceful and calm, and wise and skilful,\\
Not proud and demanding in nature.

Let them not do the slightest thing\\
That the wise would later reprove,\\
Wishing: In gladness and in safety,\\
May all beings be at ease.

Whatever living beings there may be,\\
Whether they are weak or strong, omitting none,\\
The great or the mighty, medium, short, or small,

The seen and the unseen,\\
Those living near and far away,\\
Those born and to be born,\\
May all beings be at ease.

Let none deceive another\\
Or despise any being in any state.\\
Let none through anger or ill-will\\
Wish harm upon another.

Even as a mother protects with her life\\
Her child, her only child,\\
So with a boundless heart\\
Should one cherish all living beings,\\
Radiating kindness over the entire world:

Spreading upwards to the skies\\
And downwards to the depths,\\
Outwards and unbounded,\\
Freed from hatred and ill-will.

Whether standing or walking, seated, \\
Or lying down --- free from drowsiness ---\\
One should sustain this recollection.\\
This is said to be the sublime abiding.

By not holding to fixed views,\\
The pure-hearted one, having clarity of vision,\\
Being freed from all sense-desires,\\
Is not born again into this world.

\suttaRef{Snp 1.8}

\subsection{Khandha-parittaṃ}
\label{virupakkhehi}

\firstline{Virūpakkhehi me mettaṃ mettaṃ erāpathehi me}
\firstline{Appamāṇo buddho appamāṇo dhammo}

\begin{twochants}
Virūpakkhehi me mettaṃ & mettaṃ erāpathehi me\\
Chabyā-puttehi me mettaṃ & mettaṃ kaṇhā-gotamakehi ca\\
Apādakehi me mettaṃ & mettaṃ dipādakehi me\\
Catuppadehi me mettaṃ & mettaṃ bahuppadehi me\\
Mā maṃ apādako hiṃsi & mā maṃ hiṃsi dipādako\\
Mā maṃ catuppado hiṃsi & mā maṃ hiṃsi bahuppado\\
Sabbe sattā sabbe pāṇā & sabbe bhūtā ca kevalā\\
Sabbe bhadrāni passantu & mā kiñci pāpam-āgamā\\
Appamāṇo buddho & appamāṇo dhammo\\
Appamāṇo saṅgho & pamāṇavantāni siriṃsapāni\\
Ahi-vicchikā sata-padī & uṇṇā-nābhī sarabhū mūsikā\\
Katā me rakkhā katā me parittā & paṭikkamantu bhūtāni\\
So'haṃ namo bhagavato & namo sattannaṃ\\
Sammā-sambuddhānaṃ & \\
 & \suttaRef{A.II.72-73} \\
\end{twochants}

% Bodhivana suggests AN 4.67 as source

\subsection{The Group Protection}

I have goodwill for the Virupakkhas, the Erapathas,\\
goodwill for the Chabya descendants, and the Black Gotamakas.

I have goodwill for footless beings, two-footed beings,\\
goodwill for four-footed, and many-footed beings.

May footless beings, two-footed beings do me no harm.\\
May four-footed beings and many-footed beings do me no harm.

May all creatures, all breathing things, all beings -- each and every one --\\
meet with good fortune. May none of them come to any evil.

Limitless is the Buddha, limitless the Dhamma, limitless the Saṅgha.

There is a limit to creeping things -- snakes, scorpions, centipedes, spiders,
lizards and rats.

I have made this protection, I have made this spell.\\
May the beings depart.\\
I pay homage to the Blessed One,\\
homage to the seven Rightly Self-awakened Ones.

\subsection{Chaddanta-parittaṃ}
\label{vadhissamenanti}

% The Great Elephant Protection
% The Ivory Protection

\firstline{Vadhissamenanti parāmasanto}

\begin{paritta}

Vadhissamenanti parāmasanto\\
Kāsāvamaddakkhi dhajaṃ isīnaṃ\\
Dukkhena phuṭṭhassudapādi saññā
Arahaddhajo sabbhi avajjharūpo

Sallena viddho byathitopi santo\\
Kāsāvavatthamhi manaṅ na dussayi.\\
Sace imaṃ nāgavarena saccaṃ,\\
Mā maṇ vane bālamigā agañchunti.

\end{paritta}

\subsection{Mora-parittaṃ}
\label{udetayan-cakkhuma}

\firstline{Udet'ayañ-cakkhumā eka-rājā}
\firstline{Apet'ayañ-cakkhumā eka-rājā}

\instr{a.m.}

Udet'ayañ-cakkhumā eka-rājā,\\
Harissa-vaṇṇo paṭhavippabhāso;\\
Taṃ taṃ namassāmi harissa-vaṇṇaṃ paṭhavippabhāsaṃ,\\
Tay'ajja guttā viharemu divasaṃ.

Ye brāhmaṇā veda-gu sabba-dhamme,\\
Te me namo, te ca maṃ pālayantu;\\
Nam'atthu Buddhānaṃ, nam'atthu bodhiyā,\\
Namo vimuttānaṃ, namo vimuttiyā.\\
Imaṃ so parittaṃ katvā,\\
Moro carati esanā'ti.

\instr{p.m.}

Apet'ayañ-cakkhumā eka-rājā,\\
Harissa-vaṇṇo paṭhavippabhāso;\\
Taṃ taṃ namassāmi harissa-vaṇṇaṃ paṭhavippabhāsaṃ,\\
Tay'ajja guttā viharemu rattiṃ.

Ye brāhmaṇā veda-gu sabba-dhamme,\\
Te me namo, te ca maṃ pālayantu;\\
Nam'atthu Buddhānaṃ, nam'atthu bodhiyā,\\
Namo vimuttānaṃ, namo vimuttiyā.\\
Imaṃ so parittaṃ katvā,\\
Moro vāsam-akappayī'ti.

\suttaRef{J.159}

\subsection{The Peacock's Protection}

The One King, rising, with Vision,\\
golden-hued, illuminating the Earth: I pay homage to you,\\
golden-hued, illuminating the Earth.\\
Guarded today by you, may I live through the day.

Those Brahmans who are knowers of all truths,\\
I pay homage to them; may they keep watch over me.\\
Homage to the Awakened Ones. Homage to Awakening.\\
Homage to the Released Ones. Homage to Release.

Having made this protection, the peacock sets out in search for food.

The One King, setting, with Vision,\\
golden-hued, illuminating the Earth: I pay homage to you,\\
golden-hued, illuminating the Earth.\\
Guarded today by you, may I live through the night.

Those Brahmans who are knowers of all truths,\\
I pay homage to them; may they keep watch over me.\\
Homage to the Awakened Ones. Homage to Awakening.\\
Homage to the Released Ones. Homage to Release.

Having made this protection, the peacock arranges his nest.

\subsection{Vaṭṭaka-parittaṃ}
\label{atthi-loke}

\firstline{Atthi loke sīla-guṇo saccaṃ soceyy'anuddayā}

\begin{twochants}
Atthi loke sīla-guṇo & saccaṃ soceyy'anuddayā\\
Tena saccena kāhāmi & sacca-kiriyam-anuttaraṃ\\
Āvajjitvā dhamma-balaṃ & saritvā pubbake jine\\
Sacca-balam-avassāya & sacca-kiriyam-akās'ahaṃ\\
Santi pakkhā apattanā & santi pādā avañcanā\\
Mātā pitā ca nikkhantā & jāta-veda paṭikkama\\
Saha sacce kate mayhaṃ & mahā-pajjalito sikhī\\
Vajjesi soḷasa karīsāni & udakaṃ patvā yathā sikhī\\
Saccena me samo n'atthi & esā me sacca-pāramī ti\\
\end{twochants}

\suttaRef{Cariyāpiṭaka vv.319-322}

\subsection{The Baby Quail's Protection}

There is in this world the quality of virtue,\\
truth, purity, tenderness.\\
In accordance with this truth I will make\\
an unsurpassed vow of truth.

Sensing the strength of the Dhamma,\\
calling to mind the victors of the past,\\
in dependence on the strength of truth,\\
I made an unsurpassed vow of truth:

Here are wings with no feathers;\\
here are feet that can't walk.\\
My mother and father have left me.\\
Fire, go back!

When I made my vow with truth,\\
the great crested flames\\
avoided the sixteen acres around me\\
as if they had come to a body of water.\\
My truth has no equal:\\
Such is my perfection of truth.

\subsection{Buddha-dhamma-saṅgha-guṇā}
\label{iti-pi-so}

\firstline{Iti pi so bhagavā arahaṃ sammā-sambuddho}

\begin{paritta}

Iti pi so bhagavā arahaṃ sammā-sambuddho\\
Vijjā-caraṇa-sampanno sugato loka-vidū\\
Anuttaro purisa-damma-sārathi\\
Satthā devamanussānaṃ buddho bhagavā'ti

Svākkhāto bhagavatā dhammo sandiṭṭhiko\\
\vin akāliko ehi-passiko\\
Opanayiko paccattaṃ veditabbo viññūhī'ti

Supaṭipanno bhagavato sāvaka-saṅgho\\
Uju-paṭipanno bhagavato sāvaka-saṅgho\\
Ñāya-paṭipanno bhagavato sāvaka-saṅgho\\
Sāmīci-paṭipanno bhagavato sāvaka-saṅgho\\
Yad-idaṃ cattāri purisa-yugāni aṭṭha purisa-puggalā\\
Esa bhagavato sāvaka-saṅgho\\
Āhuneyyo pāhuneyyo dakkhiṇeyyo añjali-karaṇīyo\\
Anuttaraṃ puññakkhettaṃ lokassā'ti

\firstline{Araññe rukkha-mūle vā}

\sidepar{\pointerMark}%
Araññe rukkha-mūle vā\\
Suññāgāre va bhikkhavo\\
Anussaretha Sambuddhaṃ\\
Bhayaṃ tumhāka no siyā.\\
No ce Buddhaṃ sareyyātha\\
Loka-jeṭṭhaṃ nar'āsabhaṃ\\
Atha dhammaṃ sareyyātha\\
Niyyānikaṃ sudesitaṃ.\\
No ce dhammaṃ sareyyātha\\
Niyyānikaṃ sudesitaṃ\\
Atha saṅghaṃ sareyyātha\\
Puññakkhettaṃ anuttaraṃ.\\
Evam-Buddhaṃ sarantānaṃ\\
Dhammaṃ saṅghañ-ca bhikkhavo\\
Bhayaṃ vā chambhitattaṃ vā\\
Loma-haṃso na hessatī-ti.

\suttaRef{S.I.219-220}

\end{paritta}

\subsection{Āṭānāṭiya Paritta (short)}
\label{vipassissa}

\firstline{Vipassissa nam'atthu cakkhumantassa sirīmato}

\begin{twochants}
Vipassissa nam'atthu & cakkhumantassa sirīmato\\
Sikhissa pi nam'atthu & sabba-bhūtānukampino\\
Vessabhussa nam'atthu & nhātakassa tapassino\\
Nam'atthu kakusandhassa & māra-senappamaddino\\
Konāgamanassa nam'atthu & brāhmaṇassa vusīmato\\
Kassapassa nam'atthu & vippamuttassa sabbadhi\\
Aṅgīrasassa nam'atthu & sakya-puttassa sirīmato\\
Yo imaṃ dhammam-adesesi & sabba-dukkhāpanūdanaṃ\\
Ye cāpi nibbutā loke & yathā-bhūtaṃ vipassisuṃ\\
Te janā apisuṇā & mahantā vīta-sāradā\\
Hitaṃ deva-manussānaṃ & yaṃ namassanti gotamaṃ\\
Vijjā-caraṇa-sampannaṃ & mahantaṃ vīta-sāradaṃ\\
Vijjā-caraṇa-sampannaṃ & buddhaṃ vandāma gotaman'ti\\
\end{twochants}

\suttaRef{D.III.195-196}

\subsection{Homage to the Seven Past Buddhas}

% English source: Bodhivana

Homage to Vipassī, possessed of vision and splendor.

Homage to Sikhī, sympathetic to all beings.

Homage to Vesabhū, cleansed, austere.

Homage to Kakusandha, crusher of Māra's host.

Homage to Konāgamana, the Brahman who lived the life perfected.

Homage to Kassapa, everywhere released.

Homage to Aṅgīrasa, splendid son of the Sakyans,

Who taught this Dhamma -- the dispelling of all stress.

Those unbound in the world, who have seen things as they have come to be,

Great Ones of gentle speech, thoroughly mature:

Even they pay homage to Gotama, the benefit of human and heavenly beings,

consummate in knowledge and conduct, the Great One, thoroughly mature.

We revere the Buddha Gotama, consummate in knowledge and conduct.

\subsection{Sacca-kiriyā-gāthā}
\label{natthi-me}

\firstline{Natthi me saraṇaṃ aññaṃ}

Natthi me saraṇaṃ aññaṃ buddho me saraṇaṃ varaṃ\\
Etena sacca-vajjena sotthi te/me hotu sabbadā

Natthi me saraṇaṃ aññaṃ dhammo me saraṇaṃ varaṃ\\
Etena sacca-vajjena sotthi te/me hotu sabbadā

Natthi me saraṇaṃ aññaṃ saṅgho me saraṇaṃ varaṃ\\
Etena sacca-vajjena sotthi te/me hotu sabbadā

\subsection{Yaṅkiñci ratanaṃ loke}
\label{yankinci-ratanam}

\firstline{Yaṅkiñci ratanaṃ loke}

\begin{twochants}
  Yaṅkiñci ratanaṃ loke & vijjati vividhaṃ puthu\\
  Ratanaṃ buddhasamaṃ & natthi tasmā sotthī bhavantu te\\
  Yaṅkiñci ratanaṃ loke & vijjati vividhaṃ puthu\\
  Ratanaṃ dhammasamaṃ & natthi tasmā sotthī bhavantu te\\
  Yaṅkiñci ratanaṃ loke & vijjati vividhaṃ puthu\\
  Ratanaṃ saṅghasamaṃ & natthi tasmā sotthī bhavantu te\\
\end{twochants}

\subsection{Sakkatvā buddharatanaṃ}
\label{sakkatva}

\firstline{Sakkatvā buddharatanaṃ}

\begin{twochants}
  Sakkatvā buddharatanaṃ & osathaṃ uttamaṃ varaṃ\\
  Hitaṃ devamanussānaṃ & buddhatejena sotthinā\\
  Nassantupaddavā sabbe & dukkhā vūpasamentu te\\
  Sakkatvā dhammaratanaṃ & osathaṃ uttamaṃ varaṃ\\
  Pariḷāhūpasamanaṃ & dhammatejena sotthinā\\
  Nassantupaddavā sabbe & bhayā vūpasamentu te\\
  Sakkatvā saṅgharatanaṃ & osathaṃ uttamaṃ varaṃ\\
  Āhuneyyaṃ pāhuneyyaṃ & saṅghatejena sotthinā\\
  Nassantupaddavā sabbe & rogā vūpasamentu te\\
\end{twochants}

{\centering
  \emph{The \emph{djet-damnahn} sequence ends here and continues with the closing sequence.}
\par}

\subsection{Having Revered}

% Sakkatvā buddharatanaṃ

% English source: Bodhivana

Having revered the jewel of the Buddha, the highest, most excellent medicine,
the welfare of human and heavenly beings: Through the Buddha's majesty and
safety, may all obstacles vanish. May your sufferings grow totally calm.

Having revered the jewel of the Dhamma, the highest, most excellent medicine,
the stiller of feverish passion: Through the Dhamma's majesty and safety, may
all obstacles vanish. May your fears grow totally calm.

Having revered the jewel of the Saṅgha, the highest, most excellent medicine,
worthy of gifts, worthy of hospitality: Through the Saṅgha's majesty and safety,
may all obstacles vanish. May your diseases grow totally calm.

\subsection{Aṅguli-māla-parittaṃ}
\label{yato-ham-bhagini}

\firstline{Yato'haṃ bhagini ariyāya jātiyā jāto}

\begin{paritta}
Yato'haṃ bhagini ariyāya jātiyā jāto\\
Nābhijānāmi sañcicca pāṇaṃ jīvitā voropetā\\
Tena saccena sotthi te hotu sotthi gabbhassa\\
\suttaRef{M.II.103}
\end{paritta}

% MN 86:14

% English source: Bodhivana

\begin{english}
  Sister, since being born in the Noble Birth,\\
  I am not aware that I have intentionally deprived a being of life.\\
  By this truth may you be well,\\
  and so may the child in your womb.
\end{english}

\subsection{Bojjh'aṅga-parittaṃ}
\label{bojjhango}

\firstline{Bojjh'aṅgo sati-saṅkhāto}

\begin{twochants}
Bojjh'aṅgo sati-saṅkhāto & dhammānaṃ vicayo tathā\\
Viriyam-pīti-passaddhi & bojjh'aṅgā ca tathā'pare\\
Samādh'upekkha-bojjh'aṅgā & satt'ete sabba-dassinā\\
Muninā sammad-akkhātā & bhāvitā bahulīkatā\\
Saṃvattanti abhiññāya & nibbānāya ca bodhiyā\\
Etena sacca-vajjena & sotthi te hotu sabbadā\\
Ekasmiṃ samaye nātho & moggallānañ-ca kassapaṃ\\
Gilāne dukkhite disvā & bojjh'aṅge satta desayi\\
Te ca taṃ abhinanditvā & rogā mucciṃsu taṅ-khaṇe\\
Etena sacca-vajjena & sotthi te hotu sabbadā\\
Ekadā dhamma-rājā pi & gelaññenābhipīḷito\\
Cundattherena tañ-ñeva & bhaṇāpetvāna sādaraṃ\\
Sammoditvā ca ābādhā & tamhā vuṭṭhāsi ṭhānaso\\
Etena sacca-vajjena & sotthi te hotu sabbadā\\
Pahīnā te ca ābādhā & tiṇṇannam-pi mahesinaṃ\\
Magg'āhata-kilesā va & pattānuppatti-dhammataṃ\\
Etena sacca-vajjena & sotthi te hotu sabbadā\\
\end{twochants}

\suttaRef{S.V.80f}

% SN 46.14

\subsection{The Factors for Awakening Protection}

% English source: Bodhivana

The factors for Awakening include: mindfulness, analysis of qualities,
persistence, rapture, and calm as factors for Awakening, plus concentration and
equanimity.

These seven, which the All-seeing Sage has rightly taught, when developed and
matured, bring about heightened knowledge, Unbinding and Awakening.

By the utterance of this truth, may you always be well.

At one time, our Protector -- seeing that Moggallāna and Kassapa were sick and
in pain -- taught them the seven factors for Awakening.

They, delighting in that, were instantly freed from their illness.

By the utterance of this truth, may you always be well.

Once, when the Dhamma King was afflicted with fever, he had the Elder Cunda
recite that very teaching with devotion. And as he approved, he rose up from
that disease.

By the utterance of this truth, may you always be well.

Those diseases were abandoned by the three great seers, just as defilements are
demolished by the Path in accordance with step-by-step attainment.

By the utterance of this truth, may you always be well.

\subsection{Abhaya-parittaṃ}
\label{yan-dunnimittam}

\firstline{Yan-dunnimittaṃ avamaṅgalañ-ca}

\begin{paritta}
Yan-dunnimittaṃ avamaṅgalañ-ca\\
Yo cāmanāpo sakuṇassa saddo\\
Pāpaggaho dussupinaṃ akantaṃ\\
Buddhānubhāvena vināsamentu

Yan-dunnimittaṃ avamaṅgalañ-ca\\
Yo cāmanāpo sakuṇassa saddo\\
Pāpaggaho dussupinaṃ akantaṃ\\
Dhammānubhāvena vināsamentu

Yan-dunnimittaṃ avamaṅgalañ-ca\\
Yo cāmanāpo sakuṇassa saddo\\
Pāpaggaho dussupinaṃ akantaṃ\\
Saṅghānubhāvena vināsamentu \suttaRef{Trad.}
\end{paritta}

{\centering
  \emph{The \emph{sipsong-damnahn} sequence ends here and continues with the closing sequence.}
\par}

\subsection{The Danger-free Protection}

% English source: Bodhivana

Whatever unlucky portents and ill omens,\\
and whatever distressing bird calls,\\
evil planets, upsetting nightmares:

By the Buddha's power may they be destroyed.

Whatever unlucky portents and ill omens,\\
and whatever distressing bird calls,\\
evil planets, upsetting nightmares:

By the Dhamma's power may they be destroyed.

Whatever unlucky portents and ill omens,\\
and whatever distressing bird calls,\\
evil planets, upsetting nightmares:

By the Saṅgha's power may they be destroyed.

\section{Closing Sequence}

\subsection{Devatā-uyyojana-gāthā}
\label{dukkhappatta}

\firstline{Dukkhappattā ca niddukkhā}
\firstline{Sabbe buddhā balappattā}

\begin{twochants}
Dukkhappattā ca niddukkhā & bhayappattā ca nibbhayā\\
Sokappattā ca nissokā & hontu sabbe pi pāṇino\\
Ettāvatā ca amhehi & sambhataṃ puñña-sampadaṃ\\
Sabbe devānumodantu & sabba-sampatti-siddhiyā\\
Dānaṃ dadantu saddhāya & sīlaṃ rakkhantu sabbadā\\
Bhāvanābhiratā hontu & gacchantu devatā-gatā\\\relax
[Sabbe buddhā] balappattā & paccekānañ-ca yaṃ balaṃ\\
Arahantānañ-ca tejena & rakkhaṃ bandhāmi sabbaso\\
\end{twochants}

%\suttaRef{MJG}

\subsection{Inciting the Devas}

% English source: Bodhivana

May all beings: who have fallen into suffering be without suffering,\\
who have fallen into danger be without danger,\\
who have fallen into sorrow be without sorrow.

For the sake of all attainment and success, may all heavenly beings\\
rejoice in the extent to which we have gathered a consummation of merit.

May they give gifts with conviction, may they always maintain virtue.\\
May they delight in meditation. May they go to a heavenly destination.

From the strength attained by all the Buddhas,\\
the strength of the Private Buddhas,\\
by the majesty of the arahants,\\
I bind this protection all around.

\subsection{Jaya-maṅgala-aṭṭha-gāthā}
\label{bahum}

\firstline{Bāhuṃ sahassam-abhinimmita sāvudhan-taṃ}

\begin{paritta}
Bāhuṃ sahassam-abhinimmita sāvudhan-taṃ\\
Grīmekhalaṃ udita-ghora-sasena-māraṃ\\
Dān'ādi-dhamma-vidhinā jitavā mun'indo\\
Tan-tejasā bhavatu te jaya-maṅgalāni

Mārātirekam-abhiyujjhita-sabba-rattiṃ\\
Ghoram-pan'āḷavakam-akkhama-thaddha-yakkhaṃ\\
Khantī-sudanta-vidhinā jitavā mun'indo\\
Tan-tejasā bhavatu te jaya-maṅgalāni

Nāḷāgiriṃ gaja-varaṃ atimatta-bhūtaṃ\\
Dāv'aggi-cakkam-asanīva sudāruṇan-taṃ\\
Mett'ambu-seka-vidhinā jitavā mun'indo\\
Tan-tejasā bhavatu te jaya-maṅgalāni

Ukkhitta-khaggam-atihattha-sudāruṇan-taṃ\\
Dhāvan-ti-yojana-path'aṅguli- mālavantaṃ\\
Iddhī'bhisaṅkhata-mano jitavā mun'indo\\
Tan-tejasā bhavatu te jaya-maṅgalāni

Katvāna kaṭṭham-udaraṃ iva gabbhinīyā\\
Ciñcāya duṭṭha-vacanaṃ jana-kāya majjhe\\
Santena soma-vidhinā jitavā mun'indo\\
Tan-tejasā bhavatu te jaya-maṅgalāni

Saccaṃ vihāya-mati-saccaka-vāda-ketuṃ\\
Vādābhiropita-manaṃ ati-andha-bhūtaṃ\\
Paññā-padīpa-jalito jitavā mun'indo\\
Tan-tejasā bhavatu te jaya-maṅgalāni

Nandopananda-bhujagaṃ vibudhaṃ mah'iddhiṃ\\
Puttena thera-bhujagena damāpayanto\\
Iddhūpadesa-vidhinā jitavā mun'indo\\
Tan-tejasā bhavatu te jaya-maṅgalāni

Duggāha-diṭṭhi-bhujagena sudaṭṭha-hatthaṃ\\
Brahmaṃ visuddhi-jutim-iddhi-bakābhidhānaṃ\\
Ñāṇāgadena vidhinā jitavā mun'indo\\
Tan-tejasā bhavatu te jaya-maṅgalāni

Etā pi buddha-jaya-maṅgala-aṭṭha-gāthā\\
Yo vācano dina-dine saratem-atandī\\
Hitvān'aneka-vividhāni c'upaddavāni\\
Mokkhaṃ sukhaṃ adhigameyya naro sapañño \suttaRef{Trad.}
\end{paritta}

\subsection{The Verses of the Buddha's Blessings of Victory}

% English source: Bodhivana

Creating a form with a thousand arms, each equipped with a weapon,\\
Māra, on the elephant Girimekhala,\\
uttered a frightening roar together with his troops.\\
The Lord of Sages defeated him by means of such qualities as generosity:\\
By the majesty of this, may you have blessings of victory.

Even more frightful than Māra making war all night,\\
was Āḷavaka, the arrogant unstable ogre.\\
The Lord of Sages defeated him by means of well-trained endurance:\\
By the majesty of this, may you have blessings of victory.

Nāḷāgiri, the excellent elephant, when maddened,\\
was very horrific, like a forest fire, a flaming discus, a lightning bolt.\\
The Lord of Sages defeated him by sprinkling the water of goodwill:\\
By the majesty of this, may you have blessings of victory.

Very horrific, with a sword upraised in his expert hand,\\
Garlanded-with-Fingers ran three leages along the path.\\
The Lord of Sages defeated him with mind-fashioned marvels:\\
By the majesty of this, may you have blessings of victory.

Having made a wooden belly to appear pregnant,\\
Ciñcā made a lewd accusation in the midst of the gathering.\\
The Lord of Sages defeated her with peaceful, gracious means:\\
By the majesty of this, may you have blessings of victory.

Saccaka, whose provocative views had abandoned the truth,\\
his mind delighting in argument, had become thoroughly blind.\\
The Lord of Sages defeated him with the light of discernment:\\
By the majesty of this, may you have blessings of victory.

Nandopananda was a serpent with great power but wrong views.\\
The Lord of Sages defeated him by means of a display of marvels,\\
sending his son (Moggallāna), the serpent-elder, to tame him:\\
By the majesty of this, may you have blessings of victory.

His hands bound tight by the serpent of wrongly held views,\\
Baka, the Brahmā, thought himself pure in his radiance and power.\\
The Lord of Sages defeated him by means of his words of knowledge:
By the majesty of this, may you have blessings of victory.

These eight verses of the Buddha's blessings of victory:\\
Whatever person of discernment\\
recites or recalls them day after day without lapsing,\\
destroying all kinds of obstacles,\\
will attain liberation and happiness.

\subsection{Jaya-parittaṃ}
\label{maha-karuniko}

\firstline{Mahā-kāruṇiko nātho hitāya sabba-pāṇinaṃ}

\begin{twochants}
Mahā-kāruṇiko nātho & hitāya sabba-pāṇinaṃ\\
Pūretvā pāramī sabbā & patto sambodhim-uttamaṃ\\
Etena sacca-vajjena & hotu te jaya-maṅgalaṃ\\
Jayanto bodhiyā mūle & sakyānaṃ nandi-vaḍḍhano\\
Evaṃ tvaṃ vijayo hohi & jayassu jaya-maṅgale\\
Aparājita-pallaṅke & sīse paṭhavi-pokkhare\\
Abhiseke sabba-buddhānaṃ & aggappatto pamodati\\
Sunakkhattaṃ sumaṅgalaṃ & supabhātaṃ suhuṭṭhitaṃ\\
Sukhaṇo sumuhutto ca & suyiṭṭhaṃ brahma-cārisu\\
Padakkhiṇaṃ kāya-kammaṃ & vācā-kammaṃ padakkhiṇaṃ\\
Padakkhiṇaṃ mano-kammaṃ & paṇidhi te padakkhiṇā\\
Padakkhiṇāni katvāna & labhant'atthe padakkhiṇe
\end{twochants}

\suttaRef{A.I.294}

% AN 3.156

\subsection{Victory Protection}

% English source: Bodhivana

(The Buddha), our protector, with great compassion,\\
for the welfare of all beings,\\
having fulfilled all the perfections,\\
attained the highest self-awakening.\\
By the utterance of this truth,\\
may you have a blessing of victory.

Victorious at the foot of the Bodhi tree,\\
was he who increased the Sakyans' delight.\\
May you have the same sort of victory.\\
May you win blessings of victory.

At the head of the lotus leaf of the world\\
on the undefeated seat\\
consecrated by all the Buddhas,\\
he rejoiced in the utmost attainment.

A lucky star it is, a lucky blessing,\\
a lucky dawn, a lucky sacrifice,\\
a lucky instant, a lucky moment,\\
a lucky offering: i.e., a rightful bodily act\\
a rightful verbal act, a rightful mental act,\\
your rightful intentions\\
with regard to those who lead the holy life.\\
Doing these rightful things,
your rightful aims are achieved.

\subsection{So attha-laddho}

\firstline{So attha-laddho sukhito viruḷho buddha-sāsane}

\begin{twochants}
So attha-laddho sukhito & viruḷho buddha-sāsane;\\
Arogo sukhito hohi & saha sabbehi ñātibhi. (×3)\\
\end{twochants}

\begin{english}
  May he gain in his aims, be happy, and flourish in the Buddha's teachings. May
  you, together with all your relatives, be happy and free from disease.
\end{english}

\subsection{Sā attha-laddhā}

\firstline{Sā attha-laddhā sukhitā viruḷhā buddha-sāsane}

\begin{twochants}
Sā attha-laddhā sukhitā & viruḷhā buddha-sāsane;\\
Arogā sukhitā hohi & saha sabbehi ñātibhi. (×3)\\
\end{twochants}

\subsection{Te attha-laddhā sukhitā}
\label{te-attha-laddha}

\firstline{Te attha-laddhā sukhitā viruḷhā buddha-sāsane}

\begin{twochants}
Te attha-laddhā sukhitā & viruḷhā buddha-sāsane;\\
Arogā sukhitā hotha & saha sabbehi ñātibhi. (×3)\\
\end{twochants}

\suttaRef{A.I.294}

\subsection{Bhavatu sabba-maṅgalaṃ}
\label{bhavatu}

\firstline{Bhavatu sabba-maṅgalaṃ}

Bhavatu sabba-maṅgalaṃ, rakkhantu sabba-devatā\\
Sabba-buddhānubhāvena, sadā sotthī bhavantu te

Bhavatu sabba-maṅgalaṃ, rakkhantu sabba-devatā\\
Sabba-dhammānubhāvena, sadā sotthī bhavantu te

Bhavatu sabba-maṅgalaṃ, rakkhantu sabba-devatā\\
Sabba-saṅghānubhāvena, sadā sotthī bhavantu te

\section{Mahā-kāruṇiko nātho ti ādikā gāthā}

\firstline{Mahā-kāruṇiko nātho atthāya sabba-pāṇinaṃ}

\begin{paritta}
Mahā-kāruṇiko nātho\\
Atthāya sabba-pāṇinaṃ\\
Hitāya sabba-pāṇinaṃ\\
Sukhāya sabba-pāṇinaṃ

Pūretvā pāramī sabbā\\
Patto sambodhim-uttamaṃ\\
Etena sacca-vajjena\\
Mā hontu sabb'upaddavā.
\end{paritta}

%\suttaRef{MJG}

\clearpage

\section{Āṭānāṭiya Paritta (long)}

% The Twenty-Eight Buddhas' Protection
% Āṭānāṭiya Paritta

\begin{leader}
\soloinstr{(Solo introduction)}

\firstline{Appasannehi nāthassa sāsane sādhusammate}

\begin{solotwochants}
  Appasannehi nāthassa & sāsane sādhusammate\\
  Amanussehi caṇḍehi & sadā kibbisakāribhi\\
  Parisānañca-tassannam & ahiṃsāya ca guttiyā\\
  Yandesesi mahāvīro & parittan-tam bhaṇāma se.\\
\end{solotwochants}
\end{leader}

\firstline{Namo me sabbabuddhānaṃ}

\instr{(If started with \emph{Vipassissa\ldots}, continue below without the solo
  introduction.)}

\enlargethispage{\baselineskip}

\begin{twochants}
  [Namo me sabbabuddhānaṃ] & uppannānaṃ mahesinaṃ\\
  Taṇhaṅkaro mahāvīro & medhaṅkaro mahāyaso\\
  Saraṇaṅkaro lokahito & dīpaṅkaro jutindharo\\
  Koṇḍañño janapāmokkho & maṅgalo purisāsabho\\
  Sumano sumano dhīro & revato rativaḍḍhano\\
  Sobhito guṇasampanno & anomadassī januttamo\\
  Padumo lokapajjoto & nārado varasārathī\\
  Padumuttaro sattasāro & sumedho appaṭipuggalo\\
  Sujāto sabbalokaggo & piyadassī narāsabho\\
  Atthadassī kāruṇiko & dhammadassī tamonudo\\
  Siddhattho asamo loke & tisso ca vadataṃ varo\\
  Phusso ca varado buddho & vipassī ca anūpamo\\
  Sikhī sabbahito satthā & vessabhū sukhadāyako\\
  Kakusandho satthavāho & koṇāgamano raṇañjaho\\
  Kassapo sirisampanno & gotamo sakyapuṅgavo\\
\end{twochants}

\clearpage

\enlargethispage{\baselineskip}

\begin{twochants}
  Ete caññe ca sambuddhā & anekasatakoṭayo\\
  Sabbe buddhā asamasamā & sabbe buddhā mahiddhikā\\
  Sabbe dasabalūpetā & vesārajjehupāgatā\\
  Sabbe te paṭijānanti & āsabhaṇṭhānamuttamaṃ\\
  Sīhanādaṃ nadantete & parisāsu visāradā\\
  Brahmacakkaṃ pavattenti & loke appaṭivattiyaṃ\\
  Upetā buddhadhammehi & aṭṭhārasahi nāyakā\\
  Dvattiṃsa-lakkhaṇūpetā & sītyānubyañjanādharā\\
  Byāmappabhāya suppabhā & sabbe te munikuñjarā\\
  Buddhā sabbaññuno ete & sabbe khīṇāsavā jinā\\
  Mahappabhā mahātejā & mahāpaññā mahabbalā\\
  Mahākāruṇikā dhīrā & sabbesānaṃ sukhāvahā\\
  Dīpā nāthā patiṭṭhā & ca tāṇā leṇā ca pāṇinaṃ\\
  Gatī bandhū mahassāsā & saraṇā ca hitesino\\
  Sadevakassa lokassa & sabbe ete parāyanā\\
  Tesāhaṃ sirasā pāde & vandāmi purisuttame\\
  Vacasā manasā ceva & vandāmete tathāgate\\
  Sayane āsane ṭhāne & gamane cāpi sabbadā\\
  Sadā sukhena rakkhantu & buddhā santikarā tuvaṃ\\
  Tehi tvaṃ rakkhito santo & mutto sabbabhayena ca\\
\end{twochants}

\clearpage

\savenotes

\firstline{Tesaṃ saccena sīlena khantimettābalena ca}

\begin{twochants}
  Sabba-rogavinimutto & sabba-santāpavajjito\\
  Sabba-veramatikkanto & nibbuto ca tuvaṃ bhava\\
  Tesaṃ saccena sīlena & khantimettābalena ca\\
  Tepi tumhe%
  \footnote{If chanting for oneself, change \textit{tumhe} to \textit{amhe} here and in the lines below.}
  anurakkhantu & ārogyena sukhena ca\\
  Puratthimasmiṃ disābhāge & santi bhūtā mahiddhikā\\
  Tepi tumhe anurakkhantu & ārogyena sukhena ca\\
  Dakkhiṇasmiṃ disābhāge & santi devā mahiddhikā\\
  Tepi tumhe anurakkhantu & ārogyena sukhena ca\\
  Pacchimasmiṃ disābhāge & santi nāgā mahiddhikā\\
  Tepi tumhe anurakkhantu & ārogyena sukhena ca\\
  Uttarasmiṃ disābhāge & santi yakkhā mahiddhikā\\
  Tepi tumhe anurakkhantu & ārogyena sukhena ca\\
  Purimadisaṃ dhataraṭṭho & dakkhiṇena viruḷhako\\
  Pacchimena virūpakkho & kuvero uttaraṃ disaṃ\\
  Cattāro te mahārājā & lokapālā yasassino\\
  Tepi tumhe anurakkhantu & ārogyena sukhena ca\\
  Ākāsaṭṭhā ca bhummaṭṭhā & devā nāgā mahiddhikā\\
  Tepi tumhe anurakkhantu & ārogyena sukhena ca\\
\end{twochants}

\spewnotes

\subsection{Natthi me saraṇaṃ aññaṃ}

\firstline{Natthi me saraṇaṃ aññaṃ}

\savenotes

\begin{twochants}
  Natthi me saraṇaṃ aññaṃ & buddho me saraṇaṃ varaṃ\\
  Etena saccavajjena & hotu te%
  \footnote{If chanting for oneself, change \textit{te} to \textit{me} here and in the lines below.}
  jayamaṅgalaṃ\\
  Natthi me saraṇaṃ aññaṃ & dhammo me saraṇaṃ varaṃ\\
  Etena saccavajjena & hotu te jayamaṅgalaṃ\\
  Natthi me saraṇaṃ aññaṃ & saṅgho me saraṇaṃ varaṃ\\
  Etena saccavajjena & hotu te jayamaṅgalaṃ\\
\end{twochants}

\spewnotes

\subsection{Yaṅkiñci ratanaṃ loke}

\firstline{Yaṅkiñci ratanaṃ loke vijjati vividhaṃ puthu}

\begin{twochants}
  Yaṅkiñci ratanaṃ loke & vijjati vividhaṃ puthu\\
  Ratanaṃ buddhasamaṃ & natthi tasmā sotthī bhavantu te\\
  Yaṅkiñci ratanaṃ loke & vijjati vividhaṃ puthu\\
  Ratanaṃ dhammasamaṃ & natthi tasmā sotthī bhavantu te\\
  Yaṅkiñci ratanaṃ loke & vijjati vividhaṃ puthu\\
  Ratanaṃ saṅghasamaṃ & natthi tasmā sotthī bhavantu te\\
\end{twochants}

\subsection{Sakkatvā}

\firstline{Sakkatvā buddha-ratanaṃ osathaṃ uttamaṃ varaṃ}

\begin{twochants}
  Sakkatvā buddharatanaṃ & osathaṃ uttamaṃ varaṃ\\
  Hitaṃ devamanussānaṃ & buddhatejena sotthinā\\
  Nassantupaddavā sabbe & dukkhā vūpasamentu te\\
  Sakkatvā dhammaratanaṃ & osathaṃ uttamaṃ varaṃ\\
  Pariḷāhūpasamanaṃ & dhammatejena sotthinā\\
  Nassantupaddavā sabbe & bhayā vūpasamentu te\\
  Sakkatvā saṅgharatanaṃ & osathaṃ uttamaṃ varaṃ\\
  Āhuneyyaṃ pāhuneyyaṃ & saṅghatejena sotthinā\\
  Nassantupaddavā sabbe & rogā vūpasamentu te\\
\end{twochants}

\subsection{Sabbītiyo vivajjantu}

\firstline{Sabbītiyo vivajjantu sabbarogo vinassatu}

\begin{twochants}
  Sabbītiyo vivajjantu & sabbarogo vinassatu\\
  Mā te bhavatvantarāyo & sukhī dīghāyuko bhava\\
  Abhivādanasīlissa & niccaṃ vuḍḍhāpacāyino\\
  Cattāro dhammā vaḍḍhanti & āyu vaṇṇo sukhaṃ balaṃ\\
\end{twochants}

%\suttaRef{MJG}


\section{The Twenty-Eight Buddhas' Protection}

{\setlength{\parskip}{0pt}%
  \soloinstr{Solo introduction}

  \begin{soloonechants}
    We will now recite the discourse given by the Great Hero\\
    (the Buddha), as a protection for virtue-loving human beings,\\
    Against harm from all evil-doing, malevolent non-humans who are\\
    displeased with the Buddha's Teachings.\\
  \end{soloonechants}%
}

Homage to all Buddhas, the mighty who have arisen:\\
Taṇhaṅkara, the great hero, Medhaṅkara, the renowned,\\
Saraṇaṅkara, who guarded the world, Dīpaṅkara, the light-bearer,\\
Koṇḍañña, liberator of people, Maṅgala, great leader of people,\\
Sumana, kindly and wise, Revata, increaser of joy,\\
Sobhita, perfected in virtues, Anomadassī, greatest of beings,\\
Paduma, illuminer of the world, Nārada, true charioteer,\\
Padumuttara, most excellent of beings, Sumedha, the unequalled one,\\
Sujāta, summit of the world,  Piyadassī, great leader of men,\\
Atthadassī, the compassionate, Dhammadassī, destroyer of darkness,\\
Siddhattha, unequalled in the world,  and Tissa, speaker of Truth,\\
Phussa, bestower of blessings, Vipassī, the incomparable,\\
Sikhī, the bliss-bestowing teacher, Vessabhū, giver of happiness,\\
Kakusandha, the caravan leader, Koṇāgamana, abandoner of ills,\\
Kassapa, perfect in glory, Gotama, chief of the Sakyans.

These and all self-enlightened Buddhas are also peerless ones,\\
All the Buddhas together, all of mighty power,\\
All endowed with the Ten Powers, attained to highest knowledge,\\
All of these are accorded the supreme place of leadership.\\
They roar the lion's roar with confidence among their followers,\\
They observe with the divine eye, unhindered, all the world.\\
The leaders endowed with the eighteen kinds of Buddha-Dhamma,\\
The thirty-two major and eighty minor marks of a great being,\\
Shining with fathom-wide haloes, all these elephant-like sages,\\
All these omniscient Buddhas, conquerors free of corruption,\\
Of mighty brilliance, mighty power, of mighty wisdom, mighty strength,\\
Of mighty compassion and wisdom, bearing bliss to all,\\
Islands, guardians and supports, shelters and caves for all beings,\\
Resorts, kinsmen and comforters, benevolent givers of refuge,\\
These are all the final resting place for the world with its deities.\\
With my head at their feet I salute these greatest of humans.\\
With both speech and thought I venerate those Tathāgatas,\\
Whether lying down, seated or standing, or walking anywhere.\\
May they ever guard your happiness, the Buddhas, bringers of peace,\\
And may you, guarded by them, at peace, freed from all fear,\\
Released from all illness, safe from all torments,\\
Having transcended hatred, may you gain cessation.

By the power of their truth, their virtue and love,\\
May they protect and guard you in health and happiness.\\
In the Eastern quarter are beings of great power,\\
May they protect and guard you in health and happiness.\\
In the Southern quarter are deities of great power,\\
May they protect and guard you in health and happiness.\\
In the Western quarter are dragons of great power,\\
May they protect and guard you in health and happiness.\\
In the Northern quarter are spirits of great power,\\
May they protect and guard you in health and happiness.\\
In the East is Dhataraṭṭha, in the South is Viruḷhaka,\\
In the West is Virūpakkha, Kuvera rules the North.\\
These Four Mighty Kings, far-famed guardians of the world,\\
May they all be your protectors in health and happiness.\\
Sky-dwelling and earth-dwelling gods and dragons of great power,\\
May they all be your protectors in health and happiness.\\
For me there is no other refuge, the Buddha is my excellent refuge:\\
By this declaration of truth may the blessings of victory be yours.\\
For me there is no other refuge, the Dhamma is my excellent refuge:\\
By this declaration of truth may the blessings of victory be yours.\\
For me there is no other refuge, the Saṅgha is my excellent refuge:\\
By this declaration of truth may the blessings of victory be yours.

Whatever jewel may be found in the world, however splendid,\\
There is no jewel equal to the Buddha, therefore may you be blessed.\\
Whatever jewel may be found in the world, however splendid,\\
There is no jewel equal to the Dhamma, therefore may you be blessed.\\
Whatever jewel may be found in the world, however splendid,\\
There is no jewel equal to the Saṅgha, therefore may you be blessed.\\
If you venerate the Buddha jewel, the supreme, excellent protection,\\
Which benefits gods and humans, then in safety, by the Buddha's power,\\
All dangers will be prevented, your sorrows will pass away.\\
If you venerate the Dhamma jewel, the supreme, excellent protection,\\
Which calms all fevered states, then in safety, by the Dhamma's power,\\
All dangers will be prevented, your fears will pass away.\\
If you venerate the Saṅgha jewel, the supreme, excellent protection,\\
Worthy of gifts and hospitality, then in safety, by the Saṅgha's power,\\
All dangers will be prevented, your sicknesses will pass away.\\
May all calamities be avoided, may all illness pass away,\\
May no dangers threaten you, may you be happy and long-lived,\\
Greeted kindly and welcome everywhere.\\
May four things accrue to you: long life, beauty, bliss, and strength.

\section{Pabbatopama-gāthā}

\firstline{Yathā pi selā vipulā nabhaṃ āhacca pabbatā}

\begin{twochants}
Yathā pi selā vipulā & nabhaṃ āhacca pabbatā;\\
Samantā anupariyeyyuṃ & nippothentā catuddisā;\\
Evaṃ jarā ca maccu ca & adhivattanti pāṇino;\\
Khattiye brāhmaṇe vesse & sudde caṇḍāla-pukkuse;\\
Na kiñci parivajjeti & sabbam-evābhimaddati;\\
Na tattha hatthīnaṃ bhūmi & na rathānaṃ na pattiyā;\\
Na cāpi manta-yuddhena & sakkā jetuṃ dhanena vā;\\
Tasmā hi paṇḍito poso & sampassaṃ attham-attano;\\
Buddhe Dhamme ca Saṅghe ca & dhīro saddhaṃ nivesaye;\\
Yo Dhamma-cārī kāyena & vācāya uda cetasā;\\
Idh'eva naṃ pasaṃsanti & pecca sagge pamodati.
\end{twochants}

\suttaRef{S.I.102}

\section{Verses on the Burden}

\firstline{Bhārā have pañcakkhandhā}

\begin{leader}
  [Handa mayaṃ bhāra-sutta-gāthāyo bhaṇāmase]
\end{leader}

\begin{twochants}
  Bhārā have pañcakkhandhā & bhāra-hāro ca puggalo \\
  Bhār'ādānaṃ dukkhaṃ loke & bhāra-nikkhepanaṃ sukhaṃ \\
\end{twochants}

\begin{english}
  The five aggregates indeed are burdens,\\
  The beast of burden though is man.\\
  In this world to take up burdens is dukkha.\\
  Putting them down brings happiness.
\end{english}

\begin{twochants}
  Nikkhipitvā garuṃ bhāraṃ & aññaṃ bhāraṃ anādiya \\
  Samūlaṃ taṇhaṃ abbuyha & nicchāto parinibbuto \\
\end{twochants}

\begin{english}
  A heavy burden cast away,\\
  Not taking on another load,\\
  With craving pulled out from the root,\\
  Desires stilled, one is released.
\end{english}

% SN 22.22
\suttaRef{S.III.26}

\section{True and False Refuges}

% Khemākhema-saraṇa-gamana-paridīpikā-gāthā

\firstline{Bahuṃ ve saraṇaṃ yanti pabbatāni vanāni ca}

\begin{leader}
  [Handa mayaṃ khemākhema-saraṇa-gamana-\\
  -paridīpikā-gāthāyo bhaṇāmase]
\end{leader}

\begin{twochants}
  Bahuṃ ve saraṇaṃ yanti & pabbatāni vanāni ca \\
  Ārāma-rukkha-cetyāni & manussā bhaya-tajjitā \\
\end{twochants}

\begin{english}
  To many refuges they go ---\\
  To mountain slopes and forest glades,\\
  To parkland shrines and sacred sites ---\\
  People overcome by fear.
\end{english}

\begin{twochants}
  N'etaṃ kho saraṇaṃ khemaṃ & n'etaṃ saraṇam-uttamaṃ \\
  N'etaṃ saraṇam-āgamma & sabba-dukkhā pamuccati \\
\end{twochants}

\begin{english}
  Such a refuge is not secure,\\
  Such a refuge is not supreme,\\
  Such a refuge does not bring\\
  Complete release from suffering.
\end{english}

\begin{twochants}
  Yo ca Buddhañca Dhammañca & saṅghañca saraṇaṃ gato \\
  Cattāri ariya-saccāni & sammappaññāya passati \\
\end{twochants}

\begin{english}
  Whoever goes to refuge\\
  In the Triple Gem\\
  Sees with right discernment\\
  The Four Noble Truths:
\end{english}

\begin{twochants}
  Dukkhaṃ dukkha-samuppādaṃ & dukkhassa ca atikkamaṃ \\
  Ariyañ-c'aṭṭh'aṅgikaṃ maggaṃ & dukkhūpasama-gāminaṃ \\
\end{twochants}

\begin{english}
  Suffering and its origin\\
  And that which lies beyond ---\\
  The Noble Eightfold Path\\
  That leads the way to suff'ring's end.
\end{english}

\begin{twochants}
  Etaṃ kho saraṇaṃ khemaṃ & etaṃ saraṇam-uttamaṃ \\
  Etaṃ saraṇam-āgamma & sabba-dukkhā pamuccati \\
\end{twochants}

\begin{english}
  Such a refuge is secure,\\
  Such a refuge is supreme,\\
  Such a refuge truly brings\\
  Complete release from all suffering.
\end{english}

\suttaRef{Dhp 188-192.}

\section{Verses on a Shining Night of Prosperity}

\begin{leader}
  [Handa mayaṃ bhadd'eka-ratta-gāthāyo bhaṇāmase]
\end{leader}

\firstline{Atītaṃ nānvāgameyya nappaṭikaṅkhe anāgataṃ}

\begin{twochants}
  Atītaṃ nānvāgameyya & nappaṭikaṅkhe anāgataṃ \\
  Yad'atītaṃ pahīnan-taṃ & appattañca anāgataṃ \\
\end{twochants}

\begin{english}
  One should not revive the past\\
  Nor speculate on what's to come;\\
  The past is left behind,\\
  The future is un-realized.
\end{english}

\begin{twochants}
  Paccuppannañca yo dhammaṃ & tattha tattha vipassati \\
  Asaṃhiraṃ asaṅkuppaṃ & taṃ viddhām-anubrūhaye \\
\end{twochants}

\begin{english}
  In every presently arisen state\\
  There, just there, one clearly sees;\\
  Unmoved, unagitated,\\
  Such insight is one's strength.
\end{english}

\begin{twochants}
  Ajj'eva kiccam-ātappaṃ & ko jaññā maraṇaṃ suve \\
  Na hi no saṅgaran-tena & mahā-senena maccunā \\
\end{twochants}

\begin{english}
  Ardently doing one's task today,\\
  Tomorrow, who knows, death may come;\\
  Facing the mighty hordes of death,\\
  Indeed one cannot strike a deal.
\end{english}

\begin{twochants}
  Evaṃ vihārim-ātāpiṃ & aho-rattam-atanditaṃ \\
  Taṃ ve bhadd'eka-ratto'ti & santo ācikkhate muni \\
\end{twochants}

\begin{english}
  To dwell with energy aroused\\
  Thus for a night of non-decline,\\
  That is a `night of shining prosperity.'\\
  So it was taught by the Peaceful Sage.
\end{english}

% MN 131:3
\suttaRef{M.III.187}

\section{Verses on the Three Characteristics}

\begin{leader}
  [Handa mayaṃ ti-lakkhaṇ'ādi-gāthāyo bhaṇāmase]
\end{leader}

\firstline{Sabbe saṅkhārā aniccā'ti yadā paññāya passati}

\begin{twochants}
  Sabbe saṅkhārā aniccā'ti & yadā paññāya passati \\
  Atha nibbindati dukkhe & esa maggo visuddhiyā \\
  Sabbe saṅkhārā dukkhā'ti & yadā paññāya passati \\
  Atha nibbindati dukkhe & esa maggo visuddhiyā \\
  Sabbe dhammā anattā'ti & yadā paññāya passati \\
  Atha nibbindati dukkhe & esa maggo visuddhiyā \\
\end{twochants}

\suttaRef{Dhp 277-279}

\begin{twochants}
  Appakā te manussesu & ye janā pāra-gāmino \\
  Athāyaṃ itarā pajā & tīram-evānudhāvati \\
  Ye ca kho sammad-akkhāte & dhamme dhammānuvattino \\
  Te janā pāram-essanti & maccu-dheyyaṃ suduttaraṃ \\
  Kaṇhaṃ dhammaṃ vippahāya & sukkaṃ bhāvetha paṇḍito \\
  Okā anokam-āgamma & viveke yattha dūramaṃ \\
  Tatrābhiratim-iccheyya & hitvā kāme akiñcano \\
  Pariyodapeyya attānaṃ, & citta-klesehi paṇḍito\\
  Yesaṃ sambodhi-y-aṅgesu, & sammā cittaṃ subhāvitaṃ\\
  Ādāna-paṭinissagge, & anupādāya ye ratā\\
  Khīṇ'āsavā jutimanto, & te loke parinibbutā-ti.
\end{twochants}

\suttaRef{Dhp 85-89}

\section{Verses on the Three Characteristics (English)}

`Impermanent are all conditioned things' ---\\
When with wisdom this is seen\\
One feels weary of all dukkha;\\
This is the path to purity.

`Dukkha are all conditioned things' ---\\
When with wisdom this is seen\\
One feels weary of all dukkha;\\
This is the path to purity.

`There is no self in anything' ---\\
When with wisdom this is seen\\
One feels weary of all dukkha;\\
This is the path to purity.

Few amongst humankind\\
Are those who go beyond,\\
Yet there are the many folks\\
Ever wand'ring on this shore.

Wherever Dhamma is well-taught,\\
Those who train in line with it\\
Are the ones who will cross over\\
The realm of death so hard to flee.

Abandoning the darker states,\\
The wise pursue the bright;\\
From the floods dry land they reach\\
Living withdrawn so hard to do.\\
Such rare delight one should desire,\\
Sense pleasures cast away,\\
Not having anything.

\section{Verses on Respect for the Dhamma}

\begin{leader}
  [Handa mayaṃ dhamma-gārav'ādi-gāthāyo bhaṇāmase]
\end{leader}

\firstline{Ye ca atītā sambuddhā ye ca buddhā anāgatā}

\begin{twochants}
  Ye ca atītā sambuddhā & ye ca buddhā anāgatā \\
  Yo c'etarahi sambuddho & bahunnaṃ soka-nāsano \\
\end{twochants}

\begin{english}
  All the Buddhas of the past,\\
  All the Buddhas yet to come,\\
  The Buddha of this current age ---\\
  Dispellers of much sorrow.
\end{english}

\begin{twochants}
  Sabbe saddhamma-garuno & vihariṃsu viharanti ca \\
  Atho pi viharissanti & esā buddhāna dhammatā \\
\end{twochants}

\begin{english}
  Those having lived or living now,\\
  Those living in the future,\\
  All do revere the True Dhamma ---\\
  That is the nature of all Buddhas.
\end{english}

\begin{twochants}
  Tasmā hi atta-kāmena & mahattam-abhikaṅkhatā \\
  Saddhammo garu-kātabbo & saraṃ buddhāna sāsanaṃ \\
\end{twochants}

\begin{english}
  Therefore desiring one's own welfare,\\
  Pursuing greatest aspirations,\\
  One should revere the True Dhamma ---\\
  Recollecting the Buddha's teaching.
\end{english}

\suttaRef{S.I.140}

\begin{paritta}
  Na hi dhammo adhammo ca\\
  Ubho sama-vipākino \\
  Adhammo nirayaṃ neti\\
  Dhammo pāpeti suggatiṃ
\end{paritta}

\begin{english}
  What is true Dhamma and what not\\
  Will never have the same results,\\
  While lack of Dhamma leads to hell-realms ---\\
  True Dhamma takes one on a good course.
\end{english}

\begin{paritta}
  Dhammo have rakkhati dhamma-cāriṃ\\
  Dhammo suciṇṇo sukham-āvahāti\\
  Esānisaṃso dhamme suciṇṇe\\
  Na duggatiṃ gacchati dhamma-cārī.
\end{paritta}

\begin{english}
  The Dhamma guards who lives in line with it\\
  And leads to happiness when practised well ---\\
  This is the blessing of well-practised Dhamma.
\end{english}

% AN 21.4; Thag 303-304

\suttaRef{Thag 303-304}

\section{Verses on Respect}

% Gārav'ādi gāthā

% Pali and English source: Bodhivana

Satthu-garu dhamma-garu,\\
Saṅghe ca tibba-gāravo,\\
Samādhi-garu ātāpī,\\
Sikkhāya tibba-gāravo,\\
Appamāda-garu bhikkhu,\\
Paṭisanthāra-gāravo:\\
Abhabbo parihānāya,\\
Nibbānasseva santike.

\begin{english}
  One with respect for the Buddha and Dhamma,\\
  and strong respect for the Saṅgha,\\
  one who is ardent, with respect for concentration,\\
  and strong respect for the Training,\\
  one who sees danger and respects being heedful,\\
  and shows respect in welcoming guests.\\
  A person like this cannot decline,\\
  stands right in the presence of Nibbāna.
\end{english}

% AN 7.32
\suttaRef{AN 7.32}

\section{Verses on the Buddha's First Exclamation}

\begin{leader}
  [Handa mayaṃ paṭhama-buddha-bhāsita-gāthāyo bhaṇāmase]
\end{leader}

\firstline{Aneka-jāti-saṃsāraṃ sandhāvissaṃ anibbisaṃ}

\begin{twochants}
  Aneka-jāti-saṃsāraṃ & sandhāvissaṃ anibbisaṃ \\
  Gaha-kāraṃ gavesanto & dukkhā jāti punappunaṃ \\
\end{twochants}

\begin{english}
  For many lifetimes in the round of birth,\\
  Wandering on endlessly,\\
  For the builder of this house I searched ---\\
  How painful is repeated birth.
\end{english}

\begin{twochants}
  Gaha-kāraka diṭṭho'si & puna gehaṃ na kāhasi \\
  Sabbā te phāsukā bhaggā & gaha-kūṭaṃ visaṅkhataṃ \\
  Visaṅkhāra-gataṃ cittaṃ & taṇhānaṃ khayam-ajjhagā \\
\end{twochants}

\begin{english}
  House-builder you've been seen,\\
  Another home you will not build,\\
  All your rafters have been snapped,\\
  Dismantled is your ridge-pole;\\
  The non-constructing mind\\
  Has come to craving's end.
\end{english}

\suttaRef{Dhp 153-154}

\section{Verses on the Last Instructions}

\firstline{Handa dāni bhikkhave āmantayāmi vo}

\begin{leader}
  [Handa mayaṃ pacchima-ovāda-gāthāyo bhaṇāmase]
\end{leader}

Handa dāni bhikkhave āmantayāmi vo

\begin{english}
  Now bhikkhus I declare to you,
\end{english}

Vaya-dhammā saṅkhārā

\begin{english}
  Change is the nature of conditioned things;
\end{english}

Appamādena sampādethā'ti

\begin{english}
  Perfect yourselves, not being negligent:
\end{english}

Ayaṃ tathāgatassa pacchimā vācā

\begin{english}
  These are the Tathāgata's final words.
\end{english}

\suttaRef{DN 16:6.8}

\section{Arising From a Cause}

\firstline{Ye dhammā hetuppabhavā}

\begin{paritta}
  Ye dhammā hetuppabhavā\\
  Tesaṃ hetuṃ tathāgato āha\\
  Tesañca yo nirodho\\
  Evaṃ-vādī mahāsamaṇo'ti.
\end{paritta}

\begin{english}
  Whatever phenomena arise from a cause,\\
  The Tathāgata has explained their cause,\\
  And also their cessation.\\
  That is the teaching of the Great Ascetic.
\end{english}

\suttaRef{Mv.1.23.5}

\section{Nakkhattayakkha}

\instr{The paritta chanting may be closed with the following:}

\firstline{Nakkhatta-yakkha-bhūtānaṃ}

\begin{twochants}
  Nakkhatta-yakkha-bhūtānaṃ & pāpa-ggaha-nivāraṇā\\
  Parittassānubhāvena & hantvā tesaṃ upaddave\\
\end{twochants}

\instr{(Three times)}
