\chapter{Paritta Chants}

\section{Requesting Paritta Chanting}

\begin{paritta}
\instr{(After bowing three times, with hands joined in añjali,\\
  recite the following)}

Vipatti-paṭibāhāya sabba-sampatti-siddhiyā\\
Sabbadukkha-vināsāya\\
Parittaṃ brūtha maṅgalaṃ

Vipatti-paṭibāhāya sabba-sampatti-siddhiyā\\
Sabbabhaya-vināsāya\\
Parittaṃ brūtha maṅgalaṃ

Vipatti-paṭibāhāya sabba-sampatti-siddhiyā\\
Sabbaroga-vināsāya\\
Parittaṃ brūtha maṅgalaṃ

\instr{(Bow three times)}
\end{paritta}

\clearpage

\section{Invitation to the Devas}

\begin{paritta}
% TODO is this an alternative beginning?

[Sarajjaṃ sasenaṃ sabandhuṃ nar'indaṃ\\
Paritt'ānubhavo sadā rakkhatū-ti]

\sidepar{A.}%
Pharitvāna mettaṃ samettā bhadantā\\
Avikkhitta-cittā parittaṃ bhaṇantu

\sidepar{B.}%
Samantā cakka-vāḷesu\\
Atr'āgacchantu devatā

Sagge kāme ca rūpe\\
Giri-sikhara-taṭe c'antalikkhe vimāne\\
Dīpe raṭṭhe ca gāme\\
Taru-vana-gahane geha-vatthumhi khette\\
Bhummā c'āyantu devā\\
Jala-thala-visame yakkha-gandhabba-nāgā\\
Tiṭṭhantā santike yaṃ\\
Muni-vara-vacanaṃ sādhavo me suṇantu

Dhammassavana-kālo ayam-bhadantā

\instr{(Three times, or)}

Buddha-dassana-kālo ayam-bhadantā\\
Dhammassavana-kālo ayam-bhadantā\\
Saṅgha-payirūpāsana-kālo ayam-bhadantā
\end{paritta}

\section{Pubba-bhāga-nama-kāra-pāṭho}

Namo tassa bhagavato arahato sammā-sambuddhassa\\
Namo tassa bhagavato arahato sammā-sambuddhassa\\
Namo tassa bhagavato arahato sammā-sambuddhassa

\section{Saraṇa-gamana-pāṭho}

\begin{paritta}
Buddhaṃ saraṇaṃ gacchāmi\\
Dhammaṃ saraṇaṃ gacchāmi\\
Saṅghaṃ saraṇaṃ gacchāmi

Dutiyam pi buddhaṃ saraṇaṃ gacchāmi\\
Dutiyam pi dhammaṃ saraṇaṃ gacchāmi\\
Dutiyam pi saṅghaṃ saraṇaṃ gacchāmi

Tatiyam pi buddhaṃ saraṇaṃ gacchāmi\\
Tatiyam pi dhammaṃ saraṇaṃ gacchāmi\\
Tatiyam pi saṅghaṃ saraṇaṃ gacchāmi
\end{paritta}

\enlargethispage{\baselineskip}

\section{Nama-kāra-siddhi-gāthā}

\begin{paritta}
Yo cakkhumā moha-malāpakaṭṭho\\
Sāmaṃ va buddho sugato vimutto\\
Mārassa pāsā vinimocayanto\\
Pāpesi khemaṃ janataṃ vineyyaṃ\\
Buddhaṃ varan-taṃ sirasā namāmi\\
Lokassa nāthañ-ca vināyakañ-ca\\
Tan-tejasā te jaya-siddhi hotu\\
Sabb'antarāyā ca vināsamentu

Dhammo dhajo yo viya tassa satthu\\
Dassesi lokassa visuddhi-maggaṃ\\
Niyyāniko dhamma-dharassa dhārī\\
Sāt'āvaho santi-karo suciṇṇo\\
Dhammaṃ varan-taṃ sirasā namāmi\\
Mohappadālaṃ upasanta-dāhaṃ\\
Tan-tejasā te jaya-siddhi hotu\\
Sabb'antarāyā ca vināsamentu

Saddhamma-senā sugatānugo yo\\
Lokassa pāpūpakilesa-jetā\\
Santo sayaṃ santi-niyojako ca\\
Svākkhāta-dhammaṃ viditaṃ karoti\\
Saṅghaṃ varan-taṃ sirasā namāmi\\
Buddhānubuddhaṃ sama-sīla-diṭṭhiṃ\\
Tan-tejasā te jaya-siddhi hotu\\
Sabb'antarāyā ca vināsamentu
\end{paritta}

\section{Sacca-kiriyā-gāthā}

Natthi me saraṇaṃ aññaṃ buddho me saraṇaṃ varaṃ\\
Etena sacca-vajjena sotthi te/me hotu sabbadā

Natthi me saraṇaṃ aññaṃ dhammo me saraṇaṃ varaṃ\\
Etena sacca-vajjena sotthi te/me hotu sabbadā

Natthi me saraṇaṃ aññaṃ saṅgho me saraṇaṃ varaṃ\\
Etena sacca-vajjena sotthi te/me hotu sabbadā

\section{Mahā-kāruṇiko nātho ti ādikā gāthā}

\begin{paritta}
Mahā-kāruṇiko nātho\\
Atthāya sabba-pāṇinaṃ\\
Hitāya sabba-pāṇinaṃ\\
Sukhāya sabba-pāṇinaṃ

Pūretvā pāramī sabbā\\
Patto sambodhim-uttamaṃ\\
Etena sacca-vajjena\\
Mā hontu sabb'upaddavā.
\end{paritta}

\vspace*{-0.6\baselineskip}

\enlargethispage{3\baselineskip}

\section{Namo-kāra-aṭṭhaka}

\begin{paritta}
Namo arahato sammā\\
Sambuddhassa mahesino\\
Namo uttama-dhammassa\\
Svākkhātass'eva ten'idha\\
Namo mahā-saṅghassāpi\\
Visuddha-sīla-diṭṭhino\\
Namo omāty-āraddhassa\\
Ratanattayassa sādhukaṃ\\
Namo omakātītassa\\
Tassa vatthuttayassa-pi\\
Namo-kārappabhāvena\\
Vigacchantu upaddavā\\
Namo-kārānubhāvena\\
Suvatthi hotu sabbadā\\
Namo-kārassa tejena\\
Vidhimhi homi tejavā\\
\end{paritta}

\section{Maṅgala-sutta}

[Evam-me sutaṃ: ekaṃ samayaṃ bhagavā, sāvatthiyaṃ viharati, jeta-vane
anāthapiṇḍikassa ārāme. Atha kho aññatarā devatā abhikkantāya rattiyā
abhikkanta-vaṇṇā kevala-kappaṃ jetavanaṃ obhāsetvā, yena bhagavā ten'upasaṅkami.
Upasaṅkamitvā bhagavantaṃ abhivādetvā ekam-antaṃ aṭṭhāsi. Ekam-antaṃ ṭhitā kho
sā devatā bhagavantaṃ gāthāya ajjhabhāsi:

Bahū devā manussā ca,\\
Maṅgalāni acintayuṃ;\\
Ākaṅkhamānā sotthānaṃ,\\
Brūhi maṅgalam-uttamaṃ.]

\bigskip

\begin{paritta}
Asevanā ca bālānaṃ\\
Paṇḍitānañ-ca sevanā\\
Pūjā ca pūjanīyānaṃ\\
Etam maṅgalam-uttamaṃ

Paṭirūpa-desa-vāso ca\\
Pubbe ca kata-puññatā\\
Atta-sammā-paṇidhi ca\\
Etam maṅgalam-uttamaṃ

Bāhu-saccañ-ca sippañ-ca,\\
Vinayo ca susikkhito\\
Subhāsitā ca yā vācā\\
Etam maṅgalam-uttamaṃ

Mātā-pitu-upaṭṭhānaṃ\\
Putta-dārassa saṅgaho\\
Anākulā ca kammantā\\
Etam maṅgalam-uttamaṃ

Dānañ-ca dhamma-cariyā ca\\
Ñātakānañ-ca saṅgaho\\
Anavajjāni kammāni\\
Etam maṅgalam-uttamaṃ

Āratī viratī pāpā\\
Majja-pānā ca saññamo\\
Appamādo ca dhammesu\\
Etam maṅgalam-uttamaṃ

Gāravo ca nivāto ca\\
Santuṭṭhī ca kataññutā\\
Kālena dhammassavanaṃ\\
Etam maṅgalam-uttamaṃ

Khantī ca sovacassatā\\
Samaṇānañ-ca dassanaṃ\\
Kālena dhamma-sākacchā\\
Etam maṅgalam-uttamaṃ

Tapo ca brahma-cariyañ-ca\\
Ariya-saccāna-dassanaṃ\\
Nibbāna-sacchikiriyā ca\\
Etam maṅgalam-uttamaṃ

Phuṭṭhassa loka-dhammehi\\
Cittaṃ yassa na kampati\\
Asokaṃ virajaṃ khemaṃ\\
Etam maṅgalam-uttamaṃ

Etādisāni katvāna\\
Sabbattham-aparājitā\\
Sabbattha sotthiṃ gacchanti\\
Tan-tesaṃ maṅgalam-uttaman'ti

\end{paritta}

\section{Ratana Sutta}

\begin{paritta}

Yānīdha bhūtāni samāgatāni,\\
Bhummāni vā yāni va antalikkhe.\\
Sabb'eva bhūtā sumanā bhavantu,\\
Atho pi sakkacca suṇantu bhāsitaṃ.\\
Tasmā hi bhūtā nisāmetha sabbe,\\
Mettaṃ karotha mānusiyā pajāya.\\
Divā ca ratto ca haranti ye baliṃ,\\
Tasmā hi ne rakkhatha appamattā.

\sidepar{1.}%
Yaṅkiñci vittaṃ idha vā huraṃ vā\\
Saggesu vā yaṃ ratanaṃ paṇītaṃ\\
Na no samaṃ atthi tathāgatena\\
Idam-pi buddhe ratanaṃ paṇītaṃ\\
Etena saccena suvatthi hotu

\clearpage

\sidepar{2.}%
Khayaṃ virāgaṃ amataṃ paṇītaṃ\\
Yad-ajjhagā sakya-munī samāhito\\
Na tena dhammena sam'atthi kiñci\\
Idam-pi dhamme ratanaṃ paṇītaṃ\\
Etena saccena suvatthi hotu

\sidepar{3.}%
Yam buddha-seṭṭho parivaṇṇayī suciṃ\\
Samādhim-ānantarikaññam-āhu\\
Samādhinā tena samo na vijjati\\
Idam-pi dhamme ratanaṃ paṇītaṃ\\
Etena saccena suvatthi hotu

\sidepar{4.}%
Ye puggalā aṭṭha sataṃ pasaṭṭhā\\
Cattāri etāni yugāni honti\\
Te dakkhiṇeyyā sugatassa sāvakā\\
Etesu dinnāni mahapphalāni\\
Idam-pi saṅghe ratanaṃ paṇītaṃ\\
Etena saccena suvatthi hotu

\sidepar{5.}%
Ye suppayuttā manasā daḷhena\\
Nikkāmino gotama-sāsanamhi\\
Te patti-pattā amataṃ vigayha\\
Laddhā mudhā nibbutiṃ bhuñjamānā\\
Idam-pi saṅghe ratanaṃ paṇītaṃ\\
Etena saccena suvatthi hotu

Yath'inda-khīlo paṭhaviṃ sito siyā,\\
Catubbhi vātebhi asampakampiyo.\\
Tathūpamaṃ sappurisaṃ vadāmi,\\
Yo ariya-saccāni avecca passati.\\
Idam-pi Saṅghe ratanaṃ paṇītaṃ,\\
Etena saccena suvatthi hotu.

Ye ariya-saccāni vibhāvayanti,\\
Gambhīra-paññena sudesitāni.\\
Kiñ-cāpi te honti bhusappamattā,\\
Na te bhavaṃ aṭṭhamam-ādiyanti.\\
Idam-pi Saṅghe ratanaṃ paṇītaṃ,\\
Etena saccena suvatthi hotu.

Sahā v'assa dassana-sampadāya,\\
Tay'assu dhammā jahitā bhavanti.\\
Sakkāya-diṭṭhi vicikicchitañ-ca,\\
Sīlabbataṃ vā pi yad-atthi kiñci.\\
Catūh'apāyehi ca vippamutto,\\
Cha cābhiṭhānāni abhabbo kātuṃ.\\
Idam-pi Saṅghe ratanaṃ paṇītaṃ,\\
Etena saccena suvatthi hotu.

Kiñ-cāpi so kammaṃ karoti pāpakaṃ,\\
Kāyena vācā uda cetasā vā.\\
Abhabbo so tassa paṭicchadāya,\\
Abhabbatā diṭṭha-padassa vuttā.\\
Idam-pi Saṅghe ratanaṃ paṇītaṃ,\\
Etena saccena suvatthi hotu.

Vanappagumbe yathā phussi-t-agge,\\
Gimhāna-māse paṭhamasmiṃ gimhe.\\
Tathūpamaṃ dhamma-varaṃ adesayi,\\
Nibbāna-gāmiṃ paramaṃ hitāya.\\
Idam-pi Buddhe ratanaṃ paṇītaṃ,\\
Etena saccena suvatthi hotu.

Varo varañ-ñū vara-do var'āharo,\\
Anuttaro dhamma-varaṃ adesayi.\\
Idam-pi Buddhe ratanaṃ paṇītaṃ,\\
Etena saccena suvatthi hotu.

\sidepar{6.}%
Khīṇaṃ purāṇaṃ navaṃ n'atthi sambhavaṃ\\
Viratta-citt'āyatike bhavasmiṃ\\
Te khīṇa-bījā aviruḷhi-chandā\\
Nibbanti dhīrā yathā'yam padīpo\\
Idam-pi saṅghe ratanaṃ paṇītaṃ\\
Etena saccena suvatthi hotu

Yānīdha bhūtāni samāgatāni,\\
Bhummāni vā yāni va antalikkhe.\\
Tathāgataṃ deva-manussa-pūjitaṃ,\\
Buddhaṃ namassāma suvatthi hotu.

Yānīdha bhūtāni samāgatāni,\\
Bhummāni vā yāni va antalikkhe.\\
Tathāgataṃ deva-manussa-pūjitaṃ,\\
Dhammaṃ namassāma suvatthi hotu.

Yānīdha bhūtāni samāgatāni,\\
Bhummāni vā yāni va antalikkhe.\\
Tathāgataṃ deva-manussa-pūjitaṃ,\\
Saṅghaṃ namassāma suvatthi hotū-ti.

\end{paritta}

\section{Karaṇīya-metta-sutta}

\begin{paritta}

Karaṇīyam-attha-kusalena\\
Yan-taṃ santaṃ padaṃ abhisamecca\\
Sakko ujū ca suhujū ca\\
Suvaco c'assa mudu anatimānī

Santussako ca subharo ca\\
Appakicco ca sallahuka-vutti\\
Sant'indriyo ca nipako ca\\
Appagabbho kulesu ananugiddho

Na ca khuddaṃ samācare kiñci\\
Yena viññū pare upavadeyyuṃ\\
Sukhino vā khemino hontu\\
Sabbe sattā bhavantu sukhit'attā

Ye keci pāṇa-bhūt'atthi\\
Tasā vā thāvarā vā anavasesā\\
Dīghā vā ye mahantā vā\\
Majjhimā rassakā aṇuka-thūlā

Diṭṭhā vā ye ca adiṭṭhā\\
Ye ca dūre vasanti avidūre\\
Bhūtā vā sambhavesī vā\\
Sabbe sattā bhavantu sukhit'attā

Na paro paraṃ nikubbetha\\
Nātimaññetha katthaci naṃ kiñci\\
Byārosanā paṭighasaññā\\
Nāññam-aññassa dukkham-iccheyya

Mātā yathā niyaṃ puttaṃ\\
Āyusā eka-puttam-anurakkhe\\
Evam'pi sabba-bhūtesu\\
Mānasam-bhāvaye aparimāṇaṃ

Mettañ-ca sabba-lokasmiṃ\\
Mānasam-bhāvaye aparimāṇaṃ\\
Uddhaṃ adho ca tiriyañ-ca\\
Asambādhaṃ averaṃ asapattaṃ

Tiṭṭhañ-caraṃ nisinno vā\\
Sayāno vā yāvat'assa vigata-middho\\
Etaṃ satiṃ adhiṭṭheyya\\
Brahmam-etaṃ vihāraṃ idham-āhu

Diṭṭhiñca anupagamma\\
Sīlavā dassanena sampanno\\
Kāmesu vineyya gedhaṃ\\
Na hi jātu gabbha-seyyaṃ punaretī'ti

\end{paritta}

\section{Khandha-parittaṃ}

\begin{twochants}
Virūpakkhehi me mettaṃ & mettaṃ erāpathehi me\\
Chabyā-puttehi me mettaṃ & mettaṃ kaṇhā-gotamakehi ca\\
Apādakehi me mettaṃ & mettaṃ dipādakehi me\\
Catuppadehi me mettaṃ & mettaṃ bahuppadehi me\\
Mā maṃ apādako hiṃsi & mā maṃ hiṃsi dipādako\\
Mā maṃ catuppado hiṃsi & mā maṃ hiṃsi bahuppado\\
Sabbe sattā sabbe pāṇā & sabbe bhūtā ca kevalā\\
Sabbe bhadrāni passantu & mā kiñci pāpam-āgamā\\
Appamāṇo buddho & appamāṇo dhammo\\
Appamāṇo saṅgho & pamāṇavantāni siriṃsapāni\\
Ahi-vicchikā sata-padī & uṇṇā-nābhī sarabhū mūsikā\\
Katā me rakkhā katā me parittā & paṭikkamantu bhūtāni\\
So'haṃ namo bhagavato & namo sattannaṃ\\
Sammā-sambuddhānaṃ & \\
\end{twochants}

\section{Mora-parittaṃ}

\begin{paritta}

\sidepar{a.m.}%
Udet'ayañ-cakkhumā eka-rājā,\\
\sidepar{p.m.}%
Apet'ayañ-cakkhumā eka-rājā,

Harissa-vaṇṇo paṭhavippabhāso;\\
Taṃ taṃ namassāmi harissa-vaṇṇaṃ\\
paṭhavippabhāsaṃ,

\sidepar{a.m.}%
Tay'ajja guttā viharemu divasaṃ.\\
\sidepar{p.m.}%
Tay'ajja guttā viharemu rattiṃ.

Ye brāhmaṇā veda-gu sabba-dhamme,\\
Te me namo, te ca maṃ pālayantu;\\
Nam'atthu Buddhānaṃ, nam'atthu bodhiyā,\\
Namo vimuttānaṃ, namo vimuttiyā.\\
Imaṃ so parittaṃ katvā,

\sidepar{a.m.}%
Moro carati esanā'ti.\\
\sidepar{p.m.}%
Moro vāsam-akappayī'ti.

\suttaRef{J.159}

\end{paritta}

\section{Vaṭṭaka-parittaṃ}

\begin{twochants}
Atthi loke sīla-guṇo & saccaṃ soceyy'anuddayā\\
Tena saccena kāhāmi & sacca-kiriyam-anuttaraṃ\\
Āvajjitvā dhamma-balaṃ & saritvā pubbake jine\\
Sacca-balam-avassāya & sacca-kiriyam-akās'ahaṃ\\
Santi pakkhā apattanā & santi pādā avañcanā\\
Mātā pitā ca nikkhantā & jāta-veda paṭikkama\\
Saha sacce kate mayhaṃ & mahā-pajjalito sikhī\\
Vajjesi soḷasa karīsāni & udakaṃ patvā yathā sikhī\\
Saccena me samo n'atthi & esā me sacca-pāramī ti\\
\end{twochants}

\section{Buddha-dhamma-saṅgha-guṇā}

\begin{paritta}

Iti pi so bhagavā arahaṃ sammā-sambuddho\\
Vijjā-caraṇa-sampanno sugato loka-vidū\\
Anuttaro purisa-damma-sārathi\\
Satthā devamanussānaṃ buddho bhagavā'ti

Svākkhāto bhagavatā dhammo sandiṭṭhiko\\
\vin akāliko ehi-passiko\\
Opanayiko paccattaṃ veditabbo viññūhī'ti

Supaṭipanno bhagavato sāvaka-saṅgho\\
Uju-paṭipanno bhagavato sāvaka-saṅgho\\
Ñāya-paṭipanno bhagavato sāvaka-saṅgho\\
Sāmīci-paṭipanno bhagavato sāvaka-saṅgho\\
Yad-idaṃ cattāri purisa-yugāni aṭṭha purisa-puggalā\\
Esa bhagavato sāvaka-saṅgho\\
Āhuneyyo pāhuneyyo dakkhiṇeyyo añjali-karaṇīyo\\
Anuttaraṃ puññakkhettaṃ lokassā'ti

\sidepar{\pointerMark}%
Araññe rukkha-mūle vā\\
Suññāgāre va bhikkhavo\\
Anussaretha Sambuddhaṃ\\
Bhayaṃ tumhāka no siyā.\\
No ce Buddhaṃ sareyyātha\\
Loka-jeṭṭhaṃ nar'āsabhaṃ\\
Atha dhammaṃ sareyyātha\\
Niyyānikaṃ sudesitaṃ.\\
No ce dhammaṃ sareyyātha\\
Niyyānikaṃ sudesitaṃ\\
Atha saṅghaṃ sareyyātha\\
Puññakkhettaṃ anuttaraṃ.\\
Evam-Buddhaṃ sarantānaṃ\\
Dhammaṃ saṅghañ-ca bhikkhavo\\
Bhayaṃ vā chambhitattaṃ vā\\
Loma-haṃso na hessatī-ti.

\end{paritta}

%\section{Āṭānāṭiya Paritta (short)}
%
%\begin{twochants}
%Vipassissa nam'atthu & cakkhumantassa sirīmato\\
%Sikhissa pi nam'atthu & sabba-bhūtānukampino\\
%Vessabhussa nam'atthu & nhātakassa tapassino\\
%Nam'atthu kakusandhassa & māra-senappamaddino\\
%Konāgamanassa nam'atthu & brāhmaṇassa vusīmato\\
%Kassapassa nam'atthu & vippamuttassa sabbadhi\\
%Aṅgīrasassa nam'atthu & sakya-puttassa sirīmato\\
%Yo imaṃ dhammam-adesesi & sabba-dukkhāpanūdanaṃ\\
%Ye cāpi nibbutā loke & yathā-bhūtaṃ vipassisuṃ\\
%Te janā apisuṇā & mahantā vīta-sāradā\\
%Hitaṃ deva-manussānaṃ & yaṃ namassanti gotamaṃ\\
%Vijjā-caraṇa-sampannaṃ & mahantaṃ vīta-sāradaṃ\\
%Vijjā-caraṇa-sampannaṃ & buddhaṃ vandāma gotaman'ti\\
%\end{twochants}
%
%\section{Āṭānāṭiya Paritta (long)}
%
%\begin{leader}
%\soloinstr{Solo introduction}
%
%\begin{solotwochants}
%Appasannehi nāthassa & sāsane sādhusammate\\
%Amanussehi caṇḍehi & sadā kibbisakāribhi\\
%Parisānañca-tassannam & ahiṃsāya ca guttiyā\\
%Yandesesi mahāvīro & parittan-tam bhaṇāma se.\\
%\end{solotwochants}
%\end{leader}
%
%\begin{twochants}
%[Namo me sabbabuddhānaṃ] & uppannānaṃ mahesinaṃ\\
%Taṇhaṅkaro mahāvīro & medhaṅkaro mahāyaso\\
%Saraṇaṅkaro lokahito & dīpaṅkaro jutindharo\\
%Koṇḍañño janapāmokkho & maṅgalo purisāsabho\\
%Sumano sumano dhīro & revato rativaḍḍhano\\
%Sobhito guṇasampanno & anomadassī januttamo\\
%Padumo lokapajjoto & nārado varasārathī\\
%Padumuttaro sattasāro & sumedho appaṭipuggalo\\
%Sujāto sabbalokaggo & piyadassī narāsabho\\
%Atthadassī kāruṇiko & dhammadassī tamonudo\\
%Siddhattho asamo loke & tisso ca vadataṃ varo\\
%Phusso ca varado buddho & vipassī ca anūpamo\\
%Sikhī sabbahito satthā & vessabhū sukhadāyako\\
%Kakusandho satthavāho & koṇāgamano raṇañjaho\\
%Kassapo sirisampanno & gotamo sakyapuṅgavo\\
%Ete caññe ca sambuddhā & anekasatakoṭayo\\
%Sabbe buddhā asamasamā & sabbe buddhā mahiddhikā\\
%Sabbe dasabalūpetā & vesārajjehupāgatā\\
%Sabbe te paṭijānanti & āsabhaṇṭhānamuttamaṃ\\
%Sīhanādaṃ nadantete & parisāsu visāradā\\
%Brahmacakkaṃ pavattenti & loke appaṭivattiyaṃ\\
%Upetā buddhadhammehi & aṭṭhārasahi nāyakā\\
%Dvattiṃsa-lakkhaṇūpetā & sītyānubyañjanādharā\\
%Byāmappabhāya suppabhā & sabbe te munikuñjarā\\
%Buddhā sabbaññuno ete & sabbe khīṇāsavā jinā\\
%Mahappabhā mahātejā & mahāpaññā mahabbalā\\
%Mahākāruṇikā dhīrā & sabbesānaṃ sukhāvahā\\
%Dīpā nāthā patiṭṭhā & ca tāṇā leṇā ca pāṇinaṃ\\
%Gatī bandhū mahassāsā & saraṇā ca hitesino\\
%Sadevakassa lokassa & sabbe ete parāyanā\\
%Tesāhaṃ sirasā pāde & vandāmi purisuttame\\
%Vacasā manasā ceva & vandāmete tathāgate\\
%Sayane āsane ṭhāne & gamane cāpi sabbadā\\
%Sadā sukhena rakkhantu & buddhā santikarā tuvaṃ\\
%Tehi tvaṃ rakkhito santo & mutto sabbabhayena ca\\
%Sabba-rogavinimutto & sabba-santāpavajjito\\
%Sabba-veramatikkanto & nibbuto ca tuvaṃ bhava\\
%\end{twochants}
%
%\savenotes
%
%\begin{twochants}
%Tesaṃ saccena sīlena & khantimettābalena ca\\
%Tepi tumhe%
%\footnote{If chanting for oneself, change \textit{tumhe} to \textit{amhe} here and in the lines below.}
%anurakkhantu & ārogyena sukhena ca\\
%Puratthimasmiṃ disābhāge & santi bhūtā mahiddhikā\\
%Tepi tumhe anurakkhantu & ārogyena sukhena ca\\
%Dakkhiṇasmiṃ disābhāge & santi devā mahiddhikā\\
%Tepi tumhe anurakkhantu & ārogyena sukhena ca\\
%Pacchimasmiṃ disābhāge & santi nāgā mahiddhikā\\
%Tepi tumhe anurakkhantu & ārogyena sukhena ca\\
%Uttarasmiṃ disābhāge & santi yakkhā mahiddhikā\\
%Tepi tumhe anurakkhantu & ārogyena sukhena ca\\
%Purimadisaṃ dhataraṭṭho & dakkhiṇena viruḷhako\\
%Pacchimena virūpakkho & kuvero uttaraṃ disaṃ\\
%Cattāro te mahārājā & lokapālā yasassino\\
%Tepi tumhe anurakkhantu & ārogyena sukhena ca\\
%Ākāsaṭṭhā ca bhummaṭṭhā & devā nāgā mahiddhikā\\
%Tepi tumhe anurakkhantu & ārogyena sukhena ca\\
%Natthi me saraṇaṃ aññaṃ & buddho me saraṇaṃ varaṃ\\
%Etena saccavajjena & hotu te%
%\footnote{If chanting for oneself, change \textit{te} to \textit{me} here and in the lines below.}
%jayamaṅgalaṃ\\
%Natthi me saraṇaṃ aññaṃ & dhammo me saraṇaṃ varaṃ\\
%Etena saccavajjena & hotu te jayamaṅgalaṃ\\
%Natthi me saraṇaṃ aññaṃ & saṅgho me saraṇaṃ varaṃ\\
%Etena saccavajjena & hotu te jayamaṅgalaṃ\\
%\end{twochants}
%
%\spewnotes
%
%\begin{twochants}
%Yaṅkiñci ratanaṃ loke & vijjati vividhaṃ puthu\\
%Ratanaṃ buddhasamaṃ & natthi tasmā sotthī bhavantu te\\
%Yaṅkiñci ratanaṃ loke & vijjati vividhaṃ puthu\\
%Ratanaṃ dhammasamaṃ & natthi tasmā sotthī bhavantu te\\
%Yaṅkiñci ratanaṃ loke & vijjati vividhaṃ puthu\\
%Ratanaṃ saṅghasamaṃ & natthi tasmā sotthī bhavantu te\\
%Sakkatvā buddharatanaṃ & osathaṃ uttamaṃ varaṃ\\
%Hitaṃ devamanussānaṃ & buddhatejena sotthinā\\
%Nassantupaddavā sabbe & dukkhā vūpasamentu te\\
%Sakkatvā dhammaratanaṃ & osathaṃ uttamaṃ varaṃ\\
%Pariḷāhūpasamanaṃ & dhammatejena sotthinā\\
%Nassantupaddavā sabbe & bhayā vūpasamentu te\\
%Sakkatvā saṅgharatanaṃ & osathaṃ uttamaṃ varaṃ\\
%Āhuneyyaṃ pāhuneyyaṃ & saṅghatejena sotthinā\\
%Nassantupaddavā sabbe & rogā vūpasamentu te\\
%Sabbītiyo vivajjantu & sabbarogo vinassatu\\
%Mā te bhavatvantarāyo & sukhī dīghāyuko bhava\\
%Abhivādanasīlissa & niccaṃ vuḍḍhāpacāyino\\
%Cattāro dhammā vaḍḍhanti & āyu vaṇṇo sukhaṃ balaṃ\\
%\end{twochants}
%
%\section{Aṅguli-māla-parittaṃ}
%
%\begin{paritta}
%Yato'haṃ bhagini ariyāya jātiyā jāto\\
%Nābhijānāmi sañcicca pāṇaṃ jīvitā voropetā\\
%Tena saccena sotthi te hotu sotthi gabbhassa
%
%\instr{Three times}
%
%\end{paritta}
%
%\section{Bojjh'aṅga-parittaṃ}
%
%\begin{twochants}
%Bojjh'aṅgo sati-saṅkhāto & dhammānaṃ vicayo tathā\\
%Viriyam-pīti-passaddhi & bojjh'aṅgā ca tathā'pare\\
%Samādh'upekkha-bojjh'aṅgā & satt'ete sabba-dassinā\\
%Muninā sammad-akkhātā & bhāvitā bahulīkatā\\
%Saṃvattanti abhiññāya & nibbānāya ca bodhiyā\\
%Etena sacca-vajjena & sotthi te hotu sabbadā\\
%Ekasmiṃ samaye nātho & moggallānañ-ca kassapaṃ\\
%Gilāne dukkhite disvā & bojjh'aṅge satta desayi\\
%Te ca taṃ abhinanditvā & rogā mucciṃsu taṅ-khaṇe\\
%Etena sacca-vajjena & sotthi te hotu sabbadā\\
%Ekadā dhamma-rājā pi & gelaññenābhipīḷito\\
%Cundattherena tañ-ñeva & bhaṇāpetvāna sādaraṃ\\
%Sammoditvā ca ābādhā & tamhā vuṭṭhāsi ṭhānaso\\
%Etena sacca-vajjena & sotthi te hotu sabbadā\\
%Pahīnā te ca ābādhā & tiṇṇannam-pi mahesinaṃ\\
%Magg'āhata-kilesā va & pattānuppatti-dhammataṃ\\
%Etena sacca-vajjena & sotthi te hotu sabbadā\\
%\end{twochants}
%
%\section{Abhaya-parittaṃ}
%
%\begin{paritta}
%Yan-dunnimittaṃ avamaṅgalañ-ca\\
%Yo cāmanāpo sakuṇassa saddo\\
%Pāpaggaho dussupinaṃ akantaṃ\\
%Buddhānubhāvena vināsamentu
%
%Yan-dunnimittaṃ avamaṅgalañ-ca\\
%Yo cāmanāpo sakuṇassa saddo\\
%Pāpaggaho dussupinaṃ akantaṃ\\
%Dhammānubhāvena vināsamentu
%
%Yan-dunnimittaṃ avamaṅgalañ-ca\\
%Yo cāmanāpo sakuṇassa saddo\\
%Pāpaggaho dussupinaṃ akantaṃ\\
%Saṅghānubhāvena vināsamentu
%\end{paritta}
%
%\section{Devatā-uyyojana-gāthā}
%
%\begin{twochants}
%Dukkhappattā ca niddukkhā & bhayappattā ca nibbhayā\\
%Sokappattā ca nissokā & hontu sabbe pi pāṇino\\
%Ettāvatā ca amhehi & sambhataṃ puñña-sampadaṃ\\
%Sabbe devānumodantu & sabba-sampatti-siddhiyā\\
%Dānaṃ dadantu saddhāya & sīlaṃ rakkhantu sabbadā\\
%Bhāvanābhiratā hontu & gacchantu devatā-gatā\\\relax
%[Sabbe buddhā] balappattā & paccekānañ-ca yaṃ balaṃ\\
%Arahantānañ-ca tejena & rakkhaṃ bandhāmi sabbaso\\
%\end{twochants}
%
%\section{Jaya-maṅgala-aṭṭha-gāthā}
%
%\begin{paritta}
%Bāhuṃ sahassam-abhinimmita sāvudhan-taṃ\\
%Grīmekhalaṃ udita-ghora-sasena-māraṃ\\
%Dān'ādi-dhamma-vidhinā jitavā mun'indo\\
%Tan-tejasā bhavatu te jaya-maṅgalāni
%
%Mārātirekam-abhiyujjhita-sabba-rattiṃ\\
%Ghoram-pan'āḷavakam-akkhama-thaddha-yakkhaṃ\\
%Khantī-sudanta-vidhinā jitavā mun'indo\\
%Tan-tejasā bhavatu te jaya-maṅgalāni
%
%Nāḷāgiriṃ gaja-varaṃ atimatta-bhūtaṃ\\
%Dāv'aggi-cakkam-asanīva sudāruṇan-taṃ\\
%Mett'ambu-seka-vidhinā jitavā mun'indo\\
%Tan-tejasā bhavatu te jaya-maṅgalāni
%
%Ukkhitta-khaggam-atihattha-sudāruṇan-taṃ\\
%Dhāvan-ti-yojana-path'aṅguli- mālavantaṃ\\
%Iddhī'bhisaṅkhata-mano jitavā mun'indo\\
%Tan-tejasā bhavatu te jaya-maṅgalāni
%
%Katvāna kaṭṭham-udaraṃ iva gabbhinīyā\\
%Ciñcāya duṭṭha-vacanaṃ jana-kāya majjhe\\
%Santena soma-vidhinā jitavā mun'indo\\
%Tan-tejasā bhavatu te jaya-maṅgalāni
%
%Saccaṃ vihāya-mati-saccaka-vāda-ketuṃ\\
%Vādābhiropita-manaṃ ati-andha-bhūtaṃ\\
%Paññā-padīpa-jalito jitavā mun'indo\\
%Tan-tejasā bhavatu te jaya-maṅgalāni
%
%Nandopananda-bhujagaṃ vibudhaṃ mah'iddhiṃ\\
%Puttena thera-bhujagena damāpayanto\\
%Iddhūpadesa-vidhinā jitavā mun'indo\\
%Tan-tejasā bhavatu te jaya-maṅgalāni
%
%Duggāha-diṭṭhi-bhujagena sudaṭṭha-hatthaṃ\\
%Brahmaṃ visuddhi-jutim-iddhi-bakābhidhānaṃ\\
%Ñāṇāgadena vidhinā jitavā mun'indo\\
%Tan-tejasā bhavatu te jaya-maṅgalāni
%
%Etā pi buddha-jaya-maṅgala-aṭṭha-gāthā\\
%Yo vācano dina-dine saratem-atandī\\
%Hitvān'aneka-vividhāni c'upaddavāni\\
%Mokkhaṃ sukhaṃ adhigameyya naro sapañño
%\end{paritta}
%
%\section{Jaya-parittaṃ}
%
%\begin{paritta}
%
%Mahā-kāruṇiko nātho\\
%Hitāya sabba-pāṇinaṃ\\
%Pūretvā pāramī sabbā\\
%Patto sambodhim-uttamaṃ\\
%Etena sacca-vajjena\\
%Hotu te jaya-maṅgalaṃ\\
%Jayanto bodhiyā mūle\\
%Sakyānaṃ nandi-vaḍḍhano\\
%Evaṃ tvaṃ vijayo hohi\\
%Jayassu jaya-maṅgale\\
%Aparājita-pallaṅke\\
%Sīse paṭhavi-pokkhare\\
%Abhiseke sabba-buddhānaṃ\\
%Aggappatto pamodati\\
%Sunakkhattaṃ sumaṅgalaṃ\\
%Supabhātaṃ suhuṭṭhitaṃ\\
%Sukhaṇo sumuhutto ca\\
%Suyiṭṭhaṃ brahma-cārisu\\
%Padakkhiṇaṃ kāya-kammaṃ\\
%Vācā-kammaṃ padakkhiṇaṃ\\
%Padakkhiṇaṃ mano-kammaṃ\\
%Paṇidhi te padakkhiṇā\\
%Padakkhiṇāni katvāna\\
%Labhant'atthe padakkhiṇe
%
%(♦) So attha-laddho sukhito,\\
%Viruḷho buddha-sāsane;\\
%Arogo sukhito hohi,\\
%Saha sabbehi ñātibhi.\\
%Sā attha-laddhā sukhitā,\\
%Viruḷhā buddha-sāsane;\\
%Arogā sukhitā hohi,\\
%Saha sabbehi ñātibhi.\\
%Te attha-laddhā sukhitā,\\
%Viruḷhā buddha-sāsane;\\
%Arogā sukhitā hotha,\\
%Saha sabbehi ñātibhi.
%
%(♦) Sakkatvā Buddha-ratanaṃ,\\
%Osathaṃ uttamaṃ varaṃ;\\
%Hitaṃ deva-manussānaṃ,\\
%Buddha-tejena sotthinā;\\
%Nassant'upaddavā sabbe,\\
%Dukkhā vūpasamentu te.\\
%Sakkatvā Dhamma-ratanaṃ,\\
%Osathaṃ uttamaṃ varaṃ;\\
%Pariḷāhūpasamanaṃ,\\
%Dhamma-tejena sotthinā;\\
%Nassant'upaddavā sabbe,\\
%Bhayā vūpasamentu te.\\
%Sakkatvā Saṅgha-ratanaṃ,\\
%Osathaṃ uttamaṃ varaṃ;\\
%Āhuneyyaṃ pāhuneyyaṃ,\\
%Saṅgha-tejena sotthinā;\\
%Nassant'upaddavā sabbe,\\
%Rogā vūpasamentu te.
%
%(♦) N'atthi me saraṇaṃ aññaṃ,\\
%Buddho me saraṇaṃ varaṃ;\\
%Etena sacca-vajjena,\\
%Sotthi te hotu sabbadā.
%
%N'atthi me saraṇaṃ aññaṃ,\\
%Dhammo me saraṇaṃ varaṃ;\\
%Etena sacca-vajjena,\\
%Sotthi te hotu sabbadā.
%
%N'atthi me saraṇaṃ aññaṃ,\\
%Saṅgho me saraṇaṃ varaṃ;\\
%Etena sacca-vajjena,\\
%Sotthi te hotu sabbadā.
%
%(♦) Yaṅ kiñci ratanaṃ loke\\
%vijjati vividhaṃ puthu\\
%Ratanaṃ Buddha-samaṃ n'atthi\\
%tasmā sotthī bhavantu te.\\
%Yaṅ kiñci ratanaṃ loke\\
%vijjati vividhaṃ puthu\\
%Ratanaṃ Dhamma-samaṃ n'atthi\\
%tasmā sotthī bhavantu te.\\
%Yaṅ kiñci ratanaṃ loke\\
%vijjati vividhaṃ puthu\\
%Ratanaṃ Saṅgha-samaṃ n'atthi\\
%tasmā sotthī bhavantu te.
%
%\end{paritta}
%
%\section{Pabbatopama-gāthā}
%
%Yathā pi selā vipulā,\\
%Nabhaṃ āhacca pabbatā;\\
%Samantā anupariyeyyuṃ,\\
%Nippothentā catuddisā;\\
%Evaṃ jarā ca maccu ca,\\
%Adhivattanti pāṇino;\\
%Khattiye brāhmaṇe vesse,\\
%Sudde caṇḍāla-pukkuse;\\
%Na kiñci parivajjeti,\\
%Sabbam-evābhimaddati;\\
%Na tattha hatthīnaṃ bhūmi,\\
%Na rathānaṃ na pattiyā;\\
%Na cāpi manta-yuddhena,\\
%Sakkā jetuṃ dhanena vā;\\
%Tasmā hi paṇḍito poso,\\
%Sampassaṃ attham-attano;\\
%Buddhe Dhamme ca Saṅghe ca,\\
%Dhīro saddhaṃ nivesaye;\\
%Yo Dhamma-cārī kāyena,\\
%Vācāya uda cetasā;\\
%Idh'eva naṃ pasaṃsanti,\\
%Pecca sagge pamodati.
%
%\section{Verses on the Burden}
%
%\begin{leader}
%  [Handa mayaṃ bhāra-sutta-gāthāyo bhaṇāmase]
%\end{leader}
%
%\begin{twochants}
%Bhārā have pañcakkhandhā & bhāra-hāro ca puggalo \\
%Bhār'ādānaṃ dukkhaṃ loke & bhāra-nikkhepanaṃ sukhaṃ \\
%Nikkhipitvā garuṃ bhāraṃ & aññaṃ bhāraṃ anādiya \\
%Samūlaṃ taṇhaṃ abbuyha & nicchāto parinibbuto \\
%\end{twochants}
%
%\section{Khemākhema-saraṇa-gamana-paridīpikā-gāthā}
%
%Bahuṃ ve saraṇaṃ yanti,\\
%Pabbatāni vanāni ca;\\
%Ārāma-rukkha-cetyāni,\\
%Manussā bhaya-tajjitā.\\
%N'etaṃ kho saraṇaṃ khemaṃ,\\
%N'etaṃ saraṇam-uttamaṃ;\\
%N'etaṃ saraṇam-āgamma,\\
%Sabba-dukkhā pamuccati.\\
%Yo ca Buddhañ-ca Dhammañ-ca,\\
%Saṅghañ-ca saraṇaṃ gato;\\
%Cattāri ariya-saccāni,\\
%Sammappaññāya passati.\\
%Dukkhaṃ dukkha-samuppādaṃ,\\
%Dukkhassa ca atikkamaṃ;\\
%Ariyañ-c'aṭṭh'aṅgikaṃ maggaṃ,\\
%Dukkhūpasama-gāminaṃ.\\
%Etaṃ kho saraṇaṃ khemaṃ,\\
%Etaṃ saraṇam-uttamaṃ;\\
%Etaṃ saraṇam-āgamma,\\
%Sabba-dukkhā pamuccatī-ti.
%
%\section{Verses on a Shining Night of Prosperity}
%
%\begin{leader}
%  [Handa mayaṃ bhadd'eka-ratta-gāthāyo bhaṇāmase]
%\end{leader}
%
%\begin{twochants}
%  Atītaṃ nānvāgameyya & nappaṭikaṅkhe anāgataṃ \\
%  Yad'atītaṃ pahīnan-taṃ & appattañca anāgataṃ \\
%  Paccuppannañca yo dhammaṃ & tattha tattha vipassati \\
%  Asaṃhiraṃ asaṅkuppaṃ & taṃ viddhām-anubrūhaye \\
%  Ajj'eva kiccam-ātappaṃ & ko jaññā maraṇaṃ suve \\
%  Na hi no saṅgaran-tena & mahā-senena maccunā \\
%  Evaṃ vihārim-ātāpiṃ & aho-rattam-atanditaṃ \\
%  Taṃ ve bhadd'eka-ratto'ti & santo ācikkhate muni \\
%\end{twochants}
%
%\section{Verses on the Three Characteristics}
%
%\begin{leader}
%  [Handa mayaṃ ti-lakkhaṇ'ādi-gāthāyo bhaṇāmase]
%\end{leader}
%
%\begin{twochants}
%  Sabbe saṅkhārā aniccā'ti & yadā paññāya passati \\
%  Atha nibbindati dukkhe & esa maggo visuddhiyā \\
%  Sabbe saṅkhārā dukkhā'ti & yadā paññāya passati \\
%  Atha nibbindati dukkhe & esa maggo visuddhiyā \\
%  Sabbe dhammā anattā'ti & yadā paññāya passati \\
%  Atha nibbindati dukkhe & esa maggo visuddhiyā \\
%  Appakā te manussesu & ye janā pāra-gāmino \\
%  Athāyaṃ itarā pajā & tīram-evānudhāvati \\
%  Ye ca kho sammad-akkhāte & dhamme dhammānuvattino \\
%  Te janā pāram-essanti & maccu-dheyyaṃ suduttaraṃ \\
%  Kaṇhaṃ dhammaṃ vippahāya & sukkaṃ bhāvetha paṇḍito \\
%  Okā anokam-āgamma & viveke yattha dūramaṃ \\
%  Tatrābhiratim-iccheyya & hitvā kāme akiñcano \\
%\end{twochants}
%
%Pariyodapeyya attānaṃ,\\
%Citta-klesehi paṇḍito.\\
%Yesaṃ sambodhi-y-aṅgesu,\\
%Sammā cittaṃ subhāvitaṃ;\\
%Ādāna-paṭinissagge,\\
%Anupādāya ye ratā;\\
%Khīṇ'āsavā jutimanto,\\
%Te loke parinibbutā-ti.
%
%\section{Verses on Respect for the Dhamma}
%
%\begin{leader}
%  [Handa mayaṃ dhamma-gārav'ādi-gāthāyo bhaṇāmase]
%\end{leader}
%
%\begin{twochants}
%  Ye ca atītā sambuddhā & ye ca buddhā anāgatā \\
%  Yo c'etarahi sambuddho & bahunnaṃ soka-nāsano \\
%  Sabbe saddhamma-garuno & vihariṃsu viharanti ca \\
%  Atho pi viharissanti & esā buddhāna dhammatā \\
%  Tasmā hi atta-kāmena & mahattam-abhikaṅkhatā \\
%  Saddhammo garu-kātabbo & saraṃ buddhāna sāsanaṃ \\
%  Na hi dhammo adhammo ca & ubho sama-vipākino \\
%  Adhammo nirayaṃ neti & dhammo pāpeti suggatiṃ \\
%\end{twochants}
%
%Dhammo have rakkhati dhamma-cāriṃ\\
%Dhammo suciṇṇo sukham-āvahāti\\
%Esānisaṃso dhamme suciṇṇe
%
%(NOTE: this line is not present in the community chanting book)
%
%Na duggatiṃ gacchati dhamma-cārī.
%
%\section{Verses on the Buddha's First Exclamation}
%
%\begin{leader}
%  [Handa mayaṃ paṭhama-buddha-bhāsita-gāthāyo bhaṇāmase]
%\end{leader}
%
%\begin{twochants}
%  Aneka-jāti-saṃsāraṃ & sandhāvissaṃ anibbisaṃ \\
%  Gaha-kāraṃ gavesanto & dukkhā jāti punappunaṃ \\
%  Gaha-kāraka diṭṭho'si & puna gehaṃ na kāhasi \\
%  Sabbā te phāsukā bhaggā & gaha-kūṭaṃ visaṅkhataṃ \\
%  Visaṅkhāra-gataṃ cittaṃ & taṇhānaṃ khayam-ajjhagā \\
%\end{twochants}
%
%\section{Chants Used in Sri Lanka}
%
%\section{Salutation to the Three Main Objects of Venerations}
%
%Vandāmi cetiyaṃ sabbaṃ\\
%Sabba-ṭhānesu patiṭṭhitaṃ\\
%Sārīrīka-dhātu-Mahā-bodhiṃ\\
%Buddha-rūpaṃ sakalaṃ sadā.
%
%\section{Salutation to the Bodhi-Tree}
%
%Yassa mūle nissino va\\
%Sabbāri vijayaṃ akā,\\
%Patto sabbaññutaṃ Satthā\\
%Vande taṃ Bodhi-pādapaṃ.\\
%Ime ete Mahā-Bodhi\\
%Loka-nāthena pūjitā,\\
%Aham-pi te namassāmi\\
%Bodhi-Rājā nam'atthu te!
%
%\section{Offering of Lights}
%
%Ghana-sārappadittena\\
%Dīpena tama-dhaṃsinā\\
%Tīloka-dīpam sambuddhaṃ\\
%Pūjayāmi tamo-nudaṃ.
%
%\section{Offering of Incense}
%
%Gandha-sambhāra-yuttena\\
%Dhūpenāhaṃ sugandhinā\\
%Pūjaye pūjaneyyan-taṃ\\
%Pūjā-bhājanam-uttamaṃ.
%
%\section{Offering of Flowers}
%
%Vaṇṇa-gandha-guṇopetaṃ\\
%Etaṃ kusuma-santatiṃ.\\
%Pūjayāmi munindassa\\
%Sirīpāda-saroruhe.\\
%Pūjemi Buddhaṃ kusumena'nena\\
%Puññenam-etena ca hotu mokkhaṃ\\
%Pupphaṃ milāyāti yathā idaṃ me\\
%Kāyo tathā yāti vināsa-bhāvaṃ.
%
%\section{Transference of Merit to Devas}
%
%Ākāsatthā ca bhummatthā\\
%Devā nāgā mah'iddhikā\\
%Puññaṃ taṃ anumoditvā
%
%Ciraṃ rakkhantu ...
%
%% FIXME formatting
%
%%  /loka/ sāsanaṃ.
%% Ciraṃ rakkhantu   desanaṃ
%%  maṃ paraṃ
%
%Ettāvatā ca amhehi\\
%Sambhataṃ puñña-sampadaṃ\\
%Sabbe devā/ bhūtā/ sattā anumodantu\\
%Sabba-sampatti siddhiyā.
%
%\section{Blessing to the World}
%
%Devo vassatu kālena\\
%Sassa-sampatti-hetu ca\\
%Phīto bhavatu loko ca\\
%Rajā bhavatu dhammiko.
%
%\section{Transference of Merits to Departed Ones}
%
%Idam te...
%
%% Idam te/ vo/ no/ me * ñātīnam hotu
%% sukhitā hontu ñātayo.
%% (×3)
%
%\section{The Aspirations}
%
%Iminā puñña-kammena\\
%Mā me bāla-samāgamo,\\
%Sataṃ samāgamo hotu,\\
%Yāva nibbāna-pattiyā.\\
%Kāyena vācā-cittena\\
%Pamādena mayā kataṃ\\
%Accayaṃ khama me bhante\\
%Bhūri-pañña Tathāgata.
%
%\section{Blessing and Protection}
%
%Sabb'ītiyo vivajjantu,\\
%Sabba-rogo vinassatu;\\
%Mā me/no bhavatvantarāyo,\\
%Sukhī dīghāyuko bhava.\\
%/Sukhī dīghāyukā homa.\\
%Bhavatu sabba-maṅgalaṃ.\\
%Rakkhantu sabba-devatā.
%
%When chanting for one person use ‘te’; when for laypeople use ‘vo’; when chanting together in a group use
%‘no’; when alone use ‘me’.
%
%Sabba-buddhānubhāvena,\\
%Sadā sotthi bhavantu me.\\
%Bhavatu sabba-maṅgalaṃ.\\
%Rakkhantu sabba-devatā.\\
%Sabba-dhammānunbhāvena,\\
%Sadā sotthi bhavantu me.\\
%Bhavatu sabba-maṅgalaṃ.\\
%Rakkhantu sabba-devatā.\\
%Sabba-saṅghānubhāvena,\\
%Sadā sotthi bhavantu me.\\
%Nakkhatta-yakkha-bhūtānaṃ\\
%Pāpaggaha-nivāraṇā\\
%Parittassānubhāvena\\
%Hantvā mayhaṃ/amhe upaddave.\\
%Devo vassatu kālena.\\
%Sassa-sampatti-hetu ca.\\
%Phīto bhavatu loko ca.\\
%Rājā bhavatu dhammiko.\\
%Sabbe buddhā balappattā,\\
%Paccekānañca yaṃ balaṃ\\
%Arahantānañca tejena,\\
%Rakkhaṃ bandhāmi sabbaso.
%
%\section{Mettā Bhāvanā}
%
%1. Attūpamāya sabbesaṃ\\
%Sattānaṃ sukhakāmataṃ,\\
%Passitvā kamato mettaṃ\\
%Sabbasattesu bhāvaye.\\
%2. Sukhi bhaveyyaṃ niddukkho\\
%Ahaṃ niccaṃ ahaṃ viya,\\
%Hitā ca me sukhī hontu\\
%Majjhatthā c'atha verino.\\
%3. Imamhi gāmakkhettamhi\\
%Sattā hontu sukhī sadā,\\
%Tato parañ ca-rajjesu\\
%Cakkavāḷesu jantuno.\\
%4. Samantā cakkavāḷesu\\
%Sattānan-tesu pāṇino,\\
%Sukhino puggala bhūtā\\
%Attabhāvagatā siyuṃ.\\
%5. Tathā itthī pumā ce'va\\
%Ariya anariya’ pi ca,\\
%Devā narā apāyaṭṭhā\\
%Tathā dasa disāsu cā-ti.
%
%\section{Pattanumodana}
%
%(Sharing Merits)
%
%Idaṃ te...
%
%% Idaṃ te/ vo/ no/ me* ñātīnaṃ hotu
%% Sukhitā hontu ñātayo (×3)
%
%Yathā vāri-vahā pūrā\\
%Paripūrenti sāgaraṃ,\\
%Evaṃ eva ito dinnaṃ\\
%Petānaṃ upakappatu.\\
%Unname udakaṃ vattaṃ\\
%Yathā ninnaṃ pavattati,\\
%Evaṃ eva ito dinnaṃ\\
%Petānaṃ upakappatu.\\
%Āyūr-arogya-sampatti\\
%Sagga-sampattiṃ eva ca,\\
%Atho nibbāna-sampatti,\\
%Iminā te/* samijjhatu.\\
%Icchitaṃ patthitaṃ tuyhaṃ\\
%Sabbam-eva samijjhatu,\\
%Pūrentu citta-saṅkappā\\
%Maṇi-joti-raso yathā.\\
%Icchitaṃ patthitaṃ tuyhaṃ,\\
%Sabbam-eva samijjhatu,\\
%Pūrentu citta-saṅkappā\\
%Cando paṇṇa-rasī yathā.\\
%Icchitaṃ patthitaṃ tuyhaṃ\\
%Khippam-eva samijjhatu,\\
%Sabbe pūrentu saṅkappā\\
%Cando paṇṇa-rasī yathā.

