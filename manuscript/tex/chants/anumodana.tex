\chapter{Anumodanā}

\chapter{Just as Rivers}

Yathā vāri-vahā pūrā paripūrenti sāgaraṃ\\
Evam-eva ito dinnaṃ petānaṃ upakappati\\
Icchitaṃ patthitaṃ tumhaṃ\\
Khippam-eva samijjhatu\\
Sabbe pūrentu saṅkappā\\
Cando paṇṇaraso yathā\\
Maṇi jotiraso yathā\\
Sabb'ītiyo vivajjantu\\
Sabba-rogo vinassatu\\
Mā te bhavatv-antarāyo\\
Sukhī dīgh'āyuko bhava\\
Abhivādana-sīlissa niccaṃ vuḍḍhāpacāyino\\
Cattāro dhammā vaḍḍhanti\\
Āyu vaṇṇo sukhaṃ balaṃ

Bhavatu sabba-maṅgalaṃ\\
Rakkhantu sabba-devatā\\
Sabba-buddhānubhāvena\\
Sadā sotthī bhavantu te

Bhavatu sabba-maṅgalaṃ\\
Rakkhantu sabba-devatā\\
Sabba-dhammānubhāvena\\
Sadā sotthī bhavantu te

Bhavatu sabba-maṅgalaṃ\\
Rakkhantu sabba-devatā\\
Sabba-saṅghānubhāvena\\
Sadā sotthī bhavantu te

(NOTE: alternative chant)

[/Sabba-roga-vinimutto,]\\
Sabba-santāpa-vajjito;\\
Sabba-veram-atikkanto,\\
Nibbuto ca tuvam-bhava;/]\\
Sabb’ītiyo vivajjantu,\\
Sabba-rogo vinassatu;\\
Mā te bhavatv-antarāyo,\\
Sukhī dīgh’āyuko bhava; /(×3)\\
Abhivādana-sīlissa,\\
Niccaṃ vuḍḍhāpacāyino;\\
Cattāro dhammā vaḍḍhanti,\\
Āyu vaṇṇo sukhaṃ balaṃ.

\chapter{Ratanattay’ānubhāv’ādi-gāthā}

Ratanattay’ānubhāvena\\
Ratanattaya-tejasā\\
Dukkha-roga-bhayā verā\\
Sokā sattu c’upaddavā\\
Anekā antarāyā pi\\
Vinassantu asesato\\
Jaya-siddhi dhanaṃ lābhaṃ\\
Sotthi bhāgyaṃ sukhaṃ balaṃ\\
Siri āyu ca vaṇṇo ca\\
Bhogaṃ vuḍḍhī ca yasavā\\
Sata-vassā ca āyu ca\\
Jīva-siddhī bhavantu te.

\chapter{Saṅgha-vatthu-gāthā}

Dānañ-ca peyya-vajjañ-ca\\
Attha-cariyā ca yā idha\\
Samānattatā ca dhammesu\\
Tattha tattha yathā’rahaṃ\\
Ete kho saṅgahā loke\\
Rathass’āṇīva yāyato\\
Ete ca saṅgahā nāssu\\
Na mātā putta-kāraṇā\\
Labhetha mānaṃ pūjaṃ vā\\
Pitā vā putta-kāraṇā\\
Yasmā ca saṇgahā ete\\
Samavekkhanti paṇḍitā\\
Tasmā mahattaṃ papponti\\
Pāsaṃsā ca bhavanti te-ti.

\chapter{Bhojana-dānānumodanā}

Āyu-do bala-do dhīro,\\
Vaṇṇa-do paṭibhāṇa-do;\\
Sukhassa dātā medhāvī,\\
Sukhaṃ so adhigacchati.\\
Āyuṃ datvā balaṃ vaṇṇaṃ,\\
Sukhañ-ca paṭibhāna-do;\\
Dīgh’āyu yasavā hoti,\\
Yattha yatthūpapajjatī-ti.

\chapter{Ādiya-sutta-gāthā}

Bhuttā bhogā bhaṭā bhaccā,\\
Vitiṇṇā āpadāsu me;\\
Uddhaggā dakkhiṇā dinnā,\\
Atho pañca balī katā;\\
Upaṭṭhitā sīlavanto,\\
Saññatā brahma-cārino;\\
Yad-atthaṃ bhogam-iccheyya,\\
Paṇḍito gharam-āvasaṃ;\\
So me attho anuppatto,\\
Kataṃ ananutāpiyaṃ:\\
Etaṃ anussaraṃ macco,\\
Ariya-dhamme ṭhito naro;\\
Idh’eva naṃ pasaṃsanti,\\
Pecca sagge ca pamodatī-ti.

\chapter{Culla-maṅgala-cakka-vāḷa}

Sabba-buddh’ānubhāvena sabba-dhamm’ānubhāvena sabba-saṅgh’ānubhāvena Buddharatanaṃ dhamma-ratanaṃ saṅgha-ratanaṃ\\
Tiṇṇaṃ ratanānaṃ ānubhāvena\\
Catur-āsīti-sahassa-dhammakkhandh’ānubhāvena\\
Piṭakattay’ānubhāvena\\
Jina-sāvak’ānubhāvena\\
Sabbe te rogā\\
Sabbe te bhayā\\
Sabbe te antarāyā\\
Sabbe te upaddavā\\
Sabbe te dunnimittā\\
Sabbe te avamaṅgalā vinassantu\\
āyu-vaḍḍhako /āyu-vaḍḍhakā *\\
dhana-vaḍḍhako/ā\\
siri-vaḍḍhako/ā\\
yasa-vaḍḍhako/ā\\
bala-vaḍḍhako/ā\\
vaṇṇa-vaḍḍhako/ā\\
sukha-vaḍḍhako/ā\\
hotu sabbadā.

Dukkha-roga-bhayā verā,\\
Sokā sattu c’upaddavā;\\
Anekā antarāyā pi,\\
Vinassantu ca tejasā;\\
Jaya-siddhi dhanaṃ lābhaṃ,\\
Sotthi bhāgyaṃ sukhaṃ balaṃ;\\
Siri āyu ca vaṇṇo ca,\\
Bhogaṃ vuḍḍhī ca yasavā;\\
Sata-vassā ca āyū ca,\\
Jīva-siddhī bhavantu te.

/when chanting for women.

Bhavatu sabba-maṅgalaṃ...

\chapter{Aggappasāda-sutta-gāthā}

Aggato ve pasannānaṃ,\\
Aggaṃ dhammaṃ vijānataṃ;\\
Agge Buddhe pasannānaṃ,\\
Dakkhiṇeyye anuttare;\\
Agge dhamme pasannānaṃ,\\
Virāgūpasame sukhe;\\
Agge saṅghe pasannānaṃ,\\
Puññakkhette anuttare.\\
Aggasmiṃ dānaṃ dadataṃ,\\
Aggaṃ puññaṃ pavaḍḍhati;\\
Aggaṃ āyu ca vaṇṇo ca,\\
Yaso kitti sukhaṃ balaṃ;\\
Aggassa dātā medhāvī,\\
Agga-dhamma-samāhito;\\
Deva-bhūto manusso vā,\\
Aggappatto pamodatī-ti.

\chapter{Ariya-dhana-gāthā}

Yassa saddhā Tathāgate,\\
Acalā supatiṭṭhitā ,\\
Sīlañ-ca yassa kalyāṇaṃ,\\
Ariya-kantaṃ pasaṃsitaṃ;\\
Saṅghe pasādo yass’atthi,\\
Uju-bhūtañ-ca dassanaṃ;\\
Adaliddo-ti taṃ āhu,\\
Amoghaṃ tassa jīvitaṃ;\\
Tasmā saddhañ-ca sīlañ-ca,\\
Pasādaṃ dhamma-dassanaṃ;\\
Anuyuñjetha medhāvī,\\
Saraṃ buddhāna sāsanan-ti.

\chapter{Devat’ādissa-dakkhiṇā’numodanā-gāthā}

Yasmiṃ padese kappeti,\\
Vāsaṃ paṇḍita-jātiyo;\\
Sīlavant’ettha bhojetvā,\\
Saññate brahma-cārino;\\
Yā tattha devatā āsuṃ,\\
Tāsaṃ dakkhiṇam-ādise;\\
Tā pūjitā pūjayanti,\\
Mānitā mānayanti naṃ;\\
Tato naṃ anukampanti,\\
Mātā puttaṃ va orasaṃ;\\
Devatā’nukampito poso,\\
Sadā bhadrāni passati.

\chapter{Kāla-dāna-sutta-gāthā}

Kāle dadanti sapaññā,\\
Vadaññū vīta-maccharā;\\
Kālena dinnaṃ ariyesu,\\
Uju-bhūtesu tādisu;\\
Vippasanna-manā tassa,\\
Vipulā hoti dakkhiṇā.\\
Ye tattha anumodanti,\\
Veyyāvaccaṃ karonti vā;\\
Na tena dakkhiṇā onā,\\
Te pi puññassa bhāgino.\\
Tasmā dade appaṭivāna-citto,\\
Yattha dinnaṃ mahapphalaṃ;\\
Puññāni para-lokasmiṃ,\\
Patiṭṭhā honti pāṇinan-ti.

\chapter{So Attha-laddho}

So attha-laddho sukhito,\\
Viruḷho Buddha-sāsane;\\
Arogo sukhito hohi,\\
Saha sabbehi ñātibhi.\\
Sā attha-laddhā sukhitā,\\
Viruḷhā Buddha-sāsane;\\
Arogā sukhitā hohi,\\
Saha sabbehi ñātibhi.\\
Te attha-laddhā sukhitā,\\
Viruḷhā Buddha-sāsane;\\
Arogā sukhitā hotha,\\
Saha sabbehi ñātibhi.

\chapter{Adāsi-me ādi-gāthā}

(Tiro-kuḍḍa-kaṇḍaṃ)

Adāsi me akāsi me,\\
Ñāti-mittā sakhā ca me;\\
Petānaṃ dakkhiṇaṃ dajjā,\\
Pubbe katam-anussaraṃ.\\
Na hi ruṇṇaṃ vā soko vā,\\
Yā v’aññā paridevanā;\\
Na taṃ petānam-atthāya,\\
Evaṃ tiṭṭhanti ñātayo.

(♦) Ayañ-ca kho dakkhiṇā dinnā,\\
Saṅghamhi supatiṭṭhitā ;\\
Dīgha-rattaṃ hitāy’assa,\\
Ṭhānaso upakappati.\\
So ñāti-dhammo ca ayaṃ nidassito,\\
Petāna’pūjā ca katā uḷārā;\\
Balañ-ca bhikkhūnam-anuppadinnaṃ,\\
Tumhehi puññaṃ pasutaṃ anappakan-ti.

