\chapter{Anumodanā}

\section{Just as Rivers}

\firstline{Yathā vāri-vahā pūrā}

\begin{paritta}

\sidepar{A.}%
Yathā vāri-vahā pūrā\\
Paripūrenti sāgaraṃ\\
Evam-eva ito dinnaṃ\\
Petānaṃ upakappati\hfill \suttaRef{Khp.VII.v8}

\begin{english}
  Just as rivers full of water entirely fill up the sea\\
  So will what's here been given bring blessings to departed spirits.
\end{english}

Icchitaṃ patthitaṃ tumhaṃ\\
Khippam-eva samijjhatu\\
Sabbe pūrentu saṅkappā\\
Cando paṇṇaraso yathā\\
Maṇi jotiraso yathā\hfill \suttaRef{DhpA.I.198}

\begin{english}
  May all your hopes and all your longings\\
  Come true in no long time.\\
  May all your wishes be fulfilled\\
  Like on the fifteenth day the moon\\
  Or like a bright and shining gem.
\end{english}

\firstline{Sabba-roga-vinimutto}

\sidepar{B.}%
Sabba-roga-vinimutto\\
Sabba-santāpa-vajjito\\
Sabba-veram-atikkanto\\
Nibbuto ca tuvam-bhava

\begin{english}
  (TODO: add translation)\todo{add translation}
\end{english}

\firstline{Sabb'ītiyo vivajjantu sabba-rogo vinassatu}

Sabb'ītiyo vivajjantu\\
Sabba-rogo vinassatu\\
Mā te bhavatv-antarāyo\\
Sukhī dīgh'āyuko bhava

\begin{english}
  May all misfortunes be avoided,\\
  May all illness be dispelled,\\
  May you never meet with dangers,\\
  May you be happy and live long.
\end{english}

\firstline{Abhivādana-sīlissa}

Abhivādana-sīlissa\\
Niccaṃ vuḍḍhāpacāyino\\
Cattāro dhammā vaḍḍhanti\\
Āyu vaṇṇo sukhaṃ balaṃ\hfill \suttaRef{Dhp 109}

\begin{english}
  For those who are respectful, who always honour the elders,\\
  Four are the qualities which will increase:\\
  Life, beauty, happiness and strength.
\end{english}

\firstline{Bhavatu sabba-maṅgalaṃ}

Bhavatu sabba-maṅgalaṃ, rakkhantu sabba-devatā\\
Sabba-buddhānubhāvena, sadā sotthī bhavantu te

\clearpage

Bhavatu sabba-maṅgalaṃ, rakkhantu sabba-devatā\\
Sabba-dhammānubhāvena, sadā sotthī bhavantu te

Bhavatu sabba-maṅgalaṃ, rakkhantu sabba-devatā\\
Sabba-saṅghānubhāvena, sadā sotthī bhavantu te

\begin{english}
  May every blessing come to be\\
  And all good spirits guard you well.\\
  Through the power of all Buddhas\\
  \ldots\ Dhammas \ldots\ Saṅghas\\
  May you always be at ease.
\end{english}

\end{paritta}

\section{Ratanattay'ānubhāv'ādi-gāthā}

\firstline{Ratanattay'ānubhāvena ratanattaya-tejasā}

\begin{twochants}
Ratanattay'ānubhāvena & ratanattaya-tejasā\\
Dukkha-roga-bhayā verā & sokā sattu c'upaddavā\\
Anekā antarāyā pi & vinassantu asesato\\
Jaya-siddhi dhanaṃ lābhaṃ & sotthi bhāgyaṃ sukhaṃ balaṃ\\
Siri āyu ca vaṇṇo ca & bhogaṃ vuḍḍhī ca yasavā\\
Sata-vassā ca āyu ca & jīva-siddhī bhavantu te.
\end{twochants}

\section{Bhojana-dānānumodanā}

\firstline{Āyu-do bala-do dhīro vaṇṇa-do paṭibhāṇa-do}

\begin{twochants}
Āyu-do bala-do dhīro & vaṇṇa-do paṭibhāṇa-do;\\
Sukhassa dātā medhāvī & sukhaṃ so adhigacchati.\\
Āyuṃ datvā balaṃ vaṇṇaṃ & sukhañ-ca paṭibhāna-do;\\
Dīgh'āyu yasavā hoti & yattha yatthūpapajjatī-ti.
\end{twochants}

\suttaRef{A.III.42}

\vspace*{-\baselineskip}

\section{Saṅgha-vatthu-gāthā}

\firstline{Dānañ-ca peyya-vajjañ-ca attha-cariyā ca yā idha}

\begin{twochants}
Dānañ-ca peyya-vajjañ-ca & attha-cariyā ca yā idha\\
Samānattatā ca dhammesu & tattha tattha yathā'rahaṃ\\
Ete kho saṅgahā loke & rathass'āṇīva yāyato\\
Ete ca saṅgahā nāssu & na mātā putta-kāraṇā\\
Labhetha mānaṃ pūjaṃ vā & pitā vā putta-kāraṇā\\
Yasmā ca saṇgahā ete & samavekkhanti paṇḍitā\\
Tasmā mahattaṃ papponti & pāsaṃsā ca bhavanti te-ti.
\end{twochants}

\suttaRef{A.II.32}

\vspace*{-\baselineskip}

\section{Ādiya-sutta-gāthā}

\firstline{Bhuttā bhogā bhaṭā bhaccā vitiṇṇā āpadāsu me}

\begin{twochants}
Bhuttā bhogā bhaṭā bhaccā & vitiṇṇā āpadāsu me;\\
Uddhaggā dakkhiṇā dinnā & atho pañca balī katā;\\
Upaṭṭhitā sīlavanto & saññatā brahma-cārino;\\
Yad-atthaṃ bhogam-iccheyya & paṇḍito gharam-āvasaṃ;\\
So me attho anuppatto & kataṃ ananutāpiyaṃ:\\
Etaṃ anussaraṃ macco & ariya-dhamme ṭhito naro;\\
Idh'eva naṃ pasaṃsanti & pecca sagge ca pamodatī-ti.
\end{twochants}

\suttaRef{A.III.46}

\section{Culla-maṅgala-cakka-vāḷa}

\firstline{Sabba-buddh'ānubhāvena}

\begin{paritta}

Sabba-buddh'ānubhāvena, sabba-dhamm'ānubhāvena, sabba-saṅgh'ānubhāvena

Buddha-ratanaṃ, dhamma-ratanaṃ, saṅgha-ratanaṃ

Tiṇṇaṃ ratanānaṃ ānubhāvena\\
Catur-āsīti-sahassa-dhammakkhandh'ānubhāvena\\
Piṭakattay'ānubhāvena\\
Jina-sāvak'ānubhāvena

Sabbe te rogā, sabbe te bhayā, sabbe te antarāyā, sabbe te upaddavā, sabbe te
dunnimittā, sabbe te avamaṅgalā vinassantu

Āyu-vaḍḍhako, dhana-vaḍḍhako, siri-vaḍḍhako, yasa-vaḍḍhako, bala-vaḍḍhako,
vaṇṇa-vaḍḍhako, sukha-vaḍḍhako, hotu sabbadā

Dukkha-roga-bhayā verā, sokā sattu c'upaddavā\\
Anekā antarāyā pi, vinassantu ca tejasā\\
\mbox{Jaya-siddhi dhanaṃ lābhaṃ, sotthi bhāgyaṃ sukhaṃ balaṃ}\\
Siri āyu ca vaṇṇo ca, bhogaṃ vuḍḍhī ca yasavā\\
Sata-vassā ca āyū ca, jīva-siddhī bhavantu te

Bhavatu sabba-maṅgalaṃ\ldots{}

\end{paritta}

%\suttaRef{MJG}

\section{Ariya-dhana-gāthā}

\firstline{Yassa saddhā tathāgate acalā supatiṭṭhitā}

\begin{twochants}
Yassa saddhā tathāgate & acalā supatiṭṭhitā\\
Sīlañ-ca yassa kalyāṇaṃ & ariya-kantaṃ pasaṃsitaṃ\\
Saṅghe pasādo yass'atthi & uju-bhūtañ-ca dassanaṃ\\
Adaliddo-ti taṃ āhu & amoghaṃ tassa jīvitaṃ\\
Tasmā saddhañ-ca sīlañ-ca & pasādaṃ dhamma-dassanaṃ\\
Anuyuñjetha medhāvī & saraṃ buddhāna sāsanan-ti
\end{twochants}

\suttaRef{A.III.54}

\clearpage

\section{Aggappasāda-sutta-gāthā}

\firstline{Aggato ve pasannānaṃ}

\begin{twochants}
Aggato ve pasannānaṃ & aggaṃ dhammaṃ vijānataṃ\\
Agge Buddhe pasannānaṃ & dakkhiṇeyye anuttare\\
Agge dhamme pasannānaṃ & virāgūpasame sukhe\\
Agge saṅghe pasannānaṃ & puññakkhette anuttare\\
Aggasmiṃ dānaṃ dadataṃ & aggaṃ puññaṃ pavaḍḍhati\\
Aggaṃ āyu ca vaṇṇo ca & yaso kitti sukhaṃ balaṃ\\
Aggassa dātā medhāvī & agga-dhamma-samāhito\\
Deva-bhūto manusso vā & aggappatto pamodatī-ti
\end{twochants}

\suttaRef{A.II.35; A.III.36}

\vspace*{-\baselineskip}

\section{Devat'ādissa-dakkhiṇā'numodanā-gāthā}

\firstline{Yasmiṃ padese kappeti vāsaṃ paṇḍita-jātiyo}

\begin{twochants}
Yasmiṃ padese kappeti & vāsaṃ paṇḍita-jātiyo\\
Sīlavant'ettha bhojetvā & saññate brahma-cārino\\
Yā tattha devatā āsuṃ & tāsaṃ dakkhiṇam-ādise\\
Tā pūjitā pūjayanti & mānitā mānayanti naṃ\\
Tato naṃ anukampanti & mātā puttaṃ va orasaṃ\\
Devatā'nukampito poso & sadā bhadrāni passati
\end{twochants}

\suttaRef{Vin.I.229f}

\vspace*{-\baselineskip}

\section{Adāsi-me ādi-gāthā (Tiro-kuḍḍa-kaṇḍaṃ)}

\firstline{Adāsi me akāsi me}
\firstline{Ayañ-ca kho dakkhiṇā dinnā}

\sidepar{\vspace*{-\onelineskip}\vspace*{0.4pt}\pointerMark}% Ayañ-ca kho dakkhiṇā dinnā
\begin{twochants}
Adāsi me akāsi me & ñāti-mittā sakhā ca me\\
Petānaṃ dakkhiṇaṃ dajjā & pubbe katam-anussaraṃ\\
Na hi ruṇṇaṃ vā soko vā & yā v'aññā paridevanā\\
Na taṃ petānam-atthāya & evaṃ tiṭṭhanti ñātayo\\
Ayañ-ca kho dakkhiṇā dinnā & saṅghamhi supatiṭṭhitā\\
Dīgha-rattaṃ hitāy'assa & ṭhānaso upakappati\\
\end{twochants}

So ñāti-dhammo ca ayaṃ nidassito,\\
Petāna'pūjā ca katā uḷārā;\\
Balañ-ca bhikkhūnam-anuppadinnaṃ,\\
Tumhehi puññaṃ pasutaṃ anappakan-ti.

\suttaRef{Khp.VII.v10-13}

\section{Kāla-dāna-sutta-gāthā}

\firstline{Kāle dadanti sapaññā vadaññū vīta-maccharā}

\begin{twochants}
Kāle dadanti sapaññā & vadaññū vīta-maccharā\\
Kālena dinnaṃ ariyesu & uju-bhūtesu tādisu\\
Vippasanna-manā tassa & vipulā hoti dakkhiṇā\\
Ye tattha anumodanti & veyyāvaccaṃ karonti vā\\
Na tena dakkhiṇā onā & te pi puññassa bhāgino\\
Tasmā dade appaṭivāna-citto & yattha dinnaṃ mahapphalaṃ\\
Puññāni para-lokasmiṃ & patiṭṭhā honti pāṇinan-ti
\end{twochants}

\suttaRef{A.III.41}

\section{Vihāradāna-gāthā}

\begin{twochants}
  Sītaṃ uṇhaṃ paṭihanti & tato vāḷamigāni ca;\\
  sariṃsape ca makase & sisire cāpi vuṭṭhiyo.\\
  Tato vātātapo ghoro & sañjāto paṭihaññati.\\
  Leṇatthañ ca sukhatthañ ca & jhāyituñ ca vipassituṃ.\\
  Vihāradānaṃ saṅghassa & aggaṃ buddhehi vaṇṇitaṃ;\\
  Tasmā hi paṇḍito poso & sampassaṃ attham attano.\\
  Vihāre kāraye ramme & vāsayettha bahu-ssute;\\
  Tesaṃ annañ ca pānañ ca & vattha-senāsanāni ca;\\
  Dadeyya uju-bhūtesu & vippasannena cetasā.\\
  Te tassa dhammaṃ desenti & sabbadukkhāpanūdanaṃ\\
  Yaṃ so dhammaṃ idhaññāya & parinibbātayanāsavo ti.
\end{twochants}

% Source: Chomtong chanting book

