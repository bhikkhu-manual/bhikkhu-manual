\chapter{Reflection on the Four Requisites}

\begin{leader}
  [Handa mayaṃ taṅkhaṇika-paccavekkhaṇa-pāṭhaṃ bhaṇāmase]
\end{leader}

[Paṭisaṅkhā] yoniso cīvaraṃ paṭisevāmi, yāvadeva sītassa\\
paṭighātāya, uṇhassa paṭighātāya, ḍaṃsa-makasa-vātātapa-siriṃsapa-\\
-samphassānaṃ paṭighātāya, yāvadeva hirikopina-paṭicchādanatthaṃ

[Paṭisaṅkhā] yoniso piṇḍapātaṃ paṭisevāmi, neva davāya, na madāya, na maṇḍanāya, na vibhūsanāya, yāvadeva imassa kāyassa ṭhitiyā, yāpanāya, vihiṃsūparatiyā, brahmacariyānuggahāya, iti purāṇañca vedanaṃ paṭihaṅkhāmi, navañca vedanaṃ na uppādessāmi, yātrā ca me bhavissati anavajjatā ca phāsuvihāro cā'ti

[Paṭisaṅkhā] yoniso senāsanaṃ paṭisevāmi, yāvadeva sītassa\\
paṭighātāya, uṇhassa paṭighātāya, ḍaṃsa-makasa-vātātapa-siriṃsapa-\\
-samphassānaṃ paṭighātāya, yāvadeva utuparissaya vinodanaṃ paṭisallānārāmatthaṃ

[Paṭisaṅkhā] yoniso gilāna-paccaya-bhesajja-parikkhāraṃ paṭisevāmi, yāvadeva uppannānaṃ veyyābādhikānaṃ vedanānaṃ paṭighātāya, abyāpajjha-paramatāyā'ti

\chapter{Five Subjects for Frequent Recollection}

\begin{leader}
  [Handa mayaṃ abhiṇha-paccavekkhaṇa-pāṭhaṃ bhaṇāmase]
\end{leader}

\sidepar{Men Chant}%
[Jarā-dhammomhi] jaraṃ anatīto

\sidepar{Women Chant}%
[Jarā-dhammāmhi] jaraṃ anatītā

\sidepar{m.}%
Byādhi-dhammomhi byādhiṃ anatīto

\sidepar{w.}%
Byādhi-dhammāmhi byādhiṃ anatītā

\sidepar{m.}%
Maraṇa-dhammomhi maraṇaṃ anatīto

\sidepar{w.}%
Maraṇa-dhammāmhi maraṇaṃ anatītā

Sabbehi me piyehi manāpehi nānābhāvo vinābhāvo

\sidepar{m.}%
Kammassakomhi kammadāyādo kammayoni kammabandhu kammapaṭisaraṇo\\
Yaṃ kammaṃ karissāmi, kalyāṇaṃ vā pāpakaṃ vā, tassa dāyādo bhavissāmi

\sidepar{w.}%
Kammassakāmhi kammadāyādā kammayoni kammabandhu kammapaṭisaraṇā\\
Yaṃ kammaṃ karissāmi, kalyāṇaṃ vā pāpakaṃ vā, tassa dāyādā bhavissāmi

Evaṃ amhehi abhiṇhaṃ paccavekkhitabbaṃ

\chapter{Patti-dāna-gāthā}

\begin{leader}
  [Handa mayaṃ patti-dāna-gāthāyo bhaṇāmase.]
\end{leader}

Yā devatā santi vihāra-vāsinī\\
Thūpe ghare bodhi-ghare tahiṃ tahiṃ\\
Tā dhamma-dānena bhavantu pūjitā\\
Sotthiṃ karonte’dha vihāra-maṇḍale\\
Therā ca majjhā navakā ca bhikkhavo\\
Sārāmikā dāna-patī upāsakā\\
Gāmā ca desā nigamā ca issarā\\
Sappāṇa-bhūtā sukhitā bhavantu te\\
Jalābu-jā ye pi ca aṇḍa-sambhavā\\
Saṃseda-jātā atha-v-opapātikā\\
Niyyānikaṃ dhamma-varaṃ paṭicca te\\
Sabbe pi dukkhassa karontu saṅkhayaṃ.\\
Ṭhātu ciraṃ sataṃ dhammo\\
Dhamma-dharā ca puggalā\\
Saṅgho hotu samaggo va\\
Atthāya ca hitāya ca\\
Amhe rakkhatu saddhammo\\
Sabbe pi dhamma-cārino\\
Vuḍḍhiṃ sampāpuṇeyyāma\\
Dhamme ariyappavedite.

(♦) Pasannā hontu sabbe pi\\
Pāṇino Buddha-sāsane.\\
Sammā-dhāraṃ pavecchanto\\
Kāle devo pavassatu.\\
Vuḍḍhi-bhāvāya sattānaṃ\\
Samiddhaṃ netu medaniṃ.\\
Mātā-pitā ca atra-jaṃ\\
Niccaṃ rakkhanti puttakaṃ.\\
Evaṃ dhammena rājāno\\
Pajaṃ rakkhantu sabbadā.

